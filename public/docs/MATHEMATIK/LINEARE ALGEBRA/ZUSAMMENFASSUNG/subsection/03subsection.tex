\section{Das Spatprodukt}
Betrachte man drei Vektoren $\overrightarrow{OA}$, $\overrightarrow{OB}$ und $\overrightarrow{OC}$. Das Parallelepiped aufgespannt durch die drei Vektoren nennt man Spat. Für die Vektoren heisst die Zahl Spatprodukt.
\begin{equation}
\boxed{\det\left(\overrightarrow{OA}, \overrightarrow{OB}, \overrightarrow{OC}\right)=\overrightarrow{OA}\bullet \left(\overrightarrow{OB}\times \overrightarrow{OC}\right)}
\end{equation}
Der Vektor $\overrightarrow{OB}\times \overrightarrow{OC}$ ist senkrecht auf das Parallelogram aufgespannt durch $\overrightarrow{OB}$ und $\overrightarrow{OC}$. Die Fläche des Parallelograms ist $\Big\vert\overrightarrow{OB}\times \overrightarrow{OC}\Big\vert$.
\newline\newline
Durch das Skalarprodukt wird der Schatten von $\overrightarrow{OA}$ auf $\overrightarrow{OB}\times \overrightarrow{OC}$ berechnet
\begin{equation}
\boxed{\overrightarrow{OA}\bullet \left(\overrightarrow{OB}\times \overrightarrow{OC}\right)=\cos\left(\varphi\right)\cdot \Big\vert\overrightarrow{OA}\Big\vert\cdot \Big\vert\overrightarrow{OB}\times \overrightarrow{OC}\Big\vert}
\end{equation}
Der Betrag des Spatprodukts ist gleich dem Volumen des Spats. Dies ist genau die Höhe des Körpers aufgespannt durch die Vektoren und wird mit der Grundfläche multipliziert.
\newline \newline
Das Volumen des Spats ist unabhängig von der Reihenfolge. Nur das Vorzeichen kann eventuell ändern, wenn die Reihenfolge vertauscht wird.
\begin{equation}
\boxed{\Big[\overrightarrow{OA}, \overrightarrow{OB}, \overrightarrow{OC}\Big]=\Big[\overrightarrow{OC}, \overrightarrow{OA}, \overrightarrow{OB}\Big]=\Big[\overrightarrow{OB}, \overrightarrow{OC}, \overrightarrow{OA}\Big]}
\end{equation}
\begin{equation}
\boxed{\Big[\overrightarrow{OA}, \overrightarrow{OB}, \overrightarrow{OC}\Big]=-\Big[\overrightarrow{OA}, \overrightarrow{OC}, \overrightarrow{OB}\Big]}
\end{equation}
Falls das Volumen des Spats bzw. der Betrag des Spatprodukts null ist, sind die Vektoren linear abhängig und liegen auf einer Ebene.
\lstinputlisting[language=Matlab, caption={Das Spatprodukt}]{../programs/spatproduct.m}