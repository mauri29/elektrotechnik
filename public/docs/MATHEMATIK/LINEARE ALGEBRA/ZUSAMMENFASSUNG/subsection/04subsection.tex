\section{Geometrie}
\subsection{Die Gerade}
Die \textbf{Parameterdarstellung} der Gerade, mit $\overrightarrow{OA}$ als Bezugspunkt und $\overrightarrow{AB}$ als Richtungsvektor mit $\lambda\in \mathbb{R}$ als Parameter, lautet
\begin{equation}
\boxed{g:\overrightarrow{r}=\overrightarrow{OA}+\lambda\cdot \overrightarrow{AB}}
\end{equation}
Die \textbf{Hessesche Normalform} der Gerade, mit $\overrightarrow{OA}$ als Bezugspunkt und $\overrightarrow{n}_E$ als Normalensvektor der Geraden, lautet
\begin{equation}
\boxed{\left(\overrightarrow{r}-\overrightarrow{OA}\right)\bullet \overrightarrow{n}_E=0}
\end{equation}
Die \textbf{Koordinatenform} der Gerade, mit $n_i$ Komponenten des Normalenvektors $\overrightarrow{n}_E$ zur Richtungsvektor der Gerade, wobei diese Form zweidimensionell ist, lautet
\begin{equation}
\boxed{n_x\cdot x+n_y\cdot y+d=0}
\end{equation}
Von der parameterform in die Hessesche Normalenform der Gerade muss man aus dem Richtungsvektor $\overrightarrow{AB}$ der Gerade den Normalenvektor $\overrightarrow{n}_E$ erzeugen.\newline\newline
Von der hessesche Normalenform im Koordinatenform muss man das Skalarprodukt und Terme berechnen. 
\newline\newline
Aus dem Koordinatenform in die Hessesche Normalenform muss aus den Koeffizienten den $\overrightarrow{n}_E$ auslesen. $\overrightarrow{OA}$ generieren durch die Wahl einer Koordinate und die andere mit der Koordinatengleichung berehcnen. 
\subsection{Die Ebene}
Die Parameterdarstellung der Ebene, wobei $\overrightarrow{OA}$ der Bezugspunkt und $\overrightarrow{AB}$ und $\overrightarrow{AC}$ die Richtungsvektoren mit $\lambda$, $\mu\in\mathbb{R}$ als Parametern, ausserdem die Richtungsvektoren müssen nicht unbedingt zueinander senkrecht bzw. nciht kollinear liegen, lautet
\begin{equation}
\boxed{E:\overrightarrow{r}=\overrightarrow{OA}+\lambda\cdot \overrightarrow{AB}+\mu\cdot \overrightarrow{AC}}
\end{equation}
Für die Ebene definiert man den Normalenvektor $\overrightarrow{n}'_E$ und den Ortsvektor des Bezugspunktes$\overrightarrow{OA}$. Somit wird die Normalenform der Ebene erzeugt.
\begin{equation}
\boxed{\left(\overrightarrow{r}-\overrightarrow{OA}\right)\bullet\overrightarrow{n}'_E=\left[\begin{pmatrix}x\\y\\z\end{pmatrix}-\overrightarrow{OA}\right]\bullet \left(\overrightarrow{AB}\times \overrightarrow{AC}\right)=0}
\end{equation}
Die Ebene $E$ ist definiert durch den normierten Normalenvektor $\overrightarrow{n}_E$ und den Bezugspunkt $\overrightarrow{OA}$. Somit wird die Hessesche Normalenform erzeugt.
\begin{equation} 
\boxed{\left(\overrightarrow{r}-\overrightarrow{OA}\right)\bullet \dfrac{\overrightarrow{n}'_E}{\Big\vert \overrightarrow{n}'_E\Big\vert}=\left[\begin{pmatrix}x\\y\\z\end{pmatrix}-\overrightarrow{OA}\right]\bullet \dfrac{\overrightarrow{AB}\times \overrightarrow{AC}}{\Big\vert\overrightarrow{AB}\times \overrightarrow{AC}\Big\vert}=0}
\end{equation} 
Die Ebene ist definiert durch die vier Parameter $n_1'$, $n_2'$, $n_3'$ und $n_4'$ und es gilt $(n_1')^2+(n_2')^2+(n_3')^2=1$. Der Wert $\vert n_4'\vert$ ist der Abstand zum Ursprung. Für den allgemeinen Punkt $\overrightarrow{r}$ in der Ebene gilt
\begin{equation}
\boxed{E:n_1'\cdot x+n_2'\cdot y+n_3'\cdot z+n_4'=0}
\end{equation}
Die Achsenabschnittsform der Ebene lautet
\begin{equation}
\boxed{E:\dfrac{1}{c_1}x+\dfrac{1}{c_2}y+\dfrac{1}{c_3}z-1=0}\quad \boxed{c_i=\dfrac{-n_4'}{n_i'}\quad \text{mit }i=1,2,3}
\end{equation}
Von der Koordinatenform in die Parameterform erfinde man drei Vektoren und erzeugt man daraus ein Bezugspunkt und zwei Richtungsvektoren.
\begin{equation}
\boxed{\overrightarrow{OA}=\begin{pmatrix}0\\0\\-n_4'/n_3'\end{pmatrix}, \overrightarrow{OB}=\begin{pmatrix}-n_4'/n_1'\\0\\0\end{pmatrix}, \overrightarrow{OC}=\begin{pmatrix}0\\-n_4'/n_2'\\0\end{pmatrix}}
\end{equation}
Von der Koordinatenform in die Normalenform erfinde man einen Bezugspunkt und aus den Konstanten $n_i'$ deb Normalenvektor. Mit diesen Informationen kann man die Normalenform und die Hessesche Normalenform der Ebene erzeugen.
\begin{equation}
\boxed{\overrightarrow{OA}=\begin{pmatrix}0\\0\\-n_4'/n_3'\end{pmatrix},\quad \overrightarrow{n}'_E=\begin{pmatrix}n_1'\\n_2'\\n_3'\end{pmatrix}}
\end{equation}
Von der Parameterform in die Koordinatenform berechnet man aus den Richtungsvektoren den Normalenvektor. Die Komponenten des Normalenvektors $\overrightarrow{n}'_E$ und das Einsetzen des Bezugspunktes $\overrightarrow{OA}$ erzeugt die Koordinatenform der Ebene.
\newline\newline
Von der Parameterform in die Normalenform erzeugt man den Normalenvektor aus den Richtungsvektoren. Mit den Bezugspunkt und den Normalenvektor erzeugt man die Normalenform der Ebene.
\newline\newline
Von der Normalenform in die Koordinatenform multipliziert man alles aus und so wird die gewnschte Form erzeugt.
\newline\newline
Von der Normalenform in die Parameterform erhält man, indem man die Normalenform in die Koordinatenform umwandelt ud dann diese in die Parameterform.
\subsection{Abstände}
Der Abstand eines Punktes $\overrightarrow{OP}$ zur Gerade $g:\overrightarrow{OA}+\lambda\cdot \overrightarrow{AB}$ erhält man aus der Berechnung der Fläche des Parallelograms. 
\begin{equation}
\boxed{d=\dfrac{\Big\vert\overrightarrow{AB}\times\left(\overrightarrow{OP}-\overrightarrow{OA}\right)\Big\vert}{\Big\vert\overrightarrow{AB}\Big\vert}=\left\vert\dfrac{\left(\overrightarrow{OP}-\overrightarrow{OA}\right)\bullet \overrightarrow{n}'}{\Big\vert\overrightarrow{n}'\Big\vert}\right\vert}
\end{equation}
Für die Berechnung des Fusspunktes eines Punktes $\overrightarrow{OP}$ auf einer Gerade $g:\overrightarrow{OA}+\lambda\cdot \overrightarrow{AB}$ sei den Fusspunkt $\overrightarrow{OF}$ gegeben. Man zerlegt den Vektor $\overrightarrow{AP}$ erstens in einer parallelen Komponente mit Betrag $\Big\vert\overrightarrow{AF}\Big\vert$ und Richtung $\overrightarrow{AB}$ und zweitens in eine senkrechte Komponente mit Betrag $\Big\vert\overrightarrow{FP}\Big\vert$ mit Richtung senkrecht zu $\overrightarrow{AB}$.
\newline\newline
Die Projektion von $\overrightarrow{AP}$ auf $\overrightarrow{AB}$ und Fusspunkt lauten
\begin{equation}
\boxed{\overrightarrow{OF}=\overrightarrow{OA}+\overrightarrow{AF}=\overrightarrow{OA}+\left[\overrightarrow{AP}\bullet \dfrac{\overrightarrow{AB}}{\Big\vert\overrightarrow{AB}\Big\vert}\right]\cdot \dfrac{\overrightarrow{AB}}{\Big\vert\overrightarrow{AB}\Big\vert}}
\end{equation}
Das Lot ist der Vektor zwischen dem Punkt $\overrightarrow{OP}$ und der Fusspunkt $\overrightarrow{OF}$
\begin{equation}
\boxed{\overrightarrow{h}_{\overrightarrow{OP}}=\overrightarrow{OP}-\overrightarrow{OF}}
\end{equation}
Der Abstand des Fusspunktes zum Punkt $\overrightarrow{OP}$ übereinstimmt mit dem Abstand Punkt-Gerade
\begin{equation}
\boxed{d=\Big\vert\overrightarrow{h}_{\overrightarrow{OP}}\Big\vert}
\end{equation}
Der Abstand eines Punktes $\overrightarrow{OP}$ von der Ebene $E$ in Parameterdarstellung lautet
\begin{equation}
\boxed{d=\left(\overrightarrow{OP}-\overrightarrow{OA}\right)\bullet \dfrac{\overrightarrow{AB}\times \overrightarrow{AC}}{\Big\vert\overrightarrow{AB}\times \overrightarrow{AC}\Big\vert}=\left(\overrightarrow{OP}-\overrightarrow{OA}\right)\bullet \overrightarrow{n}_E}
\end{equation}
\begin{equation}
\boxed{\left(\overrightarrow{OP}-\overrightarrow{OA}\right)\bullet \overrightarrow{n}_E-d=0}
\end{equation}
Die Punkte $\overrightarrow{OP}$ auf der Ebene $E$ kann man berechnen, indem $d=0$, so wird die Hessesche Form der Ebene erzeugt.
\newline\newline
Für die Projektion eines Punktes an einer Ebene wird der Punkt $\overrightarrow{OP}$ durch $\overrightarrow{OP}'$ auf die Ebene $E$ mit Bezugspunkt $\overrightarrow{OC}$ projiziert.
\begin{equation}
\boxed{\overrightarrow{OP}'=\overrightarrow{OP}-\left(\overrightarrow{CP}\bullet \overrightarrow{n}_E\right)\cdot \overrightarrow{n}_E}
\end{equation}
Durch $\overrightarrow{OP}''$ wird der Punkt $\overrightarrow{OP}$ an der Ebene mit Bezugspunkt $\overrightarrow{OC}$ gespiegelt.
\begin{equation}
\boxed{\overrightarrow{OP}''=\overrightarrow{OP}-2\cdot \left(\overrightarrow{CP}\bullet \overrightarrow{n}_E\right)\cdot \overrightarrow{n}_E}
\end{equation}
