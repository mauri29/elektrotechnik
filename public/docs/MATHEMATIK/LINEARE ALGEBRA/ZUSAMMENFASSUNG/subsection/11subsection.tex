\section{Funktionen und Transformationen}
Die Funktion $f\left(x\right):\mathbb{R}\rightarrow \mathbb{R}$ kann wie folgt transformiert werden
\begin{equation}
\boxed{-f(x):=\text{ Spiegelung an der }x\text{-Achse}}
\end{equation}
\begin{equation}
\boxed{f(-x):=\text{ Spiegelung an der }y\text{-Achse}}
\end{equation}
\begin{equation}
\boxed{f(x)+c:=\text{ Verschiebung in positive }y\text{-Achse}}
\end{equation}
\begin{equation}
\boxed{f(x)-c:=\text{ Verschiebung in negative }y\text{-Achse}}
\end{equation}
\begin{equation}
\boxed{f(x-c):=\text{ Verschiebung in positive }x\text{-Achse}}
\end{equation}
\begin{equation}
\boxed{f(x+c):=\text{ Verschiebung in negative }x\text{-Achse}}
\end{equation}
\begin{equation}
\boxed{f(c\cdot x):=\text{ Stauchung in }x\text{-Richtung mit }(a>1)}
\end{equation}
\begin{equation}
\boxed{f(c\cdot x):=\text{ Streckung in }x\text{-Richtung mit }(0<a<1)}
\end{equation}
\begin{equation}
\boxed{c\cdot f(x):=\text{ Streckung in }y\text{-Richtung mit }(a>1)}
\end{equation}
\begin{equation}
\boxed{c\cdot f(x):=\text{ Stauchung in }y\text{-Richtung mit }(0<a<1)}
\end{equation}