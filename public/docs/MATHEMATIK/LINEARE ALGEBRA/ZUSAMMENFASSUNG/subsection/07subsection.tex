\section{Determinanten}
Die Determinante einer quadratischen Matrix $A\in \mathbb{R}^{n\times n}$ berechnet den Flächeninhalt eines Parallelograms aufgespannt durch die Vektoren
\begin{equation} 
\boxed{\left\vert\begin{matrix}a_{11}&a_{12}\\a_{21}&a_{22}\end{matrix}\right\vert=a_{11}\cdot a_{22}-a_{12}\cdot a_{21}}
\end{equation} 
Die Determinante einer quadratischen Matrix $A\in \mathbb{R}^{n\times n}$ benutzt die Berechnung von Untermatrizen. Die Matrix $A_{ik}$ ist die $\left(n-1\right)\times\left(n-1\right)$-Matrix und entsteht durch Streichen der $i$-ten Zeile und $k$-ten Spalte.  
\begin{equation}
\boxed{\begin{array}{lll}
\left\vert\begin{matrix}a_{11}&a_{12}&a_{13}\\a_{21}&a_{22}&a_{23}\\a_{31}&a_{32}&a_{33}\end{matrix}\right\vert&=&\displaystyle \sum_{k=1,\,i=1}^n\left(-1\right)^{k+i}a_{ik}\det\left(A_{ik}\right)\\
&=&\left(-1\right)^{1+1}a_{11}\left\vert\begin{matrix}a_{22}&a_{23}\\a_{32}&a_{33}\end{matrix}\right\vert+\left(-1\right)^{1+2}a_{12}\left\vert\begin{matrix}a_{21}&a_{23}\\a_{31}&a_{33}\end{matrix}\right\vert+\left(-1\right)^{1+3}a_{13}\left\vert\begin{matrix}a_{21}&a_{22}\\a_{31}&a_{32}\end{matrix}\right\vert
\end{array}} 
\end{equation}
Die Determinante einer Dreiecksmatrix ist das Produkt der Diagonalelemente.
\begin{equation}
\boxed{\left\vert\begin{matrix}a_{11}&a_{12}\\0&a_{22}\end{matrix}\right\vert=a_{11}\cdot a_{22}}
\end{equation}
\begin{equation}
\boxed{\left\vert\begin{matrix}a_{11}&a_{12}&a_{13}\\0&a_{22}&a_{23}\\0&0&a_{33}\end{matrix}\right\vert=\left\vert\begin{matrix}a_{11}&0&0\\a_{21}&a_{22}&0\\a_{31}&a_{32}&a_{33}\end{matrix}\right\vert=a_{11}a_{22}a_{33}}
\end{equation}
Wenn man eine Kante des Paralllograms um den Faktor $\lambda$ verlängert, muss die Fläche des Parallelograms (Determinante) um diesen Faktor anwachsen. Man nennt dies die Homogenität der Determinante. 
\newline\newline
Wenn eine Fläche aufgespannt durch zwei Vektoren $\overrightarrow{a}$ und $\overrightarrow{b}+\overrightarrow{c}$, dann muss sich aus geometrischen Gründen die Gesamtfläche zusammensetzen aus dem kleinen Flächen aufgespannt durch $\overrightarrow{a}$ und $\overrightarrow{b}$ bzw. $\overrightarrow{a}$ und $\overrightarrow{c}$ und man nennt dies die Additivität der Determinante.
\newline\newline
Die Linearität der Determinante gilt, wenn die Homogenität und die Additivität der Determinante gelten.
\begin{equation}
\boxed{\det\begin{pmatrix}\lambda a_1&b_1\\\lambda a_2&b_2\end{pmatrix}=\lambda\begin{pmatrix}a_1&b_1\\a_2&b_2\end{pmatrix}}
\end{equation}
\begin{equation}
\boxed{\det\begin{pmatrix}a_1+c_1&b_1\\a_2+c_2&b_2\end{pmatrix}=\det\begin{pmatrix}a_1&b_1\\a_2&b_2\end{pmatrix}+\det\begin{pmatrix}c_1&b_1\\c_2&b_2\end{pmatrix}}
\end{equation}
Die Linearität dreier Vektoren einer Matrix gehört der Linearität des Spatproduktes
\begin{equation}
\boxed{\det\begin{pmatrix}\lambda\overrightarrow{a}&\overrightarrow{b}&\overrightarrow{c}\end{pmatrix}=\lambda\cdot \det\begin{pmatrix}\overrightarrow{a}&\overrightarrow{b}&\overrightarrow{c}\end{pmatrix}}
\end{equation}
\begin{equation}
\boxed{\det\begin{pmatrix}\overrightarrow{a}+\overrightarrow{d}&\overrightarrow{b}&\overrightarrow{c}\end{pmatrix}=\det\begin{pmatrix}\overrightarrow{a}&\overrightarrow{b}&\overrightarrow{c}\end{pmatrix}+\det\begin{pmatrix}\overrightarrow{d}&\overrightarrow{b}&\overrightarrow{c}\end{pmatrix}}
\end{equation}
Folgende sind Eigenschaften der Determinante im allgemeinen Sinne
\begin{enumerate}[$(i)$]
\item Werden alle Elemente einer Zeile/Spalte mit einer Konstanten $\lambda$ multipliziert, dann multipliziert sich die Determinante mit $1/\lambda$
\item Vertauschung zweier Zeilen/Spalten ändert das Vorzeichen der Determinante.
\item Die Determinante ändert sich nicht, wenn das Vielfache einer Zeile/Spalte zu einer anderen Zeile/Spalte addiert wird.
\item Die Determinante ist Null, wenn entweder alle Elemente einer Zeile/Spalte Null sind oder zwei Zeilen/Spalten identisch sind.
\item $\det\left(A^T\right)=\det\left(A\right)$
\item $\det\left(A\odot B\right)=\det\left(A\right)\cdot \det\left(B\right)$
\item $\det\left(A^{-1}\right)=\dfrac{1}{\det\left(A\right)}$
\item $\det\left(I\right)=1$
\item $\det\left(\lambda\cdot A\right)=\lambda^n\cdot \det\left(A\right)$
\end{enumerate}