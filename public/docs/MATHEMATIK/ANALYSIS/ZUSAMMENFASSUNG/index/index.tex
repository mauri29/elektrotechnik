\documentclass[9pt, a4paper]{scrreprt}
%%%%%%%%%%%%%%%%%%%%%%%% INPUT %%%%%%%%%%%%%%%%%%%%%%%%
\usepackage[utf8]{inputenc}
\usepackage[german]{babel}

%%%%%%%%%%%%%%%%%%%%%%%% FONTS %%%%%%%%%%%%%%%%%%%%%%%%
\usepackage[T1]{fontenc}
\usepackage{charter}
%\usepackage{mathptmx} %Times
%\usepackage{mathpazo} %Palatino
%\usepackage{helvet} %Helvetica
%\usepackage{avant} %Avant Garde
%\usepackage{courier} %Courier
%\usepackage{chancery} %Zapf Chancery

%%%%%%%%%%%%%%%%%%%%%%%% MARGINS %%%%%%%%%%%%%%%%%%%%%%%%
\usepackage[left=1cm,right=1cm,top=1cm,bottom=1cm,includeheadfoot, landscape]{geometry}
\usepackage{multicol}
\usepackage{graphicx}
%\usepackage{hyperref}
\usepackage{xcolor}
\usepackage{wrapfig}
\usepackage{enumerate}
\usepackage{comment} 
\usepackage[explicit]{titlesec}
\usepackage{float}
\usepackage[margin=1cm, justification=centering, singlelinecheck=no,tablename=Tab., figurename=Abb.]{caption}
\usepackage{subfigure}
\restylefloat{figure}
%%%%%%%%%%%%%%%%%%%%%%%% MATH %%%%%%%%%%%%%%%%%%%%%%%%
\usepackage{amssymb}
\usepackage{amsmath}
\usepackage{cancel}
\usepackage{esint}
\DeclareMathOperator{\arsinh}{arsinh}
\DeclareMathOperator{\arccot}{arccot}

%%%%%%%%%%%%%%%%%%%%%%%% COLORS %%%%%%%%%%%%%%%%%%%%%%%%
\usepackage{sectsty}
%\chapterfont{\color{cyan}}
\sectionfont{\color{orange}}
\subsectionfont{\color{violet}}

%%%%%%%%%%%%%%%%%%%%%%%% BOLD %%%%%%%%%%%%%%%%%%%%%%%%
\let\oldtextbf\textbf
\renewcommand{\textbf}[1]{\textcolor{cyan}{\oldtextbf{#1}}}

%%%%%%%%%%%%%%%%%%%%%%%% CHAPTER LAYOUT %%%%%%%%%%%%%%%%%%%%%%%%
\titleformat{\chapter}
  {\normalfont\Large\bfseries\color{cyan}}{\thechapter \quad \MakeUppercase{#1}}{.5em}{\vspace{.5ex}}[\titlerule]
\titlespacing*{\chapter}
  {0pt}{0pt}{15pt}

\makeatletter
\renewcommand*{\chapterformat}{%
  \mbox{\chapapp~\thechapter\autodot:\enskip}%
}
\makeatother
%%%%%%%%%%%%%%%%%%%%%%%% BOXED %%%%%%%%%%%%%%%%%%%%%%%%
\newcommand*{\boxedcolor}{cyan}
\makeatletter
\renewcommand{\boxed}[1]{\textcolor{\boxedcolor}{%
  \fbox{\normalcolor\m@th$\displaystyle#1$}}}

%%%%%%%%%%%%%%%%%%%%%%%% LSTLISTINGS %%%%%%%%%%%%%%%%%%%%%%%%
\usepackage{listings} 
\lstset{numbers=left, numberstyle=\tiny, numbersep=10pt} \lstset{language=Java} % Code einbinden

\lstdefinestyle{customc}{
  belowcaptionskip=1\baselineskip,
  breaklines=true,
  frame=L,
  xleftmargin=\parindent,
  language=C,
  showstringspaces=false,
  basicstyle=\footnotesize\ttfamily,
  keywordstyle=\bfseries\color{green!40!black},
  commentstyle=\itshape\color{purple!40!black},
  identifierstyle=\color{blue},
  stringstyle=\color{orange},
}

\lstdefinestyle{customasm}{
  belowcaptionskip=1\baselineskip,
  frame=L,
  xleftmargin=\parindent,
  language=[x86masm]Assembler,
  basicstyle=\footnotesize\ttfamily,
  commentstyle=\itshape\color{purple!40!black},
}

\lstset{escapechar=@,style=customc}

%%%%%%%%%%%%%%%%%%%%%%%% DOCUMENT %%%%%%%%%%%%%%%%%%%%%%%%
\begin{document}
\begin{multicols}{2}
\setlength{\columnseprule}{0.4pt}
\chapter{Grundlagen}
\section{Lage eines Massenpunktes}
\section{Physikalische Grössen}
\subsection{Zahlwert und Einheit}
Eine physikalische Grösse beinhaltet eine Zahl und eine Einheit. Beinhaltet eine physikalische Grösse einen grossen Zahlenwert, so kann man diese als ein Vielfaches dieser Einheit ausdrücken. Die Dimension gibt Auskunft auf eine detaillierte Charakterisierung der Grösse wie die Höhe, der Abstand der die Strecke mit Einheit Meter.
\subsection{Grundeinheiten und abgeleitete Einheiten}
Die Grundeinheiten besteht aus sieben Grundeinheiten: \textbf{Länge} (Meter $\text{m}$), \textbf{Masse} (Kilogramm $\text{kg}$), Zeit (Sekunde $\text{s}$), \textbf{Stromstärke} (Ampere $\text{A}$), \textbf{Temperatur} (Kelvin $\text{K}$), \textbf{Stoffmenge} (Mol $\text{mol}$) und \textbf{Lichtstärke} (Candela $\text{cd}$).
\newline\newline
Die abgeleitete Einheiten entstehen durch Beziehungen zwischen der Grundeinheiten wie die Geschwindigkeit oder die Beschleunigung.
\newline\newline
Werden Gleichungen mit den Grundeinheiten gerechnet, so wird das Resultat auch in einer Grundeinheit ausgedrückt. Das Ziel besteht auch darin, Resultate mit Hilfe von Zehnerpotenzen zu schreiben.
\begin{table}[H]
\centering
\begin{tabular}{llll}
\hline
Grösse&Zeichen&Name&Symbol\\\hline
Länge&$l$&Meter&$\text{m}$\\
Masse&$m$&Kilogramm&$\text{kg}$\\
Zeit&$t$&Sekunde&$\text{s}$\\
Stromstärke&$I, i$&Ampere&$\text{A}$\\\hline
\end{tabular}
\caption{Grundeinheiten}
\end{table}
\begin{table}[H]
\centering
\begin{tabular}{lllll}
\hline
Grösse&Zeichen&Name&Symbol&Ausdruck\\\hline
Kraft&$F$&Newton&$\text{N}$&$\text{N}=\text{m}\,\text{kg}\, \text{s}^{-2}$\\
Leistung&$P$&Watt&$\text{W}$&$\text{W}=\text{V}\,\text{A}=\text{N}\,\text{m}\,\text{s}^{-1}$\\
Arbeit, Energie&$W$&Joule&$\text{J}$&$\text{J}=\text{W}\,\text{s}=\text{N}\,\text{m}$\\
Spannung&$U, u$&Volt&$\text{V}$&$\text{V}=\text{W}\,\text{A}^{-1}$\\
Widerstand&$R$&Ohm&$\Omega$&$\Omega=\text{V}\,\text{A}^{-1}$\\
Spezifischer Widerstand&$\rho$&&$\Omega\,\text{m}$&$\Omega\,\text{m}=\text{V}\,\text{m}\,\text{A}^{-1}$\\
Leitwert&$G$&Siemens&$\text{S}$&$\text{S}=\Omega^{-1}$\\
Spezifische Leitfähigkeit&$\sigma$&&$\text{S}\,\text{m}^{-1}$&$\text{S}\,\text{m}^{-1}=\Omega^{-1}\,\text{m}^{-1}$\\
Ladung&$Q$&Coulomb&$\text{C}$&$\text{C}=\text{A}\,\text{s}$\\
Elektr. Verschiebungsdichte&$D$&&$\text{A}\,\text{s}\,\text{m}^2$&$\text{C}\,\text{m}^{-2}=\text{A}\,\text{s}\,\text{m}^{-2}$\\
Elektr. Feldstärke&$E$&&$\text{V}\,\text{m}^{-1}$&\\
Kapazität&$C$&Farad&F&$\text{F}=\text{A}\,\text{s}\,\text{V}^{-1}=\text{s}\,\Omega^{-1}$\\
Induktionsfluss&$\phi$&Weber&$\text{Wb}$&$\text{Wb}=\text{V}\,\text{s}$\\
Magn. Induktionsdichte&$B$&Tesla&$\text{T}$&$\text{T}=\text{V}\,\text{s}\,\text{m}^{-2}$\\
Magn. Feldstärke&$H$&&$\text{A}\,\text{m}^{-1}$&\\
Induktivität&$L$&Henry&$\text{H}$&$\text{H}=\text{V}\,\text{s}\,\text{A}^{-1}=\Omega\,\text{s}$\\\hline
\end{tabular}
\caption{Abgeleitete Einheiten}
\end{table}
\begin{table}[H]
\centering
\begin{tabular}{ll}
\hline
Definition&Wert\\\hline
Lichtgeschwindigkeit im Vakuum&$c_0=299'792'458\,\text{m}\,\text{s}^{-1}$\\
Elementarladung&$e=1.6022\cdot 10^{-19}\,\text{C}$\\
Ruhemasse Elektron&$m_0=9.1096\cdot 10^{-31}\,\text{kg}$\\
Ruhemasse Proton&$m_p=1.6726\cdot 10^{-27}\,\text{kg}$\\
Permeabilität im Vakuum&$\mu_0=4\pi\cdot 10^{-7}\,\text{V}\,\text{s}\,\text{A}^{-1}\,\text{m}^{-1}$\\
Permittivität im Vakuum&$\epsilon_0=c_0^{-2}\mu_0^{-1}=8.85419\cdot 10^{-12}\,\text{A}\,\text{s}\,\text{V}^{-1}\,\text{m}^{-1}$\\
Wellenimpendanz des freien Raumes&$\nu_0=\sqrt{\mu_0/\epsilon_0}=376.73\,\Omega$\\\hline
\end{tabular}
\caption{Universelle Konstanten}
\end{table}
\subsection{Skalare und vektorielle Grössen}
Zahlenwerte mit Einheiten bezeichnet man als skalare Grössen. Zahlenwerte mit Enheiten, die in einer bestimmten Richtung des Raumes wirken bezeichnet man als vektorielle Grössen und haben einen Vektorpfeil über das Symbol und ihr Betrag ist die Länge des Vektors.
\section{Elektrizität und ihre Wirkungen}
\subsection{Elektrische Ladung und elektrischer Strom}
Die elektrische Ladung $Q$ ist eine physikalische Grösse und benötigt einen Träger. Der Raum befindet sich in einem elektrischen Feld. Körper, die eine elektrische Ladung tragen, üben eine Kraftwirkung aufeinander aus.
\newline\newline
Elektrische Ladungen könenn sich dabei anziehen oder abstossen. Man unterscheidet zwischen positiver und negativer Ladung. Gleichnamige Ladungen stossen sich ab, während ungleichnamige Ladungen ziehen sich an. Elektrizität entsteht durch Trennen von Ladungen verschiedenen Vorzeichens.
\newline\newline
Im elektrischen neutralen Zustand heben sich die Wirkungen positiver und negativer Ladungen gegenseitig auf. Elektrische Ladungen lassen sich durch Berührung übertragen. Die Elektrizität besteht aus Elektronen.
\newline\newline
Elektrizitätsträger können sich je nach Material übertragen. Ladungsträger bewegen sich in einem Leitungsstrom bzw. elektrischen Strom. Dieser elektrischen Strom ist das Verhältnis der elektrischen Ladung pro Zeit.
\begin{equation}
\boxed{I=\dfrac{Q}{t}}
\end{equation}
Bewegte Ladungen haben thermische (Leiter erwärmt sich bis zum Schmelztemperatur), magnetische (Kräfte entstehen durch Magnete auf Leiter) und chemische Wirkungen (Durch Stromvorgang werden Stoffe verändert).
\subsection{Aufbau der Materie}
Moleküle können in Atome aufgeteilt werden. Um den positiven Atomkern kreisen die negativ geladenen Elektronen. Diese Elektronen bilden die Elektronenhülle, welche Elektronen durch ihre Energie zu Gruppen (Elektronenschalen) aufgeteilt werden. Ein Elektron kann sich nur auf eine Quantenbahn aufhalten. Ein Atom ist elektrisch neutral. Die Elektronen in der äusseren Schale sind Valenzelektronen und bestimmen das chemische Verhalten des Atoms.
\newline\newline
Das Atomkern besteht aus Protonen und Neutronen bzw. aus Nukleonen. Neutronen sind für die Kernspaltung von grösser Bedeutung. Die Beeinflussung der Elektronenhülle erfolgt durch Ionisierungsenergie und bildet Ionen durch Elektronenabgabe oder -zugabe. Ein Ion ist ein elektrisch geladenes Atom.
\subsection{Leiter und Nichtteiler}
Materialien werden in Leiter und Nichtteiler unterteilt. Zu den Leiter zählen die Metalle, aber verändern nicht das Material durch Stromdurchgang. Säuren, Basen und Salzlösungen werden beim Stromdurchgang verändert.
\newline\newline
Zu den Nichtleiter zählen Gummi, Seide, Kunststoffe, Porzellan, Glas, Glimmer, usw. In diesen Stoffen stehen nahezu keine Elektronen zur Verfügung.




\section{Bewegung eines Massenpunktes}
\section{Elemente der Programmiersprache Java}
\subsection{Bytecode}
Der Java-Compiler erzeugt aus den Quellcode-Dateien den so genannten \textbf{Bytecode}. Dieser Code ist binär und Ausgangspunkt für die virtuelle Machine zur Ausführung. Der Bytecode ist wie ein Prozessor, der Anweiungen wie arithmetische Operationen, Sprünge und Weiteres kennt.
\subsection{Java Virtual Machine}
Die Java Virtual Machine (JVM) kümmert sich um den Bytecode, den Quellcode auszuführen. Die Laufzeitumgebung lädt den Bytecode, prüft ihn und führt ihn in einer kontrollierten Umgebung aus. Java ist Plattform- und Betriebssystemunabhängig. Zu der JVM und der Programmiersprache kommen Standardbibliotheken für Datenstrukturen, Zeichenkettenverarbeitung, Datumverarbeitung, grafische Oberflächen, Ein- und Ausgabe, Netzwerkoperationen und mehr.
\subsection{Objektorientierung}
Eine Laufzeitumgebung eliminiert viele Fehler. Objektorientierte Programmierung versucht, die Komplexität des Softwareproblems besser zu modellieren. Menschen denken objektorientiert, darum Java bildet diese ab. Objekte bestehen aus \textbf{Eigenschaften}, also Dinge, die ein Objekt ``hat'' und ``kann''. Objekte entstehen aus \textbf{Klassen}, das sind Beschreibungen für den Aufbau von Objekten.
\\\\
Primitive Datentypen für numerische Zahlen oder Unicode-Zeichen werden nicht als Objekte betrachtet. Das \textbf{Java-Security-Modell} sicherstellt den Programmablauf. Der \textbf{Verifier} liest Code und überprüft die Korrektheit und Typsicherheit. Treten Sicherheitsprobleme auf, werden sie durch Exceptions zur Laufzeit gemeldet. Das Security-Manager überwächt Zugriffe auf das Dateisystem, die Netzwerk-Ports, externe Prozesse und weitere Systemressourcen.
\\\\
In Java gibt es keine Zeiger auf Speicherbereiche, dagegen führt Java \textbf{Referenzen} ein. Eine Referenz repräsentiert ein Objekt, und eine Variable speichert diese Referenz, sie wird Referenzvariable genannt. JVM verbindet die Referenz mit einem Speicherbereich und einem Referenztyp; der Zugriff, Dereferenzierung genannt, ist indirekt. Referenz und Speicherblock sind getrennt.
\\\\
In Java gibt es keine benutzerdefinierten überladenen Operatoren. Da das Operatorzeichen auf unterschiedlichen Datentypen gültig ist, nennt sich so ein Operator \textbf{Überladen}. Bei Zeichenketten werden Pluszeichen als \textbf{Konkatenation} angewendet. Java braucht keine \textbf{Präprozessoren}.
\subsection{Java Platform}
Mit dem Java Development Kit (JDK) lassen sich Java SE-Applikationen entwickeln. Dem JDK sind Hilfsprogramme beigelegt, die für die Java-Entwicklung nötig sind. Dazu zählen der essenzielle Compiler, aber auch andere Hilfsprogramme, etwa zur Signierung von Java-Archiven oder zum Start einer Management-Konsole.
\\\\
Das Java SE Runtime (JRE) enthält genau das, was zur Ausführung von Java-Programmen nötig ist. Die Distribution umfasst nur die JVM und Java-Bibliotheken, aber weder den Quellcode der Java-Bibliotheken noch Tools wie Management-Tools.
\subsection{Das erste Programm compilieren und testen}
\lstinputlisting[language=Java]{../../PROJEKTE/000001HelloWorld/src/Squares.java}
Ein Compiler übersetzt bzw. transformiert das geschriebene Programm in eine andere Repräsentation nämlich den Bytecode und erzeugt aus dem Program mit Endung \texttt{.java} die Datei \texttt{.class}, welche Bytecode enthält.
\\\\
Wenn der Compiler aufgrund eines syntaktischen Fehlers eine Übersetzung in Java-Bytecode nicht durchführen kann, spricht man von einem Compilerfehler.
\\\\
Eine Laufzeitumgebung liest die Bytecode-Datei Anweisung für Anweisung aus und führt sie auf den konkreten Mikroprozessor aus. Der Interpreter bringt das Programm zur Ausführung.
\\\\
Ein Java-Projekt braucht eine ordentliche Ordnerstruktur, und hier gibt es zur Organisation der Dateien unterschiedliche Ansätze. Die einfachste Form ist, Quellen, Klassendateien und Ressourcen in ein Verzeichnis zu setzen. Es gibt zwei Verzeichnisse \texttt{src} für die Quellen und \texttt{bin} für die erzeugten Klassendateien. Ein eigener Ordner \texttt{lib} ist sinnvoll für Java-Bibliotheken.
\\\\
Das Programm sitzt in einer Klasse, die drei Methoden enthält. Die Methode $\texttt{quadrat(int)}$, bekommt als Übergangsparameter eine ganze Zahl und berechnet daraus die Quadratzahl, die sie anschliessend zurückgibt. Eine weitere Methode übernimmt die Ausgabe der Quadratzahlen bis zu einer vorgegebenen Grenze. Die Methode \texttt{main()}, als Anfangspunkt, ruft die Methode \texttt{ausgabe(int)} auf.
\section{Imperative Sprachkonzepte}
\subsection{Elemente der Programmiersprache Java}
Unter dem Begriff \textbf{Semantik} versteht man die Lexikalik, Syntax und Semantik eines Programms. Der Compiler verläuft diese Schritte bevor er den Bytecode erzeugt.
\\\\
Ein \textbf{Token} ist eine lexikalische Einheit, die dem Compiler die Bausteine des Programms liefert. Der Compiler erkennt an der Grammatik einer Sprache, welche Folgen von Zeichen ein Token bilden.
\\\\
\textbf{Whitespaces} sind Leerzeichen, Tabulatoren, Zeilenvorschub und Seitenvorschubzeichen.
\\\\
Neben den Trennern gibt es noch zwölf ASCII-Zeichen geformte Tokens, die als \textbf{Separator} definiert werden: \texttt{( ) \{ \} [ ] ; , . ... @ ::}
\\\\
Für Variablen, Methoden, Klassen und Schnittstellen werden \textbf{Bezeichner}, auch \textbf{Identifizierer} genannt, vergeben. Unter \textbf{Variablen} sind dann Daten verfügbar. \textbf{Methoden} sind die Unterprogramme in objektorientierten Programmiersprachen, und \textbf{Klassen} sind die Bausteine objektorientierter Programme. Ein Bezeichner ist eine Folge von Zeichen, die fast beliebig sein kann. Die Zeichen sind Elemente aus dem Unicode-Zeichensatz. Der Bezeichner muss mit einem Java-Buchstaben beginnen. String ist eine Klasse und kein Datentyp.
\\\\
Ein Java-Buchstabe umfasst unsere lateinische Buchstaben ``A'' bis ``Z'', ``a'' bis ``z'', sondern auch viele Zeichen aus dem Unicode-Alphabet, den Unterstrich, Währungszeichen, griechische oder arabische Buchstaben, Akzente. Java unterscheidet zwischen Gross- und Kleinschreibung. Nicht erlaubt sind Zahlen am Anfang, Leerzeichen, Ausrufezeichen, reservierte Wörter oder reservierte Schlüsselwörter.
\\\\
Ein \textbf{Literal} ist ein konstanter Ausdruck wie die Wahrheitswerte \texttt{true} und \texttt{false}, Integrale Literale für Zahlen, Fliesskommaliterale, Zeichenliterale wie $\backslash$n, String-Literale für Zeichenketten wie ``Hello World'', Referenztypen wie \texttt{null}.
\\\\
Bestimmte Wörter sind reservierte Schlüsselwörter com Compiler besonders behandelt. Schlüsselwörter bestimmen die Sprache eines Compilers. Es können keine eigenen Schlüsselwörter hinzugefügt werden. Schlüsselwörter sind:
\\\\
\texttt{abstract}, \texttt{assert}, \texttt{boolean}, \texttt{break}, \texttt{byte}, \texttt{case}, \texttt{catch}, \texttt{char}, \texttt{class}, \texttt{const}, \texttt{continue}, \texttt{default}, \texttt{do}, \texttt{double}, \texttt{else}, \texttt{enum}, \texttt{extends}, \texttt{final}, \texttt{finally}, \texttt{float}, \texttt{for}, \texttt{goto}, \texttt{if}, \texttt{implements}, \texttt{import}, \texttt{instanceof}, \texttt{int}, \texttt{interface}, \texttt{long}, \texttt{native}, \texttt{new}, \texttt{package}, \texttt{private}, \texttt{protected}, \texttt{public}, \texttt{return}, \texttt{short}, \texttt{static}, \texttt{strictfp}, \texttt{super}, \texttt{switch}, \texttt{synchronized}, \texttt{this}, \texttt{throw}, \texttt{throws}, \texttt{transient}, \texttt{try}, \texttt{void}, \texttt{volatile}, \texttt{while}
\\\\
Der Compiler überliest alle Kommentare und die Trennzeichen bringen den Compiler von Token zu Token. \textbf{Zeilenkommentare} kann man mit Schrägsstrichen \boxed{\textbf{\texttt{//}}} und kommentieren den Rest einer Zeile bis Zeilenumbruchzeichen aus. \textbf{Blockkommentare} (``Wie'') kommentiert in Blöcke mit \boxed{\textbf{\texttt{/* */}}} aus. \textbf{Javadoc-Kommentare} (``Was'') sind besondere Blockkommentare mit \boxed{\textbf{\texttt{/** */}}} und beschreibt die Methode oder die Parameter, aus denen sich später die API generieren lässt. Kein Kommentar kommt in den Bytecode.
\subsection{Anweisungen}
Programme sind Ablauffolgen, die im Kern aus \textbf{Anweisungen} bestehen. Sie werden zu grösseren Bausteinen zusammengesetzt, den Methoden, die wiederum Klassen bilden. Klassen selbst werden in Paketen gesammelt, und eine Sammlung von Paketen wird als Java-Archiv ausgeliefert.
\\\\
Durch Anweisungen werden \textbf{Algorithmen} geschrieben. Anweisungen können Ausdrucksanweisungen für Zuweisungen oder Methodenaufrufe, auch  Fallunterscheidungen, oder Schleifen für Wiederholungen sein.
\\\\
Anweisungen müssen in einen Rahmen gepackt werden. Dieser Rahmen heisst \textbf{Kompilationseinheit} und deklariert eine Klasse mit ihren Methoden und Variablen. Anweisungen ausserhalb von Klassen sind nicht erlaubt. Der Klassenname ist ein Bezeichner und beinhaltet die gleiche Dateiname. Klassennamen beginnen mit Grossbuchstabe und Methoden sind kleingeschrieben. Zwischen den geschweiften Klammern folgen Deklarationen von Methoden und zwischen den Methoden die Anweisungen.
\\\\
Eine besondere Methode ist \boxed{\textbf{\texttt{public static void(String[] args)\{\}}}}. Die Methode ist für die Laufzeitumgebung etwas Besonders, denn beim Aufruf des Java-Interpreters mit einem Klassennamen wird diese Methode als Erstes ausgeführt. Demnach werden die Anweisungen ausgeführt, die innerhalb der geschweiften Klammern stehen. Der Parameter \texttt{args} wird immer verwendet.
\\\\
Haltet man sich nicht an die Syntax für den Startpunkt, so kann der Interpreter die Ausführung nicht beginnen und man hätte einen semantischen Fehler produziert, obwohl die Methode korrekt gebildet ist.
\\\\
Die Methode \boxed{\textbf{\texttt{println(...)}}} gibt Meldungen auf der Konsole aus. Innerhalb der Klammern können Argumente angegeben werden wie Zeichenketten oder \textbf{Strings} oder eine Folge von Buchstaben, Ziffern oder Sonderzeichen in doppelten Anführungszeichen. Die Methode \texttt{println(...)} gehört zum Typ \textbf{\texttt{out}} und diese zu \textbf{\texttt{System}}.
\lstinputlisting[language=Java]{../../PROJEKTE/000001HelloWorld/src/PrimeraClase.java}
Java erlaubt Methoden, die gleich heissen, denen aber unterschiedliche Dinge übergeben werden können; diese Methoden nennt man \textbf{überladen}. Viele \texttt{println()}-Methoden akzeptieren zahlartige Argumente und sind überladen.
\lstinputlisting[language=Java]{../../PROJEKTE/000001HelloWorld/src/OverloadedPrintln.java}
Die Methode \boxed{\textbf{\texttt{printf()}}} ermöglicht variable Argumentenlisten gemäss einer Formatierungsanweisung. Die Formatierungsanweisung \boxed{\textbf{\texttt{$\backslash$n}}} setzt einen Zeilenumbruch, \boxed{\textbf{\texttt{$\backslash$d}}} ist ein Platzhalter für eine ganze Zahl, \boxed{\textbf{\texttt{$\backslash$f}}} ist ein Platzhalter für eine Fliesskommazahl, \boxed{\textbf{\texttt{$\backslash$s}}} ist eine Zeichenkette oder etwas, was in einen String konvertiert werden soll.
\lstinputlisting[language=Java]{../../PROJEKTE/000001HelloWorld/src/VarArgs.java}
Methodenaufrufe lassen sich als Anweisungen einsetzen, wenn sie mit einem Semikolon abegschlossen sind, man spricht von einer \textbf{Ausdrucksanweisung} (expression statement). Jeder Methodenaufruf mit Semikolon bildet eine Ausdrucksanweisung. Dabei ist es egal, ob die Methode selbst eine Rückgabe liefert oder nicht.
\\\\
Die Methode \boxed{\textbf{\texttt{Math.random()}}} liefert eine Fliesskommazahl zwischen 0 (inklusiv) und 1 (exklusiv). In einer objektorientierte Programmiersprache sind alle Methoden an bestimmte Objekte mit einem Zustand gebunden. Alle Operationen und Zustände sind an Objekte bzw. Klassen gebunden. Der Aufruf einer Methode auf einem Objekt richtet die Anfrage genau an dieses bestimmte Objekt.
\\\\
Die Deklaration einer Klasse oder Methode kann einen oder mehrere \textbf{Modifizierer} enthalten, die zum Beispiel die Nutzung einschränken oder parallelen Zugriff synchronisieren. Der Modifizierer \boxed{\textbf{\texttt{public}}} ist ein Sichtbarkeitsmodifizierer. Er bestimmt, onb die Klasse bzw. die Methode für Programmcode anderer Klassen sichtbar ist oder nicht. Der Modifizierer \boxed{\textbf{\texttt{static}}} zwingt den Programmierer nicht dazu, vor dem Methodenaufruf ein Objekt der Klasse zu bilden. Dieser Modifizierer bestimmt die Eigenschaft, ob sich eine Methode nur über ein konkretes Objekt aufrufen lässt oder eine Eigenschaft der Klasse ist, sodass für den Aufruf kein Objekt der Klasse nötig wird.
\\\\
Ein \textbf{Block} fasst eine Gruppe von Anweisungen,die hintereinander ausgeführt werden. Ein Block \boxed{\textbf{\texttt{\{\}}}} ist eine Anweisung, die in geschweiften Klammern eine Folge von Anweisungen zu einer neuen Anweisung zusammenfasst. Ein Block kann überall dort verwendet werden, wo auch eine einzelne Anweisung stehen kann. Der neue Block hat jedoch eine Besonderheit in Bezug auf Variablen, da er einen lokalen Bereich für die darin befindlichen Anweisungen inklusive der Variablen bildet.
\\\\
Ein Block ohne Anweisung nennt sich ein leerer Block. Er verhält sich wie eine leere Anweisung, also wie ein Semikolon. Es gibt innere und äussere Blöcke. Blöcke fassen Anweisungen zusammen.
\section{Datentypen, Variablen und Zuweisungen}
Java speichert Variablen. Eine Variable ist ein reservierter Speicherbereich und belegt eine feste Anzahl von Bytes. Variablen und Ausdrücke haben einen \textbf{Datentyp} und einen \textbf{Datenwert}. Der Datentyp bestimmt die zulässigen Operationen. Java ist eine streng typisierte Programmiersprache. Datentypen werden unterteilt in \textbf{primitive Datentypen} (Zahlen, Unicode-Zeichen und Wahrheitswerte) und \textbf{Referenztypen} (Zeichenketten, Datenstrukturen, Zwergpinscher) und Bytecode durch den Compiler einfacher erzeugt.
\begin{table}[H]
\centering
\begin{tabular}{lll}
\hline
Typ&Grösse&Belegung (Wertebereich)\\\hline
boolean&1 Bit&\texttt{true} oder \texttt{false}\\
char& 16Bit&$\text{0x0000 \dotso 0xFFFF}$\\\hline
byte*&8 Bit&$-2^7$ bis $2^7-1$\\
short*&16 Bit&$-2^{15}$ bis $2^{15}-1$\\
int*&32 Bit&$-2^{31}$ bis $2^{31}-1$ \\
long*&64 Bit&$-2^{63}$ bis $2^{63}-1$\\\hline
float&32 Bit&$1,4023\cdot 10^{-45} \dotso 3,4028\cdot10^{38}$\\
double&64 Bit&$4,9406\cdot 10^{-324} \dotso 1,7976\cdot 10^{308}$\\\hline
\end{tabular}
\caption{Java-Datentypen, Grössen und Formate. *\textit{Zweierkomplement}}
\end{table}
\noindent Es gibt mehr negative Werte als positive Werte, das liegt an der Kodierung im Zweierkomplement. Bei \textbf{\texttt{float}} und \textbf{\texttt{double}} ist das Vorzeichen nicht angegeben, die Wertebereiche unterscheiden sich nicht, die kleinsten und grössten darstellbaren Zahlen können sowohl positiv als auch negativ sein.
\subsection{Variablendeklarationen}
Mit Variablen lassen sich Daten speichern, die vom Programm gelesen und geschrieben werden können. Variablen müssen deklariert werden. Hinter dem Typnamen folgt der Name der Variablen. Die \textbf{Deklaration} ist eine Anweisung und wird daher mit einem Semikolon abgeschlossen.
\lstinputlisting[language=Java]{../../PROJEKTE/000001HelloWorld/src/FirstVariable.java}
Gleich bei der Deklaration lassen sich Variablen mit einem Anfangswert initialisieren. Hinter einem Gleichheitszeichen steht der Wert, der oft ein Literal ist.
\newline\newline
Eine Konsoleneingabe. Eine Variante ist die Klasse \texttt{java.util.Scanner}. Folgende Tabelle zeigt die Eingabe von drei verschiedenen Datentypen.
\begin{table}[H]
\centering
\begin{tabular}{ll}
\hline
Eingabe & Anweisung\\\hline
\texttt{String}&\texttt{String s = new java.util.Scanner(System.in).nextLine();}\\
\texttt{int}&\texttt{int i = new java.util.Scanner(System.in).nextInt();}\\
\texttt{double}&\texttt{double d = new java.util.Scanner(System.in).nextDouble();}\\\hline
\end{tabular}
\caption{Einlesen einer Zeichenkette, Ganz- und Fliesskommazahl von der Konsole}
\end{table}
\noindent Folgendes Beispiel zeigt eine Anwendung aller Eingabenmöglichkeiten mit der Klasse \texttt{java.util.Scanner}.
\lstinputlisting[language=Java]{../../PROJEKTE/000001HelloWorld/src/SmallConversation.java}
\subsection{Fliesskommazahlen}
Java bietet die Datentypen \textbf{\texttt{float}} und \textbf{\texttt{double}}. Fliesskommazahl können einen Vorkommateil und einen Nachkommateil besitzen, die durch einen Dezimalzahl getrennt sind. Standardmässig sind die Fliesskommaliterale vom Typ \textbf{\texttt{double}}. Ein nachgestelltes \textbf{\texttt{f}} oder \textbf{\texttt{F}} zeigt dem Computer an, dass es sich um einen \texttt{float} handelt.
\newline\newline
So ist beispielsweise \texttt{1+2+4.0} eine Addition aus \texttt{1+2} dann in \texttt{double} transformiert und anschliessend \texttt{3.0+4.0}. Die Standardbibliothek \textbf{\texttt{java.math}} bietet die Klasse \textbf{\texttt{BigDecimal}} an. Diese Klasse eignet sich gut für gute Genauigkeit wie Währungen.
\subsection{Ganzzahlige Datentypen}
Java stellt fünf ganzzahlige Datentypen zur Verfügung: \textbf{\texttt{byte}}, \textbf{\texttt{short}}, \textbf{\texttt{char}}, \textbf{\texttt{int}} und \textbf{\texttt{long}}. Ganzzahlige Datentypen sind immer vorzeichenbehaftet (mit Ausnahme von \texttt{char}). Einen Modifizierer \texttt{unsigned} gibt es nicht. Java reserviert nicht so viele Bits wie benötigt und wählt nicht automatisch den passenden Wertebereich. Dabei ist \textbf{\texttt{System.out.println( 122323423434525345345435);}} fehlerbehaftet. Der Datentyp \textbf{\texttt{int}} ist in Java standardmässig.
\newline\newline
An das Ende von Ganzzahlliteralen vom Typ \textbf{\texttt{long}} wird ein \textbf{\texttt{l}} oder ein \textbf{\texttt{L}} gesetzt. Dabei wird \textbf{\texttt{System.out.println( 122323423434525345345435L);}} gültig.
\newline\newline
Ein \textbf{\texttt{byte}} ist ein Datentyp mit einem kleineren Wertebereich. Eine Initialisierung \textbf{\texttt{byte b = 200;}} ist fehlerbehaftet. Eine explizite Typumwandlung lässt Zahlen in einem \textbf{\texttt{byte}} speichern und zwar \textbf{\texttt{byte b = (byte) 200;$\Longrightarrow$ -56}}.
\newline\newline
Der Datentyp \textbf{\texttt{short}} stehen 16 Bits, (1 Bit für das Vorzeichen und 15 Bit für die Zahlen) Speicher zur Verfügung. Ein \texttt{short} ohne Vorzeichen kann folgendermassen initialisiert werden: \textbf{\texttt{short s = (short) 3300;$\Longrightarrow$ -32536}}
\subsection{Wahrheitswerte}
Der Datentyp \textbf{\texttt{boolean}} beschreibt einen Wahrheitswert, der entweder \textbf{\texttt{true}} oder \textbf{\texttt{false}} ist. Diese sind reservierte Wörter und bilden neben konstanten Strings und primitiven Datentypen Literale. Numerische Werte werden nicht als Wahrheitswerte interpretiert. Der boolesche Typ wird für Bedingungen, Verzweigungen oder Schleifen benötigt. Ein Wahrheitswerte ergibt sich aus Vergleichen.
\subsection{Unterstriche in Zahlen}
Eine Variante um grosse Zahlen mit viele Nullen zu schreiben ist es, Unterstriche in Zahlen einzusetzen, denn ein Unterstrich gliedert die Zahl in Blöcke. Unterstriche machen Tausender-Blöcke gut sichtbar. Hilfreich ist die Schreibweise auch bei Literalen in Binär- und Hexadezimaldarstellung. Mit \textbf{\texttt{0b}} beginnt ein Literal in Binärschreibweise und mit \textbf{\texttt{0x}} beginnt ein Literal in Hexadezimalschreibweise. Zwei aufeinanderfolgende Unterstriche sind aber nicht erlaubt und er darf nicht am Anfang stehen.
\subsection{Alphanumerische Zeichen}
Der alphanumerische Datentyp \textbf{\texttt{char}} ist 2 Byte gross und nimmt ein Unicode-Zeichen auf. Ein \texttt{char} ist nicht vorzeichenbehaftet. Die Literale werden in Hochkommata (nicht Anführungszeichen) gesetzt. Ein \texttt{char} kann automatisch in ein \texttt{int} konvertiert werden.
\subsection{Initialisierung von lokalen Variablen}
Die Laufzeitumgebung bzw. der Compiler initialisiert lokale Variablen nicht automatisch mit einem Nullwert bzw. einen \texttt{false}. Sind Variablen nicht initialisiert, so gibt es Fehlermeldungen.
\section{Ausdrücke, Operanden und Operatoren}
Mathematische Ausdrücke bestehen aus \textbf{Operanden} und \textbf{Operatoren}. Ein Operand ist eine Varaible, ein Literal oder Rückgabe eines Methodenaufrufs. Die Operatoren verknüpfen die Operanden. Je nach Anzahl der Operanden unterscheidet man folgende Arten von Operatoren:
\begin{itemize}
\item Ist ein Operator auf genau einem Operanden definiert, so nennt er sich unärer Operator. Bsp: Negatives Vorzeichen.
\item Die üblichen Operatoren für mathematische Ausdrücke sind binäre Operatoren.
\item Das Fragezeichen-Operator für bedingte Ausdrücke ist ein tertiäres Operator.
\end{itemize}
\subsection{Zuweisungsoperator}
Das Gleichheitszeichen \textbf{\texttt{=}} dient in Java der Zuweisung. Der Zuweisungsoperator ist ein binärer Operator, bei dem auf der linken Seite due zu belegende Variable steht und auf der rechten Seite ein Ausdruck. Erst nach dem Auswerten des Ausdrucks kopiert der Zuweisungsoperator das Ergebnis in die Variable. Division durch Null, so gibt es keinen Schreibzugriff auf die Variable. Zuweisungen können geschachtelt werden.
\subsection{Arithmetische Operatoren}
Ein arithmetischer Operator verknüpft die Operanden mit den Operatoren Addition (\textbf{\texttt{+}}), Subtraktion (\textbf{\texttt{-}}), Multiplikation (\textbf{\texttt{*}}), Division (\textbf{\texttt{/}}) und den Rest-Operator (\textbf{\texttt{\%}}). Die arithmetische Operatoren sind binär.
\newline\newline
Bei Ausdrücken mit unterschiedlichen numerischen Datentypen, bringt der Compiler vor der Anwendung der Operation alle Operanden auf den umfassenderen Typ. Vor der Auswertung von \texttt{1+2.0} wird die Ganzzahl \texttt{1} in ein \texttt{double} konvertiert und dann die Addition vorgenommen - das Ergebnis ist auch vom Typ \texttt{double}. Das nennt sich \textbf{numerische Umwandlung}. Die Operation wird ausgeführt, und der Ergebnistyp entspricht dem umfassenden Typ.
\newline\newline
Der binäre Operator bildet den Quotienten aus Dividend und Divisor. Die Division ist für Ganzzahlen und für Fliesskomazahlen definiert. Bei der Ganzzahldivision wird zu null hin gerundet und das Ergebnis ist keine Fliesskomazahl. Den Datentyp des Ergebnisses bestimmen die Operanden und nicht der Operator. Soll das Ergebnis vom Typ \texttt{double} sein, muss mindestens ein Operand ebenfalls \text{double} sein.
\subsection{Der Restwert-Operator \%}
Der Restwert-Operator liefert der Rest einer Division zweier Ganzzahlen und Fliesskomazahlen. Die DIvision und der Restwert richten sich nach einer einfachen Formel: \texttt{(int)(a/b)*b+(a\%b)=a}. Das Ergebnis ist nur dann negativ, wenn der Dividend negativ ist; das Ergebnis ist nur dann positiv, wenn der Dividend positiv ist. Um mit \texttt{value\%2 == 1} zu testem, ob \texttt{value} eine ungerade Zahl ist, muss \texttt{value} positiv sein.
\subsection{Präfix- oder Postfix-Inkrement und -Dekrement}
Die Operatoren \textbf{\texttt{++}} und \textbf{\texttt{--}} kürzen die Programmzeilen zum Inkrement und Dekrement ab. Eine lokale Variable muss allerdings vorher initialisiert sein, da ein Lesezugriff vor einem Schreibzugriff stattfindet. Beide Operatoren erfüllen somit zwei Aufgaben: Neben der Wertrückgabe gibt es eine Veränderung der Variablen.
\begin{table}[H]
\centering
\begin{tabular}{lll}
\hline
&Präfix&Postfix\\\hline
Inkrement & Prä-Inkrement, \textbf{\texttt{++i}}&Post-Inkrement, \textbf{\texttt{i++}}\\
Dekrement & Prä-Dekrement, \textbf{\texttt{--i}}&Post-Dekrement, \textbf{\texttt{i--}}\\\hline
\end{tabular}
\caption{Präfix- oder Postfix-Inkrement und -Dekrement}
\end{table}
\noindent Die beiden Operatoren liefern einen Ausdruck und geben daher einen Wert zurück. Es macht jedoch einen feinen Unterschied, wo dieser Operator platziert wird: Er kann vor der Variablen stehen, wie \texttt{++i} oder dahinter wie \texttt{i++}. Der \textbf{Präfix-Operator} verändert die Variable vor der Auswertung des Ausdrucks, und der \textbf{Postfix-Operator} verändert die Variable nach der Auswertung des Ausdrucks.
\lstinputlisting[language=Java]{../../PROJEKTE/000001HelloWorld/src/Prefixen.java}
\subsection{Auswertung bei Array-Zugriffen}
Falls die linke Seite beim Verbundoperator ein Array-Zugriff ist, wird die Indexberechnung nur einmal vorgenommen. Dies ist wichtig beim Einsatz vom Präfix-/Postfix-Operator oder von Methodenaufrufen, die Nebenwirkungen besitzen, also etwa Zustände wie einen Zähler verändern.
\subsection{Zuweisung mit Operation (Verbundoperator)}
Zuweisungen lassen sich mit numerischen Operatoren kombinieren. Für einen binären Operator (symbolisch \textbf{\texttt{\#}} genannt) im Ausdruck \textbf{\texttt{a = a\#(b)}} kürzt der Verbundoperator den Ausdruck zu \textbf{\texttt{a\#b}} ab. Der Verbundoperator erlaubt eine kompakte Schreibweise.
\subsection{Relationale und Gleichheitsoperatoren}
Relationale Operatoren sind Vergleichsoperatoren, die Ausdrücke miteinander vergleichen und einen Wahrheitswert vom Typ \texttt{boolean} ergeben. Die numerische Vergleiche sind: grösser (\textbf{\texttt{>}}), kleiner (\textbf{\texttt{<}}), grösser/gleich (\textbf{\texttt{$\geq$}}), kleiner/gleich (\textbf{\texttt{$\leq$}}), Gleichheit (\textbf{\texttt{==}}), Ungleichheit (\textbf{\texttt{!=}}).
\subsection{Logische Operatoren}
Die Programmierung ist an Bedingungen verknüpft. Diese Bedingungen sind komplex zusammengesetzt, wobei drei Operatoren am häufigsten vorkommen. \textbf{Nicht \texttt{!}:} (Negation) dreht die Aussage um; aus \texttt{wahr} wird \texttt{falsch} und aus \texttt{falsch} wird \texttt{wahr}. \textbf{Und \texttt{\&\&}:} (Konjunktion) beide Aussagen müssen \texttt{wahr} sein, damit die Gesamtaussage \texttt{wahr} wird. \textbf{Oder \texttt{||}:} (Disjunktion) eine der beiden Aussagen muss \texttt{wahr} sein, damit die Gesamtaussage \texttt{wahr} wird. \textbf{Xor \texttt{\^}:} (Exklusives Oder) Operation, die nur dann \texttt{wahr} liefert, wenn genau einer der beiden Operanden \texttt{wahr} ist. Sind beide Operanden gleich, so ist das Ergebnis \texttt{false}.
\begin{table}[H]
\centering
\begin{tabular}{llllll}
\hline
boolean a& boolean b&\texttt{!a}&\texttt{a\&\&b}&\texttt{a||b}&\texttt{a\^\,b}\\\hline
\texttt{true}&\texttt{true}&\texttt{false}&\texttt{true}&\texttt{true}&\texttt{false}\\
\texttt{true}&\texttt{false}&\texttt{false}&\texttt{false}&\texttt{true}&\texttt{true}\\
\texttt{false}&\texttt{true}&\texttt{true}&\texttt{false}&\texttt{true}&\texttt{true}\\
\texttt{false}&\texttt{false}&\texttt{true}&\texttt{false}&\texttt{false}&\texttt{true}\\\hline
\end{tabular}
\caption{Verknüpfungen der logischen Operatoren.}
\end{table}
\subsection{Rang der Operatoren}
Neben Plus und Mail gibt es eine Vielzahl von Operatoren., die alle ihre eigenenn Vorrangregeln besitzen. Der Multiplikationsoperator besitzt eine höhere Priorität als der Plus-Operator. Der \textbf{arithmetische Typ} steht für Ganz- und Fliesskommazahlen, der \textbf{integrale Typ} für \texttt{char} und Ganzzahlen und der Eintrag primitiv für jegliche primitiven Datentypen.
\begin{table}[H]
\centering
\begin{tabular}{llll}
\hline
Operator&Rang&Typ&Beschreibung\\\hline
\texttt{++}, \texttt{--}&1&arithmetisch&Inkrement und Dekrement\\
\texttt{+}, \texttt{-}&1&arithmetisch&unäres Plus und Minus\\
\texttt{\~}&1&integral&bitweises Komplement\\
\texttt{!}&1&boolean&logisches Komplement\\
\texttt{(Typ)}&1&jeder&Cast\\\hline
\texttt{*}, \texttt{/}, \texttt{\%}&2&arithmetisch&Multiplikation, Division, Rest\\\hline
\texttt{+}, \texttt{-}&3&arithmetisch&Addition und Subtraktion\\
\texttt{+}&3&String&String-Konkatenation\\
\texttt{<<}&4&integral&Verschiebung links\\
\texttt{>>}&4&integral&Rechtsverschiebung mit Vorzeichenerweiterung\\
\texttt{>>>}&4&integral&Rechtsverschiebung ohne Vorzeichenerweiterung\\\hline
\texttt{<}, \texttt{<=}, \texttt{>}, \texttt{>=}&5&arithmetisch&Numerische Vergleiche\\
\texttt{instanceof}&5&Objekt&Typvergleich\\
\texttt{==}, \texttt{!=}&6&primitiv&Gleich-/Ungleichheit von Werten\\
\texttt{==}, \texttt{!=}&6&Objekt&Gleich-/Ungleichheit von Referenzen\\
\texttt{\&}&7&integral&bitweises Und\\
\texttt{\&}&7&boolean&logisches Und\\\hline
\texttt{\^}&8&integral&bitweises XOR\\
\texttt{\^}&8&boolean&logisches XOR\\
\texttt{|}&9&integral&bitweises Oder\\
\texttt{|}&9&boolean&logisches Oder\\\hline
\texttt{\&\&}&10&boolean&logisches konditionales Und, Kurzschluss\\\hline
\texttt{||}&11&boolean&logisches konditionales Oder, Kurzschluss\\
\texttt{?:}&12&jeder&Bedingungsoperator\\
\texttt{=}&13&jeder&Zuweisung\\
\texttt{*=}, \texttt{/=}, \texttt{\%=}&13&arithmetisch&Zuweisung mit Operation\\
\texttt{+=}, \texttt{=}, \texttt{<<=}&13&arithmetisch&Zuweisung mit Operation\\
\texttt{>>=}, \texttt{>>>=}, \texttt{\&=}&13&arithmetisch&Zuweisung mit Operation\\
\texttt{\^=}, \texttt{|=}&13&arithmetisch&Zuweisung mit Operation\\
\texttt{+=}&14&String&Zuweisung mit String-Konkatenation\\\hline
\end{tabular}
\caption{Operatoren mit Rangordnung}
\end{table}
\subsection{Die Typumwandlung (Casting)}
Datentypen können konvertiert werden, dies nennt sich \textbf{Typumwandlung}. Java unterscheidet zwischen zwei Arten der Typumwandlung. EIne Typumwandlung hat eine sehr hohe Priorität. DAher muss der Ausdruck gegebenfalls geklammert werden.
\begin{itemize}
\item \textbf{Implizite Typumwandlung:} Daten eines kleineren Datentyps werden automatisch dem grösseren angepasst. Der Compiler nimmt die Anpassung selbständig vor.
\item \textbf{Explizite Typumwandlung:} Ein grösserer Typ kann einem kleineren Typ mit möglichem Verlust von Informationen zugewiesen werden.
\end{itemize}
Werte der Datentypen \texttt{byte} und \texttt{short} werden bei Rechenoperationen automatisch in den Datentyp \texttt{int} umgewandelt. Ist ein Operand vom Datentyp \texttt{long}, dann werden alle Operanden auf \texttt{long} erweitert. Wird aber \texttt{short} oder \texttt{byte} als Ergebnis verlangt, dann ist dieses durch einen expliziten Typecast anzugeben, und nur die niederwertigen Bits des Ergebniswerts werden übergeben.
\begin{table}[H]
\centering
\begin{tabular}{ll}
\hline
Vom Typ&in den Typ\\\hline
\texttt{byte} (8 Bit)&\texttt{short}, \texttt{int}, \texttt{long}, \texttt{float}, \texttt{double}\\
\texttt{short} (16 Bit)&\texttt{int}, \texttt{long}, \texttt{float}, \texttt{double}\\
\texttt{char} (16 Bit)&\texttt{int},\texttt{long}, \texttt{float}, \texttt{double}\\
\texttt{int} (32 Bit)&\texttt{long}, \texttt{float}, \texttt{double}\\
\texttt{long} (64 Bit)&\texttt{float}, \texttt{double}\\
\texttt{float} (32 Bit)&\texttt{double}\\
\texttt{double} (64 Bit)&\texttt{double}\\\hline
\end{tabular}
\caption{Implizite Typumwandlungen}
\end{table}
\noindent Die Anpassung ist eine Erweiterung des Wertebereichs (widening conversion). Der Typ \texttt{boolean} taucht nicht auf, er lässt sich in keinen anderen primitiven Typ konvertieren. Dass ein \texttt{long} auf ein \texttt{double} gebracht werden kann bzw. ein \texttt{int} auf ein \texttt{float} ist als Fehler in der Java zu sehen, denn es gehen Informationen verloren. Ein \texttt{double} kann die 64 Bit für Ganzzahlen nicht effizient nutzen wie ein \texttt{long}.
\newline\newline
Die explizite Anpassung engt einen Typ ein (narrowing conversion). Der gewünschte Typ für eine Typumwandlung wird vor den umzuwandelnden Datentyp in Klammern gesetzt. Bei jeder expliziten Typumwandlung geht Information verloren.
\newline\newline
Bei der expliziten Typumwandlung von \texttt{double} und \texttt{float} in einen Ganzzahltyp kann es selbstverständlich zum Verlust von Genauigkeit kommen sowie zur Einschränkung des Wertebereichs. Bei der konvertierung von Fliesskommazahlen verwendet Java eine Rundung gegen null, schneidet also schlicht den Nachkommaanteil ab.
\lstinputlisting[language=Java]{../../PROJEKTE/000001HelloWorld/src/Typumwandlung.java}
Die String-Konkatenation ist strikt von links nach rechts und natürlich nicht kommutativ wie die numerische Addition. Besteht der Auddruck aus mehreren Teilen, so muss die Auswertungsreihenfolge beachtet werden, andernfalls kommt es zu seltsamen Zusammensetzungen.
\lstinputlisting[language=Java]{../../PROJEKTE/000001HelloWorld/src/PlusString.java}
Nur eine Zeichenkette in doppelten Anführungszeichen ist ein String, und der Plus-Operator entfaltet seine besondere Wirkung. Ein einzelnes Zeichen in einfachen Hochkommata konvertiert Java nach den Regeln der Typumwandlung bei Berechnungen in ein \texttt{int} und Additionen sind Ganzzahl-Additionen.
\lstinputlisting[language=Java]{../../PROJEKTE/000001HelloWorld/src/PlusZeichen.java}
\section{Bedingte Anweisungen}
\subsection{Verzweigung mit der \texttt{if}-Anweisung}
Die \textbf{\texttt{if}}-Anweisung besteht aus dem Schlüsselwort \texttt{if}, dem zwingend ein Ausdruck mit dem Typ \texttt{boolean} in Klammern folgt. Es folgt eine Anweisung, die oft eine Blockanweisung ist.
\lstinputlisting[language=Java]{../../PROJEKTE/000001HelloWorld/src/WhatsYourNumber.java}
Ist das Ergebnis in der Bedingung \texttt{true}, so werden die Anweisungen in der Fallunterscheidung ausgeführt, sonst werden die \text{else}-Anweisungen ausgeführt. Eine Fallunterscheidung hat kein Semikolon. \texttt{if} und \texttt{if-else}-Anweisungen werden geschachtelt (kaskadiert).
\lstinputlisting[language=Java]{../../PROJEKTE/000001HelloWorld/src/IsLeapYear.java}
Die eingerückten Verzweigungen nennen sich auch angehäufte \texttt{if}-Anweisungen oder \texttt{if}-Kaskade.
\subsection{Der Bedingungsoperator}
Der Bedingungsoperator, auch Konditionaloperator, erlaubt es, den Wert eines Ausdrucks von einer Bedingung abhängig zu machen, ohne dass dazu eine \texttt{if}-Anweisung verwendet werden muss. Die Operanden sind durch \textbf{\texttt{?}} und \textbf{\texttt{:}} voneinander getrennt.
\lstinputlisting[language=Java]{../../PROJEKTE/000001HelloWorld/src/BedingungsOperator.java}
Drei Ausdrücke kommen in Bedingunsoperator vor. Der erste Ausdruck muss vom Typ \texttt{boolean} sein. Der Bedingungsoperator kann eingesetzt werden, wenn der zweite und dritte Operand ein numerischer Typ, boolescher Typ oder Referenztyp sind.
\subsection{Die \texttt{switch}-Anweisung}
Eine Kurzform für speziell gebaute, angehäufte \texttt{if}-Anweisungen bietet \textbf{\texttt{switch}}. Es gibt eine Reihe von unterschiedlichen Sprungzeilen, die mit \textbf{\texttt{case}} markiert sind. Die \texttt{switch}-Anweisung erlaubt die Auswahl vin Ganzzahlen, Wrapper-Typen, Aufzählungen und Strings.
\lstinputlisting[language=Java]{../../PROJEKTE/000001HelloWorld/src/Calculator.java}













\section{Geschwindigkeit eines Massenpunktes}
\section{Zweidimensionale Graphik}
\subsection{Elementare zweidimensionale Graphik}
Der Befehl \boxed{\textbf{\texttt{plot}}} öffnet ein Graphikfenster namens "figure" mit einer Nummer, in das eine Graphik engebettet werden kann. Falls für die Abszisse und für die Ordinate keine Schranken gesetzt werden, passt sich die Skalierung des Koordinatensystems den Daten automatisch an. Für jedes weitere Bild mus mit dem Befehl \boxed{\textbf{\texttt{figure}}} ein neues Graphikfenster geöffnet werden, es erhält eine neue Nummer. Andernfalls wird das alte Bild im Graphikfenster durch das neue Bild überschrieben. 
\newline\newline
Bekanntlich basiert die Grundstruktur von MATLAB auf einer \texttt{n$\times$m}-Matrix aus reellen oder komplexen Elementen. Auch Daten werden ja in Matrizen abgelegt. Für ein zweidimensaionales Bild benötigt MATLAB also mindestens zwei Kolonnenvektoren gleicher Länge.
\newline\newline
Der Befehl {\color{red}\texttt{plot(x,y)}} zeichnet den Datensatz \texttt{y} in Funktion von Datensatz \texttt{x} auf. Wie üblich wird jedem Wert von \texttt{x} ein Wert von \texttt{y} zugeordnet. \texttt{x} sind die Werte der Abszisse und \texttt{y} diejenigen auf der Ordinate. Die daraus resultierende Punkte werden mit geraden Linien verbunden (lineare Interpolation). Beide Achsen haben eine lineare Skala.
\newline\newline
Mit {\color{red}\texttt{plot(x,y,s)}} werden im String \texttt{s} der Linientyp und die Farbe der Kurve definiert. Der Befehl {\color{red}\texttt{plot(x,y,'c+:')}} plottet eine rot punktierte Linie, die jedem Datenpunkt ein rotes Plus-Zeichen hat. Der Befehl {\color{red}\texttt{plot(y)}} enthält nur einen Kolonnenvektor \texttt{y} Element von \texttt{$\mathbb{R}^{n\times 1}$}. In diesem Fall generiert MATLAB für die \texttt{x}-Achse automatisch Werte, nämlich 1 bis \texttt{n}, die Indizes der \texttt{n} Kolonnenwerte. 
\newline\newline
Handelt es sich jedoch bei \texttt{y} um einen Vektor mit komplexen Zahlen, so werden beim Befehl {\color{red}\texttt{plot(y)}} die Realteile auf der \texttt{x}-Achse und die Imaginärteile auf der \texttt{y}-Achse aufgetragen. 
\newline\newline
Der Befehl {\color{red}\texttt{plot(A)}} zeichnet für jede eine Kurve aus \texttt{n} Punkten. Die Matrix \texttt{A} Element von \texttt{$\mathbb{R}^{n\times m}$} je eine Kurve aus \texttt{n} Punkten. Die \texttt{x}-Achse zeigt wieder die Indizes 1 bis \texttt{n}. Die LInien werden zur Unterscheidung verschiedene Stile bzw. verschiedene Farben haben.
\newline\newline
Beim Befehl {\color{red}\texttt{plot(x,A)}} wird jede der \texttt{m} Kolonnen der Matrix \texttt{A} Element von \texttt{$\mathbb{R}^{n\times m}$} gegen die gemeinsame unabhängige Variable \texttt{x} aufgezeichnet. Der Vektor \texttt{x} muss die Dimension \texttt{n} haben: \texttt{x} ist ELement von \texttt{$\mathbb{R}^{n\times 1}$}
\newline\newline
Beim Befehl {\color{red}\texttt{plot(A,B)}} nilden je eine Kolonne der Matrix \texttt{A} Element von \texttt{$\mathbb{R}^{n\times m}$} und der Matrix \texttt{B} Element von \texttt{$\mathbb{R}^{n\times m}$} ein \texttt{x}-\texttt{y}-Vektorpaar. 
\newline\newline
Der Befehl \boxed{\textbf{\texttt{subplot(n,m,p)}}} unterteilt ein Graphikfenster in \texttt{n} Zeilen von je \texttt{m} Bildern. Damit können \texttt{n$\times$m} Bilder in ein Graphikfenster eingebettet werden. \texttt{p} ist der Laufindex der \texttt{n$\times$m} Bilder, wobei die Numerierung zeilenweise von links nach rechts erfolgt. Für jedes neue Bild im Graphikfenster wird der Befehl \texttt{subplot} wiederholt, jedesmal mit dem neuen Index \texttt{p}. Der eigentliche Befehl \texttt{plot} mit seinen Parametern muss dann natürlich auc noch kommen. 
\newline\newline
Die Befehle \boxed{\textbf{\texttt{semilogx}}} und {\color{red}\texttt{plot}} sind bis auf die Skalierung der \texttt{x}-Achse identisch. Beim Befehl {\color{red}\texttt{plot}} hat die Abszisse eine lineare, beim Befehl \texttt{semilogx} aber eine logarithmische Skala (Basis 10). Der Befehl {\color{red}\texttt{semilogx(x,y)}} entspricht dem Befehl {\color{red}\texttt{plot(log10(x), y)}}, doch MATLAB gibt bei \texttt{semilogx} für \texttt{x=0} keine Warnung "log for zero". Der Nullpunkt der \texttt{x}-Achse wird unterdrückt, er kann nicht gezeichnet werden. Mit dem Befehl {\color{red}\texttt{semilogy}} wird die Ordinate mit dem Zehnerlogarithmus skaliert.    
\newline\newline
Der Befehl \boxed{\textbf{\texttt{loglog}}} plottet eine zweidimensionale Graphik in einem doppeltlogarithmischen Koordinatensystem (Basis 10). Der Befehl {\color{red}\texttt{loglog(x,y),log10(y)}} entspricht dem Befehl {\color{red}\texttt{plot(log10(x))}}, doch MATLAB gibt bei \texttt{loglog} keine Warnung "log of zero", falls \texttt{x} oder \texttt{y} gleich Null ist. Der Koordinatennullpunkt wird unterdrückt.  
\newline\newline
Der Befehl \boxed{\textbf{\texttt{polar(phi,r)}}} zeichnet die beiden Vektoren \texttt{phi} und \texttt{r} in ein polares Koordinatensystem. Die Werte des Vektors \texttt{phi} sind im Bogenmass angegeben. Die WErte des Vektors \texttt{r} entsprechen dem Radius, d.h. dem Abstand zwischen dem Ursprung und dem betreffenden Punkt der Funktion.
\newline\newline
Der Befehl \boxed{\textbf{\texttt{plot(x1,x2,y1,y2)}}} versieht die Graphik mit zwei Ordinaten. Die linke \texttt{y}-Achse bezieht sich auf \texttt{y1} in Funktion von \texttt{x1} und die rechte \texttt{y}-Achse auf \texttt{y2} in Funktion von \texttt{x2}.
\subsection{Massstab}
Mit dem Befehl \boxed{\textbf{\texttt{axis([xmin xmax ymin ymax])}}} lassen sich die Grenzen der \texttt{x}- und der \texttt{y}-Achse neu definieren. Mit dem Befehl {\color{red}\texttt{axis auto}} setzt für die Achsen wieder dir ursprünglichen Werte ein. Mit dem Befehl {\color{red}\texttt{axis equal}} erhalten alle Achsen die gleiche Skalierung. Mit dem Befehl {\color{red}\texttt{axis ij}} wechselt das Vorzeichen der \texttt{y}-Achse. Die positive \texttt{y}-Achse zeigt nun nach unten. Der Befehl {\color{red}\texttt{axis xy}} macht {\color{red}\texttt{axis ij}} wieder rückgängig. Der Befehl {\color{red}\texttt{axis tight}} passt die Achsenlänge exakt dem Bild an. Der Befehl {\color{red}\texttt{axis off}} schaltet alle axis-Definitionen, die "tick marks" und den Hintergrund aus. Mit dem Befehl {\color{red}\texttt{axis on}} werden sie wieder aktiviert.
\newline\newline
Der Befehl \boxed{\textbf{\texttt{zoom on}}} aktiviert in der aktuellen Graphik die Zoom-Funktion, er kann direkt im Command Window eingegeben werden. Mit "clic and drag" wird dann im Bild ein beliebiger Ausschnitt näher herangebracht. Mit der linken Maustaste wird der Graphikausschnitt vergrössert und mit der rechten Maustaste verkleinert. Der Befehl {\color{red}\texttt{zoom(factor)}} zoomt die Achsen um den für "factor" gewählten Wert. Der Befehl {\color{red}\texttt{zoom out}} führt die Graphik in ihre default-Fenstergrösse zurück. Der Befehl {\color{red}\texttt{zoom xon}} bzw. {\color{red}\texttt{zoom yon}} aktiviert in re aktuellen Graphik die Zoom-Funktion nur für die betreffenden Achsen. Der Befehl {\color{red}\texttt{zoom(figurename, option)}} versieht die Graphik "figurename" mit einer Zoom-Funktion. Als Option kann eine der oben genannten Zoom-Funktionen gewählt werden.
\newline\newline
Der Befehl \boxed{\textbf{\texttt{grid on}}} versieht die aktuelle Graphik mit ienem Liniennetz. Mit dem Befehl {\color{red}\texttt{grid off}} werden die Linien wieder deaktiviert. Der Befehl {\color{red}\texttt{box on}} umrahmt das aktuelle Bild mit einem Rahmen aus dünnen, schwarzen LInien, der Befehl {\color{red}\texttt{box off}} entfernt ihn wieder.
\newline\newline
Der Befehl \boxed{\textbf{\texttt{hold on}}} fixiert das aktuelle Graphikfenster, so dass weitere Funktionen im bereits bestehenden Graphikfenster positioniert werden können. Die ursprünglichen Achseinstellungen bleiben unverändert, selbst dann, wenn die neue Funktion nicht gut in den Rahmen passt. Der Befehl {\color{red}\texttt{hold off}} führt zum Normalbetrieb zurück, d.h. ein neuer plot-Befehl löscht die aktuelle Funktion und fügt die neue Funktion in das Fenster ein, falls nicht mit dem Befehl {\color{red}\texttt{figure}} ein weiteres Graphikfenster geöffnet wurde.
\newline\newline
Der Befehl \boxed{\textbf{\texttt{axes('position', [links unten Breite Höhe])}}} beschreibt im Graphikfenster die POsition der linken unteren Ecke des Bildes und seine Abmessung. Mit Werten zwischen 0 und 1 kann nun die Position und dir Grösse des Bildes festgelegt werden. Das default-Graphikfenster hat eine Breite und eine Höhe 1. Der Befehl {\color{red}\texttt{axex}} generiert in einem Graphikfenster ein Koordinatensystem, in das mit {\color{red}\texttt{plot}} ein beliebiges Bild hineingelegt werden kann. Über mehrere {\color{red}\texttt{axex}}-Definitionen können im Fenster mehrere Bilder erzeugt werden.
\subsection{Beschriften von Bilder} 
Der Befehl \boxed{\textbf{\texttt{legend('st1', 'st2', ...)}}} versieht im Fenster ein Bild mit einer Legende. Er erhält die einzelnen Strings "st1", "st2", usw. Für jedes Bild in einem Graphikfenster kann ein eigener Titel gewählt werden. Der zu schreibende Text wird in Anführungszeichen genommen. Der Befehl {\color{red}\texttt{legend off}} entfernt die Legende aus dem aktuellen Bild. Der Befehl {\color{red}\texttt{legend('st1', 'st2', ..., position)}} positioniert die Legende mit der Angabe einer Position an einen definierten Ort in der Graphik: 0-beste, 1-oben-rechts, 2-oben links, 3-unten-links, 4-unten-rechts, -1-rechts. Die Legende kann verschoben werden, indem die Legende mit der rechten Maustaste angeklickt und an den gewünschten Ort gezogen wird.
\newline\newline
Der Befehl \boxed{\textbf{\texttt{title('text')}}} fügt einen Titel oberhalb der Graphik hinzu. Ein Text kann hoch \texttt{("\^\,")} und tiefgestellte \texttt{"\_\,"}, kursiv \texttt{"it"} geschrieben oder griechische Zeichen enthalten. 
\begin{equation}
\boxed{A_1e^{-\alpha t}\sin \beta t\quad \texttt{'$\backslash$itA\_\{1\}e\^\,\{$\backslash$alpha$\backslash$itt\}sin$\backslash$beta$\backslash$itt'}}
\end{equation}
Der Befehl \boxed{\textbf{\texttt{xlabel('text')}}} schreibt die Zeile "text" unter die Abszisse. Der Befehl \texttt{ylabel} ist für die Beschriftung der Ordinate.
\newline\newline
Der Befehl \boxed{\textbf{\texttt{text(x,y,'text')}}} kann innerhalb des Bildrahmes eine beliebige Textzeile an der Stelle \texttt{(x,y)} angebracht werden. \texttt{x} und \texttt{y} sind in den Koordinaten der Achsen anzugehen. Dagegen verwendet der Befehl {\color{red}\texttt{text(x,y,'text,'sc')}} die Koordinaten des Graphikfensters, nämlich (0,0) in der unteren linken Ecke und (1,1) ub der oberen rechten Ecke. Geht ein Text über mehrere Zeilen, so kann er in eine Text-Variable geschrieben werden. Ein Text kann hoch- und tiefgestellte, kursiv geschriebene oder griechische Zeichen enthalten.
\newline\newline
Mit dem Befehl \boxed{\textbf{\texttt{gtext('text')}}} kann mit der Maus im Bild irgendein Text an beliebigen Ort eingefügt werden. Im Graphifenster erscheint der Mauspfeil als Fadenkreuz. An der gewünschten Stelle kann durch Betätigung einer Maustaste der Text eingefügt werden.
\subsection{Graphiken speichern oder drucken}
Der Befehl \boxed{\textbf{\texttt{print}}} sendet eine Kopie des aktuellen Graphikfensters an den Drucker. Der Befehl {\color{red}\texttt{print filename}} speichert eine Kopie des aktuellen Graphikfensters als PostScript-Datei im aktiven Directory unter dem Dateinamen "filename". Mit dem Befehl {\color{red}\texttt{print path}} kann der genaue Pfad angegeben werden, wo das aktuelle Graphikfenster als POst-Script-Datei abgelegt werden soll. Der Befehl {\color{red}\texttt{print[-ddevice][-options]$<$filenmae$>$}} speichert die aktuelle Graphik im Format des speziell gewählten Druckertreibers und de rzusätzlichen Option im aktiven Directory unter "filename" ab. Der Befehl {\color{red}\texttt{help print}} zeigt im Command-Window alle möglichen \texttt{[-ddevice]} und \texttt{[-options]}.
\newline\newline
Der Befehl \boxed{\textbf{\texttt{orient landscape}}} druckt bei print-Befehlen die Graphikfenster im Querformat. Mit dem Befehl {\color{red}\texttt{orient portrait}} wird das Graphikfenster im Hochformat grdruckt. Der Befehl {\color{red}\texttt{orient tall}} setzt das Blattformat auf Hochformat. Zusätzliche wird das Graphikfenster auf das ganze Papierblatt vergrössert bzw. verkleinert. Der Befehl {\color{red}\texttt{orient}} sagt im Command Window, welche Orientierung momentan aktiv ist.
\section{Dreidimensionale Graphik}
\subsection{Elementare dreidimensionale Graphik}
Der Befehl \boxed{\textbf{\texttt{plot3(x,y,z)}}} plottet im dreidimensionalen Raum die Graphen, die durch die Vektoren \texttt{x}, \texttt{y} und \texttt{z} gegeben sind. Die Vektoren müssen alle dieselbe Länge haben. Mit {\color{red}\texttt{plot3(X,Y,Z)}} zeichnet pro Kolonne einen Graphen. Die Matrizen müssen alle dieselbe Grösse haben. Bei {\color{red}\texttt{plot3(x,y,z,'style')}} können zusätzlich noch der Linientyp, die Plot Symbole und die Farbe des Graphen geändert werden. {\color{red}\texttt{help plot}} listet im Command Window eine Answahl von möglichen Liniendefinitionen auf.
\newline\newline
Bei \boxed{\textbf{\texttt{mesh(Z)}}} entsprechen die Werte der Matrix \texttt{Z} Element von \texttt{$\mathbb{R}^{n\times m}$} den \texttt{z}-Werten des Netzes. Für die \texttt{x}- und \texttt{y}-Werte verwendet \texttt{mesh} die Kolonnen- bzw. die Zeilennummer.
\newline\newline
Mit dem Befehl {\color{red}\texttt{mesh(X,Y,Z,C)}} zeichnet ein Netz un der Vogelperspektive mit \texttt{Z} als Funktion von \texttt{X} und \texttt{Y}. Es handelt sich hierbei um eine FUnktion mit zwei Variablen. \texttt{X}, \texttt{Y} und \texttt{Z} sind Matrizen mit den Werten für die \texttt{x}-, \texttt{y}- und \texttt{z}-Koordinaten. \texttt{X} und \texttt{Y} können aber auch Vektoren der Länge \texttt{m} und \texttt{n} sein. Jeder \texttt{z}-Koordinate aus der Matrix \texttt{Z} werden dann die entsprechenden WErte des \texttt{x}- und \texttt{y}-Vektors zugewiesen. \texttt{C} ist ebenfalls eine Matrix und beinhaltet die Farbskala für die Graphik. Ohne \texttt{C} wird \texttt{C=Z} gesetzt.
\newline\newline
Mit dem Befehl \boxed{\textbf{\texttt{fill(X,Y,Z,C)}}} wird in den Farben der Matrix \texttt{C} ein dreidimensionales Polygon geplottet. Sind \texttt{X}, \texttt{Y} und \texttt{Z} Vektoren, so wird die Fläche unterhalb des Graphen mit Farbe ausgefüllt. Ist \texttt{C} ein Skalar, so wird die Fläche monochrom. Folgende Farben sind möglich: {\color{red}\texttt{'r'}}, {\color{red}\texttt{'g'}}, {\color{red}\texttt{'b'}}, {\color{red}\texttt{'c'}}, {\color{red}\texttt{'m'}}, {\color{red}\texttt{'y'}}, {\color{red}\texttt{'w'}} und {\color{red}\texttt{'k'}}. Mit dem Vektor {\color{red}\texttt{[rot grün blau]}} kann eine neue Farbe gemischt werden. Die Werte von liegen zwischen 0 und 1. Je nach Anteil ergibt sich eine Farbkombination. Ist \texttt{C} ein Vektor, so hat er die gleiche Länge wie \texttt{X}, \texttt{Y} und \texttt{Z}. Wird für \texttt{C} einer der drei Vektoren gewählt, dann ist die Farbabstufung der momentan aktive \texttt{colormap} proportional zur betreffenden Koordinatenachse. 
\newline\newline
Sind \texttt{X}, \texttt{Y} und \texttt{Z} Matrizen, so zeichnet \boxed{\textbf{\texttt{fill3}}} pro Kolonne ein Polygon und füllt es mit der entsprechenden Farbe aus. Die Farbgebung bleibt gleich. Falls \texttt{C} ein Zeilenvektor ist, dann hat das Polygon die Schattierung \texttt{shading flat}. Für eine Matrix wird sie {\color{red}\texttt{shading interp}}. {\color{red}\texttt{shading}} shattiert die Objekt-Oberfläche, {\color{red}\texttt{shading flat}} berechnet für jede Teilfläche einer Oberfläche, die mit den Befehlen \texttt{surf}, \texttt{mesh}, \texttt{polar}, \texttt{fill} oder \texttt{fill3} gebildet wurden, die entsprechende Farbabstufung. {\color{red}\texttt{shading interp}} interpoliert über die Farbabstufung. {\color{red}\texttt{shading faceted}} entspricht dem {\color{red}\texttt{shading flat}}. Die 3D Graphik wird jecoh zusätzlich mit schwarzen Linien versehen. 
\subsection{Projektionsarten einer Graphik}
Mit \boxed{\textbf{\texttt{view(az,el)}}} kann in einem dreidimensionalen Plot der Blickwinkel beliebig eingestellt werden, bzw. die Graphik-Box ist um zwei Achsen drehbar. {\color{red}\texttt{az}} steht für azimuth und definiert die horizontale Rotation im Grad. Für einen positiven Winkel dreht sich die Graphik entgegen dem Uhrzeigersinn um die \texttt{z}-Achse. {\color{red}\texttt{el}} beschreibt die Anheben bzw. Senken der Gtraphik in Grad. Bei einem positiven Winkel befindet sich der Betrachter in der Vogelperspektive, bei negativem Winkel in der Froschperspektive. {\color{red}\texttt{view([x y z])}} setzt den Blickwinkel in kartesischen Koordinaten. {\color{red}\texttt{view(2)}} stellt für die 2D Ansicht den vordefinierten Blickwinkel {\color{red}\texttt{view(0, 90)}} ein. {\color{red}\texttt{view(3)}} stellt für die 3D Ansicht den vordefinierten Blickwinkel {\color{red}\texttt{view(-37.5,30)}} ein. {\color{red}\texttt{T=view}} speichert diew \texttt{view} der aktuiellen Graphik in der Variable \texttt{T} als \texttt{4x4}-Matrix. {\color{red}\texttt{view(T)}} weist einer aktuellen Graphik die in der Variable \texttt{T} gespeicherte view zu.
\newline\newline
\boxed{\textbf{\texttt{T=viewmtx(az,el)}}} weist wie bei {\color{red}\texttt{T=view(az, el)}} der Variable \texttt{T} die 4x4-Transformationsmatrix zu. Die Ansicht der aktuellen Graphik wird dabei nicht verändert. Mit {\color{red}\texttt{T=viewmtx(az, el, phi)}} wird die Graphik durch ein Objektiv betrachtet. Der Linsenwinkel wird in Grad angegeben. \texttt{phi=0} Grad definiert die orthogonale Projektion. 10 Grad entspricht einem Teleobjektiv, 25 Grad einem Normalobjektiv und 60 Grad einem Weitwinkelobjektiv. Bei {\color{red}\texttt{T=viewmtx(az, el, phi, tp)}} wird mit \texttt{tp=[xp, yp, zp]} einen Fluchtpunkt gesetzt.
\newline\newline
\boxed{\textbf{\texttt{rotate3d on}}} aktiviert in der aktuellen Graphik die Maus gesteuerte 3D-Rotation. Die Graphik-Ansicht kann damit beliebig verändert werden. Mit {\color{red}\texttt{rotate3d off}} wird sie wieder deaktiviert. 
\subsection{Dreidimensionale Graphik beschriften}
\boxed{\textbf{\texttt{zlabel('text')}}} versieht die \texttt{z}-Achse mit der Aufschrift "text". Mit dem Befehl {\color{red}\texttt{colorbar('vert')}} erscheint in der aktuellen Graphik eine vertikale Farbskala. In einem 3D-Plot bezieht sie sich auf die Werte der \texttt{z}-Achse. {\color{red}\texttt{colorbar('horiz')}} zeichnet eine vertikale Farbskala. Im 3D-Plot ist die Farbgebung ebenfalls auf die \texttt{z}-Achse abgestimmt. \boxed{\textbf{\texttt{colorbar}}} alleine fügt der Graphik entweder eine vertikale Farbskala hinzu, oder die bestehende Farbskala wird aktualisiert.
\section{Spezielle Graphen}
\boxed{\textbf{\texttt{fill(x,y,c)}}} füllt diejenige Fläche mit der Farbe \texttt{c} aus, welche von der Geraden, die den Endpunkt von \texttt{f(x)} mit dem Anfang verbindet, und der Funktion \texttt{y=f(x)} selbst umgeben wird. Ist \texttt{c} ein Vektor derselben Länge wie \texttt{x} und \texttt{y}, dann verwendet MATLAB entweder die im Vektor \texttt{c} definierten Farben, oder für \texttt{c=x} bzw. \texttt{c=y} die Farbpalette der aktuellen \texttt{colormap}. Sind in \boxed{\textbf{\texttt{fill(X,Y,C)}}} \texttt{X} und \texttt{Y} Matrizen derselben Grösse, so wird pro Kolonne ein Polygon gezeichnet. \texttt{C} kann ein Vektor aber auch eine Matrix sein. Beim Vektor ist die Schattierung der Fläche {\color{red}\texttt{shading flat}}, bei der Matrix {\color{red}\texttt{shading interp}}.
\newline\newline
Mit dem Befehl \boxed{\textbf{\texttt{fplot('f',lim)}}} zeichnet eine beliebige Funktion \texttt{f=f(x)} im Bereich \texttt{lim=[x$_{\texttt{min}}$ x$_{\texttt{max}}$]}. Mit \texttt{lim=[x$_{\texttt{min}}$ x$_{\texttt{max}}$ y$_{\texttt{min}}$ y$_{\texttt{max}}$]} werden zusätzliche Schranken gesetzt. Mit dem Befehl {\color{red}\texttt{fplot('f', lim, tol)}} mit \texttt{tol$<$1} definiert die Toleranz des relativen Fehlers. Die voreingestellte Toleranz ist 2e-3 bzw. 0.2\%. Mit dem Befehl {\color{red}\texttt{fplot('f', lim, N)}} berechnet zwischen \texttt{x$_{\texttt{min}}$} für die Funktion \texttt{f=f(x)} \texttt{N+1} Punkte. Mit dem Befehl {\color{red}\texttt{fplot('f', lim, 'LineSpec')}} definiert mit LineSpec den Linien-Typ der Funktion. Alle möglichen "line specifications" werden mit \texttt{help plot} aufgelistet.    
\newline\newline
Mit dem Befehl \boxed{\textbf{\texttt{hist(x)}}} zeichnet MATLAB ein Histogramm mit den in \texttt{x} gespeicherten Daten. Es ist in 10 gleichmässig verteilte Intervalle unterteilt. Pro Intervall gibt es die Anzahl Elemente an, die es enthält. Wenn \texttt{x} eine Matrix ist, plottet \texttt{hist} pro Kolonne ein Histogramm. Mit dem Befehl {\color{red}\texttt{hist(x,n)}} definiert man zusätzlich die Anzahl \texttt{n} Intervalle. Mit dem Befehl {\color{red}\texttt{hist(x,y)}} berechnet MATLAB die Verteilung von \texttt{x} bezüglich \texttt{y}. \texttt{y} ist ein Vektor, deren Elemente in aufsteigender Ordnung aufgelistet sind. Jedes einzelne Element von \texttt{y} entspricht einem Zentrum.     
\newline\newline
Mit dem Befehl \boxed{\textbf{\texttt{pie(x)}}} stellt die Daten aus dem Vektor \texttt{x} in einem Kuchendiagramm dar. Die Elemente von \texttt{x} werden mit der Summe der \texttt{x}-Werte dividiert. Damit ist die Grösse von jedem einzelnen Kuchenstück in \% gegeben. Mit dem Befehl {\color{red}\texttt{pie(x,explode)}} zieht mit "explode" die gewünschte Stücke aus dem Kuchen, \texttt{explode} ist ein Vektor derselben Länge wie \texttt{x}. Seine Elemente haben entweder den Betrag 0, d.h. das entsprechende Stück von \texttt{x} verbleibt im Kuchen, ode rden Betrag 1, d.h. das dazugehörende Stück von \texttt{x} wird aus dem Kuchen herausgezogen.
\newline\newline
Mit dem Befehl \boxed{\textbf{\texttt{stem(y)}}} zeichnet eine Verteilung der Daten aus dem Vektor \texttt{y}. Jeder Wert aus \texttt{y} im Plot mit einem Kreis und einer Linie versehen. Auf der Abszisse wird jedes Element aus \texttt{x} fortlaufend eingereiht. Auf der Ordinate kann sein Wert abgelesen werden. Der entsprechende Wert ist mit ienem Kreis gekennzeichnet. Bei {\color{red}\texttt{stem(x,y)}} entsprechen die Elemente aus dem \texttt{x}-Vektor den \texttt{x}-Werten und diejenigen aus dem \texttt{y}-Vektor den \texttt{y}-Werten. Mit dem Befehl {\color{red}\texttt{stem(..., 'filled')}} malt den Kreis mit der entsprechenden Farbe aus. Mit dem Befehl {\color{red}\texttt{stem(..., 'linespec')}} kann der Linien-Typ gewählt werden. Alle möglichen "line specifications" werden mit {\color{red}\texttt{help plot}} aufgelistet.    
\newline\newline
Mit dem Befehl \boxed{\textbf{\texttt{contour(Z)}}} zeichnet einen Konturplot mit den WErten aus der Matrix \texttt{Z}. Die Elemente der Matrix \texttt{Z} entsprechen den Werten auf der \texttt{z}-Achse. Die Kolonnenzahlen ergeben die Koordinaten auf der \texttt{x}-Achse und die Zeilenzahlen die WErte auf der \texttt{y}-Achse. Die Höhen für die einzelnen Höhenlinien werden automatisch ausgewählt. Mit {\color{red}\texttt{contour(x,y,z)}} werden die \texttt{x}- und \texttt{y}-Koordinaten explizit mitgeliefert. {\color{red}\texttt{contour(z,N)}} und {\color{red}\texttt{contour(x,y,z,N)}} verwendet für den Konturplot \texttt{N} Höhenlinien. {\color{red}\texttt{contour(Z,v)}} und {\color{red}\texttt{contour(x,y,Z,v)}} zeichnet all jene Höhenlinien, deren Höhen im Vektor \texttt{v} angegeben werden. Mit {\color{red}\texttt{contour(Z,[v v])}} plottet MATLAB eine einzige Höhenlinie bei der Höhe \texttt{v}. Mit {\color{red}\texttt{contour(...,'linespec')}} kann der Linien-Typ gewählt werden. Alle möglichen "line specifications" werden mit {\color{red}\texttt{help plot}} aufgelistet...  
\section{Die Kinematik starrer Körper}
%%%%%%%%%%%%%%%%%%%%%%%%%%%%%%%%%%%%%%%%%%%%%%%%%%%%%%%%%%%%%%%%%%%%%%%%%%%%%%%%%%%%%%%%%%%%%%%%%%%%%%%%%%%%%%
\section{Unendliche Reihe}
\subsection{Grundbegriffe}
Aus den Gliedern einer \textbf{unendlichen Zahlenfolge} $\langle a_n\rangle=a_1, a_2,\dotso, a_n, \dotso$ werden wie folgt Partial- oder Teilsummen $s_n$ gebildet.
\begin{equation}
\boxed{s_n=a_1+a_2+a_3+\dotso + a_n=\displaystyle \sum_{k=1}^na_k}
\end{equation}
Die Folge $\langle s_n\rangle$ dieser Partialsummen heisst \textbf{unendliche Reihe}. Besitzt die Folge der Partialsummen $s_n$ einen Grenzwert $s$, $\displaystyle \lim_{n\rightarrow \infty}s_n=s$, so heisst die unendliche Reihe $\displaystyle \sum_{n=1}^{\infty}a_n$ \textbf{konvergent} mit dem Summenwert $s$. Besitzt die Partialsumme keinen Grenzwert, so heisst die unendliche Reihe \textbf{divergent}. 
\begin{equation}
\boxed{\displaystyle \sum_{n=1}^{\infty}a_n=a_1+a_2+a_3+\dotso + a_n+\dotso=s}
\end{equation}
Eine unendliche Reihe $\displaystyle \sum_{n=1}^{\infty}a_n$ heisst \textbf{absolut konvergent}, wenn die aus den Beträgen ihrer Glieder gebildete Reihe $\displaystyle \sum_{n=1}^{\infty}\Big\vert a_n\Big\vert$ konvergiert. Eine Reihe mit dem Summenwert $s=\pm\infty$ ist divergent.
\subsection{Konvergenzkriterien}
Die Bedingung $\displaystyle \lim_{n\rightarrow \infty}a_n=0$ ist zwar notwendig, nicht aber hinreichend für die Konvergenz der Reihe $\displaystyle \sum_{n=1}^{\infty}a_n$. Die Reihenglieder einer konvergenten Reihe müssen also eine \textbf{Nullfolge} bilden.
\newline\newline
Folgende Bedingungen stellen hinreichende Konvergenzbedingungen dar. Sie ermöglichen in vielen Fällen eine Entscheidung darüber, ob eine vorgegebene Reihe \textbf{konvergiert} oder \textbf{divergiert}. 
\subsubsection{Quotientenkriterium}
\begin{equation}
\boxed{\displaystyle \lim_{n\rightarrow \infty}\Big\vert\dfrac{a_{n+1}}{a_n} \Big\vert=q<1}
\end{equation}
Für $q>1$ divergiert die Reihe, für $q=1$ versagt das Kriterium, d.h. eine Entscheidung über Konvergenz oder Divergenz ist anhand dieses Kriteriums nicht möglich.
\subsubsection{Wurzelkriterium}
\begin{equation}
\boxed{\displaystyle \lim_{n\rightarrow \infty}\sqrt[n]{a_n}=q<1}
\end{equation}
Für $q>1$ divergiert die Reihe, für $q=1$ versagt das Kriterium, d.h. eine Entscheidung über Konvergenz oder Divergenz ist anhand dieses Kriteriums nicht möglich.
\subsubsection{Vergleichskriterien}
Das Konvergenzverhalten einer unendlichen Reihe $\displaystyle \sum_{n=1}^{\infty}a_n$ mit positiven Gliedern kann oft mit Hilfe einer geeigneten konvergenten bzw. divergenten Vergleichsreihe $\displaystyle \sum_{n=1}^{\infty}b_n$ bestimmt werden. Mit dem \textbf{Majorantenkriterium} kann die Konvergenz, mit dem \textbf{Minorantenkriterium} die Divergenz einer Reihe festgestellt werden. 
\subsubsection{Majorantenkriterium}
Die vorliegende Reihe konvergiert, wenn die Vergleichsreihe konvergiert und zwischen den Gliedern beider Reihen die Beziehung besteht
\begin{equation}
\boxed{a_n\leq b_n,\quad \forall n\in \mathbb{N}^*}
\end{equation}
Die konvergente Vergleichsreihe wird als \textbf{Majorante} bezeichnet. Es genügt wenn die angegebene Bedingung $a_n\leq b_n$ von einem gewissen $n_0$ an, d.h. für alle Reihenglieder mit $n\geq n_0$ erfüllt wird.
\subsubsection{Minorantenkriterium}
Die vorliegende Reihe divergiert, wenn die Vergleichsreihe divergiert und zwischen den Gliedern beider Reihen die Beziehung besteht
\begin{equation}
\boxed{a_n\geq b_n,\quad \forall n\in \mathbb{N}^*}
\end{equation}
Die divergente  Vergleichsreihe wird als \textbf{Minorante} bezeichnet. Es genügt wenn die angegebene Bedingung $a_n\geq b_n$ von einem gewissen $n_0$ an, d.h. für alle Reihenglieder mit $n\geq n_0$ erfüllt wird.
\subsubsection{Leibnizkriterien}
Eine alternierende Reihe konvergiert, wenn sie die folgenden Bedingungen erfüllt: Die Glieder eine rkonvergenten alternierende Reihe bilden dem Betrage nach eine monoton fallende Nullfolge. Die Reihe konvergiert auch dann, wenn die erste der beiden Bedingungen erst von einem bestimmten Glied an erfüllt ist.
\subsubsection{Eigenschaften}
\begin{enumerate}[$(a)$]
\item Eine konvergente Reihe bleibt konvergent, wenn man endlich viele Glieder weglässt oder hinzufügt oder abändert. Dabei kann sich jedoch der Summenwert ändern. Klammern dürfen in Allgemeinen nicht weggelassen werden, ebenso wenig darf die Reihenfolge der Glieder verändert werden.
\item Aufeinander folgende Glieder einer konvergenten Reihe dürfen durch eine Klammer zusammengefasst werden; der Summenwert der Reihe bleibt dabei erhalten.
\item Eine konvergente Reihe darf gliedweise mit einer Konstanten multipliziert werden, wobei sich auch der Summenwert der Reihe mit dieser Konstanten multipliziert.
\item Konvergente Reihen dürfen gliedweise addiert und subtrahiert werden, wobei sich ihre SUmmenwerte addieren bzw. subtrahieren.
\item Eine absolut konvergente Reihe ist stets konvergent. Für solche Reihen gelten sinngemäss die gleichen Rechenregeln wie für endliche Summen gliedweise Addition, Subtraktion und Multiplikation, beliebige Anordnung der Reihenglieder usw.
\end{enumerate}
\subsection{Spezielle konvergente Reihen}
\subsubsection{Geometrische Reihe}
Divergenz für $\Big\vert q\Big\vert\geq 1$
\begin{equation}
\boxed{\displaystyle \sum_{n=1}^{\infty}\left(a\cdot q^{n-1}\right)=a+aq+aq^2+\dotso+aq^{n-1}+\dotso=\dfrac{a}{1-q},\quad \left(\Big\vert q\Big\vert<1\right)}
\end{equation}
\subsubsection{Weitere konvergente Reihen}
\begin{enumerate}[$(a)$]
\item $1+\dfrac{1}{1!}+\dfrac{1}{2!}+\dfrac{1}{3!}+\dotso+\dfrac{1}{n!}+\dotso=e$
\item $1-\dfrac{1}{2}+\dfrac{1}{3}-\dfrac{1}{4}+-\dotso+\left(-1\right)^{n+1}\cdot \dfrac{1}{n}+\dotso=\ln\left(2\right)$
\item $1-\dfrac{1}{3}+\dfrac{1}{5}-\dfrac{1}{7}+-\dotso+\left(-1\right)^{n+1}\cdot \dfrac{1}{2n-1}+\dotso=\dfrac{\pi}{4}$
\item $\dfrac{1}{1^2}+\dfrac{1}{2^2}+\dfrac{1}{3^2}+\dfrac{1}{4^2}+\dotso+\dfrac{1}{n^2}+\dotso=\dfrac{\pi^2}{6}$
\item $\dfrac{1}{1^2}-\dfrac{1}{2^2}+\dfrac{1}{3^2}-\dfrac{1}{4^2}+-\dotso+\left(-1\right)^{n+1}\cdot \dfrac{1}{n^2}+\dotso=\dfrac{\pi^2}{12}$
\item $\dfrac{1}{1\cdot 2}+\dfrac{1}{2\cdot 3}+\dfrac{1}{3\cdot 4}+\dfrac{1}{4\cdot 5}+\dotso+\dfrac{1}{n\cdot \left(n+1\right)}+\dotso=1$
\end{enumerate}
%%%%%%%%%%%%%%%%%%%%%%%%%%%%%%%%%%%%%%%%%%%%%%%%%%%%%%%%%%%%%%%%%%%%%%%%%%%%%%%%%%%%%%%%%%%%%%%%%%%%%%%%%%%%%%
\section{Potenzreihen}
\subsection{Definition einer Potenzreihe}
\subsubsection{Entwicklung um die Stelle $x_0$}
\begin{equation}
\boxed{P\left(x\right)=\displaystyle \sum_{n=0}^{\infty}a_n \left(x-x_0\right)^n=a_0+a_1 \left(x-x_0\right)+a_2 \left(x-x_0\right)^2+\dotso+a_n \left(x-x_0\right)^n+\dotso}
\end{equation}
\subsubsection{Entwicklung um den Nullpunkt $x_0=0$}
\begin{equation}
\boxed{P\left(x\right)=\displaystyle \sum_{n=0}^{\infty}a_nx^n=a_0+a_1x+a_2x^2+\dotso+a_nx^n+\dotso}
\end{equation}
\subsection{Konvergenzradius und Konvergenzbereich}

Der Konvergenzbereich einer Potenzreihe $\displaystyle \sum_{n=0}^{\infty}a_nx^n$ besteht aus dem offenen Intervall positive Zahl $r$ heisst Konvergenzradius. Für $\Big\vert x\Big\vert>r$ divergiert die Potenzreihe.
\subsubsection{Berechnung des Konvergenzradius $r$}
Folgende Formeln gelten auch für eine um die Stelle $x_0$ entwickelte Potenzreihe. Die Reihe konvergiert dann im Intervall $\Big\vert x-x_0\Big\vert<r$, zu dem gegebenfalls noch ein oder gar beide Randpunkte hinzukommen. 
\begin{equation} 
\boxed{r=\displaystyle \lim_{n\rightarrow \infty}\Big\vert \dfrac{a_n}{a_{n+1}}\Big\vert}\quad \boxed{r=\dfrac{1}{\displaystyle \lim_{n\rightarrow \infty}\sqrt[n]{\Big\vert a_n\Big\vert}}}
\end{equation} 
\begin{enumerate}[$(i)$]
\item Sei $r=0$ so ist konvergiert die Potenzreihe nur für $n\rightarrow \infty$
\item Sei $r=\infty$, so konvergiert die Potenzreihe beständig, d.h. für jedes $x\in \mathbb{R}$ 
\end{enumerate}
\subsection{Eigenschaften einer Potenzreihe}
\begin{enumerate}[$(i)$]
\item Eine Potenzreihe konvergiert innerhalb ihres Konvergentbereiches absolut.
\item Eine Potenzreihe darf innerhalb ihres Konvernegzbereiches gliedweise differenziert und integriert werden. Die neuen Potenzreihen haben dabei denselben Konvergenzradius $r$ wie die ursprüngliche Reihe.
\item Zwei Potenzreihen dürfen im gemeinsamen Konvergenzbereich der Reihen gliedweise addiert, subtrahiert und multipliziert werden. Die neuen Potenzreihen konvergieren dann mindestens im gemeinsamen Konvergenzbereich der beiden Ausgangsreihen. 
\end{enumerate}
\section{Taylor-Reihen}
\subsection{Taylorsche und Mac Laurinsche Formel}
\subsubsection{Taylorsche Formel}
\begin{equation}
\boxed{f_n\left(x\right)=f\left(x_0\right)+\dfrac{f'\left(x_0\right)}{1!}\left(x-x_0\right)+\dfrac{f''\left(x_0\right)}{2!}\left(x-x_0\right)^2+\dotso+\dfrac{f^{\left(n\right)}\left(x_0\right)}{n!}\left(x-x_0\right)^n}
\end{equation}
\begin{equation}
\boxed{R_n\left(x\right)=\dfrac{f^{\left(n+1\right)}\left(\xi\right)}{\left(n+1\right)!}\left(x-x_0\right)^{n+1},\quad \left(x< \xi< x_0\right)}
\end{equation}
\begin{equation}
\boxed{f\left(x\right)=f_n\left(x\right)+R_n\left(x\right)}
\end{equation}
\subsubsection{Mac Laurinsche Formel}
\begin{equation}
\boxed{f_n\left(x\right)=f\left(0\right)+\dfrac{f'\left(0\right)}{1!}\left(x\right)+\dfrac{f''\left(0\right)}{2!}\left(x\right)^2+\dotso+\dfrac{f^{\left(n\right)}\left(0\right)}{n!}\left(x\right)^n}
\end{equation}
\begin{equation}
\boxed{R_n\left(x\right)=\dfrac{f^{\left(n+1\right)}\left(\theta x\right)}{\left(n+1\right)!}\left(x\right)^{n+1},\quad \left(0< \theta< 1\right)}
\end{equation}
\begin{equation}
\boxed{f\left(x\right)=f_n\left(x\right)+R_n\left(x\right)}
\end{equation}
\subsection{Taylorsche Reihe}
$f\left(x\right)$ ist in der Umgebung von $x_0$ beliebig oft differenzierbar und das Restglied $R_n\left(x\right)$ in der Taylorschen Formel verschwindet für $n\rightarrow \infty$
\begin{equation}
\boxed{
\begin{array}{lll}
f\left(x\right)&=&f\left(x_0\right)+\dfrac{f'\left(x_0\right)}{1!}\left(x-x_0\right)+\dfrac{f''\left(x_0\right)}{2!}\left(x-x_0\right)^2+\dotso\\
&=&\displaystyle \sum_{n=0}^{\infty}\dfrac{f^{\left(n\right)}\left(x_0\right)}{n!}\left(x-x_0\right)^n
\end{array}
}
\end{equation}
\subsection{Mac Laurinsche Reihe}
Die Mac Laurinsche Reihe ist eine spezielle Form der Taylorschen Reihe für das Entwicklungszentrum $x_0=0$. Bei einer geraden Funktion treten nur gerade Potenzen auf, bei einer ungeraden Funktion nur ungerade Potenzen.
\begin{equation}
\boxed{
\begin{array}{lll}
f\left(x\right)&=&f\left(0\right)+\dfrac{f'\left(0\right)}{1!}x+\dfrac{f''\left(0\right)}{2}x^2+\dotso\\
&=&\displaystyle \sum_{n=0}^{\infty}\dfrac{f^{\left(n\right)}\left(0\right)}{n!}x^n
\end{array}
}
\end{equation}
\section{Spezielle Potenzreihenentwicklungen}
\subsubsection{Allgemeine Binomische Reihe}
\begin{enumerate}[$(a)$]
\item $\left(1\pm x\right)^n=1\pm \displaystyle \binom{n}{1}x+\displaystyle \binom{n}{2}x^2\pm \displaystyle \binom{n}{3}x^3+\displaystyle \binom{n}{4}x^4\pm \dotso$
\item $\left(a\pm x\right)^n=a^n\pm \displaystyle \binom{n}{1}a^{n-1}x+\displaystyle \binom{n}{2}a^{n-2}x^2\pm \displaystyle \binom{n}{3}a^{n-3}x^3+\displaystyle \binom{n}{4}a^{n-4}x^4\pm \dotso$
\end{enumerate}
\subsubsection{Spezielle Binomische Reihen}
\begin{enumerate}[$(a)$]
\item $\left(1\pm x\right)^{1/4}=1\pm \dfrac{1}{4}x-\dfrac{1}{4}\dfrac{3}{8}x^2\pm \dfrac{1}{4}\dfrac{3}{8}\dfrac{7}{12}x^3-\dfrac{1}{4}\dfrac{3}{8}\dfrac{7}{12}\dfrac{11}{16}x^4\pm \dotso\quad \Bigg\{\begin{matrix}n>0: \Big\vert x\Big\vert\leq 1\\n<0: \Big\vert x\Big\vert< 1\end{matrix}$
\item $\left(1\pm x\right)^{1/3}=1\pm \dfrac{1}{3}x-\dfrac{1}{3}\dfrac{2}{6}x^2\pm \dfrac{1}{3}\dfrac{2}{6}\dfrac{5}{9}x^3-\dfrac{1}{3}\dfrac{2}{6}\dfrac{5}{9}\dfrac{8}{12}x^4\pm \dotso\quad \Bigg\{\begin{matrix}n>0: \Big\vert x\Big\vert\leq \Big\vert a\Big\vert\\n<0: \Big\vert x\Big\vert< \Big\vert a\Big\vert\end{matrix}$
\item $\left(1\pm x\right)^{1/2}=1\pm \dfrac{1}{2}x-\dfrac{1}{2}\dfrac{1}{4}x^2\pm \dfrac{1}{2}\dfrac{1}{4}\dfrac{3}{6}x^3-\dfrac{1}{2}\dfrac{1}{4}\dfrac{3}{6}\dfrac{5}{8}x^4\pm \dotso\quad \Big\vert x\Big\vert\leq 1$
\item $\left(1\pm x\right)^{3/2}=1\pm \dfrac{3}{2}x+\dfrac{3}{2}\dfrac{1}{4}x^2\mp \dfrac{3}{2}\dfrac{1}{4}\dfrac{1}{6}x^3+\dfrac{3}{2}\dfrac{1}{4}\dfrac{1}{6}\dfrac{3}{8}x^4\mp \dotso\quad \Big\vert x\Big\vert\leq 1$
\item $\left(1\pm x\right)^{-1/4}=1\mp \dfrac{1}{4}x+\dfrac{1}{4}\dfrac{5}{8}x^2\mp \dfrac{1}{4}\dfrac{5}{8}\dfrac{9}{12}x^3+\dfrac{1}{4}\dfrac{5}{8}\dfrac{9}{12}\dfrac{13}{16}x^4\mp \dotso\quad \Big\vert x\Big\vert< 1$
\item $\left(1\pm x\right)^{-1/3}=1\mp \dfrac{1}{3}x+\dfrac{1}{3}\dfrac{4}{6}x^2\mp \dfrac{1}{3}\dfrac{4}{6}\dfrac{7}{9}x^3+\dfrac{1}{3}\dfrac{4}{6}\dfrac{7}{9}\dfrac{10}{12}x^4\mp \dotso\quad \Big\vert x\Big\vert< 1$
\item $\left(1\pm x\right)^{-1/2}=1\mp \dfrac{1}{2}x+\dfrac{1}{2}\dfrac{3}{4}x^2\mp \dfrac{1}{2}\dfrac{3}{4}\dfrac{5}{6}x^3+\dfrac{1}{2}\dfrac{3}{4}\dfrac{5}{6}\dfrac{7}{8}x^4\mp \dotso\quad \Big\vert x\Big\vert< 1$
\item $\left(1\pm x\right)^{-1}=1\mp x+x^2\mp x^3+x^4\mp \dotso\quad \Big\vert x\Big\vert\leq 1$
\item $\left(1\pm x\right)^{-3/2}=1\mp \dfrac{3}{2}x+\dfrac{3}{2}\dfrac{5}{4}x^2\mp \dfrac{3}{2}\dfrac{5}{4}\dfrac{7}{6}x^3+\dfrac{3}{2}\dfrac{5}{4}\dfrac{7}{6}\dfrac{9}{8}x^4\mp \dotso\quad \Big\vert x\Big\vert< 1$
\item $\left(1\pm x\right)^{-2}=1\mp 2x+3x^2\mp 4x^3+5x^4\mp \dotso\quad \Big\vert x\Big\vert< 1$
\item $\left(1\pm x\right)^{-3}=1\mp \dfrac{1}{2}\left(2\cdot 3x\mp 3\cdot 4x^2+4\cdot 5x^3\mp 5\cdot 6x^4+\dotso\right)\quad \Big\vert x\Big\vert< 1$
\end{enumerate}
\subsubsection{Reihen der Exponentialfunktionen}
\begin{enumerate}[$(a)$]
\item $e^x=1+\dfrac{x}{1!}+\dfrac{x^2}{2!}+\dfrac{x^3}{3!}+\dfrac{x^4}{4!}+\dotso=\displaystyle \sum_{k=0}^{\infty}\dfrac{z^k}{k!}\quad \Big\vert x\Big\vert< \infty$
\item $e^{-x}=1-\dfrac{x}{1!}+\dfrac{x^2}{2!}-\dfrac{x^3}{3!}+\dfrac{x^4}{4!}-+\dotso\quad \Big\vert x\Big\vert< \infty$
\item $a^x=1+\dfrac{\ln\left(a\right)}{1!}x+\dfrac{\left(\ln\left(a\right)\right)^2}{2!}x^2+\dfrac{\left(\ln\left(a\right)\right)^3}{3!}x^3+\dfrac{\left(\ln\left(a\right)\right)^4}{4!}x^4+\dotso\quad \Big\vert x\Big\vert< \infty$
\end{enumerate}
\subsubsection{Reihen der logarithmischen Funktionen}
\begin{enumerate}[$(a)$]
\item $\ln\left(x\right)=\left(x-1\right)-\dfrac{1}{2}\left(x-1\right)^2+\dfrac{1}{3}\left(x-1\right)^3-\dfrac{1}{4}\left(x-1\right)^4+-\dotso\quad 0<x\leq 2$
\item $\ln\left(x\right)=2\Big[\left(\dfrac{x-1}{x+1}\right)+\dfrac{1}{3}\left(\dfrac{x-1}{x+1}\right)^3+\dfrac{1}{5}\left(\dfrac{x-1}{x+1}\right)^5+\dfrac{1}{7}\left(\dfrac{x-1}{x+1}\right)^7+\dotso\Big]\quad x>0$
\item $\ln\left(1+x\right)=x-\dfrac{x^2}{2}+\dfrac{x^3}{3}-\dfrac{x^4}{4}+-\dotso\quad -1<x\leq 1$
\item $\ln\left(1-x\right)=-\Big[x+\dfrac{x}{2}+\dfrac{x^3}{3}+\dfrac{x^4}{4}+\dotso\Big]\quad -1\leq x\leq 1$
\item $\ln\left(\dfrac{1+x}{1-x}\right)=2\Big[x+\dfrac{x^3}{3}+\dfrac{x^5}{5}+\dfrac{x^7}{7}+\dotso\Big]\quad \Big\vert x\Big\vert<1$
\end{enumerate}
\subsubsection{Reihen der trigonometrischen Funktionen}
\begin{enumerate}[$(a)$]
\item $\sin\left(x\right)=x-\dfrac{x^3}{3!}+\dfrac{x^5}{5!}-\dfrac{x^7}{7!}+-\dotso=\displaystyle \sum_{k=0}^{\infty}\left(-1\right)^k\dfrac{z^{2k+1}}{\left(2k+1\right)!}\quad \Big\vert x\Big\vert<\infty$
\item $\cos\left(x\right)=1-\dfrac{x^2}{2!}+\dfrac{x^4}{4!}-\dfrac{x^6}{6!}+-\dotso=\displaystyle \sum_{k=0}^{\infty}\left(-1\right)^k\dfrac{z^{2k}}{\left(2k\right)!}\quad \Big\vert x\Big\vert<\infty$
\item $\tan\left(x\right)=x+\dfrac{1}{3}x^3+\dfrac{2}{15}x^5+\dfrac{17}{315}x^7+\dfrac{62}{2835}x^9+\dotso\quad \Big\vert x\Big\vert<\dfrac{\pi}{2}$
\item $\cot\left(x\right)=\dfrac{1}{x}-\dfrac{1}{3}x-\dfrac{1}{45}x^3-\dfrac{2}{945}x^5-\dotso\quad 0<\Big\vert x\Big\vert<\pi$
\end{enumerate}
\subsubsection{Reihen der Arkusfunktionen}
\begin{enumerate}[$(a)$]
\item $\arcsin\left(x\right)=x+\dfrac{1}{2\cdot 3}x^3+\dfrac{1\cdot 3}{2\cdot 4\cdot 5}x^5+\dfrac{1\cdot 3\cdot 5}{2\cdot 4\cdot 6\cdot 7}x^7+\dotso\Big\vert x\Big\vert<1$
\item $\arccos\left(x\right)=\dfrac{\pi}{2}-\Big[x+\dfrac{1}{2\cdot 3}x^3+\dfrac{1\cdot 3}{2\cdot 4\cdot 5}x^5+\dfrac{1\cdot 3\cdot 5}{2\cdot 4\cdot 6\cdot 7}x^7+\dotso\Big]\quad \Big\vert x\Big\vert<1$
\item $\arctan\left(x\right)=x-\dfrac{x^3}{3}+\dfrac{x^5}{5}-\dfrac{x^7}{7}+-\dotso\quad \Big\vert x\Big\vert<1$ 
\item $\arccot\left(x\right)=\dfrac{\pi}{2}-\Big[x-\dfrac{x^3}{3}+\dfrac{x^5}{5}-\dfrac{x^7}{7}+-\dotso\Big]\quad \Big\vert x\Big\vert<1$ 
\end{enumerate}
\subsubsection{Reihen der Hyperbelfunktionen}
\begin{enumerate}[$(a)$]
\item $\sinh\left(x\right)=x+\dfrac{x^3}{3!}+\dfrac{x^5}{5!}+\dfrac{x^7}{7!}+\dotso=\displaystyle \sum_{k=0}^{\infty}\dfrac{z^{2k+1}}{\left(2k+1\right)!}\quad \Big\vert x\Big\vert<\infty$
\item $\cosh\left(x\right)=1+\dfrac{x^2}{2!}+\dfrac{x^4}{4!}+\dfrac{x^6}{6!}+\dotso=\displaystyle \sum_{k=0}^{\infty}\dfrac{z^{2k}}{\left(2k\right)!}\quad \Big\vert x\Big\vert<\infty$
\item $\tanh\left(x\right)=x-\dfrac{1}{3}x^3+\dfrac{2}{15}x^5-\dfrac{17}{315}x^7+\dfrac{62}{2835}x^9-+\dotso\quad \Big\vert x\Big\vert<\dfrac{\pi}{2}$
\item $\coth\left(x\right)=\dfrac{1}{x}+\dfrac{1}{3}x-\dfrac{1}{45}x^3+\dfrac{2}{945}x^5-+\dotso\quad 0<\Big\vert x\Big\vert<\pi$
\end{enumerate}
\subsubsection{Reihen der Areafunktionen}
\begin{enumerate}[$(a)$]
\item $\text{Arsinh}\left(x\right)=x-\dfrac{1}{2\cdot 3}x^3+\dfrac{1\cdot 3}{2\cdot 4\cdot 5}x^5-\dfrac{1\cdot 3\cdot 5}{2\cdot 4\cdot 6\cdot 7}x^7+-\dotso=\displaystyle \sum_{k=0}^{\infty}\binom{-1/2}{k}\dfrac{z^{2k+1}}{2k+1}\quad \Big\vert x\Big\vert<1$
\item $\text{Arcosh}\left(x\right)=\ln\left(2x\right)-\dfrac{1}{2\cdot 2x^2}+\dfrac{1\cdot 3}{2\cdot 4\cdot 4x^4}-\dfrac{1\cdot 3\cdot 5}{2\cdot 4\cdot 6\cdot 6x^6}-\dotso\quad \Big\vert x\Big\vert>1$
\item $\text{Artanh}\left(x\right)=x+\dfrac{x^3}{3}+\dfrac{x^5}{5}+\dfrac{x^7}{7}+\dotso=\displaystyle \sum_{k=0}^{\infty}\dfrac{z^{2k+1}}{2k+1}\quad \Big\vert x\Big\vert<1$
\item $\text{Arcoth}\left(x\right)=\dfrac{1}{x}+\dfrac{1}{3x^3}+\dfrac{1}{5x^5}+\dfrac{1}{7x^7}+\dotso\quad \Big\vert x\Big\vert>1$
\end{enumerate}
\section{Näherungspolynome einer Funktion}
Bricht man die Potenzreihenentwicklung einer Funktion $f\left(x\right)$ nach der $n$-ten POtenz ab, so erhält man ein Näherungspolynom $f_n\left(x\right)$ vom Grade $n$ für $f\left(x\right)$, das sogenannte Mac. Laurinsches bzw. Taylorsches Polynom. Funktion $f\left(x\right)$ und $f_n\left(x\right)$ stimmen an der Entwicklungsstelle $x_0$ in ihrem Funktionswert und in ihren ersten $n$ Ableitungen miteinander überein.
\subsection{Fehlerabschätzung}
Der durch den Abbruch der Potenzreihe entstandene Fehler lässt sich in Allgemeinen anhand der Lagrangeschen Restgliedformel abschätzen. Er liegt in der Grössenordnung des grössten Reihengliedes, das in der Näherung nicht mehr berücksichtigt wurde.
\subsection{Näherungspolynome spezieller Funktionen}
\begin{enumerate}[$(a)$]
\item $\left(1\pm x\right)^n=\Bigg\{\begin{matrix}=1+\pm nx,\quad \text{(1. Näherung)}\\=1\pm nx+\dfrac{n\left(n-1\right)}{2}x^2,\quad \text{(2. Näherung)}\end{matrix}$
\item $e^x=\Bigg\{\begin{matrix}=1+x,\quad \text{(1. Näherung)}\\=1+x+\dfrac{1}{2}x^2,\quad \text{(2. Näherung)}\end{matrix}$
\item $e^{-x}=\Bigg\{\begin{matrix}=1-x,\quad \text{(1. Näherung)}\\=1-x+\dfrac{1}{2}x^2,\quad \text{(2. Näherung)}\end{matrix}$
\item $a^{x}=\Bigg\{\begin{matrix}=1+\ln\left(a\right)x,\quad \text{(1. Näherung)}\\=1+\ln\left(a\right)x+\dfrac{\left(\ln\left(a\right)\right)^2}{2}x^2,\quad \text{(2. Näherung)}\end{matrix}$
\item $\ln\left(1+x\right)=\Bigg\{\begin{matrix}=x,\quad \text{(1. Näherung)}\\=x-\dfrac{1}{2}x^2,\quad \text{(2. Näherung)}\end{matrix}$
\item $\ln\left(1-x\right)=\Bigg\{\begin{matrix}=-x,\quad \text{(1. Näherung)}\\=-x-\dfrac{1}{2}x^2,\quad \text{(2. Näherung)}\end{matrix}$
\item $\ln\left(\dfrac{1+x}{1-x}\right)=\Bigg\{\begin{matrix}=2x,\quad \text{(1. Näherung)}\\=2x+\dfrac{2}{3}x^3,\quad \text{(2. Näherung)}\end{matrix}$
\item $\sin\left(x\right)=\Bigg\{\begin{matrix}=x,\quad \text{(1. Näherung)}\\=x-\dfrac{1}{6}x^3,\quad \text{(2. Näherung)}\end{matrix}$
\item $\cos\left(x\right)=\Bigg\{\begin{matrix}=1-\dfrac{1}{2}x^2,\quad \text{(1. Näherung)}\\=1-\dfrac{1}{2}x^2+\dfrac{1}{24}x^4,\quad \text{(2. Näherung)}\end{matrix}$
\item $\tan\left(x\right)=\Bigg\{\begin{matrix}=x,\quad \text{(1. Näherung)}\\=x+\dfrac{1}{3}x^3,\quad \text{(2. Näherung)}\end{matrix}$
\item $\arcsin\left(x\right)=\Bigg\{\begin{matrix}=x,\text{(1. Näherung)}\\=x+\dfrac{1}{6}x^3,\quad \text{(2. Näherung)}\end{matrix}$
\item $\arccos\left(x\right)=\Bigg\{\begin{matrix}=\dfrac{\pi}{2}-x,\text{(1. Näherung)}\\=\dfrac{\pi}{2}-x-\dfrac{1}{6}x^3,\quad \text{(2. Näherung)}\end{matrix}$
\item $\arctan\left(x\right)=\Bigg\{\begin{matrix}=x,\text{(1. Näherung)}\\=x-\dfrac{1}{3}x^3,\quad \text{(2. Näherung)}\end{matrix}$
\item $\arccot\left(x\right)=\Bigg\{\begin{matrix}=\dfrac{\pi}{2}-x,\text{(1. Näherung)}\\=\dfrac{\pi}{2}-x+\dfrac{1}{3}x^3,\quad \text{(2. Näherung)}\end{matrix}$
\item $\sinh\left(x\right)=\Bigg\{\begin{matrix}=x,\text{(1. Näherung)}\\=x+\dfrac{1}{6}x^3,\quad \text{(2. Näherung)}\end{matrix}$
\item $\cosh\left(x\right)=\Bigg\{\begin{matrix}=1+\dfrac{1}{2}x^2,\text{(1. Näherung)}\\=1+\dfrac{1}{2}x^2+\dfrac{1}{24}x^4,\quad \text{(2. Näherung)}\end{matrix}$
\item $\tanh\left(x\right)=\Bigg\{\begin{matrix}=x,\text{(1. Näherung)}\\=x-\dfrac{1}{3}x^3,\quad \text{(2. Näherung)}\end{matrix}$
\item $\text{Arsinh}\left(x\right)=\Bigg\{\begin{matrix}=x,\text{(1. Näherung)}\\=x-\dfrac{1}{6}x^3,\quad \text{(2. Näherung)}\end{matrix}$
\item $\text{Artanh}\left(x\right)=\Bigg\{\begin{matrix}=x,\text{(1. Näherung)}\\=x+\dfrac{1}{3}x^3,\quad \text{(2. Näherung)}\end{matrix}$
\end{enumerate}
\section{Fourier-Reihen}
\subsection{Fourier-Reihe einer periodischen Funktion}
Eine periodische Funktion $f\left(x\right)$ mit der Periode $p=2\pi$ lässt sich unter bestimmten Voraussetzungen in eine unendliche trigonometrische Reihe der Form entwickeln.
\begin{equation}
\boxed{f\left(x\right)=\dfrac{a_0}{2}+\displaystyle \sum_{n=1}^{\infty}\Big[a_n\cdot \cos\left(nx\right)+b_n\cdot \sin\left(nx\right)\Big]}
\end{equation}
\subsubsection{Berechnung der Fourier-Koeffizienten $a_n$ und $b_n$}
\begin{equation}
\boxed{a_0=\dfrac{1}{\pi}\displaystyle \int_0^{2\pi}f\left(x\right)\,\text{d}x}
\end{equation}
\begin{equation}
\boxed{a_n=\dfrac{1}{\pi}\displaystyle \int_0^{2\pi}f\left(x\right)\cdot \cos\left(nx\right)\,\text{d}x,\quad b_n=\dfrac{1}{\pi}\displaystyle \int_0^{2\pi}f\left(x\right)\cdot \sin\left(nx\right)\,\text{d}x}
\end{equation}
\begin{enumerate}[$(1)$]
\item Voraussetzung ist, dass die folgenden Dirichletschen Bedingungen erfüllt sind
\begin{enumerate}[$(a)$]
\item Das Periodenintervall lässt sich in endlich Teilintervalle zerlegen, in denen $f\left(x\right)$ stetig und monoton ist.
\item Besitzt die Funktion $f\left(x\right)$ im Periodenintervall Unstetigkeitsstellen, so existiert in ihnen sowohl der links-als auch der rechtsseitige Grenzwert. 
\end{enumerate}
\item In den Sprungstellen der Funktion $f\left(x\right)$ liefert die Fourier-Reihe von $f\left(x\right)$ das arithmetische Mittel aus dem links- und rechtsseitigen Grenzwert der Funktion.
\end{enumerate}
\subsubsection{Symmetriebetrachtungen}
$f\left(x\right)$ ist eine gerade Funktion:
\begin{equation}
\boxed{f\left(x\right)=\dfrac{a_0}{2}+\displaystyle \sum_{n=1}^{\infty}a_n\cdot \cos\left(nx\right),\quad \left(b_n=0\text{ für } n\in \mathbb{N}^*\right)}
\end{equation}
$f\left(x\right)$ ist eine ungerade Funktion:
\begin{equation}
\boxed{f\left(x\right)=\displaystyle \sum_{n=1}^{\infty}b_n\cdot \sin\left(nx\right),\quad \left(a_n=0\text{ für } n\in \mathbb{N}^*\right)}
\end{equation}
\subsubsection{Komplexe Darstellung der Fourier-Reihe}
\begin{equation}
\boxed{f\left(x\right)=\displaystyle \sum_{n=-\infty}^{\infty}c_n\cdot e^{jnx},\quad c_n=\dfrac{1}{2\pi}\displaystyle \int_{0}^{2\pi}f\left(x\right)\cdot e^{-jnx},\quad \left(n\in \mathbb{Z}\right)}
\end{equation}
Die komplexe Fourier-Reihe lässt sich auch wie folgt aufspalten
\begin{equation}
\boxed{f\left(x\right)=\displaystyle \sum_{n=-\infty}^{\infty}c_n\cdot e^{jnx}=c_0+\displaystyle \sum_{n=1}^\infty c_{-n}\cdot e^{-jnx}+\displaystyle \sum_{n=1}^{\infty}c_n\cdot e^{jnx}}
\end{equation}
Der Koeffizient $c_{-n}$ ist dabei konjugiert komplex zu $c_n$, d.h. $c_{-n}=c_n^*$
\subsubsection{Zusammenhang zwischen den Koeffizienten $a_n$, $b_n$ und $c_n$}
Übergang von der reellen zur komplexen Form
\begin{equation}
\boxed{c_0=\dfrac{1}{2}a_0,\quad c_n=\dfrac{1}{2}\left(a_n-jb_n\right),\quad c_{-n}=\dfrac{1}{2}\left(a_n+jb_n\right),\quad \left(n\in \mathbb{N}^*\right)}
\end{equation}
Übergang von der komplexen zur reellen Form
\begin{equation}
\boxed{a_0=2c_0,\quad a_n=c_n+c_{-n},\quad b_{n}=j\left(c_n-c_{-n}\right),\quad \left(n\in \mathbb{N}^*\right)}
\end{equation}
\subsection{Fourier einer nichtsinusförmigen Schwingung}
Eine nichtsinusförmig verlaufende Schwingung $y=y\left(t\right)$ mit der Kreisfrequenz $\omega_0$ und der Schwingungsdauer (Periodendauer) $T=2\pi/\omega$ lässt sich nach Fourier wie folgt in ihre harmonischen Bestandteile zerlegen. $\omega_0=2\pi/T$ ist die Kreisfrequenz der Grundschwingung. $n\omega_0$ sind die Kreisfrequenzen der harmonischen Oberschwingungen $\left(n=2, 3, 4, \dotso \right)$ 
\begin{equation}
\boxed{y\left(t\right)=\dfrac{a_0}{2}+\displaystyle \sum_{n=1}^{\infty}\Big[a_n\cdot \cos\left(n\omega_0t\right)+b_n\cdot \sin\left(n\omega_0t\right)\Big]}
\end{equation}
\subsubsection{Berechnung der Fourier-Koeffizienten $a_n$ und $b_n$}
\begin{equation}
\boxed{a_0=\dfrac{2}{T}\displaystyle\int_{\left(T\right)}y\left(t\right)\,\text{d}t}
\end{equation}
\begin{equation}
\boxed{a_n=\dfrac{2}{T}\displaystyle \int_{\left(T\right)}y\left(t\right)\cdot \cos\left(n\omega_0t\right)\,\text{d}t,\quad b_n=\dfrac{2}{T}\displaystyle \int_{\left(T\right)}y\left(t\right)\cdot \sin\left(n\omega_0t\right)\,\text{d}t}
\end{equation}
\subsubsection{Fourier-Zerlegung in phasenverschobene Sinusschwingungen}
\begin{equation}
\boxed{y\left(t\right)=\dfrac{a_0}{2}+\displaystyle \sum_{n=1}^{\infty}\Big[a_n\cdot \cos\left(n\omega_0t\right)+b_n\sin\left(n\omega_0t\right)\Big]=A_0+\displaystyle \sum_{n=1}^{\infty}A_n\cdot \sin\left(n\omega_0t+\varphi_n\right)}
\end{equation}
\begin{equation}
\boxed{A_0=\dfrac{a_0}{2},\quad A_n=\sqrt{a_n^2+b_n^2},\quad \tan\left(\varphi_n\right)=\dfrac{a_n}{b_n}\quad \left(n\in \mathbb{N}^*\right)}
\end{equation}
\subsubsection{Fourier-Zerlegung in komplexer Form}
\begin{equation}
\boxed{y\left(t\right)=\displaystyle \sum_{n=-\infty}^{\infty}c_n\cdot e^{jn\omega_0t}}
\end{equation}
\begin{equation}
\boxed{c_n=\dfrac{1}{T}\displaystyle \int_{0}^Ty\left(t\right)\cdot e^{-jn\omega_0t}\,\text{d}t}
\end{equation}
\subsection{Spezielle Fourier-Reihen}
\subsubsection{Rechteckskurve}
\begin{equation}
\boxed{y\left(t\right)=\Bigg\{\begin{matrix}\hat{y}&\texttt{für}&0\leq t\leq \dfrac{T}{2}\\0&\texttt{für}&\dfrac{T}{2}\leq t\leq T\end{matrix}}
\end{equation}
\begin{equation}
\boxed{y\left(t\right)=\dfrac{\hat{y}}{2}+\dfrac{2\hat{y}}{\pi}\left(\sin\left(\omega_0t\right)+\dfrac{1}{3}\sin\left(3\omega_0t\right)+\dfrac{1}{5}\sin\left(5\omega_0t\right)+\dotso\right)}
\end{equation}
\subsubsection{Rechtecksimpuls}
\begin{equation}
\boxed{b=\dfrac{T}{2}-2a}
\end{equation}
\begin{equation}
\boxed{y\left(t\right)\Bigg\{\begin{matrix}\hat{y}&\text{für}&a<t<\dfrac{T}{2}-a\\-\hat{y}&\text{für}&\dfrac{T}{2}+a<t<T-a\\0&&\text{im übrigen Intervall}\end{matrix}}
\end{equation}
\begin{equation}
\boxed{y\left(t\right)=\dfrac{4\hat{y}}{\pi}\left(\dfrac{\cos\left(\omega_0a\right)}{1}\cdot \sin\left(\omega_0t\right)+\dfrac{\cos\left(3\omega_0a\right)}{3}\cdot \sin\left(3\omega_0t\right)+\dotso\right)}
\end{equation}
\subsubsection{Dreieckskurve}
\begin{equation}
\boxed{y\left(t\right)\Bigg\{\begin{matrix}-\dfrac{2\hat{y}}{T}+\hat{y}&\text{für}&0\leq t\leq \dfrac{T}{2}\\\dfrac{2\hat{y}}{T}t-\hat{y}&\text{für}&\dfrac{T}{2}\leq t \leq T\\\end{matrix}}
\end{equation}
\begin{equation}
\boxed{y\left(t\right)=\dfrac{\hat{y}}{2}+\dfrac{4\hat{y}}{\pi^2}\left(\dfrac{1}{1^2}\cdot\cos\left(\omega_0t\right)+\dfrac{1}{3^2}\cdot \cos\left(3\omega_0t\right)+\dfrac{1}{5^2}\cdot \cos\left(5\omega_0t\right)+\dotso\right)}
\end{equation}


%\chapter{Digitale Darstellung von Signalen}
\section{Physikalische Grössen}
\subsection{Zahlwert und Einheit}
Eine physikalische Grösse beinhaltet eine Zahl und eine Einheit. Beinhaltet eine physikalische Grösse einen grossen Zahlenwert, so kann man diese als ein Vielfaches dieser Einheit ausdrücken. Die Dimension gibt Auskunft auf eine detaillierte Charakterisierung der Grösse wie die Höhe, der Abstand der die Strecke mit Einheit Meter.
\subsection{Grundeinheiten und abgeleitete Einheiten}
Die Grundeinheiten besteht aus sieben Grundeinheiten: \textbf{Länge} (Meter $\text{m}$), \textbf{Masse} (Kilogramm $\text{kg}$), Zeit (Sekunde $\text{s}$), \textbf{Stromstärke} (Ampere $\text{A}$), \textbf{Temperatur} (Kelvin $\text{K}$), \textbf{Stoffmenge} (Mol $\text{mol}$) und \textbf{Lichtstärke} (Candela $\text{cd}$).
\newline\newline
Die abgeleitete Einheiten entstehen durch Beziehungen zwischen der Grundeinheiten wie die Geschwindigkeit oder die Beschleunigung.
\newline\newline
Werden Gleichungen mit den Grundeinheiten gerechnet, so wird das Resultat auch in einer Grundeinheit ausgedrückt. Das Ziel besteht auch darin, Resultate mit Hilfe von Zehnerpotenzen zu schreiben.
\begin{table}[H]
\centering
\begin{tabular}{llll}
\hline
Grösse&Zeichen&Name&Symbol\\\hline
Länge&$l$&Meter&$\text{m}$\\
Masse&$m$&Kilogramm&$\text{kg}$\\
Zeit&$t$&Sekunde&$\text{s}$\\
Stromstärke&$I, i$&Ampere&$\text{A}$\\\hline
\end{tabular}
\caption{Grundeinheiten}
\end{table}
\begin{table}[H]
\centering
\begin{tabular}{lllll}
\hline
Grösse&Zeichen&Name&Symbol&Ausdruck\\\hline
Kraft&$F$&Newton&$\text{N}$&$\text{N}=\text{m}\,\text{kg}\, \text{s}^{-2}$\\
Leistung&$P$&Watt&$\text{W}$&$\text{W}=\text{V}\,\text{A}=\text{N}\,\text{m}\,\text{s}^{-1}$\\
Arbeit, Energie&$W$&Joule&$\text{J}$&$\text{J}=\text{W}\,\text{s}=\text{N}\,\text{m}$\\
Spannung&$U, u$&Volt&$\text{V}$&$\text{V}=\text{W}\,\text{A}^{-1}$\\
Widerstand&$R$&Ohm&$\Omega$&$\Omega=\text{V}\,\text{A}^{-1}$\\
Spezifischer Widerstand&$\rho$&&$\Omega\,\text{m}$&$\Omega\,\text{m}=\text{V}\,\text{m}\,\text{A}^{-1}$\\
Leitwert&$G$&Siemens&$\text{S}$&$\text{S}=\Omega^{-1}$\\
Spezifische Leitfähigkeit&$\sigma$&&$\text{S}\,\text{m}^{-1}$&$\text{S}\,\text{m}^{-1}=\Omega^{-1}\,\text{m}^{-1}$\\
Ladung&$Q$&Coulomb&$\text{C}$&$\text{C}=\text{A}\,\text{s}$\\
Elektr. Verschiebungsdichte&$D$&&$\text{A}\,\text{s}\,\text{m}^2$&$\text{C}\,\text{m}^{-2}=\text{A}\,\text{s}\,\text{m}^{-2}$\\
Elektr. Feldstärke&$E$&&$\text{V}\,\text{m}^{-1}$&\\
Kapazität&$C$&Farad&F&$\text{F}=\text{A}\,\text{s}\,\text{V}^{-1}=\text{s}\,\Omega^{-1}$\\
Induktionsfluss&$\phi$&Weber&$\text{Wb}$&$\text{Wb}=\text{V}\,\text{s}$\\
Magn. Induktionsdichte&$B$&Tesla&$\text{T}$&$\text{T}=\text{V}\,\text{s}\,\text{m}^{-2}$\\
Magn. Feldstärke&$H$&&$\text{A}\,\text{m}^{-1}$&\\
Induktivität&$L$&Henry&$\text{H}$&$\text{H}=\text{V}\,\text{s}\,\text{A}^{-1}=\Omega\,\text{s}$\\\hline
\end{tabular}
\caption{Abgeleitete Einheiten}
\end{table}
\begin{table}[H]
\centering
\begin{tabular}{ll}
\hline
Definition&Wert\\\hline
Lichtgeschwindigkeit im Vakuum&$c_0=299'792'458\,\text{m}\,\text{s}^{-1}$\\
Elementarladung&$e=1.6022\cdot 10^{-19}\,\text{C}$\\
Ruhemasse Elektron&$m_0=9.1096\cdot 10^{-31}\,\text{kg}$\\
Ruhemasse Proton&$m_p=1.6726\cdot 10^{-27}\,\text{kg}$\\
Permeabilität im Vakuum&$\mu_0=4\pi\cdot 10^{-7}\,\text{V}\,\text{s}\,\text{A}^{-1}\,\text{m}^{-1}$\\
Permittivität im Vakuum&$\epsilon_0=c_0^{-2}\mu_0^{-1}=8.85419\cdot 10^{-12}\,\text{A}\,\text{s}\,\text{V}^{-1}\,\text{m}^{-1}$\\
Wellenimpendanz des freien Raumes&$\nu_0=\sqrt{\mu_0/\epsilon_0}=376.73\,\Omega$\\\hline
\end{tabular}
\caption{Universelle Konstanten}
\end{table}
\subsection{Skalare und vektorielle Grössen}
Zahlenwerte mit Einheiten bezeichnet man als skalare Grössen. Zahlenwerte mit Enheiten, die in einer bestimmten Richtung des Raumes wirken bezeichnet man als vektorielle Grössen und haben einen Vektorpfeil über das Symbol und ihr Betrag ist die Länge des Vektors.
\section{Elektrizität und ihre Wirkungen}
\subsection{Elektrische Ladung und elektrischer Strom}
Die elektrische Ladung $Q$ ist eine physikalische Grösse und benötigt einen Träger. Der Raum befindet sich in einem elektrischen Feld. Körper, die eine elektrische Ladung tragen, üben eine Kraftwirkung aufeinander aus.
\newline\newline
Elektrische Ladungen könenn sich dabei anziehen oder abstossen. Man unterscheidet zwischen positiver und negativer Ladung. Gleichnamige Ladungen stossen sich ab, während ungleichnamige Ladungen ziehen sich an. Elektrizität entsteht durch Trennen von Ladungen verschiedenen Vorzeichens.
\newline\newline
Im elektrischen neutralen Zustand heben sich die Wirkungen positiver und negativer Ladungen gegenseitig auf. Elektrische Ladungen lassen sich durch Berührung übertragen. Die Elektrizität besteht aus Elektronen.
\newline\newline
Elektrizitätsträger können sich je nach Material übertragen. Ladungsträger bewegen sich in einem Leitungsstrom bzw. elektrischen Strom. Dieser elektrischen Strom ist das Verhältnis der elektrischen Ladung pro Zeit.
\begin{equation}
\boxed{I=\dfrac{Q}{t}}
\end{equation}
Bewegte Ladungen haben thermische (Leiter erwärmt sich bis zum Schmelztemperatur), magnetische (Kräfte entstehen durch Magnete auf Leiter) und chemische Wirkungen (Durch Stromvorgang werden Stoffe verändert).
\subsection{Aufbau der Materie}
Moleküle können in Atome aufgeteilt werden. Um den positiven Atomkern kreisen die negativ geladenen Elektronen. Diese Elektronen bilden die Elektronenhülle, welche Elektronen durch ihre Energie zu Gruppen (Elektronenschalen) aufgeteilt werden. Ein Elektron kann sich nur auf eine Quantenbahn aufhalten. Ein Atom ist elektrisch neutral. Die Elektronen in der äusseren Schale sind Valenzelektronen und bestimmen das chemische Verhalten des Atoms.
\newline\newline
Das Atomkern besteht aus Protonen und Neutronen bzw. aus Nukleonen. Neutronen sind für die Kernspaltung von grösser Bedeutung. Die Beeinflussung der Elektronenhülle erfolgt durch Ionisierungsenergie und bildet Ionen durch Elektronenabgabe oder -zugabe. Ein Ion ist ein elektrisch geladenes Atom.
\subsection{Leiter und Nichtteiler}
Materialien werden in Leiter und Nichtteiler unterteilt. Zu den Leiter zählen die Metalle, aber verändern nicht das Material durch Stromdurchgang. Säuren, Basen und Salzlösungen werden beim Stromdurchgang verändert.
\newline\newline
Zu den Nichtleiter zählen Gummi, Seide, Kunststoffe, Porzellan, Glas, Glimmer, usw. In diesen Stoffen stehen nahezu keine Elektronen zur Verfügung.




\section{Elemente der Programmiersprache Java}
\subsection{Bytecode}
Der Java-Compiler erzeugt aus den Quellcode-Dateien den so genannten \textbf{Bytecode}. Dieser Code ist binär und Ausgangspunkt für die virtuelle Machine zur Ausführung. Der Bytecode ist wie ein Prozessor, der Anweiungen wie arithmetische Operationen, Sprünge und Weiteres kennt.
\subsection{Java Virtual Machine}
Die Java Virtual Machine (JVM) kümmert sich um den Bytecode, den Quellcode auszuführen. Die Laufzeitumgebung lädt den Bytecode, prüft ihn und führt ihn in einer kontrollierten Umgebung aus. Java ist Plattform- und Betriebssystemunabhängig. Zu der JVM und der Programmiersprache kommen Standardbibliotheken für Datenstrukturen, Zeichenkettenverarbeitung, Datumverarbeitung, grafische Oberflächen, Ein- und Ausgabe, Netzwerkoperationen und mehr.
\subsection{Objektorientierung}
Eine Laufzeitumgebung eliminiert viele Fehler. Objektorientierte Programmierung versucht, die Komplexität des Softwareproblems besser zu modellieren. Menschen denken objektorientiert, darum Java bildet diese ab. Objekte bestehen aus \textbf{Eigenschaften}, also Dinge, die ein Objekt ``hat'' und ``kann''. Objekte entstehen aus \textbf{Klassen}, das sind Beschreibungen für den Aufbau von Objekten.
\\\\
Primitive Datentypen für numerische Zahlen oder Unicode-Zeichen werden nicht als Objekte betrachtet. Das \textbf{Java-Security-Modell} sicherstellt den Programmablauf. Der \textbf{Verifier} liest Code und überprüft die Korrektheit und Typsicherheit. Treten Sicherheitsprobleme auf, werden sie durch Exceptions zur Laufzeit gemeldet. Das Security-Manager überwächt Zugriffe auf das Dateisystem, die Netzwerk-Ports, externe Prozesse und weitere Systemressourcen.
\\\\
In Java gibt es keine Zeiger auf Speicherbereiche, dagegen führt Java \textbf{Referenzen} ein. Eine Referenz repräsentiert ein Objekt, und eine Variable speichert diese Referenz, sie wird Referenzvariable genannt. JVM verbindet die Referenz mit einem Speicherbereich und einem Referenztyp; der Zugriff, Dereferenzierung genannt, ist indirekt. Referenz und Speicherblock sind getrennt.
\\\\
In Java gibt es keine benutzerdefinierten überladenen Operatoren. Da das Operatorzeichen auf unterschiedlichen Datentypen gültig ist, nennt sich so ein Operator \textbf{Überladen}. Bei Zeichenketten werden Pluszeichen als \textbf{Konkatenation} angewendet. Java braucht keine \textbf{Präprozessoren}.
\subsection{Java Platform}
Mit dem Java Development Kit (JDK) lassen sich Java SE-Applikationen entwickeln. Dem JDK sind Hilfsprogramme beigelegt, die für die Java-Entwicklung nötig sind. Dazu zählen der essenzielle Compiler, aber auch andere Hilfsprogramme, etwa zur Signierung von Java-Archiven oder zum Start einer Management-Konsole.
\\\\
Das Java SE Runtime (JRE) enthält genau das, was zur Ausführung von Java-Programmen nötig ist. Die Distribution umfasst nur die JVM und Java-Bibliotheken, aber weder den Quellcode der Java-Bibliotheken noch Tools wie Management-Tools.
\subsection{Das erste Programm compilieren und testen}
\lstinputlisting[language=Java]{../../PROJEKTE/000001HelloWorld/src/Squares.java}
Ein Compiler übersetzt bzw. transformiert das geschriebene Programm in eine andere Repräsentation nämlich den Bytecode und erzeugt aus dem Program mit Endung \texttt{.java} die Datei \texttt{.class}, welche Bytecode enthält.
\\\\
Wenn der Compiler aufgrund eines syntaktischen Fehlers eine Übersetzung in Java-Bytecode nicht durchführen kann, spricht man von einem Compilerfehler.
\\\\
Eine Laufzeitumgebung liest die Bytecode-Datei Anweisung für Anweisung aus und führt sie auf den konkreten Mikroprozessor aus. Der Interpreter bringt das Programm zur Ausführung.
\\\\
Ein Java-Projekt braucht eine ordentliche Ordnerstruktur, und hier gibt es zur Organisation der Dateien unterschiedliche Ansätze. Die einfachste Form ist, Quellen, Klassendateien und Ressourcen in ein Verzeichnis zu setzen. Es gibt zwei Verzeichnisse \texttt{src} für die Quellen und \texttt{bin} für die erzeugten Klassendateien. Ein eigener Ordner \texttt{lib} ist sinnvoll für Java-Bibliotheken.
\\\\
Das Programm sitzt in einer Klasse, die drei Methoden enthält. Die Methode $\texttt{quadrat(int)}$, bekommt als Übergangsparameter eine ganze Zahl und berechnet daraus die Quadratzahl, die sie anschliessend zurückgibt. Eine weitere Methode übernimmt die Ausgabe der Quadratzahlen bis zu einer vorgegebenen Grenze. Die Methode \texttt{main()}, als Anfangspunkt, ruft die Methode \texttt{ausgabe(int)} auf.
\section{Imperative Sprachkonzepte}
\subsection{Elemente der Programmiersprache Java}
Unter dem Begriff \textbf{Semantik} versteht man die Lexikalik, Syntax und Semantik eines Programms. Der Compiler verläuft diese Schritte bevor er den Bytecode erzeugt.
\\\\
Ein \textbf{Token} ist eine lexikalische Einheit, die dem Compiler die Bausteine des Programms liefert. Der Compiler erkennt an der Grammatik einer Sprache, welche Folgen von Zeichen ein Token bilden.
\\\\
\textbf{Whitespaces} sind Leerzeichen, Tabulatoren, Zeilenvorschub und Seitenvorschubzeichen.
\\\\
Neben den Trennern gibt es noch zwölf ASCII-Zeichen geformte Tokens, die als \textbf{Separator} definiert werden: \texttt{( ) \{ \} [ ] ; , . ... @ ::}
\\\\
Für Variablen, Methoden, Klassen und Schnittstellen werden \textbf{Bezeichner}, auch \textbf{Identifizierer} genannt, vergeben. Unter \textbf{Variablen} sind dann Daten verfügbar. \textbf{Methoden} sind die Unterprogramme in objektorientierten Programmiersprachen, und \textbf{Klassen} sind die Bausteine objektorientierter Programme. Ein Bezeichner ist eine Folge von Zeichen, die fast beliebig sein kann. Die Zeichen sind Elemente aus dem Unicode-Zeichensatz. Der Bezeichner muss mit einem Java-Buchstaben beginnen. String ist eine Klasse und kein Datentyp.
\\\\
Ein Java-Buchstabe umfasst unsere lateinische Buchstaben ``A'' bis ``Z'', ``a'' bis ``z'', sondern auch viele Zeichen aus dem Unicode-Alphabet, den Unterstrich, Währungszeichen, griechische oder arabische Buchstaben, Akzente. Java unterscheidet zwischen Gross- und Kleinschreibung. Nicht erlaubt sind Zahlen am Anfang, Leerzeichen, Ausrufezeichen, reservierte Wörter oder reservierte Schlüsselwörter.
\\\\
Ein \textbf{Literal} ist ein konstanter Ausdruck wie die Wahrheitswerte \texttt{true} und \texttt{false}, Integrale Literale für Zahlen, Fliesskommaliterale, Zeichenliterale wie $\backslash$n, String-Literale für Zeichenketten wie ``Hello World'', Referenztypen wie \texttt{null}.
\\\\
Bestimmte Wörter sind reservierte Schlüsselwörter com Compiler besonders behandelt. Schlüsselwörter bestimmen die Sprache eines Compilers. Es können keine eigenen Schlüsselwörter hinzugefügt werden. Schlüsselwörter sind:
\\\\
\texttt{abstract}, \texttt{assert}, \texttt{boolean}, \texttt{break}, \texttt{byte}, \texttt{case}, \texttt{catch}, \texttt{char}, \texttt{class}, \texttt{const}, \texttt{continue}, \texttt{default}, \texttt{do}, \texttt{double}, \texttt{else}, \texttt{enum}, \texttt{extends}, \texttt{final}, \texttt{finally}, \texttt{float}, \texttt{for}, \texttt{goto}, \texttt{if}, \texttt{implements}, \texttt{import}, \texttt{instanceof}, \texttt{int}, \texttt{interface}, \texttt{long}, \texttt{native}, \texttt{new}, \texttt{package}, \texttt{private}, \texttt{protected}, \texttt{public}, \texttt{return}, \texttt{short}, \texttt{static}, \texttt{strictfp}, \texttt{super}, \texttt{switch}, \texttt{synchronized}, \texttt{this}, \texttt{throw}, \texttt{throws}, \texttt{transient}, \texttt{try}, \texttt{void}, \texttt{volatile}, \texttt{while}
\\\\
Der Compiler überliest alle Kommentare und die Trennzeichen bringen den Compiler von Token zu Token. \textbf{Zeilenkommentare} kann man mit Schrägsstrichen \boxed{\textbf{\texttt{//}}} und kommentieren den Rest einer Zeile bis Zeilenumbruchzeichen aus. \textbf{Blockkommentare} (``Wie'') kommentiert in Blöcke mit \boxed{\textbf{\texttt{/* */}}} aus. \textbf{Javadoc-Kommentare} (``Was'') sind besondere Blockkommentare mit \boxed{\textbf{\texttt{/** */}}} und beschreibt die Methode oder die Parameter, aus denen sich später die API generieren lässt. Kein Kommentar kommt in den Bytecode.
\subsection{Anweisungen}
Programme sind Ablauffolgen, die im Kern aus \textbf{Anweisungen} bestehen. Sie werden zu grösseren Bausteinen zusammengesetzt, den Methoden, die wiederum Klassen bilden. Klassen selbst werden in Paketen gesammelt, und eine Sammlung von Paketen wird als Java-Archiv ausgeliefert.
\\\\
Durch Anweisungen werden \textbf{Algorithmen} geschrieben. Anweisungen können Ausdrucksanweisungen für Zuweisungen oder Methodenaufrufe, auch  Fallunterscheidungen, oder Schleifen für Wiederholungen sein.
\\\\
Anweisungen müssen in einen Rahmen gepackt werden. Dieser Rahmen heisst \textbf{Kompilationseinheit} und deklariert eine Klasse mit ihren Methoden und Variablen. Anweisungen ausserhalb von Klassen sind nicht erlaubt. Der Klassenname ist ein Bezeichner und beinhaltet die gleiche Dateiname. Klassennamen beginnen mit Grossbuchstabe und Methoden sind kleingeschrieben. Zwischen den geschweiften Klammern folgen Deklarationen von Methoden und zwischen den Methoden die Anweisungen.
\\\\
Eine besondere Methode ist \boxed{\textbf{\texttt{public static void(String[] args)\{\}}}}. Die Methode ist für die Laufzeitumgebung etwas Besonders, denn beim Aufruf des Java-Interpreters mit einem Klassennamen wird diese Methode als Erstes ausgeführt. Demnach werden die Anweisungen ausgeführt, die innerhalb der geschweiften Klammern stehen. Der Parameter \texttt{args} wird immer verwendet.
\\\\
Haltet man sich nicht an die Syntax für den Startpunkt, so kann der Interpreter die Ausführung nicht beginnen und man hätte einen semantischen Fehler produziert, obwohl die Methode korrekt gebildet ist.
\\\\
Die Methode \boxed{\textbf{\texttt{println(...)}}} gibt Meldungen auf der Konsole aus. Innerhalb der Klammern können Argumente angegeben werden wie Zeichenketten oder \textbf{Strings} oder eine Folge von Buchstaben, Ziffern oder Sonderzeichen in doppelten Anführungszeichen. Die Methode \texttt{println(...)} gehört zum Typ \textbf{\texttt{out}} und diese zu \textbf{\texttt{System}}.
\lstinputlisting[language=Java]{../../PROJEKTE/000001HelloWorld/src/PrimeraClase.java}
Java erlaubt Methoden, die gleich heissen, denen aber unterschiedliche Dinge übergeben werden können; diese Methoden nennt man \textbf{überladen}. Viele \texttt{println()}-Methoden akzeptieren zahlartige Argumente und sind überladen.
\lstinputlisting[language=Java]{../../PROJEKTE/000001HelloWorld/src/OverloadedPrintln.java}
Die Methode \boxed{\textbf{\texttt{printf()}}} ermöglicht variable Argumentenlisten gemäss einer Formatierungsanweisung. Die Formatierungsanweisung \boxed{\textbf{\texttt{$\backslash$n}}} setzt einen Zeilenumbruch, \boxed{\textbf{\texttt{$\backslash$d}}} ist ein Platzhalter für eine ganze Zahl, \boxed{\textbf{\texttt{$\backslash$f}}} ist ein Platzhalter für eine Fliesskommazahl, \boxed{\textbf{\texttt{$\backslash$s}}} ist eine Zeichenkette oder etwas, was in einen String konvertiert werden soll.
\lstinputlisting[language=Java]{../../PROJEKTE/000001HelloWorld/src/VarArgs.java}
Methodenaufrufe lassen sich als Anweisungen einsetzen, wenn sie mit einem Semikolon abegschlossen sind, man spricht von einer \textbf{Ausdrucksanweisung} (expression statement). Jeder Methodenaufruf mit Semikolon bildet eine Ausdrucksanweisung. Dabei ist es egal, ob die Methode selbst eine Rückgabe liefert oder nicht.
\\\\
Die Methode \boxed{\textbf{\texttt{Math.random()}}} liefert eine Fliesskommazahl zwischen 0 (inklusiv) und 1 (exklusiv). In einer objektorientierte Programmiersprache sind alle Methoden an bestimmte Objekte mit einem Zustand gebunden. Alle Operationen und Zustände sind an Objekte bzw. Klassen gebunden. Der Aufruf einer Methode auf einem Objekt richtet die Anfrage genau an dieses bestimmte Objekt.
\\\\
Die Deklaration einer Klasse oder Methode kann einen oder mehrere \textbf{Modifizierer} enthalten, die zum Beispiel die Nutzung einschränken oder parallelen Zugriff synchronisieren. Der Modifizierer \boxed{\textbf{\texttt{public}}} ist ein Sichtbarkeitsmodifizierer. Er bestimmt, onb die Klasse bzw. die Methode für Programmcode anderer Klassen sichtbar ist oder nicht. Der Modifizierer \boxed{\textbf{\texttt{static}}} zwingt den Programmierer nicht dazu, vor dem Methodenaufruf ein Objekt der Klasse zu bilden. Dieser Modifizierer bestimmt die Eigenschaft, ob sich eine Methode nur über ein konkretes Objekt aufrufen lässt oder eine Eigenschaft der Klasse ist, sodass für den Aufruf kein Objekt der Klasse nötig wird.
\\\\
Ein \textbf{Block} fasst eine Gruppe von Anweisungen,die hintereinander ausgeführt werden. Ein Block \boxed{\textbf{\texttt{\{\}}}} ist eine Anweisung, die in geschweiften Klammern eine Folge von Anweisungen zu einer neuen Anweisung zusammenfasst. Ein Block kann überall dort verwendet werden, wo auch eine einzelne Anweisung stehen kann. Der neue Block hat jedoch eine Besonderheit in Bezug auf Variablen, da er einen lokalen Bereich für die darin befindlichen Anweisungen inklusive der Variablen bildet.
\\\\
Ein Block ohne Anweisung nennt sich ein leerer Block. Er verhält sich wie eine leere Anweisung, also wie ein Semikolon. Es gibt innere und äussere Blöcke. Blöcke fassen Anweisungen zusammen.
\section{Datentypen, Variablen und Zuweisungen}
Java speichert Variablen. Eine Variable ist ein reservierter Speicherbereich und belegt eine feste Anzahl von Bytes. Variablen und Ausdrücke haben einen \textbf{Datentyp} und einen \textbf{Datenwert}. Der Datentyp bestimmt die zulässigen Operationen. Java ist eine streng typisierte Programmiersprache. Datentypen werden unterteilt in \textbf{primitive Datentypen} (Zahlen, Unicode-Zeichen und Wahrheitswerte) und \textbf{Referenztypen} (Zeichenketten, Datenstrukturen, Zwergpinscher) und Bytecode durch den Compiler einfacher erzeugt.
\begin{table}[H]
\centering
\begin{tabular}{lll}
\hline
Typ&Grösse&Belegung (Wertebereich)\\\hline
boolean&1 Bit&\texttt{true} oder \texttt{false}\\
char& 16Bit&$\text{0x0000 \dotso 0xFFFF}$\\\hline
byte*&8 Bit&$-2^7$ bis $2^7-1$\\
short*&16 Bit&$-2^{15}$ bis $2^{15}-1$\\
int*&32 Bit&$-2^{31}$ bis $2^{31}-1$ \\
long*&64 Bit&$-2^{63}$ bis $2^{63}-1$\\\hline
float&32 Bit&$1,4023\cdot 10^{-45} \dotso 3,4028\cdot10^{38}$\\
double&64 Bit&$4,9406\cdot 10^{-324} \dotso 1,7976\cdot 10^{308}$\\\hline
\end{tabular}
\caption{Java-Datentypen, Grössen und Formate. *\textit{Zweierkomplement}}
\end{table}
\noindent Es gibt mehr negative Werte als positive Werte, das liegt an der Kodierung im Zweierkomplement. Bei \textbf{\texttt{float}} und \textbf{\texttt{double}} ist das Vorzeichen nicht angegeben, die Wertebereiche unterscheiden sich nicht, die kleinsten und grössten darstellbaren Zahlen können sowohl positiv als auch negativ sein.
\subsection{Variablendeklarationen}
Mit Variablen lassen sich Daten speichern, die vom Programm gelesen und geschrieben werden können. Variablen müssen deklariert werden. Hinter dem Typnamen folgt der Name der Variablen. Die \textbf{Deklaration} ist eine Anweisung und wird daher mit einem Semikolon abgeschlossen.
\lstinputlisting[language=Java]{../../PROJEKTE/000001HelloWorld/src/FirstVariable.java}
Gleich bei der Deklaration lassen sich Variablen mit einem Anfangswert initialisieren. Hinter einem Gleichheitszeichen steht der Wert, der oft ein Literal ist.
\newline\newline
Eine Konsoleneingabe. Eine Variante ist die Klasse \texttt{java.util.Scanner}. Folgende Tabelle zeigt die Eingabe von drei verschiedenen Datentypen.
\begin{table}[H]
\centering
\begin{tabular}{ll}
\hline
Eingabe & Anweisung\\\hline
\texttt{String}&\texttt{String s = new java.util.Scanner(System.in).nextLine();}\\
\texttt{int}&\texttt{int i = new java.util.Scanner(System.in).nextInt();}\\
\texttt{double}&\texttt{double d = new java.util.Scanner(System.in).nextDouble();}\\\hline
\end{tabular}
\caption{Einlesen einer Zeichenkette, Ganz- und Fliesskommazahl von der Konsole}
\end{table}
\noindent Folgendes Beispiel zeigt eine Anwendung aller Eingabenmöglichkeiten mit der Klasse \texttt{java.util.Scanner}.
\lstinputlisting[language=Java]{../../PROJEKTE/000001HelloWorld/src/SmallConversation.java}
\subsection{Fliesskommazahlen}
Java bietet die Datentypen \textbf{\texttt{float}} und \textbf{\texttt{double}}. Fliesskommazahl können einen Vorkommateil und einen Nachkommateil besitzen, die durch einen Dezimalzahl getrennt sind. Standardmässig sind die Fliesskommaliterale vom Typ \textbf{\texttt{double}}. Ein nachgestelltes \textbf{\texttt{f}} oder \textbf{\texttt{F}} zeigt dem Computer an, dass es sich um einen \texttt{float} handelt.
\newline\newline
So ist beispielsweise \texttt{1+2+4.0} eine Addition aus \texttt{1+2} dann in \texttt{double} transformiert und anschliessend \texttt{3.0+4.0}. Die Standardbibliothek \textbf{\texttt{java.math}} bietet die Klasse \textbf{\texttt{BigDecimal}} an. Diese Klasse eignet sich gut für gute Genauigkeit wie Währungen.
\subsection{Ganzzahlige Datentypen}
Java stellt fünf ganzzahlige Datentypen zur Verfügung: \textbf{\texttt{byte}}, \textbf{\texttt{short}}, \textbf{\texttt{char}}, \textbf{\texttt{int}} und \textbf{\texttt{long}}. Ganzzahlige Datentypen sind immer vorzeichenbehaftet (mit Ausnahme von \texttt{char}). Einen Modifizierer \texttt{unsigned} gibt es nicht. Java reserviert nicht so viele Bits wie benötigt und wählt nicht automatisch den passenden Wertebereich. Dabei ist \textbf{\texttt{System.out.println( 122323423434525345345435);}} fehlerbehaftet. Der Datentyp \textbf{\texttt{int}} ist in Java standardmässig.
\newline\newline
An das Ende von Ganzzahlliteralen vom Typ \textbf{\texttt{long}} wird ein \textbf{\texttt{l}} oder ein \textbf{\texttt{L}} gesetzt. Dabei wird \textbf{\texttt{System.out.println( 122323423434525345345435L);}} gültig.
\newline\newline
Ein \textbf{\texttt{byte}} ist ein Datentyp mit einem kleineren Wertebereich. Eine Initialisierung \textbf{\texttt{byte b = 200;}} ist fehlerbehaftet. Eine explizite Typumwandlung lässt Zahlen in einem \textbf{\texttt{byte}} speichern und zwar \textbf{\texttt{byte b = (byte) 200;$\Longrightarrow$ -56}}.
\newline\newline
Der Datentyp \textbf{\texttt{short}} stehen 16 Bits, (1 Bit für das Vorzeichen und 15 Bit für die Zahlen) Speicher zur Verfügung. Ein \texttt{short} ohne Vorzeichen kann folgendermassen initialisiert werden: \textbf{\texttt{short s = (short) 3300;$\Longrightarrow$ -32536}}
\subsection{Wahrheitswerte}
Der Datentyp \textbf{\texttt{boolean}} beschreibt einen Wahrheitswert, der entweder \textbf{\texttt{true}} oder \textbf{\texttt{false}} ist. Diese sind reservierte Wörter und bilden neben konstanten Strings und primitiven Datentypen Literale. Numerische Werte werden nicht als Wahrheitswerte interpretiert. Der boolesche Typ wird für Bedingungen, Verzweigungen oder Schleifen benötigt. Ein Wahrheitswerte ergibt sich aus Vergleichen.
\subsection{Unterstriche in Zahlen}
Eine Variante um grosse Zahlen mit viele Nullen zu schreiben ist es, Unterstriche in Zahlen einzusetzen, denn ein Unterstrich gliedert die Zahl in Blöcke. Unterstriche machen Tausender-Blöcke gut sichtbar. Hilfreich ist die Schreibweise auch bei Literalen in Binär- und Hexadezimaldarstellung. Mit \textbf{\texttt{0b}} beginnt ein Literal in Binärschreibweise und mit \textbf{\texttt{0x}} beginnt ein Literal in Hexadezimalschreibweise. Zwei aufeinanderfolgende Unterstriche sind aber nicht erlaubt und er darf nicht am Anfang stehen.
\subsection{Alphanumerische Zeichen}
Der alphanumerische Datentyp \textbf{\texttt{char}} ist 2 Byte gross und nimmt ein Unicode-Zeichen auf. Ein \texttt{char} ist nicht vorzeichenbehaftet. Die Literale werden in Hochkommata (nicht Anführungszeichen) gesetzt. Ein \texttt{char} kann automatisch in ein \texttt{int} konvertiert werden.
\subsection{Initialisierung von lokalen Variablen}
Die Laufzeitumgebung bzw. der Compiler initialisiert lokale Variablen nicht automatisch mit einem Nullwert bzw. einen \texttt{false}. Sind Variablen nicht initialisiert, so gibt es Fehlermeldungen.
\section{Ausdrücke, Operanden und Operatoren}
Mathematische Ausdrücke bestehen aus \textbf{Operanden} und \textbf{Operatoren}. Ein Operand ist eine Varaible, ein Literal oder Rückgabe eines Methodenaufrufs. Die Operatoren verknüpfen die Operanden. Je nach Anzahl der Operanden unterscheidet man folgende Arten von Operatoren:
\begin{itemize}
\item Ist ein Operator auf genau einem Operanden definiert, so nennt er sich unärer Operator. Bsp: Negatives Vorzeichen.
\item Die üblichen Operatoren für mathematische Ausdrücke sind binäre Operatoren.
\item Das Fragezeichen-Operator für bedingte Ausdrücke ist ein tertiäres Operator.
\end{itemize}
\subsection{Zuweisungsoperator}
Das Gleichheitszeichen \textbf{\texttt{=}} dient in Java der Zuweisung. Der Zuweisungsoperator ist ein binärer Operator, bei dem auf der linken Seite due zu belegende Variable steht und auf der rechten Seite ein Ausdruck. Erst nach dem Auswerten des Ausdrucks kopiert der Zuweisungsoperator das Ergebnis in die Variable. Division durch Null, so gibt es keinen Schreibzugriff auf die Variable. Zuweisungen können geschachtelt werden.
\subsection{Arithmetische Operatoren}
Ein arithmetischer Operator verknüpft die Operanden mit den Operatoren Addition (\textbf{\texttt{+}}), Subtraktion (\textbf{\texttt{-}}), Multiplikation (\textbf{\texttt{*}}), Division (\textbf{\texttt{/}}) und den Rest-Operator (\textbf{\texttt{\%}}). Die arithmetische Operatoren sind binär.
\newline\newline
Bei Ausdrücken mit unterschiedlichen numerischen Datentypen, bringt der Compiler vor der Anwendung der Operation alle Operanden auf den umfassenderen Typ. Vor der Auswertung von \texttt{1+2.0} wird die Ganzzahl \texttt{1} in ein \texttt{double} konvertiert und dann die Addition vorgenommen - das Ergebnis ist auch vom Typ \texttt{double}. Das nennt sich \textbf{numerische Umwandlung}. Die Operation wird ausgeführt, und der Ergebnistyp entspricht dem umfassenden Typ.
\newline\newline
Der binäre Operator bildet den Quotienten aus Dividend und Divisor. Die Division ist für Ganzzahlen und für Fliesskomazahlen definiert. Bei der Ganzzahldivision wird zu null hin gerundet und das Ergebnis ist keine Fliesskomazahl. Den Datentyp des Ergebnisses bestimmen die Operanden und nicht der Operator. Soll das Ergebnis vom Typ \texttt{double} sein, muss mindestens ein Operand ebenfalls \text{double} sein.
\subsection{Der Restwert-Operator \%}
Der Restwert-Operator liefert der Rest einer Division zweier Ganzzahlen und Fliesskomazahlen. Die DIvision und der Restwert richten sich nach einer einfachen Formel: \texttt{(int)(a/b)*b+(a\%b)=a}. Das Ergebnis ist nur dann negativ, wenn der Dividend negativ ist; das Ergebnis ist nur dann positiv, wenn der Dividend positiv ist. Um mit \texttt{value\%2 == 1} zu testem, ob \texttt{value} eine ungerade Zahl ist, muss \texttt{value} positiv sein.
\subsection{Präfix- oder Postfix-Inkrement und -Dekrement}
Die Operatoren \textbf{\texttt{++}} und \textbf{\texttt{--}} kürzen die Programmzeilen zum Inkrement und Dekrement ab. Eine lokale Variable muss allerdings vorher initialisiert sein, da ein Lesezugriff vor einem Schreibzugriff stattfindet. Beide Operatoren erfüllen somit zwei Aufgaben: Neben der Wertrückgabe gibt es eine Veränderung der Variablen.
\begin{table}[H]
\centering
\begin{tabular}{lll}
\hline
&Präfix&Postfix\\\hline
Inkrement & Prä-Inkrement, \textbf{\texttt{++i}}&Post-Inkrement, \textbf{\texttt{i++}}\\
Dekrement & Prä-Dekrement, \textbf{\texttt{--i}}&Post-Dekrement, \textbf{\texttt{i--}}\\\hline
\end{tabular}
\caption{Präfix- oder Postfix-Inkrement und -Dekrement}
\end{table}
\noindent Die beiden Operatoren liefern einen Ausdruck und geben daher einen Wert zurück. Es macht jedoch einen feinen Unterschied, wo dieser Operator platziert wird: Er kann vor der Variablen stehen, wie \texttt{++i} oder dahinter wie \texttt{i++}. Der \textbf{Präfix-Operator} verändert die Variable vor der Auswertung des Ausdrucks, und der \textbf{Postfix-Operator} verändert die Variable nach der Auswertung des Ausdrucks.
\lstinputlisting[language=Java]{../../PROJEKTE/000001HelloWorld/src/Prefixen.java}
\subsection{Auswertung bei Array-Zugriffen}
Falls die linke Seite beim Verbundoperator ein Array-Zugriff ist, wird die Indexberechnung nur einmal vorgenommen. Dies ist wichtig beim Einsatz vom Präfix-/Postfix-Operator oder von Methodenaufrufen, die Nebenwirkungen besitzen, also etwa Zustände wie einen Zähler verändern.
\subsection{Zuweisung mit Operation (Verbundoperator)}
Zuweisungen lassen sich mit numerischen Operatoren kombinieren. Für einen binären Operator (symbolisch \textbf{\texttt{\#}} genannt) im Ausdruck \textbf{\texttt{a = a\#(b)}} kürzt der Verbundoperator den Ausdruck zu \textbf{\texttt{a\#b}} ab. Der Verbundoperator erlaubt eine kompakte Schreibweise.
\subsection{Relationale und Gleichheitsoperatoren}
Relationale Operatoren sind Vergleichsoperatoren, die Ausdrücke miteinander vergleichen und einen Wahrheitswert vom Typ \texttt{boolean} ergeben. Die numerische Vergleiche sind: grösser (\textbf{\texttt{>}}), kleiner (\textbf{\texttt{<}}), grösser/gleich (\textbf{\texttt{$\geq$}}), kleiner/gleich (\textbf{\texttt{$\leq$}}), Gleichheit (\textbf{\texttt{==}}), Ungleichheit (\textbf{\texttt{!=}}).
\subsection{Logische Operatoren}
Die Programmierung ist an Bedingungen verknüpft. Diese Bedingungen sind komplex zusammengesetzt, wobei drei Operatoren am häufigsten vorkommen. \textbf{Nicht \texttt{!}:} (Negation) dreht die Aussage um; aus \texttt{wahr} wird \texttt{falsch} und aus \texttt{falsch} wird \texttt{wahr}. \textbf{Und \texttt{\&\&}:} (Konjunktion) beide Aussagen müssen \texttt{wahr} sein, damit die Gesamtaussage \texttt{wahr} wird. \textbf{Oder \texttt{||}:} (Disjunktion) eine der beiden Aussagen muss \texttt{wahr} sein, damit die Gesamtaussage \texttt{wahr} wird. \textbf{Xor \texttt{\^}:} (Exklusives Oder) Operation, die nur dann \texttt{wahr} liefert, wenn genau einer der beiden Operanden \texttt{wahr} ist. Sind beide Operanden gleich, so ist das Ergebnis \texttt{false}.
\begin{table}[H]
\centering
\begin{tabular}{llllll}
\hline
boolean a& boolean b&\texttt{!a}&\texttt{a\&\&b}&\texttt{a||b}&\texttt{a\^\,b}\\\hline
\texttt{true}&\texttt{true}&\texttt{false}&\texttt{true}&\texttt{true}&\texttt{false}\\
\texttt{true}&\texttt{false}&\texttt{false}&\texttt{false}&\texttt{true}&\texttt{true}\\
\texttt{false}&\texttt{true}&\texttt{true}&\texttt{false}&\texttt{true}&\texttt{true}\\
\texttt{false}&\texttt{false}&\texttt{true}&\texttt{false}&\texttt{false}&\texttt{true}\\\hline
\end{tabular}
\caption{Verknüpfungen der logischen Operatoren.}
\end{table}
\subsection{Rang der Operatoren}
Neben Plus und Mail gibt es eine Vielzahl von Operatoren., die alle ihre eigenenn Vorrangregeln besitzen. Der Multiplikationsoperator besitzt eine höhere Priorität als der Plus-Operator. Der \textbf{arithmetische Typ} steht für Ganz- und Fliesskommazahlen, der \textbf{integrale Typ} für \texttt{char} und Ganzzahlen und der Eintrag primitiv für jegliche primitiven Datentypen.
\begin{table}[H]
\centering
\begin{tabular}{llll}
\hline
Operator&Rang&Typ&Beschreibung\\\hline
\texttt{++}, \texttt{--}&1&arithmetisch&Inkrement und Dekrement\\
\texttt{+}, \texttt{-}&1&arithmetisch&unäres Plus und Minus\\
\texttt{\~}&1&integral&bitweises Komplement\\
\texttt{!}&1&boolean&logisches Komplement\\
\texttt{(Typ)}&1&jeder&Cast\\\hline
\texttt{*}, \texttt{/}, \texttt{\%}&2&arithmetisch&Multiplikation, Division, Rest\\\hline
\texttt{+}, \texttt{-}&3&arithmetisch&Addition und Subtraktion\\
\texttt{+}&3&String&String-Konkatenation\\
\texttt{<<}&4&integral&Verschiebung links\\
\texttt{>>}&4&integral&Rechtsverschiebung mit Vorzeichenerweiterung\\
\texttt{>>>}&4&integral&Rechtsverschiebung ohne Vorzeichenerweiterung\\\hline
\texttt{<}, \texttt{<=}, \texttt{>}, \texttt{>=}&5&arithmetisch&Numerische Vergleiche\\
\texttt{instanceof}&5&Objekt&Typvergleich\\
\texttt{==}, \texttt{!=}&6&primitiv&Gleich-/Ungleichheit von Werten\\
\texttt{==}, \texttt{!=}&6&Objekt&Gleich-/Ungleichheit von Referenzen\\
\texttt{\&}&7&integral&bitweises Und\\
\texttt{\&}&7&boolean&logisches Und\\\hline
\texttt{\^}&8&integral&bitweises XOR\\
\texttt{\^}&8&boolean&logisches XOR\\
\texttt{|}&9&integral&bitweises Oder\\
\texttt{|}&9&boolean&logisches Oder\\\hline
\texttt{\&\&}&10&boolean&logisches konditionales Und, Kurzschluss\\\hline
\texttt{||}&11&boolean&logisches konditionales Oder, Kurzschluss\\
\texttt{?:}&12&jeder&Bedingungsoperator\\
\texttt{=}&13&jeder&Zuweisung\\
\texttt{*=}, \texttt{/=}, \texttt{\%=}&13&arithmetisch&Zuweisung mit Operation\\
\texttt{+=}, \texttt{=}, \texttt{<<=}&13&arithmetisch&Zuweisung mit Operation\\
\texttt{>>=}, \texttt{>>>=}, \texttt{\&=}&13&arithmetisch&Zuweisung mit Operation\\
\texttt{\^=}, \texttt{|=}&13&arithmetisch&Zuweisung mit Operation\\
\texttt{+=}&14&String&Zuweisung mit String-Konkatenation\\\hline
\end{tabular}
\caption{Operatoren mit Rangordnung}
\end{table}
\subsection{Die Typumwandlung (Casting)}
Datentypen können konvertiert werden, dies nennt sich \textbf{Typumwandlung}. Java unterscheidet zwischen zwei Arten der Typumwandlung. EIne Typumwandlung hat eine sehr hohe Priorität. DAher muss der Ausdruck gegebenfalls geklammert werden.
\begin{itemize}
\item \textbf{Implizite Typumwandlung:} Daten eines kleineren Datentyps werden automatisch dem grösseren angepasst. Der Compiler nimmt die Anpassung selbständig vor.
\item \textbf{Explizite Typumwandlung:} Ein grösserer Typ kann einem kleineren Typ mit möglichem Verlust von Informationen zugewiesen werden.
\end{itemize}
Werte der Datentypen \texttt{byte} und \texttt{short} werden bei Rechenoperationen automatisch in den Datentyp \texttt{int} umgewandelt. Ist ein Operand vom Datentyp \texttt{long}, dann werden alle Operanden auf \texttt{long} erweitert. Wird aber \texttt{short} oder \texttt{byte} als Ergebnis verlangt, dann ist dieses durch einen expliziten Typecast anzugeben, und nur die niederwertigen Bits des Ergebniswerts werden übergeben.
\begin{table}[H]
\centering
\begin{tabular}{ll}
\hline
Vom Typ&in den Typ\\\hline
\texttt{byte} (8 Bit)&\texttt{short}, \texttt{int}, \texttt{long}, \texttt{float}, \texttt{double}\\
\texttt{short} (16 Bit)&\texttt{int}, \texttt{long}, \texttt{float}, \texttt{double}\\
\texttt{char} (16 Bit)&\texttt{int},\texttt{long}, \texttt{float}, \texttt{double}\\
\texttt{int} (32 Bit)&\texttt{long}, \texttt{float}, \texttt{double}\\
\texttt{long} (64 Bit)&\texttt{float}, \texttt{double}\\
\texttt{float} (32 Bit)&\texttt{double}\\
\texttt{double} (64 Bit)&\texttt{double}\\\hline
\end{tabular}
\caption{Implizite Typumwandlungen}
\end{table}
\noindent Die Anpassung ist eine Erweiterung des Wertebereichs (widening conversion). Der Typ \texttt{boolean} taucht nicht auf, er lässt sich in keinen anderen primitiven Typ konvertieren. Dass ein \texttt{long} auf ein \texttt{double} gebracht werden kann bzw. ein \texttt{int} auf ein \texttt{float} ist als Fehler in der Java zu sehen, denn es gehen Informationen verloren. Ein \texttt{double} kann die 64 Bit für Ganzzahlen nicht effizient nutzen wie ein \texttt{long}.
\newline\newline
Die explizite Anpassung engt einen Typ ein (narrowing conversion). Der gewünschte Typ für eine Typumwandlung wird vor den umzuwandelnden Datentyp in Klammern gesetzt. Bei jeder expliziten Typumwandlung geht Information verloren.
\newline\newline
Bei der expliziten Typumwandlung von \texttt{double} und \texttt{float} in einen Ganzzahltyp kann es selbstverständlich zum Verlust von Genauigkeit kommen sowie zur Einschränkung des Wertebereichs. Bei der konvertierung von Fliesskommazahlen verwendet Java eine Rundung gegen null, schneidet also schlicht den Nachkommaanteil ab.
\lstinputlisting[language=Java]{../../PROJEKTE/000001HelloWorld/src/Typumwandlung.java}
Die String-Konkatenation ist strikt von links nach rechts und natürlich nicht kommutativ wie die numerische Addition. Besteht der Auddruck aus mehreren Teilen, so muss die Auswertungsreihenfolge beachtet werden, andernfalls kommt es zu seltsamen Zusammensetzungen.
\lstinputlisting[language=Java]{../../PROJEKTE/000001HelloWorld/src/PlusString.java}
Nur eine Zeichenkette in doppelten Anführungszeichen ist ein String, und der Plus-Operator entfaltet seine besondere Wirkung. Ein einzelnes Zeichen in einfachen Hochkommata konvertiert Java nach den Regeln der Typumwandlung bei Berechnungen in ein \texttt{int} und Additionen sind Ganzzahl-Additionen.
\lstinputlisting[language=Java]{../../PROJEKTE/000001HelloWorld/src/PlusZeichen.java}
\section{Bedingte Anweisungen}
\subsection{Verzweigung mit der \texttt{if}-Anweisung}
Die \textbf{\texttt{if}}-Anweisung besteht aus dem Schlüsselwort \texttt{if}, dem zwingend ein Ausdruck mit dem Typ \texttt{boolean} in Klammern folgt. Es folgt eine Anweisung, die oft eine Blockanweisung ist.
\lstinputlisting[language=Java]{../../PROJEKTE/000001HelloWorld/src/WhatsYourNumber.java}
Ist das Ergebnis in der Bedingung \texttt{true}, so werden die Anweisungen in der Fallunterscheidung ausgeführt, sonst werden die \text{else}-Anweisungen ausgeführt. Eine Fallunterscheidung hat kein Semikolon. \texttt{if} und \texttt{if-else}-Anweisungen werden geschachtelt (kaskadiert).
\lstinputlisting[language=Java]{../../PROJEKTE/000001HelloWorld/src/IsLeapYear.java}
Die eingerückten Verzweigungen nennen sich auch angehäufte \texttt{if}-Anweisungen oder \texttt{if}-Kaskade.
\subsection{Der Bedingungsoperator}
Der Bedingungsoperator, auch Konditionaloperator, erlaubt es, den Wert eines Ausdrucks von einer Bedingung abhängig zu machen, ohne dass dazu eine \texttt{if}-Anweisung verwendet werden muss. Die Operanden sind durch \textbf{\texttt{?}} und \textbf{\texttt{:}} voneinander getrennt.
\lstinputlisting[language=Java]{../../PROJEKTE/000001HelloWorld/src/BedingungsOperator.java}
Drei Ausdrücke kommen in Bedingunsoperator vor. Der erste Ausdruck muss vom Typ \texttt{boolean} sein. Der Bedingungsoperator kann eingesetzt werden, wenn der zweite und dritte Operand ein numerischer Typ, boolescher Typ oder Referenztyp sind.
\subsection{Die \texttt{switch}-Anweisung}
Eine Kurzform für speziell gebaute, angehäufte \texttt{if}-Anweisungen bietet \textbf{\texttt{switch}}. Es gibt eine Reihe von unterschiedlichen Sprungzeilen, die mit \textbf{\texttt{case}} markiert sind. Die \texttt{switch}-Anweisung erlaubt die Auswahl vin Ganzzahlen, Wrapper-Typen, Aufzählungen und Strings.
\lstinputlisting[language=Java]{../../PROJEKTE/000001HelloWorld/src/Calculator.java}













\section{Zweidimensionale Graphik}
\subsection{Elementare zweidimensionale Graphik}
Der Befehl \boxed{\textbf{\texttt{plot}}} öffnet ein Graphikfenster namens "figure" mit einer Nummer, in das eine Graphik engebettet werden kann. Falls für die Abszisse und für die Ordinate keine Schranken gesetzt werden, passt sich die Skalierung des Koordinatensystems den Daten automatisch an. Für jedes weitere Bild mus mit dem Befehl \boxed{\textbf{\texttt{figure}}} ein neues Graphikfenster geöffnet werden, es erhält eine neue Nummer. Andernfalls wird das alte Bild im Graphikfenster durch das neue Bild überschrieben. 
\newline\newline
Bekanntlich basiert die Grundstruktur von MATLAB auf einer \texttt{n$\times$m}-Matrix aus reellen oder komplexen Elementen. Auch Daten werden ja in Matrizen abgelegt. Für ein zweidimensaionales Bild benötigt MATLAB also mindestens zwei Kolonnenvektoren gleicher Länge.
\newline\newline
Der Befehl {\color{red}\texttt{plot(x,y)}} zeichnet den Datensatz \texttt{y} in Funktion von Datensatz \texttt{x} auf. Wie üblich wird jedem Wert von \texttt{x} ein Wert von \texttt{y} zugeordnet. \texttt{x} sind die Werte der Abszisse und \texttt{y} diejenigen auf der Ordinate. Die daraus resultierende Punkte werden mit geraden Linien verbunden (lineare Interpolation). Beide Achsen haben eine lineare Skala.
\newline\newline
Mit {\color{red}\texttt{plot(x,y,s)}} werden im String \texttt{s} der Linientyp und die Farbe der Kurve definiert. Der Befehl {\color{red}\texttt{plot(x,y,'c+:')}} plottet eine rot punktierte Linie, die jedem Datenpunkt ein rotes Plus-Zeichen hat. Der Befehl {\color{red}\texttt{plot(y)}} enthält nur einen Kolonnenvektor \texttt{y} Element von \texttt{$\mathbb{R}^{n\times 1}$}. In diesem Fall generiert MATLAB für die \texttt{x}-Achse automatisch Werte, nämlich 1 bis \texttt{n}, die Indizes der \texttt{n} Kolonnenwerte. 
\newline\newline
Handelt es sich jedoch bei \texttt{y} um einen Vektor mit komplexen Zahlen, so werden beim Befehl {\color{red}\texttt{plot(y)}} die Realteile auf der \texttt{x}-Achse und die Imaginärteile auf der \texttt{y}-Achse aufgetragen. 
\newline\newline
Der Befehl {\color{red}\texttt{plot(A)}} zeichnet für jede eine Kurve aus \texttt{n} Punkten. Die Matrix \texttt{A} Element von \texttt{$\mathbb{R}^{n\times m}$} je eine Kurve aus \texttt{n} Punkten. Die \texttt{x}-Achse zeigt wieder die Indizes 1 bis \texttt{n}. Die LInien werden zur Unterscheidung verschiedene Stile bzw. verschiedene Farben haben.
\newline\newline
Beim Befehl {\color{red}\texttt{plot(x,A)}} wird jede der \texttt{m} Kolonnen der Matrix \texttt{A} Element von \texttt{$\mathbb{R}^{n\times m}$} gegen die gemeinsame unabhängige Variable \texttt{x} aufgezeichnet. Der Vektor \texttt{x} muss die Dimension \texttt{n} haben: \texttt{x} ist ELement von \texttt{$\mathbb{R}^{n\times 1}$}
\newline\newline
Beim Befehl {\color{red}\texttt{plot(A,B)}} nilden je eine Kolonne der Matrix \texttt{A} Element von \texttt{$\mathbb{R}^{n\times m}$} und der Matrix \texttt{B} Element von \texttt{$\mathbb{R}^{n\times m}$} ein \texttt{x}-\texttt{y}-Vektorpaar. 
\newline\newline
Der Befehl \boxed{\textbf{\texttt{subplot(n,m,p)}}} unterteilt ein Graphikfenster in \texttt{n} Zeilen von je \texttt{m} Bildern. Damit können \texttt{n$\times$m} Bilder in ein Graphikfenster eingebettet werden. \texttt{p} ist der Laufindex der \texttt{n$\times$m} Bilder, wobei die Numerierung zeilenweise von links nach rechts erfolgt. Für jedes neue Bild im Graphikfenster wird der Befehl \texttt{subplot} wiederholt, jedesmal mit dem neuen Index \texttt{p}. Der eigentliche Befehl \texttt{plot} mit seinen Parametern muss dann natürlich auc noch kommen. 
\newline\newline
Die Befehle \boxed{\textbf{\texttt{semilogx}}} und {\color{red}\texttt{plot}} sind bis auf die Skalierung der \texttt{x}-Achse identisch. Beim Befehl {\color{red}\texttt{plot}} hat die Abszisse eine lineare, beim Befehl \texttt{semilogx} aber eine logarithmische Skala (Basis 10). Der Befehl {\color{red}\texttt{semilogx(x,y)}} entspricht dem Befehl {\color{red}\texttt{plot(log10(x), y)}}, doch MATLAB gibt bei \texttt{semilogx} für \texttt{x=0} keine Warnung "log for zero". Der Nullpunkt der \texttt{x}-Achse wird unterdrückt, er kann nicht gezeichnet werden. Mit dem Befehl {\color{red}\texttt{semilogy}} wird die Ordinate mit dem Zehnerlogarithmus skaliert.    
\newline\newline
Der Befehl \boxed{\textbf{\texttt{loglog}}} plottet eine zweidimensionale Graphik in einem doppeltlogarithmischen Koordinatensystem (Basis 10). Der Befehl {\color{red}\texttt{loglog(x,y),log10(y)}} entspricht dem Befehl {\color{red}\texttt{plot(log10(x))}}, doch MATLAB gibt bei \texttt{loglog} keine Warnung "log of zero", falls \texttt{x} oder \texttt{y} gleich Null ist. Der Koordinatennullpunkt wird unterdrückt.  
\newline\newline
Der Befehl \boxed{\textbf{\texttt{polar(phi,r)}}} zeichnet die beiden Vektoren \texttt{phi} und \texttt{r} in ein polares Koordinatensystem. Die Werte des Vektors \texttt{phi} sind im Bogenmass angegeben. Die WErte des Vektors \texttt{r} entsprechen dem Radius, d.h. dem Abstand zwischen dem Ursprung und dem betreffenden Punkt der Funktion.
\newline\newline
Der Befehl \boxed{\textbf{\texttt{plot(x1,x2,y1,y2)}}} versieht die Graphik mit zwei Ordinaten. Die linke \texttt{y}-Achse bezieht sich auf \texttt{y1} in Funktion von \texttt{x1} und die rechte \texttt{y}-Achse auf \texttt{y2} in Funktion von \texttt{x2}.
\subsection{Massstab}
Mit dem Befehl \boxed{\textbf{\texttt{axis([xmin xmax ymin ymax])}}} lassen sich die Grenzen der \texttt{x}- und der \texttt{y}-Achse neu definieren. Mit dem Befehl {\color{red}\texttt{axis auto}} setzt für die Achsen wieder dir ursprünglichen Werte ein. Mit dem Befehl {\color{red}\texttt{axis equal}} erhalten alle Achsen die gleiche Skalierung. Mit dem Befehl {\color{red}\texttt{axis ij}} wechselt das Vorzeichen der \texttt{y}-Achse. Die positive \texttt{y}-Achse zeigt nun nach unten. Der Befehl {\color{red}\texttt{axis xy}} macht {\color{red}\texttt{axis ij}} wieder rückgängig. Der Befehl {\color{red}\texttt{axis tight}} passt die Achsenlänge exakt dem Bild an. Der Befehl {\color{red}\texttt{axis off}} schaltet alle axis-Definitionen, die "tick marks" und den Hintergrund aus. Mit dem Befehl {\color{red}\texttt{axis on}} werden sie wieder aktiviert.
\newline\newline
Der Befehl \boxed{\textbf{\texttt{zoom on}}} aktiviert in der aktuellen Graphik die Zoom-Funktion, er kann direkt im Command Window eingegeben werden. Mit "clic and drag" wird dann im Bild ein beliebiger Ausschnitt näher herangebracht. Mit der linken Maustaste wird der Graphikausschnitt vergrössert und mit der rechten Maustaste verkleinert. Der Befehl {\color{red}\texttt{zoom(factor)}} zoomt die Achsen um den für "factor" gewählten Wert. Der Befehl {\color{red}\texttt{zoom out}} führt die Graphik in ihre default-Fenstergrösse zurück. Der Befehl {\color{red}\texttt{zoom xon}} bzw. {\color{red}\texttt{zoom yon}} aktiviert in re aktuellen Graphik die Zoom-Funktion nur für die betreffenden Achsen. Der Befehl {\color{red}\texttt{zoom(figurename, option)}} versieht die Graphik "figurename" mit einer Zoom-Funktion. Als Option kann eine der oben genannten Zoom-Funktionen gewählt werden.
\newline\newline
Der Befehl \boxed{\textbf{\texttt{grid on}}} versieht die aktuelle Graphik mit ienem Liniennetz. Mit dem Befehl {\color{red}\texttt{grid off}} werden die Linien wieder deaktiviert. Der Befehl {\color{red}\texttt{box on}} umrahmt das aktuelle Bild mit einem Rahmen aus dünnen, schwarzen LInien, der Befehl {\color{red}\texttt{box off}} entfernt ihn wieder.
\newline\newline
Der Befehl \boxed{\textbf{\texttt{hold on}}} fixiert das aktuelle Graphikfenster, so dass weitere Funktionen im bereits bestehenden Graphikfenster positioniert werden können. Die ursprünglichen Achseinstellungen bleiben unverändert, selbst dann, wenn die neue Funktion nicht gut in den Rahmen passt. Der Befehl {\color{red}\texttt{hold off}} führt zum Normalbetrieb zurück, d.h. ein neuer plot-Befehl löscht die aktuelle Funktion und fügt die neue Funktion in das Fenster ein, falls nicht mit dem Befehl {\color{red}\texttt{figure}} ein weiteres Graphikfenster geöffnet wurde.
\newline\newline
Der Befehl \boxed{\textbf{\texttt{axes('position', [links unten Breite Höhe])}}} beschreibt im Graphikfenster die POsition der linken unteren Ecke des Bildes und seine Abmessung. Mit Werten zwischen 0 und 1 kann nun die Position und dir Grösse des Bildes festgelegt werden. Das default-Graphikfenster hat eine Breite und eine Höhe 1. Der Befehl {\color{red}\texttt{axex}} generiert in einem Graphikfenster ein Koordinatensystem, in das mit {\color{red}\texttt{plot}} ein beliebiges Bild hineingelegt werden kann. Über mehrere {\color{red}\texttt{axex}}-Definitionen können im Fenster mehrere Bilder erzeugt werden.
\subsection{Beschriften von Bilder} 
Der Befehl \boxed{\textbf{\texttt{legend('st1', 'st2', ...)}}} versieht im Fenster ein Bild mit einer Legende. Er erhält die einzelnen Strings "st1", "st2", usw. Für jedes Bild in einem Graphikfenster kann ein eigener Titel gewählt werden. Der zu schreibende Text wird in Anführungszeichen genommen. Der Befehl {\color{red}\texttt{legend off}} entfernt die Legende aus dem aktuellen Bild. Der Befehl {\color{red}\texttt{legend('st1', 'st2', ..., position)}} positioniert die Legende mit der Angabe einer Position an einen definierten Ort in der Graphik: 0-beste, 1-oben-rechts, 2-oben links, 3-unten-links, 4-unten-rechts, -1-rechts. Die Legende kann verschoben werden, indem die Legende mit der rechten Maustaste angeklickt und an den gewünschten Ort gezogen wird.
\newline\newline
Der Befehl \boxed{\textbf{\texttt{title('text')}}} fügt einen Titel oberhalb der Graphik hinzu. Ein Text kann hoch \texttt{("\^\,")} und tiefgestellte \texttt{"\_\,"}, kursiv \texttt{"it"} geschrieben oder griechische Zeichen enthalten. 
\begin{equation}
\boxed{A_1e^{-\alpha t}\sin \beta t\quad \texttt{'$\backslash$itA\_\{1\}e\^\,\{$\backslash$alpha$\backslash$itt\}sin$\backslash$beta$\backslash$itt'}}
\end{equation}
Der Befehl \boxed{\textbf{\texttt{xlabel('text')}}} schreibt die Zeile "text" unter die Abszisse. Der Befehl \texttt{ylabel} ist für die Beschriftung der Ordinate.
\newline\newline
Der Befehl \boxed{\textbf{\texttt{text(x,y,'text')}}} kann innerhalb des Bildrahmes eine beliebige Textzeile an der Stelle \texttt{(x,y)} angebracht werden. \texttt{x} und \texttt{y} sind in den Koordinaten der Achsen anzugehen. Dagegen verwendet der Befehl {\color{red}\texttt{text(x,y,'text,'sc')}} die Koordinaten des Graphikfensters, nämlich (0,0) in der unteren linken Ecke und (1,1) ub der oberen rechten Ecke. Geht ein Text über mehrere Zeilen, so kann er in eine Text-Variable geschrieben werden. Ein Text kann hoch- und tiefgestellte, kursiv geschriebene oder griechische Zeichen enthalten.
\newline\newline
Mit dem Befehl \boxed{\textbf{\texttt{gtext('text')}}} kann mit der Maus im Bild irgendein Text an beliebigen Ort eingefügt werden. Im Graphifenster erscheint der Mauspfeil als Fadenkreuz. An der gewünschten Stelle kann durch Betätigung einer Maustaste der Text eingefügt werden.
\subsection{Graphiken speichern oder drucken}
Der Befehl \boxed{\textbf{\texttt{print}}} sendet eine Kopie des aktuellen Graphikfensters an den Drucker. Der Befehl {\color{red}\texttt{print filename}} speichert eine Kopie des aktuellen Graphikfensters als PostScript-Datei im aktiven Directory unter dem Dateinamen "filename". Mit dem Befehl {\color{red}\texttt{print path}} kann der genaue Pfad angegeben werden, wo das aktuelle Graphikfenster als POst-Script-Datei abgelegt werden soll. Der Befehl {\color{red}\texttt{print[-ddevice][-options]$<$filenmae$>$}} speichert die aktuelle Graphik im Format des speziell gewählten Druckertreibers und de rzusätzlichen Option im aktiven Directory unter "filename" ab. Der Befehl {\color{red}\texttt{help print}} zeigt im Command-Window alle möglichen \texttt{[-ddevice]} und \texttt{[-options]}.
\newline\newline
Der Befehl \boxed{\textbf{\texttt{orient landscape}}} druckt bei print-Befehlen die Graphikfenster im Querformat. Mit dem Befehl {\color{red}\texttt{orient portrait}} wird das Graphikfenster im Hochformat grdruckt. Der Befehl {\color{red}\texttt{orient tall}} setzt das Blattformat auf Hochformat. Zusätzliche wird das Graphikfenster auf das ganze Papierblatt vergrössert bzw. verkleinert. Der Befehl {\color{red}\texttt{orient}} sagt im Command Window, welche Orientierung momentan aktiv ist.
\section{Dreidimensionale Graphik}
\subsection{Elementare dreidimensionale Graphik}
Der Befehl \boxed{\textbf{\texttt{plot3(x,y,z)}}} plottet im dreidimensionalen Raum die Graphen, die durch die Vektoren \texttt{x}, \texttt{y} und \texttt{z} gegeben sind. Die Vektoren müssen alle dieselbe Länge haben. Mit {\color{red}\texttt{plot3(X,Y,Z)}} zeichnet pro Kolonne einen Graphen. Die Matrizen müssen alle dieselbe Grösse haben. Bei {\color{red}\texttt{plot3(x,y,z,'style')}} können zusätzlich noch der Linientyp, die Plot Symbole und die Farbe des Graphen geändert werden. {\color{red}\texttt{help plot}} listet im Command Window eine Answahl von möglichen Liniendefinitionen auf.
\newline\newline
Bei \boxed{\textbf{\texttt{mesh(Z)}}} entsprechen die Werte der Matrix \texttt{Z} Element von \texttt{$\mathbb{R}^{n\times m}$} den \texttt{z}-Werten des Netzes. Für die \texttt{x}- und \texttt{y}-Werte verwendet \texttt{mesh} die Kolonnen- bzw. die Zeilennummer.
\newline\newline
Mit dem Befehl {\color{red}\texttt{mesh(X,Y,Z,C)}} zeichnet ein Netz un der Vogelperspektive mit \texttt{Z} als Funktion von \texttt{X} und \texttt{Y}. Es handelt sich hierbei um eine FUnktion mit zwei Variablen. \texttt{X}, \texttt{Y} und \texttt{Z} sind Matrizen mit den Werten für die \texttt{x}-, \texttt{y}- und \texttt{z}-Koordinaten. \texttt{X} und \texttt{Y} können aber auch Vektoren der Länge \texttt{m} und \texttt{n} sein. Jeder \texttt{z}-Koordinate aus der Matrix \texttt{Z} werden dann die entsprechenden WErte des \texttt{x}- und \texttt{y}-Vektors zugewiesen. \texttt{C} ist ebenfalls eine Matrix und beinhaltet die Farbskala für die Graphik. Ohne \texttt{C} wird \texttt{C=Z} gesetzt.
\newline\newline
Mit dem Befehl \boxed{\textbf{\texttt{fill(X,Y,Z,C)}}} wird in den Farben der Matrix \texttt{C} ein dreidimensionales Polygon geplottet. Sind \texttt{X}, \texttt{Y} und \texttt{Z} Vektoren, so wird die Fläche unterhalb des Graphen mit Farbe ausgefüllt. Ist \texttt{C} ein Skalar, so wird die Fläche monochrom. Folgende Farben sind möglich: {\color{red}\texttt{'r'}}, {\color{red}\texttt{'g'}}, {\color{red}\texttt{'b'}}, {\color{red}\texttt{'c'}}, {\color{red}\texttt{'m'}}, {\color{red}\texttt{'y'}}, {\color{red}\texttt{'w'}} und {\color{red}\texttt{'k'}}. Mit dem Vektor {\color{red}\texttt{[rot grün blau]}} kann eine neue Farbe gemischt werden. Die Werte von liegen zwischen 0 und 1. Je nach Anteil ergibt sich eine Farbkombination. Ist \texttt{C} ein Vektor, so hat er die gleiche Länge wie \texttt{X}, \texttt{Y} und \texttt{Z}. Wird für \texttt{C} einer der drei Vektoren gewählt, dann ist die Farbabstufung der momentan aktive \texttt{colormap} proportional zur betreffenden Koordinatenachse. 
\newline\newline
Sind \texttt{X}, \texttt{Y} und \texttt{Z} Matrizen, so zeichnet \boxed{\textbf{\texttt{fill3}}} pro Kolonne ein Polygon und füllt es mit der entsprechenden Farbe aus. Die Farbgebung bleibt gleich. Falls \texttt{C} ein Zeilenvektor ist, dann hat das Polygon die Schattierung \texttt{shading flat}. Für eine Matrix wird sie {\color{red}\texttt{shading interp}}. {\color{red}\texttt{shading}} shattiert die Objekt-Oberfläche, {\color{red}\texttt{shading flat}} berechnet für jede Teilfläche einer Oberfläche, die mit den Befehlen \texttt{surf}, \texttt{mesh}, \texttt{polar}, \texttt{fill} oder \texttt{fill3} gebildet wurden, die entsprechende Farbabstufung. {\color{red}\texttt{shading interp}} interpoliert über die Farbabstufung. {\color{red}\texttt{shading faceted}} entspricht dem {\color{red}\texttt{shading flat}}. Die 3D Graphik wird jecoh zusätzlich mit schwarzen Linien versehen. 
\subsection{Projektionsarten einer Graphik}
Mit \boxed{\textbf{\texttt{view(az,el)}}} kann in einem dreidimensionalen Plot der Blickwinkel beliebig eingestellt werden, bzw. die Graphik-Box ist um zwei Achsen drehbar. {\color{red}\texttt{az}} steht für azimuth und definiert die horizontale Rotation im Grad. Für einen positiven Winkel dreht sich die Graphik entgegen dem Uhrzeigersinn um die \texttt{z}-Achse. {\color{red}\texttt{el}} beschreibt die Anheben bzw. Senken der Gtraphik in Grad. Bei einem positiven Winkel befindet sich der Betrachter in der Vogelperspektive, bei negativem Winkel in der Froschperspektive. {\color{red}\texttt{view([x y z])}} setzt den Blickwinkel in kartesischen Koordinaten. {\color{red}\texttt{view(2)}} stellt für die 2D Ansicht den vordefinierten Blickwinkel {\color{red}\texttt{view(0, 90)}} ein. {\color{red}\texttt{view(3)}} stellt für die 3D Ansicht den vordefinierten Blickwinkel {\color{red}\texttt{view(-37.5,30)}} ein. {\color{red}\texttt{T=view}} speichert diew \texttt{view} der aktuiellen Graphik in der Variable \texttt{T} als \texttt{4x4}-Matrix. {\color{red}\texttt{view(T)}} weist einer aktuellen Graphik die in der Variable \texttt{T} gespeicherte view zu.
\newline\newline
\boxed{\textbf{\texttt{T=viewmtx(az,el)}}} weist wie bei {\color{red}\texttt{T=view(az, el)}} der Variable \texttt{T} die 4x4-Transformationsmatrix zu. Die Ansicht der aktuellen Graphik wird dabei nicht verändert. Mit {\color{red}\texttt{T=viewmtx(az, el, phi)}} wird die Graphik durch ein Objektiv betrachtet. Der Linsenwinkel wird in Grad angegeben. \texttt{phi=0} Grad definiert die orthogonale Projektion. 10 Grad entspricht einem Teleobjektiv, 25 Grad einem Normalobjektiv und 60 Grad einem Weitwinkelobjektiv. Bei {\color{red}\texttt{T=viewmtx(az, el, phi, tp)}} wird mit \texttt{tp=[xp, yp, zp]} einen Fluchtpunkt gesetzt.
\newline\newline
\boxed{\textbf{\texttt{rotate3d on}}} aktiviert in der aktuellen Graphik die Maus gesteuerte 3D-Rotation. Die Graphik-Ansicht kann damit beliebig verändert werden. Mit {\color{red}\texttt{rotate3d off}} wird sie wieder deaktiviert. 
\subsection{Dreidimensionale Graphik beschriften}
\boxed{\textbf{\texttt{zlabel('text')}}} versieht die \texttt{z}-Achse mit der Aufschrift "text". Mit dem Befehl {\color{red}\texttt{colorbar('vert')}} erscheint in der aktuellen Graphik eine vertikale Farbskala. In einem 3D-Plot bezieht sie sich auf die Werte der \texttt{z}-Achse. {\color{red}\texttt{colorbar('horiz')}} zeichnet eine vertikale Farbskala. Im 3D-Plot ist die Farbgebung ebenfalls auf die \texttt{z}-Achse abgestimmt. \boxed{\textbf{\texttt{colorbar}}} alleine fügt der Graphik entweder eine vertikale Farbskala hinzu, oder die bestehende Farbskala wird aktualisiert.
\section{Spezielle Graphen}
\boxed{\textbf{\texttt{fill(x,y,c)}}} füllt diejenige Fläche mit der Farbe \texttt{c} aus, welche von der Geraden, die den Endpunkt von \texttt{f(x)} mit dem Anfang verbindet, und der Funktion \texttt{y=f(x)} selbst umgeben wird. Ist \texttt{c} ein Vektor derselben Länge wie \texttt{x} und \texttt{y}, dann verwendet MATLAB entweder die im Vektor \texttt{c} definierten Farben, oder für \texttt{c=x} bzw. \texttt{c=y} die Farbpalette der aktuellen \texttt{colormap}. Sind in \boxed{\textbf{\texttt{fill(X,Y,C)}}} \texttt{X} und \texttt{Y} Matrizen derselben Grösse, so wird pro Kolonne ein Polygon gezeichnet. \texttt{C} kann ein Vektor aber auch eine Matrix sein. Beim Vektor ist die Schattierung der Fläche {\color{red}\texttt{shading flat}}, bei der Matrix {\color{red}\texttt{shading interp}}.
\newline\newline
Mit dem Befehl \boxed{\textbf{\texttt{fplot('f',lim)}}} zeichnet eine beliebige Funktion \texttt{f=f(x)} im Bereich \texttt{lim=[x$_{\texttt{min}}$ x$_{\texttt{max}}$]}. Mit \texttt{lim=[x$_{\texttt{min}}$ x$_{\texttt{max}}$ y$_{\texttt{min}}$ y$_{\texttt{max}}$]} werden zusätzliche Schranken gesetzt. Mit dem Befehl {\color{red}\texttt{fplot('f', lim, tol)}} mit \texttt{tol$<$1} definiert die Toleranz des relativen Fehlers. Die voreingestellte Toleranz ist 2e-3 bzw. 0.2\%. Mit dem Befehl {\color{red}\texttt{fplot('f', lim, N)}} berechnet zwischen \texttt{x$_{\texttt{min}}$} für die Funktion \texttt{f=f(x)} \texttt{N+1} Punkte. Mit dem Befehl {\color{red}\texttt{fplot('f', lim, 'LineSpec')}} definiert mit LineSpec den Linien-Typ der Funktion. Alle möglichen "line specifications" werden mit \texttt{help plot} aufgelistet.    
\newline\newline
Mit dem Befehl \boxed{\textbf{\texttt{hist(x)}}} zeichnet MATLAB ein Histogramm mit den in \texttt{x} gespeicherten Daten. Es ist in 10 gleichmässig verteilte Intervalle unterteilt. Pro Intervall gibt es die Anzahl Elemente an, die es enthält. Wenn \texttt{x} eine Matrix ist, plottet \texttt{hist} pro Kolonne ein Histogramm. Mit dem Befehl {\color{red}\texttt{hist(x,n)}} definiert man zusätzlich die Anzahl \texttt{n} Intervalle. Mit dem Befehl {\color{red}\texttt{hist(x,y)}} berechnet MATLAB die Verteilung von \texttt{x} bezüglich \texttt{y}. \texttt{y} ist ein Vektor, deren Elemente in aufsteigender Ordnung aufgelistet sind. Jedes einzelne Element von \texttt{y} entspricht einem Zentrum.     
\newline\newline
Mit dem Befehl \boxed{\textbf{\texttt{pie(x)}}} stellt die Daten aus dem Vektor \texttt{x} in einem Kuchendiagramm dar. Die Elemente von \texttt{x} werden mit der Summe der \texttt{x}-Werte dividiert. Damit ist die Grösse von jedem einzelnen Kuchenstück in \% gegeben. Mit dem Befehl {\color{red}\texttt{pie(x,explode)}} zieht mit "explode" die gewünschte Stücke aus dem Kuchen, \texttt{explode} ist ein Vektor derselben Länge wie \texttt{x}. Seine Elemente haben entweder den Betrag 0, d.h. das entsprechende Stück von \texttt{x} verbleibt im Kuchen, ode rden Betrag 1, d.h. das dazugehörende Stück von \texttt{x} wird aus dem Kuchen herausgezogen.
\newline\newline
Mit dem Befehl \boxed{\textbf{\texttt{stem(y)}}} zeichnet eine Verteilung der Daten aus dem Vektor \texttt{y}. Jeder Wert aus \texttt{y} im Plot mit einem Kreis und einer Linie versehen. Auf der Abszisse wird jedes Element aus \texttt{x} fortlaufend eingereiht. Auf der Ordinate kann sein Wert abgelesen werden. Der entsprechende Wert ist mit ienem Kreis gekennzeichnet. Bei {\color{red}\texttt{stem(x,y)}} entsprechen die Elemente aus dem \texttt{x}-Vektor den \texttt{x}-Werten und diejenigen aus dem \texttt{y}-Vektor den \texttt{y}-Werten. Mit dem Befehl {\color{red}\texttt{stem(..., 'filled')}} malt den Kreis mit der entsprechenden Farbe aus. Mit dem Befehl {\color{red}\texttt{stem(..., 'linespec')}} kann der Linien-Typ gewählt werden. Alle möglichen "line specifications" werden mit {\color{red}\texttt{help plot}} aufgelistet.    
\newline\newline
Mit dem Befehl \boxed{\textbf{\texttt{contour(Z)}}} zeichnet einen Konturplot mit den WErten aus der Matrix \texttt{Z}. Die Elemente der Matrix \texttt{Z} entsprechen den Werten auf der \texttt{z}-Achse. Die Kolonnenzahlen ergeben die Koordinaten auf der \texttt{x}-Achse und die Zeilenzahlen die WErte auf der \texttt{y}-Achse. Die Höhen für die einzelnen Höhenlinien werden automatisch ausgewählt. Mit {\color{red}\texttt{contour(x,y,z)}} werden die \texttt{x}- und \texttt{y}-Koordinaten explizit mitgeliefert. {\color{red}\texttt{contour(z,N)}} und {\color{red}\texttt{contour(x,y,z,N)}} verwendet für den Konturplot \texttt{N} Höhenlinien. {\color{red}\texttt{contour(Z,v)}} und {\color{red}\texttt{contour(x,y,Z,v)}} zeichnet all jene Höhenlinien, deren Höhen im Vektor \texttt{v} angegeben werden. Mit {\color{red}\texttt{contour(Z,[v v])}} plottet MATLAB eine einzige Höhenlinie bei der Höhe \texttt{v}. Mit {\color{red}\texttt{contour(...,'linespec')}} kann der Linien-Typ gewählt werden. Alle möglichen "line specifications" werden mit {\color{red}\texttt{help plot}} aufgelistet...  
%\chapter{Integralrechnung}
\section{Zweidimensionale Graphik}
\subsection{Elementare zweidimensionale Graphik}
Der Befehl \boxed{\textbf{\texttt{plot}}} öffnet ein Graphikfenster namens "figure" mit einer Nummer, in das eine Graphik engebettet werden kann. Falls für die Abszisse und für die Ordinate keine Schranken gesetzt werden, passt sich die Skalierung des Koordinatensystems den Daten automatisch an. Für jedes weitere Bild mus mit dem Befehl \boxed{\textbf{\texttt{figure}}} ein neues Graphikfenster geöffnet werden, es erhält eine neue Nummer. Andernfalls wird das alte Bild im Graphikfenster durch das neue Bild überschrieben. 
\newline\newline
Bekanntlich basiert die Grundstruktur von MATLAB auf einer \texttt{n$\times$m}-Matrix aus reellen oder komplexen Elementen. Auch Daten werden ja in Matrizen abgelegt. Für ein zweidimensaionales Bild benötigt MATLAB also mindestens zwei Kolonnenvektoren gleicher Länge.
\newline\newline
Der Befehl {\color{red}\texttt{plot(x,y)}} zeichnet den Datensatz \texttt{y} in Funktion von Datensatz \texttt{x} auf. Wie üblich wird jedem Wert von \texttt{x} ein Wert von \texttt{y} zugeordnet. \texttt{x} sind die Werte der Abszisse und \texttt{y} diejenigen auf der Ordinate. Die daraus resultierende Punkte werden mit geraden Linien verbunden (lineare Interpolation). Beide Achsen haben eine lineare Skala.
\newline\newline
Mit {\color{red}\texttt{plot(x,y,s)}} werden im String \texttt{s} der Linientyp und die Farbe der Kurve definiert. Der Befehl {\color{red}\texttt{plot(x,y,'c+:')}} plottet eine rot punktierte Linie, die jedem Datenpunkt ein rotes Plus-Zeichen hat. Der Befehl {\color{red}\texttt{plot(y)}} enthält nur einen Kolonnenvektor \texttt{y} Element von \texttt{$\mathbb{R}^{n\times 1}$}. In diesem Fall generiert MATLAB für die \texttt{x}-Achse automatisch Werte, nämlich 1 bis \texttt{n}, die Indizes der \texttt{n} Kolonnenwerte. 
\newline\newline
Handelt es sich jedoch bei \texttt{y} um einen Vektor mit komplexen Zahlen, so werden beim Befehl {\color{red}\texttt{plot(y)}} die Realteile auf der \texttt{x}-Achse und die Imaginärteile auf der \texttt{y}-Achse aufgetragen. 
\newline\newline
Der Befehl {\color{red}\texttt{plot(A)}} zeichnet für jede eine Kurve aus \texttt{n} Punkten. Die Matrix \texttt{A} Element von \texttt{$\mathbb{R}^{n\times m}$} je eine Kurve aus \texttt{n} Punkten. Die \texttt{x}-Achse zeigt wieder die Indizes 1 bis \texttt{n}. Die LInien werden zur Unterscheidung verschiedene Stile bzw. verschiedene Farben haben.
\newline\newline
Beim Befehl {\color{red}\texttt{plot(x,A)}} wird jede der \texttt{m} Kolonnen der Matrix \texttt{A} Element von \texttt{$\mathbb{R}^{n\times m}$} gegen die gemeinsame unabhängige Variable \texttt{x} aufgezeichnet. Der Vektor \texttt{x} muss die Dimension \texttt{n} haben: \texttt{x} ist ELement von \texttt{$\mathbb{R}^{n\times 1}$}
\newline\newline
Beim Befehl {\color{red}\texttt{plot(A,B)}} nilden je eine Kolonne der Matrix \texttt{A} Element von \texttt{$\mathbb{R}^{n\times m}$} und der Matrix \texttt{B} Element von \texttt{$\mathbb{R}^{n\times m}$} ein \texttt{x}-\texttt{y}-Vektorpaar. 
\newline\newline
Der Befehl \boxed{\textbf{\texttt{subplot(n,m,p)}}} unterteilt ein Graphikfenster in \texttt{n} Zeilen von je \texttt{m} Bildern. Damit können \texttt{n$\times$m} Bilder in ein Graphikfenster eingebettet werden. \texttt{p} ist der Laufindex der \texttt{n$\times$m} Bilder, wobei die Numerierung zeilenweise von links nach rechts erfolgt. Für jedes neue Bild im Graphikfenster wird der Befehl \texttt{subplot} wiederholt, jedesmal mit dem neuen Index \texttt{p}. Der eigentliche Befehl \texttt{plot} mit seinen Parametern muss dann natürlich auc noch kommen. 
\newline\newline
Die Befehle \boxed{\textbf{\texttt{semilogx}}} und {\color{red}\texttt{plot}} sind bis auf die Skalierung der \texttt{x}-Achse identisch. Beim Befehl {\color{red}\texttt{plot}} hat die Abszisse eine lineare, beim Befehl \texttt{semilogx} aber eine logarithmische Skala (Basis 10). Der Befehl {\color{red}\texttt{semilogx(x,y)}} entspricht dem Befehl {\color{red}\texttt{plot(log10(x), y)}}, doch MATLAB gibt bei \texttt{semilogx} für \texttt{x=0} keine Warnung "log for zero". Der Nullpunkt der \texttt{x}-Achse wird unterdrückt, er kann nicht gezeichnet werden. Mit dem Befehl {\color{red}\texttt{semilogy}} wird die Ordinate mit dem Zehnerlogarithmus skaliert.    
\newline\newline
Der Befehl \boxed{\textbf{\texttt{loglog}}} plottet eine zweidimensionale Graphik in einem doppeltlogarithmischen Koordinatensystem (Basis 10). Der Befehl {\color{red}\texttt{loglog(x,y),log10(y)}} entspricht dem Befehl {\color{red}\texttt{plot(log10(x))}}, doch MATLAB gibt bei \texttt{loglog} keine Warnung "log of zero", falls \texttt{x} oder \texttt{y} gleich Null ist. Der Koordinatennullpunkt wird unterdrückt.  
\newline\newline
Der Befehl \boxed{\textbf{\texttt{polar(phi,r)}}} zeichnet die beiden Vektoren \texttt{phi} und \texttt{r} in ein polares Koordinatensystem. Die Werte des Vektors \texttt{phi} sind im Bogenmass angegeben. Die WErte des Vektors \texttt{r} entsprechen dem Radius, d.h. dem Abstand zwischen dem Ursprung und dem betreffenden Punkt der Funktion.
\newline\newline
Der Befehl \boxed{\textbf{\texttt{plot(x1,x2,y1,y2)}}} versieht die Graphik mit zwei Ordinaten. Die linke \texttt{y}-Achse bezieht sich auf \texttt{y1} in Funktion von \texttt{x1} und die rechte \texttt{y}-Achse auf \texttt{y2} in Funktion von \texttt{x2}.
\subsection{Massstab}
Mit dem Befehl \boxed{\textbf{\texttt{axis([xmin xmax ymin ymax])}}} lassen sich die Grenzen der \texttt{x}- und der \texttt{y}-Achse neu definieren. Mit dem Befehl {\color{red}\texttt{axis auto}} setzt für die Achsen wieder dir ursprünglichen Werte ein. Mit dem Befehl {\color{red}\texttt{axis equal}} erhalten alle Achsen die gleiche Skalierung. Mit dem Befehl {\color{red}\texttt{axis ij}} wechselt das Vorzeichen der \texttt{y}-Achse. Die positive \texttt{y}-Achse zeigt nun nach unten. Der Befehl {\color{red}\texttt{axis xy}} macht {\color{red}\texttt{axis ij}} wieder rückgängig. Der Befehl {\color{red}\texttt{axis tight}} passt die Achsenlänge exakt dem Bild an. Der Befehl {\color{red}\texttt{axis off}} schaltet alle axis-Definitionen, die "tick marks" und den Hintergrund aus. Mit dem Befehl {\color{red}\texttt{axis on}} werden sie wieder aktiviert.
\newline\newline
Der Befehl \boxed{\textbf{\texttt{zoom on}}} aktiviert in der aktuellen Graphik die Zoom-Funktion, er kann direkt im Command Window eingegeben werden. Mit "clic and drag" wird dann im Bild ein beliebiger Ausschnitt näher herangebracht. Mit der linken Maustaste wird der Graphikausschnitt vergrössert und mit der rechten Maustaste verkleinert. Der Befehl {\color{red}\texttt{zoom(factor)}} zoomt die Achsen um den für "factor" gewählten Wert. Der Befehl {\color{red}\texttt{zoom out}} führt die Graphik in ihre default-Fenstergrösse zurück. Der Befehl {\color{red}\texttt{zoom xon}} bzw. {\color{red}\texttt{zoom yon}} aktiviert in re aktuellen Graphik die Zoom-Funktion nur für die betreffenden Achsen. Der Befehl {\color{red}\texttt{zoom(figurename, option)}} versieht die Graphik "figurename" mit einer Zoom-Funktion. Als Option kann eine der oben genannten Zoom-Funktionen gewählt werden.
\newline\newline
Der Befehl \boxed{\textbf{\texttt{grid on}}} versieht die aktuelle Graphik mit ienem Liniennetz. Mit dem Befehl {\color{red}\texttt{grid off}} werden die Linien wieder deaktiviert. Der Befehl {\color{red}\texttt{box on}} umrahmt das aktuelle Bild mit einem Rahmen aus dünnen, schwarzen LInien, der Befehl {\color{red}\texttt{box off}} entfernt ihn wieder.
\newline\newline
Der Befehl \boxed{\textbf{\texttt{hold on}}} fixiert das aktuelle Graphikfenster, so dass weitere Funktionen im bereits bestehenden Graphikfenster positioniert werden können. Die ursprünglichen Achseinstellungen bleiben unverändert, selbst dann, wenn die neue Funktion nicht gut in den Rahmen passt. Der Befehl {\color{red}\texttt{hold off}} führt zum Normalbetrieb zurück, d.h. ein neuer plot-Befehl löscht die aktuelle Funktion und fügt die neue Funktion in das Fenster ein, falls nicht mit dem Befehl {\color{red}\texttt{figure}} ein weiteres Graphikfenster geöffnet wurde.
\newline\newline
Der Befehl \boxed{\textbf{\texttt{axes('position', [links unten Breite Höhe])}}} beschreibt im Graphikfenster die POsition der linken unteren Ecke des Bildes und seine Abmessung. Mit Werten zwischen 0 und 1 kann nun die Position und dir Grösse des Bildes festgelegt werden. Das default-Graphikfenster hat eine Breite und eine Höhe 1. Der Befehl {\color{red}\texttt{axex}} generiert in einem Graphikfenster ein Koordinatensystem, in das mit {\color{red}\texttt{plot}} ein beliebiges Bild hineingelegt werden kann. Über mehrere {\color{red}\texttt{axex}}-Definitionen können im Fenster mehrere Bilder erzeugt werden.
\subsection{Beschriften von Bilder} 
Der Befehl \boxed{\textbf{\texttt{legend('st1', 'st2', ...)}}} versieht im Fenster ein Bild mit einer Legende. Er erhält die einzelnen Strings "st1", "st2", usw. Für jedes Bild in einem Graphikfenster kann ein eigener Titel gewählt werden. Der zu schreibende Text wird in Anführungszeichen genommen. Der Befehl {\color{red}\texttt{legend off}} entfernt die Legende aus dem aktuellen Bild. Der Befehl {\color{red}\texttt{legend('st1', 'st2', ..., position)}} positioniert die Legende mit der Angabe einer Position an einen definierten Ort in der Graphik: 0-beste, 1-oben-rechts, 2-oben links, 3-unten-links, 4-unten-rechts, -1-rechts. Die Legende kann verschoben werden, indem die Legende mit der rechten Maustaste angeklickt und an den gewünschten Ort gezogen wird.
\newline\newline
Der Befehl \boxed{\textbf{\texttt{title('text')}}} fügt einen Titel oberhalb der Graphik hinzu. Ein Text kann hoch \texttt{("\^\,")} und tiefgestellte \texttt{"\_\,"}, kursiv \texttt{"it"} geschrieben oder griechische Zeichen enthalten. 
\begin{equation}
\boxed{A_1e^{-\alpha t}\sin \beta t\quad \texttt{'$\backslash$itA\_\{1\}e\^\,\{$\backslash$alpha$\backslash$itt\}sin$\backslash$beta$\backslash$itt'}}
\end{equation}
Der Befehl \boxed{\textbf{\texttt{xlabel('text')}}} schreibt die Zeile "text" unter die Abszisse. Der Befehl \texttt{ylabel} ist für die Beschriftung der Ordinate.
\newline\newline
Der Befehl \boxed{\textbf{\texttt{text(x,y,'text')}}} kann innerhalb des Bildrahmes eine beliebige Textzeile an der Stelle \texttt{(x,y)} angebracht werden. \texttt{x} und \texttt{y} sind in den Koordinaten der Achsen anzugehen. Dagegen verwendet der Befehl {\color{red}\texttt{text(x,y,'text,'sc')}} die Koordinaten des Graphikfensters, nämlich (0,0) in der unteren linken Ecke und (1,1) ub der oberen rechten Ecke. Geht ein Text über mehrere Zeilen, so kann er in eine Text-Variable geschrieben werden. Ein Text kann hoch- und tiefgestellte, kursiv geschriebene oder griechische Zeichen enthalten.
\newline\newline
Mit dem Befehl \boxed{\textbf{\texttt{gtext('text')}}} kann mit der Maus im Bild irgendein Text an beliebigen Ort eingefügt werden. Im Graphifenster erscheint der Mauspfeil als Fadenkreuz. An der gewünschten Stelle kann durch Betätigung einer Maustaste der Text eingefügt werden.
\subsection{Graphiken speichern oder drucken}
Der Befehl \boxed{\textbf{\texttt{print}}} sendet eine Kopie des aktuellen Graphikfensters an den Drucker. Der Befehl {\color{red}\texttt{print filename}} speichert eine Kopie des aktuellen Graphikfensters als PostScript-Datei im aktiven Directory unter dem Dateinamen "filename". Mit dem Befehl {\color{red}\texttt{print path}} kann der genaue Pfad angegeben werden, wo das aktuelle Graphikfenster als POst-Script-Datei abgelegt werden soll. Der Befehl {\color{red}\texttt{print[-ddevice][-options]$<$filenmae$>$}} speichert die aktuelle Graphik im Format des speziell gewählten Druckertreibers und de rzusätzlichen Option im aktiven Directory unter "filename" ab. Der Befehl {\color{red}\texttt{help print}} zeigt im Command-Window alle möglichen \texttt{[-ddevice]} und \texttt{[-options]}.
\newline\newline
Der Befehl \boxed{\textbf{\texttt{orient landscape}}} druckt bei print-Befehlen die Graphikfenster im Querformat. Mit dem Befehl {\color{red}\texttt{orient portrait}} wird das Graphikfenster im Hochformat grdruckt. Der Befehl {\color{red}\texttt{orient tall}} setzt das Blattformat auf Hochformat. Zusätzliche wird das Graphikfenster auf das ganze Papierblatt vergrössert bzw. verkleinert. Der Befehl {\color{red}\texttt{orient}} sagt im Command Window, welche Orientierung momentan aktiv ist.
\section{Dreidimensionale Graphik}
\subsection{Elementare dreidimensionale Graphik}
Der Befehl \boxed{\textbf{\texttt{plot3(x,y,z)}}} plottet im dreidimensionalen Raum die Graphen, die durch die Vektoren \texttt{x}, \texttt{y} und \texttt{z} gegeben sind. Die Vektoren müssen alle dieselbe Länge haben. Mit {\color{red}\texttt{plot3(X,Y,Z)}} zeichnet pro Kolonne einen Graphen. Die Matrizen müssen alle dieselbe Grösse haben. Bei {\color{red}\texttt{plot3(x,y,z,'style')}} können zusätzlich noch der Linientyp, die Plot Symbole und die Farbe des Graphen geändert werden. {\color{red}\texttt{help plot}} listet im Command Window eine Answahl von möglichen Liniendefinitionen auf.
\newline\newline
Bei \boxed{\textbf{\texttt{mesh(Z)}}} entsprechen die Werte der Matrix \texttt{Z} Element von \texttt{$\mathbb{R}^{n\times m}$} den \texttt{z}-Werten des Netzes. Für die \texttt{x}- und \texttt{y}-Werte verwendet \texttt{mesh} die Kolonnen- bzw. die Zeilennummer.
\newline\newline
Mit dem Befehl {\color{red}\texttt{mesh(X,Y,Z,C)}} zeichnet ein Netz un der Vogelperspektive mit \texttt{Z} als Funktion von \texttt{X} und \texttt{Y}. Es handelt sich hierbei um eine FUnktion mit zwei Variablen. \texttt{X}, \texttt{Y} und \texttt{Z} sind Matrizen mit den Werten für die \texttt{x}-, \texttt{y}- und \texttt{z}-Koordinaten. \texttt{X} und \texttt{Y} können aber auch Vektoren der Länge \texttt{m} und \texttt{n} sein. Jeder \texttt{z}-Koordinate aus der Matrix \texttt{Z} werden dann die entsprechenden WErte des \texttt{x}- und \texttt{y}-Vektors zugewiesen. \texttt{C} ist ebenfalls eine Matrix und beinhaltet die Farbskala für die Graphik. Ohne \texttt{C} wird \texttt{C=Z} gesetzt.
\newline\newline
Mit dem Befehl \boxed{\textbf{\texttt{fill(X,Y,Z,C)}}} wird in den Farben der Matrix \texttt{C} ein dreidimensionales Polygon geplottet. Sind \texttt{X}, \texttt{Y} und \texttt{Z} Vektoren, so wird die Fläche unterhalb des Graphen mit Farbe ausgefüllt. Ist \texttt{C} ein Skalar, so wird die Fläche monochrom. Folgende Farben sind möglich: {\color{red}\texttt{'r'}}, {\color{red}\texttt{'g'}}, {\color{red}\texttt{'b'}}, {\color{red}\texttt{'c'}}, {\color{red}\texttt{'m'}}, {\color{red}\texttt{'y'}}, {\color{red}\texttt{'w'}} und {\color{red}\texttt{'k'}}. Mit dem Vektor {\color{red}\texttt{[rot grün blau]}} kann eine neue Farbe gemischt werden. Die Werte von liegen zwischen 0 und 1. Je nach Anteil ergibt sich eine Farbkombination. Ist \texttt{C} ein Vektor, so hat er die gleiche Länge wie \texttt{X}, \texttt{Y} und \texttt{Z}. Wird für \texttt{C} einer der drei Vektoren gewählt, dann ist die Farbabstufung der momentan aktive \texttt{colormap} proportional zur betreffenden Koordinatenachse. 
\newline\newline
Sind \texttt{X}, \texttt{Y} und \texttt{Z} Matrizen, so zeichnet \boxed{\textbf{\texttt{fill3}}} pro Kolonne ein Polygon und füllt es mit der entsprechenden Farbe aus. Die Farbgebung bleibt gleich. Falls \texttt{C} ein Zeilenvektor ist, dann hat das Polygon die Schattierung \texttt{shading flat}. Für eine Matrix wird sie {\color{red}\texttt{shading interp}}. {\color{red}\texttt{shading}} shattiert die Objekt-Oberfläche, {\color{red}\texttt{shading flat}} berechnet für jede Teilfläche einer Oberfläche, die mit den Befehlen \texttt{surf}, \texttt{mesh}, \texttt{polar}, \texttt{fill} oder \texttt{fill3} gebildet wurden, die entsprechende Farbabstufung. {\color{red}\texttt{shading interp}} interpoliert über die Farbabstufung. {\color{red}\texttt{shading faceted}} entspricht dem {\color{red}\texttt{shading flat}}. Die 3D Graphik wird jecoh zusätzlich mit schwarzen Linien versehen. 
\subsection{Projektionsarten einer Graphik}
Mit \boxed{\textbf{\texttt{view(az,el)}}} kann in einem dreidimensionalen Plot der Blickwinkel beliebig eingestellt werden, bzw. die Graphik-Box ist um zwei Achsen drehbar. {\color{red}\texttt{az}} steht für azimuth und definiert die horizontale Rotation im Grad. Für einen positiven Winkel dreht sich die Graphik entgegen dem Uhrzeigersinn um die \texttt{z}-Achse. {\color{red}\texttt{el}} beschreibt die Anheben bzw. Senken der Gtraphik in Grad. Bei einem positiven Winkel befindet sich der Betrachter in der Vogelperspektive, bei negativem Winkel in der Froschperspektive. {\color{red}\texttt{view([x y z])}} setzt den Blickwinkel in kartesischen Koordinaten. {\color{red}\texttt{view(2)}} stellt für die 2D Ansicht den vordefinierten Blickwinkel {\color{red}\texttt{view(0, 90)}} ein. {\color{red}\texttt{view(3)}} stellt für die 3D Ansicht den vordefinierten Blickwinkel {\color{red}\texttt{view(-37.5,30)}} ein. {\color{red}\texttt{T=view}} speichert diew \texttt{view} der aktuiellen Graphik in der Variable \texttt{T} als \texttt{4x4}-Matrix. {\color{red}\texttt{view(T)}} weist einer aktuellen Graphik die in der Variable \texttt{T} gespeicherte view zu.
\newline\newline
\boxed{\textbf{\texttt{T=viewmtx(az,el)}}} weist wie bei {\color{red}\texttt{T=view(az, el)}} der Variable \texttt{T} die 4x4-Transformationsmatrix zu. Die Ansicht der aktuellen Graphik wird dabei nicht verändert. Mit {\color{red}\texttt{T=viewmtx(az, el, phi)}} wird die Graphik durch ein Objektiv betrachtet. Der Linsenwinkel wird in Grad angegeben. \texttt{phi=0} Grad definiert die orthogonale Projektion. 10 Grad entspricht einem Teleobjektiv, 25 Grad einem Normalobjektiv und 60 Grad einem Weitwinkelobjektiv. Bei {\color{red}\texttt{T=viewmtx(az, el, phi, tp)}} wird mit \texttt{tp=[xp, yp, zp]} einen Fluchtpunkt gesetzt.
\newline\newline
\boxed{\textbf{\texttt{rotate3d on}}} aktiviert in der aktuellen Graphik die Maus gesteuerte 3D-Rotation. Die Graphik-Ansicht kann damit beliebig verändert werden. Mit {\color{red}\texttt{rotate3d off}} wird sie wieder deaktiviert. 
\subsection{Dreidimensionale Graphik beschriften}
\boxed{\textbf{\texttt{zlabel('text')}}} versieht die \texttt{z}-Achse mit der Aufschrift "text". Mit dem Befehl {\color{red}\texttt{colorbar('vert')}} erscheint in der aktuellen Graphik eine vertikale Farbskala. In einem 3D-Plot bezieht sie sich auf die Werte der \texttt{z}-Achse. {\color{red}\texttt{colorbar('horiz')}} zeichnet eine vertikale Farbskala. Im 3D-Plot ist die Farbgebung ebenfalls auf die \texttt{z}-Achse abgestimmt. \boxed{\textbf{\texttt{colorbar}}} alleine fügt der Graphik entweder eine vertikale Farbskala hinzu, oder die bestehende Farbskala wird aktualisiert.
\section{Spezielle Graphen}
\boxed{\textbf{\texttt{fill(x,y,c)}}} füllt diejenige Fläche mit der Farbe \texttt{c} aus, welche von der Geraden, die den Endpunkt von \texttt{f(x)} mit dem Anfang verbindet, und der Funktion \texttt{y=f(x)} selbst umgeben wird. Ist \texttt{c} ein Vektor derselben Länge wie \texttt{x} und \texttt{y}, dann verwendet MATLAB entweder die im Vektor \texttt{c} definierten Farben, oder für \texttt{c=x} bzw. \texttt{c=y} die Farbpalette der aktuellen \texttt{colormap}. Sind in \boxed{\textbf{\texttt{fill(X,Y,C)}}} \texttt{X} und \texttt{Y} Matrizen derselben Grösse, so wird pro Kolonne ein Polygon gezeichnet. \texttt{C} kann ein Vektor aber auch eine Matrix sein. Beim Vektor ist die Schattierung der Fläche {\color{red}\texttt{shading flat}}, bei der Matrix {\color{red}\texttt{shading interp}}.
\newline\newline
Mit dem Befehl \boxed{\textbf{\texttt{fplot('f',lim)}}} zeichnet eine beliebige Funktion \texttt{f=f(x)} im Bereich \texttt{lim=[x$_{\texttt{min}}$ x$_{\texttt{max}}$]}. Mit \texttt{lim=[x$_{\texttt{min}}$ x$_{\texttt{max}}$ y$_{\texttt{min}}$ y$_{\texttt{max}}$]} werden zusätzliche Schranken gesetzt. Mit dem Befehl {\color{red}\texttt{fplot('f', lim, tol)}} mit \texttt{tol$<$1} definiert die Toleranz des relativen Fehlers. Die voreingestellte Toleranz ist 2e-3 bzw. 0.2\%. Mit dem Befehl {\color{red}\texttt{fplot('f', lim, N)}} berechnet zwischen \texttt{x$_{\texttt{min}}$} für die Funktion \texttt{f=f(x)} \texttt{N+1} Punkte. Mit dem Befehl {\color{red}\texttt{fplot('f', lim, 'LineSpec')}} definiert mit LineSpec den Linien-Typ der Funktion. Alle möglichen "line specifications" werden mit \texttt{help plot} aufgelistet.    
\newline\newline
Mit dem Befehl \boxed{\textbf{\texttt{hist(x)}}} zeichnet MATLAB ein Histogramm mit den in \texttt{x} gespeicherten Daten. Es ist in 10 gleichmässig verteilte Intervalle unterteilt. Pro Intervall gibt es die Anzahl Elemente an, die es enthält. Wenn \texttt{x} eine Matrix ist, plottet \texttt{hist} pro Kolonne ein Histogramm. Mit dem Befehl {\color{red}\texttt{hist(x,n)}} definiert man zusätzlich die Anzahl \texttt{n} Intervalle. Mit dem Befehl {\color{red}\texttt{hist(x,y)}} berechnet MATLAB die Verteilung von \texttt{x} bezüglich \texttt{y}. \texttt{y} ist ein Vektor, deren Elemente in aufsteigender Ordnung aufgelistet sind. Jedes einzelne Element von \texttt{y} entspricht einem Zentrum.     
\newline\newline
Mit dem Befehl \boxed{\textbf{\texttt{pie(x)}}} stellt die Daten aus dem Vektor \texttt{x} in einem Kuchendiagramm dar. Die Elemente von \texttt{x} werden mit der Summe der \texttt{x}-Werte dividiert. Damit ist die Grösse von jedem einzelnen Kuchenstück in \% gegeben. Mit dem Befehl {\color{red}\texttt{pie(x,explode)}} zieht mit "explode" die gewünschte Stücke aus dem Kuchen, \texttt{explode} ist ein Vektor derselben Länge wie \texttt{x}. Seine Elemente haben entweder den Betrag 0, d.h. das entsprechende Stück von \texttt{x} verbleibt im Kuchen, ode rden Betrag 1, d.h. das dazugehörende Stück von \texttt{x} wird aus dem Kuchen herausgezogen.
\newline\newline
Mit dem Befehl \boxed{\textbf{\texttt{stem(y)}}} zeichnet eine Verteilung der Daten aus dem Vektor \texttt{y}. Jeder Wert aus \texttt{y} im Plot mit einem Kreis und einer Linie versehen. Auf der Abszisse wird jedes Element aus \texttt{x} fortlaufend eingereiht. Auf der Ordinate kann sein Wert abgelesen werden. Der entsprechende Wert ist mit ienem Kreis gekennzeichnet. Bei {\color{red}\texttt{stem(x,y)}} entsprechen die Elemente aus dem \texttt{x}-Vektor den \texttt{x}-Werten und diejenigen aus dem \texttt{y}-Vektor den \texttt{y}-Werten. Mit dem Befehl {\color{red}\texttt{stem(..., 'filled')}} malt den Kreis mit der entsprechenden Farbe aus. Mit dem Befehl {\color{red}\texttt{stem(..., 'linespec')}} kann der Linien-Typ gewählt werden. Alle möglichen "line specifications" werden mit {\color{red}\texttt{help plot}} aufgelistet.    
\newline\newline
Mit dem Befehl \boxed{\textbf{\texttt{contour(Z)}}} zeichnet einen Konturplot mit den WErten aus der Matrix \texttt{Z}. Die Elemente der Matrix \texttt{Z} entsprechen den Werten auf der \texttt{z}-Achse. Die Kolonnenzahlen ergeben die Koordinaten auf der \texttt{x}-Achse und die Zeilenzahlen die WErte auf der \texttt{y}-Achse. Die Höhen für die einzelnen Höhenlinien werden automatisch ausgewählt. Mit {\color{red}\texttt{contour(x,y,z)}} werden die \texttt{x}- und \texttt{y}-Koordinaten explizit mitgeliefert. {\color{red}\texttt{contour(z,N)}} und {\color{red}\texttt{contour(x,y,z,N)}} verwendet für den Konturplot \texttt{N} Höhenlinien. {\color{red}\texttt{contour(Z,v)}} und {\color{red}\texttt{contour(x,y,Z,v)}} zeichnet all jene Höhenlinien, deren Höhen im Vektor \texttt{v} angegeben werden. Mit {\color{red}\texttt{contour(Z,[v v])}} plottet MATLAB eine einzige Höhenlinie bei der Höhe \texttt{v}. Mit {\color{red}\texttt{contour(...,'linespec')}} kann der Linien-Typ gewählt werden. Alle möglichen "line specifications" werden mit {\color{red}\texttt{help plot}} aufgelistet...  
%\chapter{Impulsverhalten von RC-Schaltungen}
%%%%%%%%%%%%%%%%%%%%%%%%%%%%%%%%%%%%%%%%%%%%%%%%%%%%%%%%%%%%%%%%%%%%%%%%%%%%%%%%%%%%%%%%%%%%%%%%%%%%%%%%%%%%%%
\section{Unendliche Reihe}
\subsection{Grundbegriffe}
Aus den Gliedern einer \textbf{unendlichen Zahlenfolge} $\langle a_n\rangle=a_1, a_2,\dotso, a_n, \dotso$ werden wie folgt Partial- oder Teilsummen $s_n$ gebildet.
\begin{equation}
\boxed{s_n=a_1+a_2+a_3+\dotso + a_n=\displaystyle \sum_{k=1}^na_k}
\end{equation}
Die Folge $\langle s_n\rangle$ dieser Partialsummen heisst \textbf{unendliche Reihe}. Besitzt die Folge der Partialsummen $s_n$ einen Grenzwert $s$, $\displaystyle \lim_{n\rightarrow \infty}s_n=s$, so heisst die unendliche Reihe $\displaystyle \sum_{n=1}^{\infty}a_n$ \textbf{konvergent} mit dem Summenwert $s$. Besitzt die Partialsumme keinen Grenzwert, so heisst die unendliche Reihe \textbf{divergent}. 
\begin{equation}
\boxed{\displaystyle \sum_{n=1}^{\infty}a_n=a_1+a_2+a_3+\dotso + a_n+\dotso=s}
\end{equation}
Eine unendliche Reihe $\displaystyle \sum_{n=1}^{\infty}a_n$ heisst \textbf{absolut konvergent}, wenn die aus den Beträgen ihrer Glieder gebildete Reihe $\displaystyle \sum_{n=1}^{\infty}\Big\vert a_n\Big\vert$ konvergiert. Eine Reihe mit dem Summenwert $s=\pm\infty$ ist divergent.
\subsection{Konvergenzkriterien}
Die Bedingung $\displaystyle \lim_{n\rightarrow \infty}a_n=0$ ist zwar notwendig, nicht aber hinreichend für die Konvergenz der Reihe $\displaystyle \sum_{n=1}^{\infty}a_n$. Die Reihenglieder einer konvergenten Reihe müssen also eine \textbf{Nullfolge} bilden.
\newline\newline
Folgende Bedingungen stellen hinreichende Konvergenzbedingungen dar. Sie ermöglichen in vielen Fällen eine Entscheidung darüber, ob eine vorgegebene Reihe \textbf{konvergiert} oder \textbf{divergiert}. 
\subsubsection{Quotientenkriterium}
\begin{equation}
\boxed{\displaystyle \lim_{n\rightarrow \infty}\Big\vert\dfrac{a_{n+1}}{a_n} \Big\vert=q<1}
\end{equation}
Für $q>1$ divergiert die Reihe, für $q=1$ versagt das Kriterium, d.h. eine Entscheidung über Konvergenz oder Divergenz ist anhand dieses Kriteriums nicht möglich.
\subsubsection{Wurzelkriterium}
\begin{equation}
\boxed{\displaystyle \lim_{n\rightarrow \infty}\sqrt[n]{a_n}=q<1}
\end{equation}
Für $q>1$ divergiert die Reihe, für $q=1$ versagt das Kriterium, d.h. eine Entscheidung über Konvergenz oder Divergenz ist anhand dieses Kriteriums nicht möglich.
\subsubsection{Vergleichskriterien}
Das Konvergenzverhalten einer unendlichen Reihe $\displaystyle \sum_{n=1}^{\infty}a_n$ mit positiven Gliedern kann oft mit Hilfe einer geeigneten konvergenten bzw. divergenten Vergleichsreihe $\displaystyle \sum_{n=1}^{\infty}b_n$ bestimmt werden. Mit dem \textbf{Majorantenkriterium} kann die Konvergenz, mit dem \textbf{Minorantenkriterium} die Divergenz einer Reihe festgestellt werden. 
\subsubsection{Majorantenkriterium}
Die vorliegende Reihe konvergiert, wenn die Vergleichsreihe konvergiert und zwischen den Gliedern beider Reihen die Beziehung besteht
\begin{equation}
\boxed{a_n\leq b_n,\quad \forall n\in \mathbb{N}^*}
\end{equation}
Die konvergente Vergleichsreihe wird als \textbf{Majorante} bezeichnet. Es genügt wenn die angegebene Bedingung $a_n\leq b_n$ von einem gewissen $n_0$ an, d.h. für alle Reihenglieder mit $n\geq n_0$ erfüllt wird.
\subsubsection{Minorantenkriterium}
Die vorliegende Reihe divergiert, wenn die Vergleichsreihe divergiert und zwischen den Gliedern beider Reihen die Beziehung besteht
\begin{equation}
\boxed{a_n\geq b_n,\quad \forall n\in \mathbb{N}^*}
\end{equation}
Die divergente  Vergleichsreihe wird als \textbf{Minorante} bezeichnet. Es genügt wenn die angegebene Bedingung $a_n\geq b_n$ von einem gewissen $n_0$ an, d.h. für alle Reihenglieder mit $n\geq n_0$ erfüllt wird.
\subsubsection{Leibnizkriterien}
Eine alternierende Reihe konvergiert, wenn sie die folgenden Bedingungen erfüllt: Die Glieder eine rkonvergenten alternierende Reihe bilden dem Betrage nach eine monoton fallende Nullfolge. Die Reihe konvergiert auch dann, wenn die erste der beiden Bedingungen erst von einem bestimmten Glied an erfüllt ist.
\subsubsection{Eigenschaften}
\begin{enumerate}[$(a)$]
\item Eine konvergente Reihe bleibt konvergent, wenn man endlich viele Glieder weglässt oder hinzufügt oder abändert. Dabei kann sich jedoch der Summenwert ändern. Klammern dürfen in Allgemeinen nicht weggelassen werden, ebenso wenig darf die Reihenfolge der Glieder verändert werden.
\item Aufeinander folgende Glieder einer konvergenten Reihe dürfen durch eine Klammer zusammengefasst werden; der Summenwert der Reihe bleibt dabei erhalten.
\item Eine konvergente Reihe darf gliedweise mit einer Konstanten multipliziert werden, wobei sich auch der Summenwert der Reihe mit dieser Konstanten multipliziert.
\item Konvergente Reihen dürfen gliedweise addiert und subtrahiert werden, wobei sich ihre SUmmenwerte addieren bzw. subtrahieren.
\item Eine absolut konvergente Reihe ist stets konvergent. Für solche Reihen gelten sinngemäss die gleichen Rechenregeln wie für endliche Summen gliedweise Addition, Subtraktion und Multiplikation, beliebige Anordnung der Reihenglieder usw.
\end{enumerate}
\subsection{Spezielle konvergente Reihen}
\subsubsection{Geometrische Reihe}
Divergenz für $\Big\vert q\Big\vert\geq 1$
\begin{equation}
\boxed{\displaystyle \sum_{n=1}^{\infty}\left(a\cdot q^{n-1}\right)=a+aq+aq^2+\dotso+aq^{n-1}+\dotso=\dfrac{a}{1-q},\quad \left(\Big\vert q\Big\vert<1\right)}
\end{equation}
\subsubsection{Weitere konvergente Reihen}
\begin{enumerate}[$(a)$]
\item $1+\dfrac{1}{1!}+\dfrac{1}{2!}+\dfrac{1}{3!}+\dotso+\dfrac{1}{n!}+\dotso=e$
\item $1-\dfrac{1}{2}+\dfrac{1}{3}-\dfrac{1}{4}+-\dotso+\left(-1\right)^{n+1}\cdot \dfrac{1}{n}+\dotso=\ln\left(2\right)$
\item $1-\dfrac{1}{3}+\dfrac{1}{5}-\dfrac{1}{7}+-\dotso+\left(-1\right)^{n+1}\cdot \dfrac{1}{2n-1}+\dotso=\dfrac{\pi}{4}$
\item $\dfrac{1}{1^2}+\dfrac{1}{2^2}+\dfrac{1}{3^2}+\dfrac{1}{4^2}+\dotso+\dfrac{1}{n^2}+\dotso=\dfrac{\pi^2}{6}$
\item $\dfrac{1}{1^2}-\dfrac{1}{2^2}+\dfrac{1}{3^2}-\dfrac{1}{4^2}+-\dotso+\left(-1\right)^{n+1}\cdot \dfrac{1}{n^2}+\dotso=\dfrac{\pi^2}{12}$
\item $\dfrac{1}{1\cdot 2}+\dfrac{1}{2\cdot 3}+\dfrac{1}{3\cdot 4}+\dfrac{1}{4\cdot 5}+\dotso+\dfrac{1}{n\cdot \left(n+1\right)}+\dotso=1$
\end{enumerate}
%%%%%%%%%%%%%%%%%%%%%%%%%%%%%%%%%%%%%%%%%%%%%%%%%%%%%%%%%%%%%%%%%%%%%%%%%%%%%%%%%%%%%%%%%%%%%%%%%%%%%%%%%%%%%%
\section{Potenzreihen}
\subsection{Definition einer Potenzreihe}
\subsubsection{Entwicklung um die Stelle $x_0$}
\begin{equation}
\boxed{P\left(x\right)=\displaystyle \sum_{n=0}^{\infty}a_n \left(x-x_0\right)^n=a_0+a_1 \left(x-x_0\right)+a_2 \left(x-x_0\right)^2+\dotso+a_n \left(x-x_0\right)^n+\dotso}
\end{equation}
\subsubsection{Entwicklung um den Nullpunkt $x_0=0$}
\begin{equation}
\boxed{P\left(x\right)=\displaystyle \sum_{n=0}^{\infty}a_nx^n=a_0+a_1x+a_2x^2+\dotso+a_nx^n+\dotso}
\end{equation}
\subsection{Konvergenzradius und Konvergenzbereich}

Der Konvergenzbereich einer Potenzreihe $\displaystyle \sum_{n=0}^{\infty}a_nx^n$ besteht aus dem offenen Intervall positive Zahl $r$ heisst Konvergenzradius. Für $\Big\vert x\Big\vert>r$ divergiert die Potenzreihe.
\subsubsection{Berechnung des Konvergenzradius $r$}
Folgende Formeln gelten auch für eine um die Stelle $x_0$ entwickelte Potenzreihe. Die Reihe konvergiert dann im Intervall $\Big\vert x-x_0\Big\vert<r$, zu dem gegebenfalls noch ein oder gar beide Randpunkte hinzukommen. 
\begin{equation} 
\boxed{r=\displaystyle \lim_{n\rightarrow \infty}\Big\vert \dfrac{a_n}{a_{n+1}}\Big\vert}\quad \boxed{r=\dfrac{1}{\displaystyle \lim_{n\rightarrow \infty}\sqrt[n]{\Big\vert a_n\Big\vert}}}
\end{equation} 
\begin{enumerate}[$(i)$]
\item Sei $r=0$ so ist konvergiert die Potenzreihe nur für $n\rightarrow \infty$
\item Sei $r=\infty$, so konvergiert die Potenzreihe beständig, d.h. für jedes $x\in \mathbb{R}$ 
\end{enumerate}
\subsection{Eigenschaften einer Potenzreihe}
\begin{enumerate}[$(i)$]
\item Eine Potenzreihe konvergiert innerhalb ihres Konvergentbereiches absolut.
\item Eine Potenzreihe darf innerhalb ihres Konvernegzbereiches gliedweise differenziert und integriert werden. Die neuen Potenzreihen haben dabei denselben Konvergenzradius $r$ wie die ursprüngliche Reihe.
\item Zwei Potenzreihen dürfen im gemeinsamen Konvergenzbereich der Reihen gliedweise addiert, subtrahiert und multipliziert werden. Die neuen Potenzreihen konvergieren dann mindestens im gemeinsamen Konvergenzbereich der beiden Ausgangsreihen. 
\end{enumerate}
\section{Taylor-Reihen}
\subsection{Taylorsche und Mac Laurinsche Formel}
\subsubsection{Taylorsche Formel}
\begin{equation}
\boxed{f_n\left(x\right)=f\left(x_0\right)+\dfrac{f'\left(x_0\right)}{1!}\left(x-x_0\right)+\dfrac{f''\left(x_0\right)}{2!}\left(x-x_0\right)^2+\dotso+\dfrac{f^{\left(n\right)}\left(x_0\right)}{n!}\left(x-x_0\right)^n}
\end{equation}
\begin{equation}
\boxed{R_n\left(x\right)=\dfrac{f^{\left(n+1\right)}\left(\xi\right)}{\left(n+1\right)!}\left(x-x_0\right)^{n+1},\quad \left(x< \xi< x_0\right)}
\end{equation}
\begin{equation}
\boxed{f\left(x\right)=f_n\left(x\right)+R_n\left(x\right)}
\end{equation}
\subsubsection{Mac Laurinsche Formel}
\begin{equation}
\boxed{f_n\left(x\right)=f\left(0\right)+\dfrac{f'\left(0\right)}{1!}\left(x\right)+\dfrac{f''\left(0\right)}{2!}\left(x\right)^2+\dotso+\dfrac{f^{\left(n\right)}\left(0\right)}{n!}\left(x\right)^n}
\end{equation}
\begin{equation}
\boxed{R_n\left(x\right)=\dfrac{f^{\left(n+1\right)}\left(\theta x\right)}{\left(n+1\right)!}\left(x\right)^{n+1},\quad \left(0< \theta< 1\right)}
\end{equation}
\begin{equation}
\boxed{f\left(x\right)=f_n\left(x\right)+R_n\left(x\right)}
\end{equation}
\subsection{Taylorsche Reihe}
$f\left(x\right)$ ist in der Umgebung von $x_0$ beliebig oft differenzierbar und das Restglied $R_n\left(x\right)$ in der Taylorschen Formel verschwindet für $n\rightarrow \infty$
\begin{equation}
\boxed{
\begin{array}{lll}
f\left(x\right)&=&f\left(x_0\right)+\dfrac{f'\left(x_0\right)}{1!}\left(x-x_0\right)+\dfrac{f''\left(x_0\right)}{2!}\left(x-x_0\right)^2+\dotso\\
&=&\displaystyle \sum_{n=0}^{\infty}\dfrac{f^{\left(n\right)}\left(x_0\right)}{n!}\left(x-x_0\right)^n
\end{array}
}
\end{equation}
\subsection{Mac Laurinsche Reihe}
Die Mac Laurinsche Reihe ist eine spezielle Form der Taylorschen Reihe für das Entwicklungszentrum $x_0=0$. Bei einer geraden Funktion treten nur gerade Potenzen auf, bei einer ungeraden Funktion nur ungerade Potenzen.
\begin{equation}
\boxed{
\begin{array}{lll}
f\left(x\right)&=&f\left(0\right)+\dfrac{f'\left(0\right)}{1!}x+\dfrac{f''\left(0\right)}{2}x^2+\dotso\\
&=&\displaystyle \sum_{n=0}^{\infty}\dfrac{f^{\left(n\right)}\left(0\right)}{n!}x^n
\end{array}
}
\end{equation}
\section{Spezielle Potenzreihenentwicklungen}
\subsubsection{Allgemeine Binomische Reihe}
\begin{enumerate}[$(a)$]
\item $\left(1\pm x\right)^n=1\pm \displaystyle \binom{n}{1}x+\displaystyle \binom{n}{2}x^2\pm \displaystyle \binom{n}{3}x^3+\displaystyle \binom{n}{4}x^4\pm \dotso$
\item $\left(a\pm x\right)^n=a^n\pm \displaystyle \binom{n}{1}a^{n-1}x+\displaystyle \binom{n}{2}a^{n-2}x^2\pm \displaystyle \binom{n}{3}a^{n-3}x^3+\displaystyle \binom{n}{4}a^{n-4}x^4\pm \dotso$
\end{enumerate}
\subsubsection{Spezielle Binomische Reihen}
\begin{enumerate}[$(a)$]
\item $\left(1\pm x\right)^{1/4}=1\pm \dfrac{1}{4}x-\dfrac{1}{4}\dfrac{3}{8}x^2\pm \dfrac{1}{4}\dfrac{3}{8}\dfrac{7}{12}x^3-\dfrac{1}{4}\dfrac{3}{8}\dfrac{7}{12}\dfrac{11}{16}x^4\pm \dotso\quad \Bigg\{\begin{matrix}n>0: \Big\vert x\Big\vert\leq 1\\n<0: \Big\vert x\Big\vert< 1\end{matrix}$
\item $\left(1\pm x\right)^{1/3}=1\pm \dfrac{1}{3}x-\dfrac{1}{3}\dfrac{2}{6}x^2\pm \dfrac{1}{3}\dfrac{2}{6}\dfrac{5}{9}x^3-\dfrac{1}{3}\dfrac{2}{6}\dfrac{5}{9}\dfrac{8}{12}x^4\pm \dotso\quad \Bigg\{\begin{matrix}n>0: \Big\vert x\Big\vert\leq \Big\vert a\Big\vert\\n<0: \Big\vert x\Big\vert< \Big\vert a\Big\vert\end{matrix}$
\item $\left(1\pm x\right)^{1/2}=1\pm \dfrac{1}{2}x-\dfrac{1}{2}\dfrac{1}{4}x^2\pm \dfrac{1}{2}\dfrac{1}{4}\dfrac{3}{6}x^3-\dfrac{1}{2}\dfrac{1}{4}\dfrac{3}{6}\dfrac{5}{8}x^4\pm \dotso\quad \Big\vert x\Big\vert\leq 1$
\item $\left(1\pm x\right)^{3/2}=1\pm \dfrac{3}{2}x+\dfrac{3}{2}\dfrac{1}{4}x^2\mp \dfrac{3}{2}\dfrac{1}{4}\dfrac{1}{6}x^3+\dfrac{3}{2}\dfrac{1}{4}\dfrac{1}{6}\dfrac{3}{8}x^4\mp \dotso\quad \Big\vert x\Big\vert\leq 1$
\item $\left(1\pm x\right)^{-1/4}=1\mp \dfrac{1}{4}x+\dfrac{1}{4}\dfrac{5}{8}x^2\mp \dfrac{1}{4}\dfrac{5}{8}\dfrac{9}{12}x^3+\dfrac{1}{4}\dfrac{5}{8}\dfrac{9}{12}\dfrac{13}{16}x^4\mp \dotso\quad \Big\vert x\Big\vert< 1$
\item $\left(1\pm x\right)^{-1/3}=1\mp \dfrac{1}{3}x+\dfrac{1}{3}\dfrac{4}{6}x^2\mp \dfrac{1}{3}\dfrac{4}{6}\dfrac{7}{9}x^3+\dfrac{1}{3}\dfrac{4}{6}\dfrac{7}{9}\dfrac{10}{12}x^4\mp \dotso\quad \Big\vert x\Big\vert< 1$
\item $\left(1\pm x\right)^{-1/2}=1\mp \dfrac{1}{2}x+\dfrac{1}{2}\dfrac{3}{4}x^2\mp \dfrac{1}{2}\dfrac{3}{4}\dfrac{5}{6}x^3+\dfrac{1}{2}\dfrac{3}{4}\dfrac{5}{6}\dfrac{7}{8}x^4\mp \dotso\quad \Big\vert x\Big\vert< 1$
\item $\left(1\pm x\right)^{-1}=1\mp x+x^2\mp x^3+x^4\mp \dotso\quad \Big\vert x\Big\vert\leq 1$
\item $\left(1\pm x\right)^{-3/2}=1\mp \dfrac{3}{2}x+\dfrac{3}{2}\dfrac{5}{4}x^2\mp \dfrac{3}{2}\dfrac{5}{4}\dfrac{7}{6}x^3+\dfrac{3}{2}\dfrac{5}{4}\dfrac{7}{6}\dfrac{9}{8}x^4\mp \dotso\quad \Big\vert x\Big\vert< 1$
\item $\left(1\pm x\right)^{-2}=1\mp 2x+3x^2\mp 4x^3+5x^4\mp \dotso\quad \Big\vert x\Big\vert< 1$
\item $\left(1\pm x\right)^{-3}=1\mp \dfrac{1}{2}\left(2\cdot 3x\mp 3\cdot 4x^2+4\cdot 5x^3\mp 5\cdot 6x^4+\dotso\right)\quad \Big\vert x\Big\vert< 1$
\end{enumerate}
\subsubsection{Reihen der Exponentialfunktionen}
\begin{enumerate}[$(a)$]
\item $e^x=1+\dfrac{x}{1!}+\dfrac{x^2}{2!}+\dfrac{x^3}{3!}+\dfrac{x^4}{4!}+\dotso=\displaystyle \sum_{k=0}^{\infty}\dfrac{z^k}{k!}\quad \Big\vert x\Big\vert< \infty$
\item $e^{-x}=1-\dfrac{x}{1!}+\dfrac{x^2}{2!}-\dfrac{x^3}{3!}+\dfrac{x^4}{4!}-+\dotso\quad \Big\vert x\Big\vert< \infty$
\item $a^x=1+\dfrac{\ln\left(a\right)}{1!}x+\dfrac{\left(\ln\left(a\right)\right)^2}{2!}x^2+\dfrac{\left(\ln\left(a\right)\right)^3}{3!}x^3+\dfrac{\left(\ln\left(a\right)\right)^4}{4!}x^4+\dotso\quad \Big\vert x\Big\vert< \infty$
\end{enumerate}
\subsubsection{Reihen der logarithmischen Funktionen}
\begin{enumerate}[$(a)$]
\item $\ln\left(x\right)=\left(x-1\right)-\dfrac{1}{2}\left(x-1\right)^2+\dfrac{1}{3}\left(x-1\right)^3-\dfrac{1}{4}\left(x-1\right)^4+-\dotso\quad 0<x\leq 2$
\item $\ln\left(x\right)=2\Big[\left(\dfrac{x-1}{x+1}\right)+\dfrac{1}{3}\left(\dfrac{x-1}{x+1}\right)^3+\dfrac{1}{5}\left(\dfrac{x-1}{x+1}\right)^5+\dfrac{1}{7}\left(\dfrac{x-1}{x+1}\right)^7+\dotso\Big]\quad x>0$
\item $\ln\left(1+x\right)=x-\dfrac{x^2}{2}+\dfrac{x^3}{3}-\dfrac{x^4}{4}+-\dotso\quad -1<x\leq 1$
\item $\ln\left(1-x\right)=-\Big[x+\dfrac{x}{2}+\dfrac{x^3}{3}+\dfrac{x^4}{4}+\dotso\Big]\quad -1\leq x\leq 1$
\item $\ln\left(\dfrac{1+x}{1-x}\right)=2\Big[x+\dfrac{x^3}{3}+\dfrac{x^5}{5}+\dfrac{x^7}{7}+\dotso\Big]\quad \Big\vert x\Big\vert<1$
\end{enumerate}
\subsubsection{Reihen der trigonometrischen Funktionen}
\begin{enumerate}[$(a)$]
\item $\sin\left(x\right)=x-\dfrac{x^3}{3!}+\dfrac{x^5}{5!}-\dfrac{x^7}{7!}+-\dotso=\displaystyle \sum_{k=0}^{\infty}\left(-1\right)^k\dfrac{z^{2k+1}}{\left(2k+1\right)!}\quad \Big\vert x\Big\vert<\infty$
\item $\cos\left(x\right)=1-\dfrac{x^2}{2!}+\dfrac{x^4}{4!}-\dfrac{x^6}{6!}+-\dotso=\displaystyle \sum_{k=0}^{\infty}\left(-1\right)^k\dfrac{z^{2k}}{\left(2k\right)!}\quad \Big\vert x\Big\vert<\infty$
\item $\tan\left(x\right)=x+\dfrac{1}{3}x^3+\dfrac{2}{15}x^5+\dfrac{17}{315}x^7+\dfrac{62}{2835}x^9+\dotso\quad \Big\vert x\Big\vert<\dfrac{\pi}{2}$
\item $\cot\left(x\right)=\dfrac{1}{x}-\dfrac{1}{3}x-\dfrac{1}{45}x^3-\dfrac{2}{945}x^5-\dotso\quad 0<\Big\vert x\Big\vert<\pi$
\end{enumerate}
\subsubsection{Reihen der Arkusfunktionen}
\begin{enumerate}[$(a)$]
\item $\arcsin\left(x\right)=x+\dfrac{1}{2\cdot 3}x^3+\dfrac{1\cdot 3}{2\cdot 4\cdot 5}x^5+\dfrac{1\cdot 3\cdot 5}{2\cdot 4\cdot 6\cdot 7}x^7+\dotso\Big\vert x\Big\vert<1$
\item $\arccos\left(x\right)=\dfrac{\pi}{2}-\Big[x+\dfrac{1}{2\cdot 3}x^3+\dfrac{1\cdot 3}{2\cdot 4\cdot 5}x^5+\dfrac{1\cdot 3\cdot 5}{2\cdot 4\cdot 6\cdot 7}x^7+\dotso\Big]\quad \Big\vert x\Big\vert<1$
\item $\arctan\left(x\right)=x-\dfrac{x^3}{3}+\dfrac{x^5}{5}-\dfrac{x^7}{7}+-\dotso\quad \Big\vert x\Big\vert<1$ 
\item $\arccot\left(x\right)=\dfrac{\pi}{2}-\Big[x-\dfrac{x^3}{3}+\dfrac{x^5}{5}-\dfrac{x^7}{7}+-\dotso\Big]\quad \Big\vert x\Big\vert<1$ 
\end{enumerate}
\subsubsection{Reihen der Hyperbelfunktionen}
\begin{enumerate}[$(a)$]
\item $\sinh\left(x\right)=x+\dfrac{x^3}{3!}+\dfrac{x^5}{5!}+\dfrac{x^7}{7!}+\dotso=\displaystyle \sum_{k=0}^{\infty}\dfrac{z^{2k+1}}{\left(2k+1\right)!}\quad \Big\vert x\Big\vert<\infty$
\item $\cosh\left(x\right)=1+\dfrac{x^2}{2!}+\dfrac{x^4}{4!}+\dfrac{x^6}{6!}+\dotso=\displaystyle \sum_{k=0}^{\infty}\dfrac{z^{2k}}{\left(2k\right)!}\quad \Big\vert x\Big\vert<\infty$
\item $\tanh\left(x\right)=x-\dfrac{1}{3}x^3+\dfrac{2}{15}x^5-\dfrac{17}{315}x^7+\dfrac{62}{2835}x^9-+\dotso\quad \Big\vert x\Big\vert<\dfrac{\pi}{2}$
\item $\coth\left(x\right)=\dfrac{1}{x}+\dfrac{1}{3}x-\dfrac{1}{45}x^3+\dfrac{2}{945}x^5-+\dotso\quad 0<\Big\vert x\Big\vert<\pi$
\end{enumerate}
\subsubsection{Reihen der Areafunktionen}
\begin{enumerate}[$(a)$]
\item $\text{Arsinh}\left(x\right)=x-\dfrac{1}{2\cdot 3}x^3+\dfrac{1\cdot 3}{2\cdot 4\cdot 5}x^5-\dfrac{1\cdot 3\cdot 5}{2\cdot 4\cdot 6\cdot 7}x^7+-\dotso=\displaystyle \sum_{k=0}^{\infty}\binom{-1/2}{k}\dfrac{z^{2k+1}}{2k+1}\quad \Big\vert x\Big\vert<1$
\item $\text{Arcosh}\left(x\right)=\ln\left(2x\right)-\dfrac{1}{2\cdot 2x^2}+\dfrac{1\cdot 3}{2\cdot 4\cdot 4x^4}-\dfrac{1\cdot 3\cdot 5}{2\cdot 4\cdot 6\cdot 6x^6}-\dotso\quad \Big\vert x\Big\vert>1$
\item $\text{Artanh}\left(x\right)=x+\dfrac{x^3}{3}+\dfrac{x^5}{5}+\dfrac{x^7}{7}+\dotso=\displaystyle \sum_{k=0}^{\infty}\dfrac{z^{2k+1}}{2k+1}\quad \Big\vert x\Big\vert<1$
\item $\text{Arcoth}\left(x\right)=\dfrac{1}{x}+\dfrac{1}{3x^3}+\dfrac{1}{5x^5}+\dfrac{1}{7x^7}+\dotso\quad \Big\vert x\Big\vert>1$
\end{enumerate}
\section{Näherungspolynome einer Funktion}
Bricht man die Potenzreihenentwicklung einer Funktion $f\left(x\right)$ nach der $n$-ten POtenz ab, so erhält man ein Näherungspolynom $f_n\left(x\right)$ vom Grade $n$ für $f\left(x\right)$, das sogenannte Mac. Laurinsches bzw. Taylorsches Polynom. Funktion $f\left(x\right)$ und $f_n\left(x\right)$ stimmen an der Entwicklungsstelle $x_0$ in ihrem Funktionswert und in ihren ersten $n$ Ableitungen miteinander überein.
\subsection{Fehlerabschätzung}
Der durch den Abbruch der Potenzreihe entstandene Fehler lässt sich in Allgemeinen anhand der Lagrangeschen Restgliedformel abschätzen. Er liegt in der Grössenordnung des grössten Reihengliedes, das in der Näherung nicht mehr berücksichtigt wurde.
\subsection{Näherungspolynome spezieller Funktionen}
\begin{enumerate}[$(a)$]
\item $\left(1\pm x\right)^n=\Bigg\{\begin{matrix}=1+\pm nx,\quad \text{(1. Näherung)}\\=1\pm nx+\dfrac{n\left(n-1\right)}{2}x^2,\quad \text{(2. Näherung)}\end{matrix}$
\item $e^x=\Bigg\{\begin{matrix}=1+x,\quad \text{(1. Näherung)}\\=1+x+\dfrac{1}{2}x^2,\quad \text{(2. Näherung)}\end{matrix}$
\item $e^{-x}=\Bigg\{\begin{matrix}=1-x,\quad \text{(1. Näherung)}\\=1-x+\dfrac{1}{2}x^2,\quad \text{(2. Näherung)}\end{matrix}$
\item $a^{x}=\Bigg\{\begin{matrix}=1+\ln\left(a\right)x,\quad \text{(1. Näherung)}\\=1+\ln\left(a\right)x+\dfrac{\left(\ln\left(a\right)\right)^2}{2}x^2,\quad \text{(2. Näherung)}\end{matrix}$
\item $\ln\left(1+x\right)=\Bigg\{\begin{matrix}=x,\quad \text{(1. Näherung)}\\=x-\dfrac{1}{2}x^2,\quad \text{(2. Näherung)}\end{matrix}$
\item $\ln\left(1-x\right)=\Bigg\{\begin{matrix}=-x,\quad \text{(1. Näherung)}\\=-x-\dfrac{1}{2}x^2,\quad \text{(2. Näherung)}\end{matrix}$
\item $\ln\left(\dfrac{1+x}{1-x}\right)=\Bigg\{\begin{matrix}=2x,\quad \text{(1. Näherung)}\\=2x+\dfrac{2}{3}x^3,\quad \text{(2. Näherung)}\end{matrix}$
\item $\sin\left(x\right)=\Bigg\{\begin{matrix}=x,\quad \text{(1. Näherung)}\\=x-\dfrac{1}{6}x^3,\quad \text{(2. Näherung)}\end{matrix}$
\item $\cos\left(x\right)=\Bigg\{\begin{matrix}=1-\dfrac{1}{2}x^2,\quad \text{(1. Näherung)}\\=1-\dfrac{1}{2}x^2+\dfrac{1}{24}x^4,\quad \text{(2. Näherung)}\end{matrix}$
\item $\tan\left(x\right)=\Bigg\{\begin{matrix}=x,\quad \text{(1. Näherung)}\\=x+\dfrac{1}{3}x^3,\quad \text{(2. Näherung)}\end{matrix}$
\item $\arcsin\left(x\right)=\Bigg\{\begin{matrix}=x,\text{(1. Näherung)}\\=x+\dfrac{1}{6}x^3,\quad \text{(2. Näherung)}\end{matrix}$
\item $\arccos\left(x\right)=\Bigg\{\begin{matrix}=\dfrac{\pi}{2}-x,\text{(1. Näherung)}\\=\dfrac{\pi}{2}-x-\dfrac{1}{6}x^3,\quad \text{(2. Näherung)}\end{matrix}$
\item $\arctan\left(x\right)=\Bigg\{\begin{matrix}=x,\text{(1. Näherung)}\\=x-\dfrac{1}{3}x^3,\quad \text{(2. Näherung)}\end{matrix}$
\item $\arccot\left(x\right)=\Bigg\{\begin{matrix}=\dfrac{\pi}{2}-x,\text{(1. Näherung)}\\=\dfrac{\pi}{2}-x+\dfrac{1}{3}x^3,\quad \text{(2. Näherung)}\end{matrix}$
\item $\sinh\left(x\right)=\Bigg\{\begin{matrix}=x,\text{(1. Näherung)}\\=x+\dfrac{1}{6}x^3,\quad \text{(2. Näherung)}\end{matrix}$
\item $\cosh\left(x\right)=\Bigg\{\begin{matrix}=1+\dfrac{1}{2}x^2,\text{(1. Näherung)}\\=1+\dfrac{1}{2}x^2+\dfrac{1}{24}x^4,\quad \text{(2. Näherung)}\end{matrix}$
\item $\tanh\left(x\right)=\Bigg\{\begin{matrix}=x,\text{(1. Näherung)}\\=x-\dfrac{1}{3}x^3,\quad \text{(2. Näherung)}\end{matrix}$
\item $\text{Arsinh}\left(x\right)=\Bigg\{\begin{matrix}=x,\text{(1. Näherung)}\\=x-\dfrac{1}{6}x^3,\quad \text{(2. Näherung)}\end{matrix}$
\item $\text{Artanh}\left(x\right)=\Bigg\{\begin{matrix}=x,\text{(1. Näherung)}\\=x+\dfrac{1}{3}x^3,\quad \text{(2. Näherung)}\end{matrix}$
\end{enumerate}
\section{Fourier-Reihen}
\subsection{Fourier-Reihe einer periodischen Funktion}
Eine periodische Funktion $f\left(x\right)$ mit der Periode $p=2\pi$ lässt sich unter bestimmten Voraussetzungen in eine unendliche trigonometrische Reihe der Form entwickeln.
\begin{equation}
\boxed{f\left(x\right)=\dfrac{a_0}{2}+\displaystyle \sum_{n=1}^{\infty}\Big[a_n\cdot \cos\left(nx\right)+b_n\cdot \sin\left(nx\right)\Big]}
\end{equation}
\subsubsection{Berechnung der Fourier-Koeffizienten $a_n$ und $b_n$}
\begin{equation}
\boxed{a_0=\dfrac{1}{\pi}\displaystyle \int_0^{2\pi}f\left(x\right)\,\text{d}x}
\end{equation}
\begin{equation}
\boxed{a_n=\dfrac{1}{\pi}\displaystyle \int_0^{2\pi}f\left(x\right)\cdot \cos\left(nx\right)\,\text{d}x,\quad b_n=\dfrac{1}{\pi}\displaystyle \int_0^{2\pi}f\left(x\right)\cdot \sin\left(nx\right)\,\text{d}x}
\end{equation}
\begin{enumerate}[$(1)$]
\item Voraussetzung ist, dass die folgenden Dirichletschen Bedingungen erfüllt sind
\begin{enumerate}[$(a)$]
\item Das Periodenintervall lässt sich in endlich Teilintervalle zerlegen, in denen $f\left(x\right)$ stetig und monoton ist.
\item Besitzt die Funktion $f\left(x\right)$ im Periodenintervall Unstetigkeitsstellen, so existiert in ihnen sowohl der links-als auch der rechtsseitige Grenzwert. 
\end{enumerate}
\item In den Sprungstellen der Funktion $f\left(x\right)$ liefert die Fourier-Reihe von $f\left(x\right)$ das arithmetische Mittel aus dem links- und rechtsseitigen Grenzwert der Funktion.
\end{enumerate}
\subsubsection{Symmetriebetrachtungen}
$f\left(x\right)$ ist eine gerade Funktion:
\begin{equation}
\boxed{f\left(x\right)=\dfrac{a_0}{2}+\displaystyle \sum_{n=1}^{\infty}a_n\cdot \cos\left(nx\right),\quad \left(b_n=0\text{ für } n\in \mathbb{N}^*\right)}
\end{equation}
$f\left(x\right)$ ist eine ungerade Funktion:
\begin{equation}
\boxed{f\left(x\right)=\displaystyle \sum_{n=1}^{\infty}b_n\cdot \sin\left(nx\right),\quad \left(a_n=0\text{ für } n\in \mathbb{N}^*\right)}
\end{equation}
\subsubsection{Komplexe Darstellung der Fourier-Reihe}
\begin{equation}
\boxed{f\left(x\right)=\displaystyle \sum_{n=-\infty}^{\infty}c_n\cdot e^{jnx},\quad c_n=\dfrac{1}{2\pi}\displaystyle \int_{0}^{2\pi}f\left(x\right)\cdot e^{-jnx},\quad \left(n\in \mathbb{Z}\right)}
\end{equation}
Die komplexe Fourier-Reihe lässt sich auch wie folgt aufspalten
\begin{equation}
\boxed{f\left(x\right)=\displaystyle \sum_{n=-\infty}^{\infty}c_n\cdot e^{jnx}=c_0+\displaystyle \sum_{n=1}^\infty c_{-n}\cdot e^{-jnx}+\displaystyle \sum_{n=1}^{\infty}c_n\cdot e^{jnx}}
\end{equation}
Der Koeffizient $c_{-n}$ ist dabei konjugiert komplex zu $c_n$, d.h. $c_{-n}=c_n^*$
\subsubsection{Zusammenhang zwischen den Koeffizienten $a_n$, $b_n$ und $c_n$}
Übergang von der reellen zur komplexen Form
\begin{equation}
\boxed{c_0=\dfrac{1}{2}a_0,\quad c_n=\dfrac{1}{2}\left(a_n-jb_n\right),\quad c_{-n}=\dfrac{1}{2}\left(a_n+jb_n\right),\quad \left(n\in \mathbb{N}^*\right)}
\end{equation}
Übergang von der komplexen zur reellen Form
\begin{equation}
\boxed{a_0=2c_0,\quad a_n=c_n+c_{-n},\quad b_{n}=j\left(c_n-c_{-n}\right),\quad \left(n\in \mathbb{N}^*\right)}
\end{equation}
\subsection{Fourier einer nichtsinusförmigen Schwingung}
Eine nichtsinusförmig verlaufende Schwingung $y=y\left(t\right)$ mit der Kreisfrequenz $\omega_0$ und der Schwingungsdauer (Periodendauer) $T=2\pi/\omega$ lässt sich nach Fourier wie folgt in ihre harmonischen Bestandteile zerlegen. $\omega_0=2\pi/T$ ist die Kreisfrequenz der Grundschwingung. $n\omega_0$ sind die Kreisfrequenzen der harmonischen Oberschwingungen $\left(n=2, 3, 4, \dotso \right)$ 
\begin{equation}
\boxed{y\left(t\right)=\dfrac{a_0}{2}+\displaystyle \sum_{n=1}^{\infty}\Big[a_n\cdot \cos\left(n\omega_0t\right)+b_n\cdot \sin\left(n\omega_0t\right)\Big]}
\end{equation}
\subsubsection{Berechnung der Fourier-Koeffizienten $a_n$ und $b_n$}
\begin{equation}
\boxed{a_0=\dfrac{2}{T}\displaystyle\int_{\left(T\right)}y\left(t\right)\,\text{d}t}
\end{equation}
\begin{equation}
\boxed{a_n=\dfrac{2}{T}\displaystyle \int_{\left(T\right)}y\left(t\right)\cdot \cos\left(n\omega_0t\right)\,\text{d}t,\quad b_n=\dfrac{2}{T}\displaystyle \int_{\left(T\right)}y\left(t\right)\cdot \sin\left(n\omega_0t\right)\,\text{d}t}
\end{equation}
\subsubsection{Fourier-Zerlegung in phasenverschobene Sinusschwingungen}
\begin{equation}
\boxed{y\left(t\right)=\dfrac{a_0}{2}+\displaystyle \sum_{n=1}^{\infty}\Big[a_n\cdot \cos\left(n\omega_0t\right)+b_n\sin\left(n\omega_0t\right)\Big]=A_0+\displaystyle \sum_{n=1}^{\infty}A_n\cdot \sin\left(n\omega_0t+\varphi_n\right)}
\end{equation}
\begin{equation}
\boxed{A_0=\dfrac{a_0}{2},\quad A_n=\sqrt{a_n^2+b_n^2},\quad \tan\left(\varphi_n\right)=\dfrac{a_n}{b_n}\quad \left(n\in \mathbb{N}^*\right)}
\end{equation}
\subsubsection{Fourier-Zerlegung in komplexer Form}
\begin{equation}
\boxed{y\left(t\right)=\displaystyle \sum_{n=-\infty}^{\infty}c_n\cdot e^{jn\omega_0t}}
\end{equation}
\begin{equation}
\boxed{c_n=\dfrac{1}{T}\displaystyle \int_{0}^Ty\left(t\right)\cdot e^{-jn\omega_0t}\,\text{d}t}
\end{equation}
\subsection{Spezielle Fourier-Reihen}
\subsubsection{Rechteckskurve}
\begin{equation}
\boxed{y\left(t\right)=\Bigg\{\begin{matrix}\hat{y}&\texttt{für}&0\leq t\leq \dfrac{T}{2}\\0&\texttt{für}&\dfrac{T}{2}\leq t\leq T\end{matrix}}
\end{equation}
\begin{equation}
\boxed{y\left(t\right)=\dfrac{\hat{y}}{2}+\dfrac{2\hat{y}}{\pi}\left(\sin\left(\omega_0t\right)+\dfrac{1}{3}\sin\left(3\omega_0t\right)+\dfrac{1}{5}\sin\left(5\omega_0t\right)+\dotso\right)}
\end{equation}
\subsubsection{Rechtecksimpuls}
\begin{equation}
\boxed{b=\dfrac{T}{2}-2a}
\end{equation}
\begin{equation}
\boxed{y\left(t\right)\Bigg\{\begin{matrix}\hat{y}&\text{für}&a<t<\dfrac{T}{2}-a\\-\hat{y}&\text{für}&\dfrac{T}{2}+a<t<T-a\\0&&\text{im übrigen Intervall}\end{matrix}}
\end{equation}
\begin{equation}
\boxed{y\left(t\right)=\dfrac{4\hat{y}}{\pi}\left(\dfrac{\cos\left(\omega_0a\right)}{1}\cdot \sin\left(\omega_0t\right)+\dfrac{\cos\left(3\omega_0a\right)}{3}\cdot \sin\left(3\omega_0t\right)+\dotso\right)}
\end{equation}
\subsubsection{Dreieckskurve}
\begin{equation}
\boxed{y\left(t\right)\Bigg\{\begin{matrix}-\dfrac{2\hat{y}}{T}+\hat{y}&\text{für}&0\leq t\leq \dfrac{T}{2}\\\dfrac{2\hat{y}}{T}t-\hat{y}&\text{für}&\dfrac{T}{2}\leq t \leq T\\\end{matrix}}
\end{equation}
\begin{equation}
\boxed{y\left(t\right)=\dfrac{\hat{y}}{2}+\dfrac{4\hat{y}}{\pi^2}\left(\dfrac{1}{1^2}\cdot\cos\left(\omega_0t\right)+\dfrac{1}{3^2}\cdot \cos\left(3\omega_0t\right)+\dfrac{1}{5^2}\cdot \cos\left(5\omega_0t\right)+\dotso\right)}
\end{equation}
%\chapter{Die komplexe Zahlen}
\section{Darstellungsformen einer komplexe Zahlen}
\subsection{Die Kartesische Form}
Eine komplexe Zahl $z$ lässt sich in der Gaussschen Zahlenebene durch einen Bildpunkt $P\left(z\right)$ oder durch einen vom Koordinatenursprung $O$ zum Bildpunkt $P\left(z\right)$ gerichteten Zeiger bildlich darstellen. Die komplexe Zahl besteht aus einem reellen $\text{Re}\left(z\right)=a$, einem imaginären Anteil $\text{Im}\left(z\right)=b$ und eine imaginäre Einheit $\text{j}^2=-1$. Die Menge ist $\mathbb{C}=\left\{z\,\vert\, z=a+\text{j}b\quad\,\text{ mit }a,b\in \mathbb{R}\right\}$
\begin{equation}
\boxed{z=a+\text{j}\,b}
\end{equation}
Die Länge des Zeigers heisst Betrag $\Big\vert z\Big\vert$ der komplexen Zahl $z$.
\begin{equation}
\boxed{\Big\vert z\Big\vert=\sqrt{a^2+b^2}}
\end{equation}
Zwei komplexe Zahlen $z_1$ und $z_2$ sind genau gleich, $z_1=z_2$, wenn ihre Bildpunkte zusammenfallen, d.h. $a_1=a_2$ und $b_1=b_2$ ist.
\newline\newline
Die zu $z$ konjugiert komplexe Zahl $\overline{z}$ liegt spiegelsymmetrisch zur rellen Achse. Die komplexe Zahl $z$ und ihre konjugiert $\overline{z}$ unterscheiden sich in ihrem Imaginärteil durch das Vorzeichen.
\begin{equation}
\boxed{\overline{z}=\overline{a+\text{j}b}=a-\text{j}b}
\end{equation}
Somit gelten folgende Ausdrücke
\begin{enumerate}[$(i)$]
\item $\text{Re}\left(\overline{z}\right)=\text{Re}\left(z\right)=a$
\item $\Big\vert \overline{z}\Big\vert=\Big\vert z\Big\vert$
\item $\overline{\left(\overline{z}\right)}=z$
\item Gilt $\overline{z}=z$, so ist $z$ reell
\end{enumerate}
%%%%%%%%%%%%%%%%%%%%%%%%%%%%%%%%%%%%%%%%%%%%%%%%%%%%%%%%%%%%%%%%%%%%%%%%%%%%%%%%%
\subsection{Die Polarform}
In der Polarform erfolgt die Darstellung einer komplexen Zahl durch die Polarkoordinaten $r$ und $\varphi$. Man beschränkt sich auf $\left[0,2\pi\right)$.
\begin{equation}
\boxed{r=\sqrt{a^2+b^2}}\quad \boxed{\varphi=\arctan\left(\dfrac{b}{a}\right)+\left\{\begin{array}{l}+0,\quad \text{(I)}\\+\pi,\quad \text{(II, III)}\\+2\pi,\quad \text{(IV)}\end{array}\right\}}
\end{equation}
Die \textbf{goniometrische Form} besteht aus dem Betrag und dem Argument von $z$. Die entsprechende konjugiert komplexe Zahl ist
\begin{equation}
\boxed{z=r\cdot \Big(\cos\left(\varphi\right)+\text{j}\sin\left(\varphi\right)\Big)}\quad
\boxed{\overline{z}=r\cdot \Big(\cos\left(\varphi\right)-\text{j}\sin\left(\varphi\right)\Big)}
\end{equation}
Die \textbf{Exponentialform} besteht aus dem Betrag und dem Argument von $z$. Die entsprechende konjugierte komplexe Zahl ist
\begin{equation}
\boxed{z=r\cdot e^{\text{j}\varphi}}\quad
\boxed{\overline{z}=r\cdot e^{-\text{j}\varphi}}
\end{equation}
Somit ergeben sich folgende Beziehungen
\begin{equation}
\boxed{e^{\text{j}\varphi}=\cos\left(\varphi\right)+\text{j}\sin\left(\varphi\right)}\quad \boxed{e^{-\text{j}\varphi}=\cos\left(\varphi\right)-\text{j}\sin\left(\varphi\right)}
\end{equation}
\section{Grundrechenarten für komplexe Zahlen}
\subsection{Addition und Subtraktion komplexer Zahlen}
Zwei komplexe Zahlen werden addiert bzw. subtrahiert, indem man ihre Real- und Imaginärteil (jeweils für sich getrennt) addiert bzw. subtrahiert. Addition und Subtraktion sind nur in der karteischen Form durchführbar. Geometrisch entspricht dem Parallelogramregel der Vektorrechnung.
\begin{equation}
\boxed{\begin{array}{lll}
z_1\pm z_2&=&\left(a_1+ \text{j}b_1\right)\pm \left(a_2+ \text{j}b_2\right)\\
&=&\left(a_1\pm a_2\right)+\text{j}\left(b_1\pm b_2\right)
\end{array}}
\end{equation}
\begin{enumerate}[$(i)$]
\item $z_1+z_2=z_2+z_1$
\item $z_1+\left(z_2+z_3\right)=\left(z_1+z_2\right)+z_3$
\end{enumerate}
\subsection{Multiplikation komplexer Zahlen}
Die Multiplikation in \textbf{kartesische Form} erfolgt, indem jeder Summand der ersten Klammer mit jedem Summand der zweiter Klammer unter Beachtung von $\text{j}^2=-1$ multipliziert. Geometrisch erfolgt eine Drehung im Gegenuhrzeigersinn falls $\varphi_2>0$ und im Uhrzeigersinn falls $\varphi_2<0$ und eine Streckung um den Faktor $r_2$.
\begin{equation}
\boxed{\begin{array}{lll}
z_1\cdot z_2&=&\left(a_1+\text{j}b_1\right)\cdot \left(a_2+\text{j}b_2\right)\\
&=&\left(a_1a_2-b_1b_2\right)+\text{j}\left(a_1b_2+a_2b_1\right)
\end{array}}
\end{equation}
Die Multiplikation in \textbf{Polarform} erfolgt, indem man ihre Beträge multipliziert und die Argumente addiert.
\begin{equation}
\boxed{\begin{array}{lll}
z_1\cdot z_2&=&\Big[r_1\Big(\cos\left(\varphi_1\right)+\text{j}\sin\left(\varphi_1\right)\Big)\Big]\cdot r_2\Big(\cos\left(\varphi_2\right)+\text{j}\sin\left(\varphi_2\right)\Big)\Big]\\
&=&\left(r_1r_2\right)\cdot \Big[\cos\left(\varphi_1+\varphi_2\right)+\text{j}\sin\left(\varphi_1+\varphi_2\right)\Big]
\end{array}}
\end{equation}
\begin{equation}
\boxed{\begin{array}{lll}
z_1\cdot z_2&=&\Big(r_1\cdot e^{\text{j}\varphi_1}\Big)\cdot \Big(r_2\cdot e^{\text{j}\varphi_2}\Big)\\
&=&\left(r_1r_2\right)\cdot e^{\text{j}\left(\varphi_1+\varphi_2\right)}
\end{array}}
\end{equation}
\begin{enumerate}[$(i)$]
\item $z_1z_2=z_2z_1$
\item $z_1\left(z_2z_3\right)=\left(z_1z_2\right)z_3$
\item $z_1\left(z_2+z_3\right)=z_1z_2+z_1z_3$
\item $z\cdot \overline{z}=a^2+b^2=\Big\vert z\Big\vert^2\Longrightarrow \Big\vert z\Big\vert=\sqrt{z\cdot \overline{z}}$
\item $\text{j}^{4n}=1,\quad \text{j}^{4n+1}=\text{j},\quad \text{j}^{4n+2}=-1,\quad \text{j}^{4n+3}=-\text{j}\quad \left(n\in \mathbb{Z}\right)$
\end{enumerate}
\subsection{Division komplexer Zahlen}
Zähler und Nenner des Quotienten  werden zunächst mit dem konjugiert komplexen Nenner, d.h. der Zahl $\overline{z}_2$ multipliziert, dadurch wird der Nenner reell. Geometrisch erfährt $z_1$ eine Zurückdrehung im Uhrzeigersinn falls $\varphi_2>0$ und im Gegenuhrzeigersinn falls $\varphi_2<0$ und eine Streckung um den Faktor $1/r_2$.
\begin{equation}
\boxed{
\begin{array}{lll}
\dfrac{z_1}{z_2}&=&\dfrac{a_1+\text{j}b_1}{a_2+\text{j}b_2}=\dfrac{\left(a_1+\text{j}b_1\right)\cdot \left(a_2+\text{j}b_2\right)}{\left(a_2+\text{j}b_2\right)\cdot \left(a_2+\text{j}b_2\right)}\\\\
&=&\dfrac{a_1a_2+b_1b_2}{a_2^2+b_2^2}+\text{j}\dfrac{a_2b_1-a_1b_2}{a_2^2+b_2^2}
\end{array}
}
\end{equation}
\begin{equation}
\boxed{
\begin{array}{lll}
\dfrac{z_1}{z_2}&=&\dfrac{r_1\Big[\cos\left(\varphi_1\right)+\text{j}\sin\left(\varphi_1\right)\Big]}{r_2\Big[\cos\left(\varphi_2\right)+\text{j}\sin\left(\varphi_2\right)\Big]}\\
&=&\left(\dfrac{r_1}{r_2}\right)\cdot \Big[\cos\left(\varphi_1-\varphi_2\right)+\text{j}\sin\left(\varphi_1-\varphi_2\right)\Big]
\end{array}
}
\end{equation}
\begin{equation}
\boxed{\begin{array}{lll}
\dfrac{z_1}{z_2}&=&\dfrac{r_1\cdot e^{\text{j}\varphi_1}}{r_2\cdot e^{\text{j}\varphi_2}}=\left(\dfrac{r_1}{r_2}\right)\cdot e^{\text{j}\left(\varphi_1-\varphi_2\right)}\end{array}}
\end{equation}
\begin{enumerate}[$(i)$]
\item $\dfrac{1}{z}=\dfrac{1}{r\cdot e^{\text{j}\varphi}}=\left(\dfrac{1}{r}\right)\cdot e^{-\text{j}\varphi}$
\item $\dfrac{1}{z}=\dfrac{1}{a+\text{j}b}=\dfrac{a}{a^2+b^2}-\text{j}\dfrac{b}{a^2+b^2}$
\item $\dfrac{1}{\text{j}}=-\text{j}$
\end{enumerate}
\section{Potenzieren komplexer Zahlen}
\subsection{Die kartesische Form}
In \textbf{kartesischer Form} ist das Potenzieren komplexer Zahlen nach dem binomischen Lehrsatz
\begin{equation}
\boxed{z^n=\left(a+\text{j}b\right)^n=a^n+\text{j}\displaystyle \binom{n}{1}a^{n-1}b+\text{j}^2\displaystyle\binom{n}{2}a^{n-2}b^2+\dotso+\text{j}^nb^n}
\end{equation}
\subsection{Die Polarform}
In \textbf{Polarform} wird der Betrag in die $n$-te Potenz erhebt und ihr Argument mit dem Exponenten $n$ multipliziert
\begin{equation}
\boxed{
\begin{array}{lll}
z^n&=&\Big[r\cdot \Big(\cos\left(\varphi\right)+\text{j}\sin\left(\varphi\right)\Big)\Big]^n\\\\
&=&r^n\cdot \Big[\cos\left(n\varphi\right)+\text{j}\sin\left(n\varphi\right)\Big]
\end{array}
}
\end{equation}
\begin{equation}
\boxed{
\begin{array}{lll}
z^n&=&\Big[r\cdot e^{\text{j}\varphi}\Big]^n=r^n\cdot e^{\text{j}n\varphi}
\end{array}
}
\end{equation}
\section{Radizieren komplexer Zahlen}
Eine komplexe Zahl $z$ heisst eine $n$-te Wurzel aus $a$, wenn sie der algebraischen Gleichung $z^n=a$ genügt $\left(a\in\mathbb{C};\quad n\in \mathbb{N}^*\right)$.
\newline\newline
Eine algebraische Gleichung $n$-ten Grades von folgendem Typ besitzt in der Menge $\mathbb{C}$ der komplexen Zahlen stets genau $n$ Lösungen. Bei ausschliesslich reellen koeffizienten $a_i$ treten komplexe Lösungen immer paarweise in Form konjugiert komplexer Zahlen auf.
\begin{equation}
\boxed{a_nz^n+a_{n-1}z^{n-1}+\dotso + a_1z+a_0=0,\quad \left(a_i:\,\text{reell oder komplex}\right)}
\end{equation}
Die $n$ Wurzeln der Gleichung $z^n=a=a_0\cdot e^{\text{j}\varphi}$ mit $a_0>0$ und $n\in \mathbb{N}^*$ lauten
\begin{equation}
\boxed{z_k=\sqrt[n]{a_0}\cdot \Big[\cos\left(\dfrac{\alpha+k\,2\pi}{n}\right)+\text{j}\sin\left(\dfrac{\alpha+k\,2\pi}{n}\right)\Big],\quad \left(k=0,\,1,\dotso, n-1\right)}
\end{equation}
Der Hauptwert ist bei $k=0$ und für $k=1, 2, \dotso, n-1$ erhält man Nebenwerte. Die Winkel können auch mîm Gradmass angegeben werden.
\newline\newline
geometrisch liegen die zugehörige Bildpunkte aufdem Mittelpunktskreis mit dem Radius $R=\sqrt[n]{a_0}$ und bilden die Ecken eines regelmässigen $n$-Ecks.
\newline\newline
Die $n$ Lösungen der Gleichung $z^n=1$ heissen $n$-te Einheitswurzeln und lauten
\begin{equation}
\boxed{z^n=1\Longrightarrow z_k=\cos\left(\dfrac{k\,2\pi}{n}\right)+\text{j}\sin\left(\dfrac{k\,2\pi}{n}\right)=e^{\text{j}\dfrac{k\,2\pi}{n}}}
\end{equation}
\section{Logarithmieren komplexer Zahlen}
Der natürliche Logarithmus einer komplexen Zahl
\begin{equation}
\boxed{z=r\cdot e^{\text{j}\varphi}=r\cdot e^{\text{j}\left(\varphi+k\,2\pi\right)},\quad \left(0\leq \varphi< 2\pi;\,k\in \mathbb{Z}\right)}
\end{equation}
ist unendlich vieldeutig, denn der Hauptwert des Winkels wird häufig auch im Intervall $-\pi<\varphi < \pi$
\begin{equation}
\boxed{\ln\left(z\right)=\ln\left(r\right)+\text{j}\left(\varphi+k\,2\pi\right),\quad \left(k\in \mathbb{Z}\right)}
\end{equation}
\section{Ortskurven}
\subsection{Komplexwertige Funktion einer reellen Variablen}
Die von einem reellen Parameter $t$ abhängige komplexe Zahl heisst komplexwertige Funktion $z\left(t\right)$ der reellen Variablen $t$.
\begin{equation}
\boxed{z=z\left(t\right)=x\left(t\right)+\text{j}y\left(t\right),\quad \left(a\leq z\leq b\right)}
\end{equation}
\subsection{Ortskurve einer parameterabhängigen komplexen Zahl}
Die von einem parameterabhängigen komplexen Zeiger $\underline{z}=\underline{z}\left(t\right)$ in der Gaussschen Zahlenebene beschriebene Bahn heisst Ortskurve
\begin{equation}
\boxed{\underline{z}\left(t\right)=x\left(t\right)+\text{j}y\left(t\right)}
\end{equation}
\subsection{Inversion einer Ortskurve}
Der Übergang von einer komplexen Zahl $z\neq 0$ zu ihrem Kehrwert $w=1/z$ heisst Inversion. Vorzeichenwechsel im Argument, Kehrwertbildung des Betrages von $z$. Geometrisch wird der Zeiger an der reellen Achse gespiegelt und dann gestreckt mit Faktor $1/r^2$.
\begin{equation}
\boxed{z=r\cdot e^{\text{j}\varphi}\rightarrow w=\dfrac{1}{z}=\left(\dfrac{1}{r}\right)\cdot e^{-\text{j}\varphi}}
\end{equation}
\begin{enumerate}[$(i)$]
\item Der Punkt mit dem kleinsten Abstand vom Nullpunkt führt zu dem Bildpunkt mit dem grössten Abstand und umgekehrt.
\item Ein Punkt oberhalb der reellen Achse führt zu einem Bildpunkt unterhalb der reellen Achse und umgekehrt.
\item Eine Gerade durch den Nullpunkt auf der $z$-Ebene erzeugt eine Gerade durch den Nullpunkt auf der $w$-Ebene.
\item Eine Gerade, die nicht durch den Nullpunkt auf der $z$-Ebene geht, erzeugt einen Kreis durch den Nullpunkt auf der $w$-Ebene.
\item Ein Mittelpunktskreis auf der $z$-Ebene erzeugt einen Mittelpunktskreis auf der $w$-Ebene.
\item Ein Kreis durch den Nullpunkt auf der $z$-Ebene erzeugt eine Gerade, die nicht durch den Nullpunkt verläuft auf der $w$-Ebene.
\item Ein Kreis, der nicht durch den Nullpunkt auf der $z$-Ebene verläuft erzeugt einen Kreis, der nicht durch den Nullpunkt verläuft, auf der $w$-Ebene.
\end{enumerate}
\section{Komplexe Funktionen}
\subsection{Definition einer komplexen Funktion}
Unter der komplexen Funktion versteht man eine Vorschrift, de jeder komplexen Zahl $z\in D$ genau eine komplexe Zahl $w\in W$ zuordnet. Symbolische Schreibweise sind $w=f\left(z\right)$. $D$ und $W$ sind Teilmengen von $\mathbb{C}$.
\subsection{Definitionsgleichungen elementarer Funktionen}
\subsubsection{Trigonometrische Funktionen}
\begin{equation}
\boxed{\sin\left(z\right)=z-\dfrac{z^3}{3!}+\dfrac{z^5}{5!}-\dotso}
\quad
\boxed{\cos\left(z\right)=1-\dfrac{z^2}{4!}+\dfrac{z^4}{4!}-\dotso}
\end{equation}
\begin{equation}
\boxed{\tan\left(z\right)=\dfrac{\sin\left(z\right)}{\cos\left(z\right)}}
\quad
\boxed{\cot\left(z\right)=\dfrac{\cos\left(z\right)}{\sin\left(z\right)}=\dfrac{1}{\tan\left(z\right)}}
\end{equation}
\subsubsection{Hyperbelfunktionen}
\begin{equation}
\boxed{\sinh\left(z\right)=z+\dfrac{z^3}{3!}+\dfrac{z^5}{5!}-\dotso}
\quad
\boxed{\cosh\left(z\right)=1+\dfrac{z^2}{4!}+\dfrac{z^4}{4!}-\dotso}
\end{equation}
\begin{equation}
\boxed{\tanh\left(z\right)=\dfrac{\sinh\left(z\right)}{\cosh\left(z\right)}}
\quad
\boxed{\coth\left(z\right)=\dfrac{\cosh\left(z\right)}{\sinh\left(z\right)}=\dfrac{1}{\tanh\left(z\right)}}
\end{equation}
\subsubsection{Exponentialfunktion}
\begin{equation}
\boxed{e^z=1+\dfrac{z}{1!}+\dfrac{z^2}{2!}+\dfrac{z^3}{3!}+\dotso}
\end{equation}
\subsection{Beziehungen}
\subsubsection{Eulersche Formeln}
\begin{equation}
\boxed{e^{\text{j}x}=\cos\left(x\right)+\text{j}\sin\left(x\right)}
\quad
\boxed{e^{-\text{j}x}=\cos\left(x\right)-\text{j}\sin\left(x\right)}
\end{equation}
\subsubsection{Beziehungen trigonometrischen und komplexen $e$-Funktion}
\begin{equation}
\boxed{\sin\left(x\right)=\dfrac{1}{2\text{j}}\left(e^{\text{j}x}-e^{-\text{j}x}\right)}
\quad
\boxed{\cos\left(x\right)=\dfrac{1}{2}\left(e^{\text{j}x}+e^{-\text{j}x}\right)}
\end{equation}
\begin{equation}
\boxed{\tan\left(x\right)=-\text{j}\dfrac{\left(e^{\text{j}x}-e^{-\text{j}x}\right)}{\left(e^{\text{j}x}+e^{-\text{j}x}\right)}}
\quad
\boxed{\cot\left(x\right)=\text{j}\dfrac{\left(e^{\text{j}x}+e^{-\text{j}x}\right)}{\left(e^{\text{j}x}-e^{-\text{j}x}\right)}}
\end{equation}
\subsubsection{Beziehungen trigonometrischen und Hyperbelfunktionen}
\begin{equation}
\boxed{\sin\left(\text{j}x\right)=\text{j}\cdot \sinh\left(x\right)}\quad \boxed{\cos\left(\text{j}x\right)=\cosh\left(x\right)}\quad \boxed{\tan\left(\text{j}x\right)=\text{j}\cdot \tanh\left (x\right)}
\end{equation}
\begin{equation}
\boxed{\sinh\left(\text{j}x\right)=\text{j}\cdot \sin\left(x\right)}\quad \boxed{\cosh\left(\text{j}x\right)=\cos\left(x\right)}\quad \boxed{\tanh\left(\text{j}x\right)=\text{j}\cdot \tan\left (x\right)}
\end{equation}
\subsubsection{Additionstheoreme der trigonometrischen und Hyperbelfunktionen}
\begin{equation}
\boxed{\begin{array}{lll}
\sin\left(x\pm \text{j}y\right)&=&\sin\left(x\right)\cdot \cosh\left(y\right)\pm \text{j}\cdot \cos\left(x\right)\cdot \sinh\left(y\right)\\
\cos\left(x\pm \text{j}y\right)&=&\cos\left(x\right)\cdot \cosh\left(y\right)\mp \text{j}\cdot \sin\left(x\right)\cdot \sinh\left(y\right)\\
\tan\left(x\pm \text{j}y\right)&=&\dfrac{\sin\left(2x\right)\pm \text{j}\cdot \sinh\left(2x\right)}{\cos\left(2x\right)+\cosh\left(2x\right)}\\
\end{array}}
\end{equation}
\begin{equation}
\boxed{\begin{array}{lll}
\sinh\left(x\pm \text{j}y\right)&=&\sinh\left(x\right)\cdot \cos\left(y\right)\pm \text{j}\cdot \cosh\left(x\right)\cdot \sin\left(y\right)\\
\cosh\left(x\pm \text{j}y\right)&=&\cosh\left(x\right)\cdot \cos\left(y\right)\pm \text{j}\cdot \sinh\left(x\right)\cdot \sin\left(y\right)\\
\tanh\left(x\pm \text{j}y\right)&=&\dfrac{\sinh\left(2x\right)\pm \text{j}\cdot \sin\left(2x\right)}{\cosh\left(2x\right)+\cos\left(2x\right)}\\
\end{array}}
\end{equation}
\subsubsection{Arkus- und Areafunktionen}
\begin{equation}
\boxed{\arcsin\left(\text{j}x\right)=\text{j}\cdot \arsinh\left(x\right)}\quad \boxed{\arccos\left(\text{j}x\right)=\text{j}\cdot \text{Arcosh}\left(x\right)}
\end{equation}
\begin{equation}
\boxed{\arsinh\left(\text{j}x\right)=\text{j}\cdot \arcsin\left(x\right)}\quad \boxed{\text{Arcosh}\left(\text{j}x\right)=\text{j}\cdot \arccos\left(x\right)}
\end{equation}
\begin{equation}
\boxed{\arctan\left(\text{j}x\right)=\text{j}\cdot \text{Artanh}\left(x\right)}\quad \boxed{\text{Artanh}\left(\text{j}x\right)=\text{j}\cdot \arctan\left(x\right)}
\end{equation}
\section{Komplexe Zahlen in MATLAB}
\subsection{Definition der imaginäre Einheit}
Folgende Befehle sind Berechnungen der komplexen Zahlen
\begin{enumerate}[$\texttt{>}\texttt{>}$]
\item {\color{red}\texttt{sqrt(-1) => 0.0000 + 1.0000i}}
\item {\color{red}\texttt{i\^\,5 => 0.0000 + 1.0000i}}
\item {\color{red}\texttt{i\^\,10 => -1}}
\end{enumerate}
\subsection{Operationen mit komplexen Zahlen}
\begin{enumerate}[$\texttt{>}\texttt{>}$]
\item {\color{red}\texttt{double(solve('x\^\,2+x+1')) => -0.5000 + 0.8660i, -0.5000-0.8660i}}
\item {\color{red}\texttt{z1=3+4*i => 3.0000 + 4.0000i}}
\item {\color{red}\texttt{z2=complex(3,4) => 3.0000 + 4.0000i}}
\item {\color{red}\texttt{real(z1) => 3}}
\item {\color{red}\texttt{imag(z1) => 4}}
\item {\color{red}\texttt{(3+4*i)+(1+2*i)/(-4+3*i) => 3.0800 + 3.5600i}}
\item {\color{red}\texttt{conj(1+2*i) => 1.0000 - 2.0000i}}
\item {\color{red}\texttt{(1+2*i)*conj(1+2*i) => 5}}
\end{enumerate}
\subsection{Gauss'sche Zahlenebene}
\begin{enumerate}[$\texttt{>}\texttt{>}$]
\item {\color{red}\texttt{plot([1+2*i,2-3*i,4-5*i,-3+i,i, -6, -3-2*i], 'r*')}}
\item {\color{red}\texttt{z=-2+i = -2.0000 + 1.0000i}}
\item {\color{red}\texttt{betrag=abs(z) => betrag =  2.2361}}
\item {\color{red}\texttt{sym=abs(z) => sym =  2.2361}}
\item {\color{red}\texttt{winkel=angle(z) => winkel =  2.6779}}
\item {\color{red}\texttt{winkel\_grad=winkel/pi*180 => winkel\_grad = 153.4349}}
\end{enumerate}
\subsection{Satz von Moivre}
\begin{enumerate}[$\texttt{>}\texttt{>}$]
\item {\color{red}\texttt{(-sqrt(3)-i)\^\,10 => 5.1200e+02 - 8.8681e+02i}}
\item {\color{red}\texttt{abs((-sqrt(3)-i)\^\,10) => 1.0240e+03}}
\item {\color{red}\texttt{angle((-sqrt(3)-i)\^\,10) => -1.0472}}
\item {\color{red}\texttt{(-sqrt(3)-i)\^\,(1/3) => 0.8099 - 0.9652i}}
\item {\color{red}\texttt{double(solve('z\^\,3-(-sqrt(3)-i)')) => 0.8099 - 0.9652i, -1.2408 - 0.2188i, 0.4309 + 1.1839i}}
\item {\color{red}\texttt{compass(double(solve('z\^\,3-(-sqrt(3)-i)')))}}
\end{enumerate}
\subsection{Eulersche Formel}
\begin{enumerate}[$\texttt{>}\texttt{>}$]
\item {\color{red}\texttt{log(-1) = 0.0000 + 3.1416i}}
\item {\color{red}\texttt{log(-10*i) = 2.3026 - 1.5708i}}
\item {\color{red}\texttt{log2(sqrt(2)-sqrt(2)*i) = 1.0000 - 1.1331i}}
\item {\color{red}\texttt{exp(1+i*pi) = -2.7183 + 0.0000i}}
\item {\color{red}\texttt{2\^i = 0.7692 + 0.6390i}}
\item {\color{red}\texttt{(-i)\^\,(-i) = 0.2079}}
\end{enumerate}
\section{Anwendungen in der Schwingungslehre}
\subsection{Darstellung einer harmonischen Schwingung durch einen rotierenden komplexen Zeiger}
Eine harmonische Schwingung vom Typ $y=A\cdot\sin\left(\omega t+\varphi\right)$ mit $A>0$ und $\omega>0$ lässt sich in der Gaussschen Zahlenebene durch einen mit der Winkelgeschwin-digkeit $\omega$ um den Nullpunkt rotierenden komplexen zeitabhängigen Zeiger der Länge $A$ darstellen
\begin{equation}
\boxed{\underline{y}\left(t\right)=A\cdot e^{\text{j}\left(\omega t+\varphi\right)}=\underline{A}\cdot e^{\text{j}\omega t}}\quad \boxed{\underline{A}=A\cdot e^{\text{j}\varphi}}
\end{equation}
Die Drehung erfolgt im Gegenuhrzeigersinn. Die komplexe Schwingungsamplitude $\underline{A}$ beschreibt dabei die Anfangslage des Zeigers $\underline{y}\left(t\right)$ zur Zeit $t=0$, d.h. es ist $\underline{y}\left(0\right)=\underline{A}$
\newline\newline
Eine in der Kosinusform vorliegende Schwingung lässt sich wie folgt ein die Sinusform umschreiben
\begin{equation}
\boxed{y=A\cdot \cos\left(\omega t+\varphi\right)=A\cdot \sin\left(\omega t+\varphi+\dfrac{\pi}{2}\right)=A\cdot \sin\left(\omega t+\varphi^*\right)}
\end{equation}
Der Nullphasenwinkel beträgt somit $\varphi^*=\varphi+\dfrac{\pi}{2}$, d.h. der Zeiger ist um $90^{\circ}$ vorzudrehen.
\subsection{Ungestörte überlagerung gleichfrequenter harmonischer Schwingungen}
Durch ungestörte Überlagerung der gleichfrequenten harmonische Schwingungen
\begin{equation}
\boxed{y_1=A_1\cdot \sin\left(\omega t+\varphi_1\right)\quad \text{und}\quad y_2=A_2\cdot \sin\left(\omega t+\varphi_2\right)}
\end{equation}
entsteht nach dem Superpositionsprinzip der Physik eine resultierende Schwingung mit derselben Frequenz, wobei $A>0$, $\omega>0$, $A_1>0$ und $A_2>0$
\begin{equation}
\boxed{y=y_1+y_2=A_1\cdot \sin\left(\omega t+\varphi_1\right)+A_2\cdot \sin\left(\omega t+\varphi_2\right)=A\cdot \sin\left(\omega t+\varphi\right)}
\end{equation}
Die Berechnung der Schwingungsamplitude $A$ und des Phasenwinkels $\varphi$ erfolgt durch Übergang von der reellen zur komplexen Form, danach durch Addition der komplexen Amplituden und Elongationen und zum Schluss Rücktransformation aus der komplexen in die reelle Form.
\begin{equation}
\boxed{y_1=A_1\cdot \sin\left(\omega t + \varphi_1\right)\rightarrow \underline{y}_1=\underline{A}_1\cdot e^{\text{j}\omega t}}\quad \boxed{\underline{A}_1=A_1\cdot e^{\text{j}\varphi_1}}
\end{equation}
\begin{equation}
\boxed{y_2=A_2\cdot \sin\left(\omega t + \varphi_2\right)\rightarrow \underline{y}_2=\underline{A}_2\cdot e^{\text{j}\omega t}}\quad \boxed{\underline{A}_2=A_2\cdot e^{\text{j}\varphi_2}}
\end{equation}
\begin{equation}
\boxed{\underline{A}=\underline{A}_1+\underline{A}_2=A\cdot e^{\text{j}\varphi}}
\end{equation}
\begin{equation}
\boxed{\underline{y}=\underline{y}_1+\underline{y}_2=\underline{A}\cdot e^{\text{j}\omega t}=\underline{A}\cdot e^{\text{j}\omega t+\varphi}}
\end{equation}
\begin{equation}
\boxed{y=y_1+y_2=\text{Im}\left(\underline{y}\right)=\text{Im}\left(\underline{A}\cdot e^{\text{j}\omega t}\right)=\text{Im}\left(A\cdot e^{\text{j}\omega t+\varphi}\right)=A\cdot \sin\left(\omega t+\varphi\right)}
\end{equation}
\begin{enumerate}[$(i)$]
\item Überlagerung einer Sinusschwingung mit einer Kosinusschwingung: Letztere erst auf die Sinusform bringen.
\item Überlagerung zweier Kosinusschwingungen: Beide erst auf die Sinusform bringen oder die resultierende Schwingung ebenfalls als Kosinusschwingung darstellen, wobei bei der Rücktransformation der Realteil von $\underline{y}=A\cdot e^{\text{j}\left(\omega t + \varphi\right)}$ zu nehmen ist.
\end{enumerate}
\section{Anwendungen komplexer Zahlen in der Wechselstromtechnik}
\subsection{Harmonische Schwingungen}
Harmonische Schwingungen sind wichtige Funktionen einer unabhängigen Zeitvariablen $t$ in der Technik.
\begin{equation}
\boxed{h_{\text{sin}}\left(t\right)=A\cdot \sin\left(\omega t+\varphi_0\right)+x_0=A\cdot \sin\left(\omega\cdot \left(t-t_0\right)\right)+x_0}
\end{equation}
\begin{equation}
\boxed{h_{\text{cos}}\left(t\right)=A\cdot \cos\left(\omega t+\varphi_0\right)+x_0=A\cdot \cos\left(\omega\cdot \left(t-t_0\right)\right)+x_0}
\end{equation}
\begin{enumerate}[$(a)$]
\item Sei $A$ die Amplitude (maximale Auslenkung aus der Ruhelage)
\item Sei $x_0$ der lineare Mittelwert (Verschiebung der Kurve in vertikaler Richtung)
\item Sei $\omega$ die Kreisfrequenz der Schwingung (Anzahl Schwingungen in der Zeitspanne $2\pi$)
\item Sei $f$ die Frequenz der Schwingung (Anzahl Schwingungen in der Zeitspanne 1). Es gilt $f=\omega/2\pi$
\item Sei $T$ der Periodendauer einer Schwingung (Zeitdauer bis sich die Schwingung wiederholt). Es gilt $T=1/f=2\pi/\omega$
\item Sei $\varphi_0$ die Phasenverschiebung (Winkeloffset zum Zeitpunkt $t=0$)
\item Sei $t_0$ die zeitliche Verschiebung der Schwingung (Zeitpunkt für den Start der Grundschwingung Sinus oder Kosinus)
\end{enumerate}
\subsection{Harmonische Schwingungen und komplexe Zahlen}
Denkt man sich ein rotierender Stab der Länge $A$, dessen eines Ende im Punkt $\left(0,x_0\right)$ befestigt ist, welcher mit eine rfesten Winkelgeschwindigkeit $\omega$ im gegenuhrzeigersinn rotiert und betrachtet man die Schattenlänge des Stabes wenn dieser von der linken Seite angestrahlt wird, so erhält man eine harmonische Schwingung.
\newline\newline
Mit Hilfe der Exponentialform kann ein solcher rotierender Stab (in der Gauss'schen Zahlenebene) wie folgt beschrieben werden
\begin{equation}
\boxed{
\begin{array}{lll}
z_S\left(t\right)&=&\sqrt{A^2-\left(h_{\text{sin}}\left(t\right)-x_0\right)^2}+\text{j}\cdot h_{\text{sin}}\left(t\right)\\
&=&\text{j}x_0+Ae^{\text{j}\left(\omega t+\varphi_0\right)}
\end{array}
}
\end{equation}
\begin{equation}
\boxed{
\begin{array}{lll}
z_C\left(t\right)&=&-\sqrt{A^2-\left(h_{\text{cos}}\left(t\right)-x_0\right)^2}+\text{j}\cdot h_{\text{cos}}\left(t\right)\\
&=&\text{j}x_0+Ae^{\text{j}\left(\omega t+\varphi_0\right)}\\
&=&\text{j}x_0+Ae^{\text{j}\left(\omega t+\varphi_0+\pi/2\right)}
\end{array}
}
\end{equation}
Im weiteren betrachtet man harmonische Schwingungen ohne linearen Mittelwert.
\begin{equation}
\boxed{
\begin{array}{lll}
z_{S_0}\left(t\right)&=&\sqrt{A^2-h^2_{\text{sin}}\left(t\right)}+\text{j}\cdot h_{\text{sin}}\left(t\right)\\
&=&A\cdot e^{\text{j}\left(\omega t+\varphi_0\right)}
\end{array}
}
\end{equation}
\begin{equation}
\boxed{
\begin{array}{lll}
z_{C_0}\left(t\right)&=&-\sqrt{A^2-h^2_{\text{cos}}\left(t\right)}+\text{j}\cdot h_{\text{cos}}\left(t\right)\\
&=&A\cdot e^{\text{j}\left(\omega t+\varphi_0+\pi/2\right)}
\end{array}
}
\end{equation}
Die Schattenlänge ist nun gleich dem Imaginärteil des rotierenden Stabes
\begin{equation}
\boxed{h_{\text{sin}}\left(t\right)=\text{Im}\left(z_{S_0}\left(t\right)\right)=\text{Im}\left(Ae^{\text{j}\left(\omega t+\varphi_0\right)}\right)}
\end{equation}
\begin{equation}
\boxed{h_{\text{cos}}\left(t\right)=\text{Im}\left(z_{C_0}\left(t\right)\right)=\text{Im}\left(Ae^{\text{j}\left(\omega t+\varphi_0+\pi/2\right)}\right)}
\end{equation}
Mit den Beziehungen $\text{Re}\left(z\right)=\dfrac{z+\overline{z}}{2}$ und $\text{Im}\left(z\right)=\dfrac{z-\overline{z}}{2\text{j}}$ lassen sich die Schwingungen auch wie folgt beschreiben
\begin{equation}
\boxed{A\cdot \sin\left(\omega t+\varphi_0\right)=\dfrac{z_{S_0}\left(t\right)-\overline{z_{S_0}\left(t\right)}}{2\text{j}}=\dfrac{A\cdot e^{\text{j}\left(\omega t+\varphi_0\right)}-Ae^{-\text{j}\left(\omega t+\varphi_0\right)}}{2\text{j}}}
\end{equation}
\begin{equation}
\boxed{A\cdot \cos\left(\omega t+\varphi_0\right)=\dfrac{z_{C_0}\left(t\right)-\overline{z_{C_0}\left(t\right)}}{2\text{j}}=\dfrac{A\cdot e^{\text{j}\left(\omega t+\varphi_0\right)}+Ae^{-\text{j}\left(\omega t+\varphi_0\right)}}{2}}
\end{equation}
Die eingefügten rotierenden Zeiger sind eine äquivalente Beschreibungsform für die harmonischen Schwingung. Neben den rotierenden Zeigern arbeitet man auch häufig mit nicht rotierenden Zeigern. Dabei ist das Ziel, die Zeitabhängigkeit aus der Beschreibung zu eliminieren. Dazu kann z.B. die folgende Transformation verwendet werden.
\begin{equation}
\boxed{h_{\text{sin}}\left(t\right)\longrightarrow H_{\text{sin}}=z_{S_0}\left(t\right)\cdot e^{-\text{j}\omega t}=A\cdot e^{\text{j}\left(\omega t+\varphi_0\right)}\cdot e^{-\text{j}\omega t}=A\cdot e^{\text{j}\varphi_0}}
\end{equation}
\begin{equation}
\boxed{h_{\text{cos}}\left(t\right)\longrightarrow H_{\text{cos}}=z_{C_0}\left(t\right)\cdot e^{-\text{j}\omega t}=A\cdot e^{\text{j}\left(\varphi_0+\dfrac{\pi}{2}\right)}}
\end{equation}
Bei dieserTransformation ist die neue Beschreibungsform nicht das Gleiche wie die ursprüngliche harmonische Schwingung. Das zeitabhängige Signal (Zeitbereich) wird durch die Transformation auf eine komplexe Zahl (nicht mehr zeitabhängig) transformiert. Man spricht von einer Beschreibung in der Modellwelt. Zu dieser Transformation gibt es auch eine Rücktransformation. Eine Anwendung dieser Transformation ist die \textbf{Überlagerung} gleichfrequenter harmonischer Schwingungen.
\begin{equation}
\boxed{h_{\text{sin}}\left(t\right)\longrightarrow H_{\text{sin}}=\text{Im}\left(H_{\text{sin}}e^{\text{j}\omega t}\right)=\text{Im}\left(A\cdot e^{\text{j}\varphi_0}e^{\text{j}\omega t}\right)}
\end{equation}
\begin{equation}
\boxed{h_{\text{cos}}\left(t\right)\longrightarrow H_{\text{cos}}=\text{Im}\left(H_{\text{cos}}e^{\text{j}\omega t}\right)=\text{Im}\left(A\cdot e^{\text{j}\left(\varphi_0+\dfrac{\pi}{2}\right)}\cdot e^{\text{j}\omega t}\right)}
\end{equation}
In der Elektrotechnik wird die Transformation in den Bildbereich wie folgt definiert. Ein Signal $s\left(t\right)=A\cdot \cos\left(\omega t+\varphi_0\right)$ wird wie folgt als komplexe Schwingung definiert
\begin{equation}
\boxed{
\begin{array}{lll}
\underline{s\left(t\right)}&=&A\cdot \Big[\cos\left(\omega t+\varphi_0\right)+\text{j}\cdot \sin\left(\omega t+\varphi_0\right)\Big]\\
&=&A\cdot e^{\text{j}\left(\omega t +\varphi_0\right)}
\end{array}
}
\end{equation}
Mathematisch kann dies wie folgt angegeben werden
\begin{equation}
\boxed{s\left(t\right)\longrightarrow \underline{s\left(t\right)}=s\left(t\right)+\text{j}\cdot \sqrt{A^2-s^2\left(t\right)}}
\end{equation}
\begin{equation}
\boxed{\underline{s\left(t\right)}\longrightarrow s\left(t\right)=\text{Re}\left(\underline{s\left(t\right)}\right)}
\end{equation}
Im Weiteren wird die Zeitabhängigkeit eliminiert, gleichzeitig wird noch die Zeigerlänge vom Amplitudenwert auf den Effektivwert normiert
\begin{equation}
\boxed{s\left(t\right)=A\cdot \cos\left(\omega t+\varphi_0\right)\rightarrow \underline{S}=\dfrac{A}{\sqrt{2}}e^{\text{j}\cdot \varphi_0}}
\end{equation}
\begin{equation}
\boxed{s\left(t\right)=A\cdot \sin\left(\omega t+\varphi_0\right)\rightarrow \underline{S}=\dfrac{A}{\sqrt{2}}e^{\text{j}\cdot \left(\varphi_0-\dfrac{\pi}{2}\right)}}
\end{equation}
\subsection{Elektrotechnische Grundkenntnisse}
Bei elektrischen Schaltungen interessiert man sich oft für die Spannung über den Bauteilen, den Strom in den Bauteilen und die Leistung die ein Bauteil bezieht. Dabei unterscheidet man verschiedene Situationen wie Gleichstrom- und Wechselstromtechnik und verschiedene Betrachtungsweisen wie Einschaltvorgänge oder Grössen bei eingeschwungenem stationären Zustand.
\subsubsection{Gesetze für Gleichstromtechnik}
\textbf{Ohm'sche Gesetz:} Fliesst durch einen ohmschen Widerstand der Strom $I$, so misst man über dem Widerstand eine zum Strom proportionale Spannung $U$. Der Proportionalitätsfaktor $R$ nennt man Widerstand
\begin{equation}
\boxed{U=R\cdot I}
\end{equation}
\textbf{Maschenregel:} In einer geschlossenen Masche ist die Summe der Spannungsabfälle gleich der Summe der Quellspannungen.
\begin{equation}
\boxed{\displaystyle \sum U_q=\displaystyle \sum U_{ab}}
\end{equation}
\textbf{Knotenregel:} Die Summe aller Ströme in einem Knoten ist gleich Null.
\begin{equation}
\boxed{\displaystyle \sum I_k=0}
\end{equation}
\textbf{Serienschaltung:} In Serie geschaltete Widerstände können durch einen Ersatzwiderstand ersetzt werden. DAbei ist der Widerstand das Ersatzwiderstandes gleich der Summe der einzelnen Widerstände.
\begin{equation}
\boxed{R_{\text{ers}}=\displaystyle \sum R_k}
\end{equation}
\textbf{Parallelschaltung:} Parallel geschaltete Widerstände können durch einen Ersatzwiderstand ersetzt werden. Dabei ist der Widerstand des Ersatzwiderstandes gleich dem Kehrwert der Summe der Kehrwerte der einzelnen Widerstände.
\begin{equation}
\boxed{R_{\text{par}}=\dfrac{1}{\displaystyle \sum \dfrac{1}{R_k}}}
\end{equation}
\subsubsection{Gesetze für Spule und Kondensator}
\textbf{Spule-Induktivität:} Der Spannungsabfall an einer Spule ist proportional zur Stromänderung. Der Faktor $L$ nennt man Induktivität der Spule.
\begin{equation}
\boxed{u_L\left(t\right)=L\cdot \dfrac{\text{d}}{\text{d}t}\Big[i\left(t\right)\Big]}
\end{equation}
\textbf{Kondensator-Kapazität:} Der Spannungsabfall an einem Kondensator ist proportional zur gespeicherten Ladung $Q$. Die gespeicherte Ladung ist gleich dem Integral des Stromes nach der Zeit. Der Faktor $C$ nennt man Kapazität des Kondensators.
\begin{equation}
\boxed{u_C\left(t\right)=\dfrac{1}{C} \displaystyle \int i\left(t\right)\,\text{d}t}
\end{equation}
\subsection{Berechnung mit Differentialgleichungen}
\textbf{Spule, Widerstand, Gleichspannungsquelle:} Eine Spule und einen ohmschen Widerstand werden in Serie an eine Gleichspannungsquelle angeschlossen und den Strom und die Spannungsabfälle an den beiden Bauteilen bestimmt. In der gegebenen Schaltung liegt eine geschlossene Masche vor und daher ist die Summe der Spannungsabfälle gleich der Summe der Quellspannungen. Dies ist eine lineare Differentialgleichung erster Ordnung mit konstanten Koeffizienten.
\begin{equation}
\boxed{u_R\left(t\right)+u_L\left(t\right)=R\cdot i\left(t\right)+L\cdot \dfrac{\text{d}}{\text{d}t}\Big[i\left(t\right)\Big]=U_q}
\end{equation}
\begin{equation}
\boxed{i\left(t\right)=\dfrac{U_q}{R}\left(1-e^{-\dfrac{Rt}{L}}\right)}
\end{equation}
\begin{equation}
\boxed{u_R\left(t\right)=U_q\left(1-e^{-\dfrac{Rt}{L}}\right)}
\end{equation}
\begin{equation}
\boxed{u_L\left(t\right)=U_q\left(e^{-\dfrac{Rt}{L}}\right)}
\end{equation}
Liegt eine Gleichspannung an einer Schaltung, so berechnet man mit der Differentialgleichung das Einschaltverhalten, d.h. den Übergang von einem stationären Zustand zu einem neuen stationären Zustand. So hat man vor dem Einschalten keinen Stromfluss und nachdem Einschalten steigt der Strom und nähert sich einem Endwert und ist nachher wieder konstant. Analoges Verhalten zeigen die Spannungen. Der eigentliche Einschaltvorgang nennt man auch das \textbf{transiente Verhalten}.
\subsubsection{MATLAB für Gleichspannungsquelle}
\begin{enumerate}[$\texttt{>}\texttt{>}$]
\item {\color{red}\texttt{syms L R Uq I}}
\item {\color{red}\texttt{dgl=`L*DI+R*I=Uq' => L*DI+R*I=Uq}}
\item {\color{red}\texttt{lgs=dsolve(dgl) => lsg=Uq/R+exp(-1/L*R*t)*C1}}
\item {\color{red}\texttt{lsg\_part=dsolve(dgl,'I(0)=0'}}
\item {\color{red}\texttt{lsg\_part=Uq/R-exp(-1/L*R*t)*Uq/R}}
\end{enumerate}
\textbf{Spule, Widerstand, Wechselspannungsquelle:} Eine Spule und einen ohmschen Widerstand in Serie an eine Wechselspannungsquelle wird angeschlossen und den Strom und die Spannungsbfälle an den beiden Bauteilen bestimmt. In der gegebene Schaltung liegt eine geschlossene Masche vor und daher ist die Summe der Spannungsabfälle gleich der Summe der Quellspannungen.
\begin{equation}
\boxed{u_R\left(t\right)+u_L\left(t\right)=R\cdot i\left(t\right)+L\cdot \dfrac{\text{d}}{\text{d}t}\Big[i\left(t\right)\Big]=u_q\left(t\right)=\hat{U}\sin\left(\omega t\right)}
\end{equation}
\begin{equation}
\boxed{i\left(t\right)=\dfrac{\hat{U}}{R^2+\left(\omega L\right)^2}\left(\omega Le^{-\dfrac{Rt}{L}}-\omega L\cos\left(\omega t\right)+R\sin\left(\omega t\right)\right)}
\end{equation}
\begin{equation}
\boxed{u_R\left(t\right)=\dfrac{\hat{U}R}{R^2+\left(\omega L\right)^2}\left(\omega L e^{-\dfrac{Rt}{L}}-\omega L\cos\left(\omega t\right)+R\sin\left(\omega t\right)\right)}
\end{equation}
\begin{equation}
\boxed{u_L\left(t\right)=\dfrac{\hat{U}\omega L}{R^2+\left(\omega L\right)^2}\left(-R e^{-\dfrac{Rt}{L}}+\omega L\sin\left(\omega t\right)+R\cos\left(\omega t\right)\right)}
\end{equation}
Hier erkennt man einen Einschaltvorgang. Betrachtet man den Strom in der Schaltung, so hat man zwei Summanden. Der erste Suzmmand beschreibt eine armonische Schwingung und der zweite Summand zeigt exponentielles Verhalten. Die Schwingung ist nicht zeitabhängig (konstante Amplitude und Kreisfrequenz) und beschreibt das Verhalten der Schaltung nachdem der Einschaltvorgang abgeschlossen ist (partikuläre Lösung der inhomogenen DGL). Diese Schwingung nennt man auch das stationäre Verhalten der Schaltung. Das zweite Signal ist zeitabhängig und beschreibt den Übergang zwischen den stationären Zuständen (transientes verhalten der Schaltung - partikuläre Lösung der homogenen DGL).
\subsubsection{MATLAB für Wechselspannungsquelle}
\begin{enumerate}[$\texttt{>}\texttt{>}$]
\item {\color{red}\texttt{syms L R U I w t}}
\item {\color{red}\texttt{dgl='L*DI+R*I=U*sin(w*t)' => dgl=L*DI+R*I=U*sin(w*t)}}
\item {\color{red}\texttt{lsg=dsolve(dgl)}}
\item {\color{red}\texttt{lsg\_part=dsolve(dgl, `I(0)=0')}}
\end{enumerate}
\subsection{Stationäres Verhalten einer Spule (Induktivität)}
\textbf{Spule, Wechselspannungsquelle:} An eine Spule legt man eine Wechselspannung an und will daraus den Stromfluss in der Spule untersuchen. Das Verhältnis der Amplituden zwischen Spannung und Strom ist konstant und gleich $\dfrac{U_L}{I_L}=X_L=\omega L$. Der Strom eilt der Spannung um eine Viertelperiode nach.
\begin{equation}
\boxed{u_L\left(t\right)=L\cdot \dfrac{\text{d}}{\text{d}t}\Big[i\left(t\right)\Big]=\hat{U}\sin\left(\omega t\right)=u_q\left(t\right)}
\end{equation}
\begin{equation}
\boxed{i\left(t\right)=\dfrac{1}{L}\displaystyle \int u_L\left(t\right)\,\text{d}t=\dfrac{1}{L}\displaystyle \int \hat{U}\sin\left(\omega t\right)\,\text{d}t=-\dfrac{\hat{U}}{\omega L}\cos\left(\omega t\right)}
\end{equation}
\subsection{Stationäres Verhalten einer Spule (Kapazität)}
\textbf{Kondensator, Wechselspannungsquelle:} An einen Kondensator legt man eine Wechselspannung an und will den Stromfluss im Kondensator untersuchen. Das Verhältnis der Amplituden zwischen Spannung und Strom ist konstant und gleich $\dfrac{u_C}{I_C}=X_C=\dfrac{1}{\omega C}$. Der Strom eilt der Spannung um eine Viertelperiode vor.
\begin{equation}
\boxed{u_C\left(t\right)=\dfrac{1}{C} \displaystyle \int i\left(t\right)\,\text{d}t=\hat{U}\sin\left(\omega t\right)=u_q\left(t\right)}
\end{equation}
\begin{equation}
\boxed{i\left(t\right)=C\cdot \dfrac{\text{d}}{\text{d}t}\Big[u_C\Big]\left(t\right)=C\cdot \dfrac{\text{d}}{\text{d}t}\Big[\hat{U}\sin\left(\omega t\right)\Big]=\omega C \hat{U}\cos\left(\omega t\right)}
\end{equation}
\subsection{Berechnung mit komplexen Zahlen - Bildbereich}
In diesem Abschnitt werden nur stationären Signale (Vernachlässigung des Einschaltenverhaltens, d.h. der transiente Vorgang) bei Schaltungen die an einer Wechselspannung angeschlossen. Anstelle der Berechnung mittels Differentialgleichungen arbeitet man mit einer Modellwelt für die Grössen und Signale. Da man weisst, dass alle Signale (Ströme und Spannungen) ebenfalls Wechselgrössen (mit der gleichen Kreisfrequenz wie die Quelle) sind, kann man in einer Modellwelt arbeiten, in der die Zeit nicht mehr vorkommt. um eine Wechselgrösse (harmonische Schwingung) zu beschreiben sind drei Angaben von Bedeutung: Amplitude und Schwingung, Kreisfrequenz der Schwingung und Anfangsphase der Schwingung.
\newline\newline
Da die Kreisfrequenz aller Signale gleich ist, muss man diese Grösse nicht in der Modellwelt mitführen. um die restlichen beiden Grössen zu beschreiben wählt man nun komplexe Zahlen für die Modellwelt. Eine komplexe Zahl in goniometrischer oder exponentieller Darstellung beinhaltet die beiden Informationen, Betrag und Argument. Nun kann man ein Signal wie folgt durch eine komplexe zahl bzw. einen Zeiger beschreiben.
\begin{equation}
\boxed{s\left(t\right)=\hat{A}\cos\left(\omega t+\varphi\right) \rightarrow \underline{S}=\dfrac{\hat{A}}{\sqrt{2}}e^{\text{j}\varphi}}
\end{equation}
\begin{equation}
\boxed{s\left(t\right)=\hat{A}\sin\left(\omega t+\varphi\right) \rightarrow \underline{S}=\dfrac{\hat{A}}{\sqrt{2}}e^{\text{j}\left(\varphi-\dfrac{\pi}{2}\right)}}
\end{equation}
Der Elektrotechniker arbeitet meist nicht mit den Amplituden sondern mit den Effektivwerten. Bei harmonischen Signalen ist die Amplitude des Signals um den Faktor $\sqrt{2}$ grösser als der Effektivwert.
\subsubsection{Impendanz eines Widerstandes}
Stom und Spannung an einem Widerstand sind in Phase und das Verhältnis zwischen Strom und Spannung ist durch den Widerstandswert gegeben. Daher definiert man die Impendanz wie folgt:
\begin{equation}
\boxed{Z_R=R\Longrightarrow U_R=Z_R\cdot I_R}
\end{equation}
\subsubsection{Impendanz einer Spule}
Die Spannung eilt in einer Spule dem Strom um eine Viertelperiode vor und das Verhältnis zwischen Strom und Spannung ist durch $X_L=\omega L$ gegeben. Daher definiert man die Impendanz wie folgt:
\begin{equation}
\boxed{\underline{Z_L}=\text{j}\omega L\longrightarrow \underline{U_L}=\underline{Z_L}\cdot \underline{I_L}}
\end{equation}
\subsubsection{Impendanz eines Kondensators}
Die Spannung eilt in einem Kondensator dem Strom um eine Viertelperiode nach und das Verhältnis zwischen Strom und Spannung ist durch $X_C=\dfrac{1}{\omega C}$ gegeben. Daher definiert man die Impendanz wie folgt:
\begin{equation}
\boxed{\underline{Z_C}=-\text{j}\dfrac{1}{\omega C}\Longrightarrow \underline{U_C}=\underline{Z_C}\cdot \underline{I_C}}
\end{equation}
\subsubsection{Wechselstromschaltung}
Nun können Wechselstromschaltungen in der Modellwelt analog zu Gleichspannungsschaltungen berechnet werden. Die Serieschaltung einer Spule mit einem Widerstand, welche an eine Wechselquelle angeschlossen sind
\begin{equation}
\boxed{u_q\left(t\right)=\hat{U}\cos\left(\omega t\right)\longrightarrow U_q=\dfrac{\hat{U}}{\sqrt{2}}}
\end{equation}
\begin{equation}
\boxed{\underline{Z_R}=R}\quad \boxed{\underline{Z_L}=\text{j}\omega L}\quad \boxed{\underline{Z_{\text{ers}}}=\underline{Z_R}+\underline{Z_L}=R+\text{j}\omega L}
\end{equation}
\begin{equation}
\boxed{i\left(t\right)=\sqrt{2}\Big\vert\underline{I}\Big\vert\cos\left(\omega t+\text{arg}\left(\underline{I}\right)\right)\longrightarrow \underline{I}=\dfrac{\underline{U_q}}{\underline{Z_{\text{ers}}}}=\dfrac{\hat{U}}{\sqrt{2}}\dfrac{R-\text{j}\omega L}{R^2+\omega^2L^2}}
\end{equation}
\begin{equation}
\boxed{u_R\left(t\right)=\sqrt{2}\Big\vert\underline{U_R}\Big\vert\cos\left(\omega t+\text{arg}\left(\underline{U_R}\right)\right)\longrightarrow \underline{U_R}=\underline{I}\cdot \underline{Z_R}=\dfrac{\hat{U}}{\sqrt{2}}\dfrac{R^2-\text{j}\omega LR}{R^2+\omega^2L^2}}
\end{equation}
\begin{equation}
\boxed{u_L\left(t\right)=\sqrt{2}\Big\vert\underline{U_L}\Big\vert\cos\left(\omega t+\text{arg}\left(\underline{U_L}\right)\right)\longrightarrow \underline{U_L}=\underline{I}\cdot \underline{Z_L}=\dfrac{\hat{U}}{\sqrt{2}}\dfrac{R\omega^2L^2+\text{j}\omega LR}{R^2+\omega^2L^2}}
\end{equation}
Sind somit Signale und Grössen gegeben, so müssen sie in eine komplexe Form transformiert werden. Das Problem kann mittels Algebra der komplexen Zahlen gelöst werden.
\begin{comment}
\section{Komplexe Zahlen in MATLAB}
\subsection{Definition der imaginäre Einheit}
Folgende Befehle sind Berechnungen der komplexen Zahlen
\begin{enumerate}[$\texttt{>}\texttt{>}$]
\item {\color{red}\texttt{sqrt(-1) => 0.0000 + 1.0000i}}
\item {\color{red}\texttt{i\^\,5 => 0.0000 + 1.0000i}}
\item {\color{red}\texttt{i\^\,10 => -1}}
\end{enumerate}
\subsection{Operationen mit komplexen Zahlen}
\begin{enumerate}[$\texttt{>}\texttt{>}$]
\item {\color{red}\texttt{double(solve('x\^\,2+x+1')) => -0.5000 + 0.8660i, -0.5000-0.8660i}}
\item {\color{red}\texttt{z1=3+4*i => 3.0000 + 4.0000i}}
\item {\color{red}\texttt{z2=complex(3,4) => 3.0000 + 4.0000i}}
\item {\color{red}\texttt{real(z1) => 3}}
\item {\color{red}\texttt{imag(z1) => 4}}
\item {\color{red}\texttt{(3+4*i)+(1+2*i)/(-4+3*i) => 3.0800 + 3.5600i}}
\item {\color{red}\texttt{conj(1+2*i) => 1.0000 - 2.0000i}}
\item {\color{red}\texttt{(1+2*i)*conj(1+2*i) => 5}}
\end{enumerate}
\subsection{Gauss'sche Zahlenebene}
\begin{enumerate}[$\texttt{>}\texttt{>}$]
\item {\color{red}\texttt{plot([1+2*i,2-3*i,4-5*i,-3+i,i, -6, -3-2*i], 'r*')}}
\item {\color{red}\texttt{z=-2+i = -2.0000 + 1.0000i}}
\item {\color{red}\texttt{betrag=abs(z) => betrag =  2.2361}}
\item {\color{red}\texttt{sym=abs(z) => sym =  2.2361}}
\item {\color{red}\texttt{winkel=angle(z) => winkel =  2.6779}}
\item {\color{red}\texttt{winkel\_grad=winkel/pi*180 => winkel\_grad = 153.4349}}
\end{enumerate}
\subsection{Satz von Moivre}
\begin{enumerate}[$\texttt{>}\texttt{>}$]
\item {\color{red}\texttt{(-sqrt(3)-i)\^\,10 => 5.1200e+02 - 8.8681e+02i}}
\item {\color{red}\texttt{abs((-sqrt(3)-i)\^\,10) => 1.0240e+03}}
\item {\color{red}\texttt{angle((-sqrt(3)-i)\^\,10) => -1.0472}}
\item {\color{red}\texttt{(-sqrt(3)-i)\^\,(1/3) => 0.8099 - 0.9652i}}
\item {\color{red}\texttt{double(solve('z\^\,3-(-sqrt(3)-i)')) => 0.8099 - 0.9652i, -1.2408 - 0.2188i, 0.4309 + 1.1839i}}
\item {\color{red}\texttt{compass(double(solve('z\^\,3-(-sqrt(3)-i)')))}}
\end{enumerate}
\subsection{Eulersche Formel}
\begin{enumerate}[$\texttt{>}\texttt{>}$]
\item {\color{red}\texttt{log(-1) = 0.0000 + 3.1416i}}
\item {\color{red}\texttt{log(-10*i) = 2.3026 - 1.5708i}}
\item {\color{red}\texttt{log2(sqrt(2)-sqrt(2)*i) = 1.0000 - 1.1331i}}
\item {\color{red}\texttt{exp(1+i*pi) = -2.7183 + 0.0000i}}
\item {\color{red}\texttt{2\^i = 0.7692 + 0.6390i}}
\item {\color{red}\texttt{(-i)\^\,(-i) = 0.2079}}
\end{enumerate}
\end{comment}
\section{Ortskurven}
\subsection{Einführung}
Eine Ortskurve ist eine parametrisierte Abbildung in die Gauss'sche Zahlenebene der Form
\begin{equation}
\boxed{
\begin{array}{lll}
f&:&\mathbb{R}\rightarrow \mathbb{C}\\
t&\mapsto&z\left(t\right)=x\left(t\right)+jy\left(t\right)\\
\end{array}
}
\end{equation}
Gegeben sei eine veränderliche komplexe Grösse. Die komplexe Grösse kann bekanntlich als Zeiger in der Gauss'schen Zahlenebene aufgefasst werden. Die Ortskurve beinhaltet nun alle Zeigerspitzen der veränderlichen komplexen Grösse.
\subsection{Kurven in der Gauss'schen Zahlenebene}
\subsubsection{Die Gerade in der Gauss'schen Zahlenebene}
Die allgemeine Gleichung einer Geraden in der Gauss'schen Zahlenebene lautet
\begin{equation}
\boxed{z\left(t\right)=z_0+f\left(t\right)z_1}
\end{equation}
Dabei bezeichnen $z_0$ und $z_1$ komplexe Zahlen (Zeiger) und $f\left(t\right)$ eine reellwertige Funktion für den Parameter. Analog der Parametergleichung
\begin{equation}
\boxed{\overrightarrow{r}=\overrightarrow{r}_0+t\overrightarrow{r}_1}
\end{equation}
Eine besonders einfache Form der Beschreibung einer Geraden in der Gauss'schen Zahlenebene liefert die folgende Gleichung
\begin{equation}
\boxed{z\left(t\right)=a\left(1+i\cdot f\left(t\right)\right)=a\left(1+i\cdot t\right)=\underbrace{\left(a\right)}_{z_0}+\underbrace{\left(i\cdot a\right)}_{z_1}\cdot t}
\end{equation}
Die komplexen Zeiger $z_0=a$ und $z_1=i\cdot a$ stehen senkrecht zueinader und die komplee Zahl $a$ ist der kürzeste Zeiger vom Ursprung auf die Gerade.
\subsubsection{Der Kreis in der Gauss'schen Zahlenebene}
Für einen beliebigen Kreis verschiebt man nun den Kreis um die komplexe Zahl $b$ und multiplizieren mit $a$. Der Mittelpunkt des Kreises lautet
\begin{equation}
\boxed{M\left(\dfrac{\text{Re}\left(a\right)}{2}+\text{Re}\left(b\right), \dfrac{\text{Im}\left(a\right)}{2}+\text{Im}\left(b\right)\right)}
\end{equation}
\begin{equation}
\boxed{z\left(t\right)=az_0\left(t\right)+b}
\end{equation}
\subsection{Inversion}
\subsubsection{Inversion einer Geraden durch den Ursprung}
Betrachte die Gerade durch den Ursprung mit der Gleichung
\begin{equation}
\boxed{z\left(t\right)=f\left(t\right)z_0}
\end{equation} \tabularnewline
Für die Inversion $w=\dfrac{1}{z}$ findet man
\begin{equation}
\boxed{w\left(t\right)=\dfrac{1}{z\left(t\right)}=\dfrac{1}{f\left(t\right)z_0}=\dfrac{1}{f\left(t\right)}\dfrac{\overline{z_0}}{z_0\overline{z_0}}}
\end{equation}
Dies ist eine Gerade durch den Ursprung mit der Parametrisierung: $g\left(t\right)=\dfrac{1}{f\left(t\right)}$ und der Richtung $w_0=\dfrac{\overline{z_0}}{z_0\overline{z_0}}$. Die invertierte Gerade ist die Spiegelung der gegebenen Geraden an der reellen Achse.
\subsubsection{Inversion einer Geraden nicht durch den Ursprung}
Betrachte die Gerade mit $a=\Big\vert a\Big\vert e^{\text{j}\varphi}$
\begin{equation}
\boxed{z\left(t\right)=a\left(1+\text{j} f\left(t\right)\right)}
\end{equation}
Die Inversion ergibt
\begin{equation}
\boxed{w\left(t\right)=\dfrac{1}{z\left(t\right)}=\dfrac{1}{a\left(1+\text{j}f\left(t\right)\right)}=\underbrace{\dfrac{1}{a}}_{\text{Drehstreckung}}\cdot \underbrace{\dfrac{1}{1+\text{j}f\left(t\right)}}_{\text{Kreis durch den Ursprung}}}
\end{equation}
Die Inversion einer Geraden nicht durch den Ursprung ergibt einen Kreis durch den Ursprung mit den Daten
\begin{equation}
\boxed{R=\dfrac{1}{2\Big\vert a \Big\vert}}\quad \boxed{M\left(\dfrac{1}{2\Big\vert a\Big\vert}\cos\left(-\varphi\right), \dfrac{1}{2\Big\vert a\Big\vert}\sin\left(-\varphi\right)\right)}
\end{equation}
\subsubsection{Inversion eines Kreises durch den Ursprung}
Die Inversion eines Kreises durch den Ursprung muss nach den Überlegungen des letzten Abschnittes eine Gerade ergeben, welche nicht durch den Ursprung geht. Sei also ein Kreis mit dem Mittelpunkt gegeben.
\begin{equation}
\boxed{M\left(\dfrac{\Big\vert a\Big\vert}{2}\cos\left(\varphi\right), \dfrac{\Big\vert a\Big\vert}{2}\sin\left(\varphi\right)\right)\Rightarrow R=\dfrac{\Big\vert a\Big\vert}{2}}
\end{equation}
Diesen Kreis kann man duch die folgende Gleichung beschrieben.
\begin{equation}
\boxed{z\left(t\right)=a\dfrac{1}{1+\text{j}f\left(t\right)}}
\end{equation}
Die Inversion ergibt nun
\begin{equation}
\boxed{w\left(t\right)=\dfrac{1}{z\left(t\right)}=\dfrac{1}{a}\left(1+\text{j}f\left(t\right)\right)=\dfrac{1}{\Big\vert a\Big\vert}e^{-\text{j}\varphi}\left(1+\text{j}f\left(t\right)\right)}
\end{equation}
Die Inversion eines Kreises durch den Ursprung ergibt eine Gerade die nicht durch den Ursprung geht. Dies ist eine Gerade durch den Punkt $z_0=\dfrac{1}{\Big\vert a\Big\vert}e^{-\text{j}\varphi}$ mit der Richtung $z_1=\text{j}z_0=\dfrac{1}{\Big\vert a\Big\vert}e^{\text{j}\dfrac{\pi}{2}-\varphi}$.
\subsubsection{Inversion eines Kreises nicht durch den Ursprung}
Die Inversion des Kreises $z\left(t\right)$ mit Mittelpunkt $z_M$ und Radius $R$ hat folgende Inversion
\begin{equation}
\boxed{z\left(t\right)=b+a\dfrac{1}{1+\text{j}f\left(t\right)}}
\end{equation}
\begin{equation}
\boxed{z_M=\dfrac{a}{2}+b}
\end{equation}
\begin{equation}
\boxed{R=\dfrac{\Big\vert a\Big\vert}{2}}
\end{equation}
\begin{equation}
\boxed{w\left(t\right)=\dfrac{1}{z\left(t\right)}=\dfrac{1}{b+a\dfrac{1}{1+\text{j}f\left(t\right)}}=\dfrac{1+\text{j}f\left(t\right)}{a+b\left(1+\text{j}f\left(t\right)\right)}}
\end{equation}
Die Inversion eines Kreises nicht durch den Ursprung ergibt wieder einen Kreis der nicht durch den Ursprung geht.
\begin{equation}
\boxed{z_{w,n}=z_M\pm R_z\dfrac{z_M}{\Big\vert z_M\Big\vert}=z_M\left(1\pm \dfrac{R_z}{\Big\vert z_M\Big\vert}\right)=\dfrac{z_M}{\Big\vert z_M\Big\vert}\left(\Big\vert z_M\Big\vert\pm R_z\right)}
\end{equation}
\begin{equation}
\boxed{w_{n,w}=\dfrac{1}{\dfrac{z_M}{\Big\vert z_M\Big\vert}\left(\Big\vert z_M\Big\vert\pm R_z\right)}=\dfrac{\overline{z_M}}{\Big\vert z_M\Big\vert}\cdot \dfrac{1}{\Big\vert z_M\Big\vert\pm R_z}}
\end{equation}
Der Mittelpunkt liegt zwischen diesen beiden Punkten.
\begin{equation}
\boxed{w_M=\dfrac{1}{2}\left(w_M+w_n\right)=\dfrac{\overline{z_M}}{2\Big\vert z_M\Big\vert}\left(\dfrac{1}{\Big\vert z_M\Big\vert-R_z}+\dfrac{1}{\Big\vert z_M\Big\vert+R_z}\right)=\dfrac{\overline{z_M}}{\Big\vert z_M\Big\vert^2-R_z^2}}
\end{equation}
Den Radius des neuen Kreises erhält man aus dem halben Betrag der Differenz der beiden Zeiger
\begin{equation}
\boxed{R_w=\dfrac{1}{2}\Big\vert W_w-W_n\Big\vert=\dfrac{R_z}{\Big\vert z_M\Big\vert^2-R_z^2}}
\end{equation}
\subsection{Ortskurven in MATLAB}
Ortskurve
\begin{enumerate}[$\texttt{>}\texttt{>}$]
\item {\color{red}\texttt{syms omega}}
\item {\color{red}\texttt{omega\_r = [0,100];}}
\item {\color{red}\texttt{omega\_w = [0,10,20,30,40,50,100];}}
\item {\color{red}\texttt{z=0.5+i*omega*0.01}}
\item {\color{red}\texttt{ezplot(real(z),imag(z),omega\_r)}}
\item {\color{red}\texttt{hold on}}
\item {\color{red}\texttt{for k=1:length(omega\_w)}}
\item $\quad${\color{red}\texttt{zz=subs(z,omega,omega\_w(k));}}
\item $\quad${\color{red}\texttt{xx=real(zz);}}
\item $\quad${\color{red}\texttt{yy=imag(zz);}}
\item $\quad${\color{red}\texttt{plot(xx,yy,'r*')}}
\item $\quad${\color{red}\texttt{text(xx,yy,strcat('$\backslash$leftarrow $\backslash$omega=', num2str(omega\_w(k)), `1/\_s'))}}
\item {\color{red}\texttt{end}}
\end{enumerate}
Inverse der Ortskurve
\begin{enumerate}[$\texttt{>}\texttt{>}$]
\item {\color{red}\texttt{z\_inv=1/z}}
\item {\color{red}\texttt{ezplot(real(z\_inv), imag(z\_inv), omega\_r);}}
\item {\color{red}\texttt{hold on}}
\item {\color{red}\texttt{for k=1:length(omega\_w)}}
\item $\quad${\color{red}\texttt{zz=subs(z\_inv, omega, omega\_w(k));}}
\item $\quad${\color{red}\texttt{xx=real(zz);}}
\item $\quad${\color{red}\texttt{yy=img(zz);}}
\item $\quad${\color{red}\texttt{plot(xx, yy, `r*')}}
\item $\quad${\color{red}\texttt{text(xx,yy,strcat('$\backslash$leftarrow $\backslash$omega=', num2str(omega\_w(k)), `1/\_s'))}}
\item {\color{red}\texttt{end}}
\end{enumerate}













%\chapter{Lineare Algebra und Geometrie}
Das Skalarprodukt aus der Kraft mit der Geschwindigkeit ergibt eine skalare Grösse, nämlich die Leistung, welche die Wirkung der Kräfte an bewegten Massensystemen charakterisiert. Die Leistung ist eine zeit- und bewegungsabhängige Grösse. Die Dimension der Leistung ist der \textbf{Watt}, also die Kraft welche 1N m s$^{-1}$
%%%%%%%%%%%%%%%%%%%%%%%%%%%%%%%%%%%%%%%%%%%%%%%%%%%%%%%%%%%%%%%%%%%%%%%%%%%%%%%%%%%%%%%%%%%%
\subsection{Leistung einer Einzelkraft}
Die Leistung einer Einzelkraft mit dem Massenangriffspunkt $M$ ist
\begin{equation}
\boxed{\mathcal{P}=\overrightarrow{F}\bullet \overrightarrow{v}_A}
\end{equation}
\begin{equation}
\boxed{\mathcal{P}=\Big\vert\overrightarrow{F}\Big\vert\cdot \Big\vert\overrightarrow{v}_A\Big\vert\cdot \cos\left(\alpha\right)}
\end{equation}
Für $\alpha<90^{\circ}$ ist $\mathcal{P}>0$ und die Kraft $\overrightarrow{F}$ wirkt als \textbf{Antriebskraft}. Ist $\alpha>90^{\circ}$ ist $\mathcal{P}<0$ und die Kraft $\overrightarrow{F}$ wirkt als \textbf{Widerstandskraft}. Ist $\alpha=90^{\circ}$, so verschwindet die Leistung $\mathcal{P}=0$ und die Kraft $\overrightarrow{F}$ ist \textbf{momentan leistungslos}.
\begin{equation}
\boxed{\mathcal{P}=\overrightarrow{F}\bullet \overrightarrow{v}_A=\overrightarrow{F}\bullet \left(\overrightarrow{\omega}\times\overrightarrow{r}_{OA}\right)=\overrightarrow{\omega}\bullet \underbrace{\left(\overrightarrow{r}_{OA}\times \overrightarrow{F}\right)}_{\overrightarrow{M}_O}=\overrightarrow{\omega}\bullet \overrightarrow{M}_O}
\end{equation}
Das Vektorprodukt zwischen dem Ortsvektor $\overrightarrow{r}_{OA}$ des Massenangriffspunkts $A$ einer Kraft und dem Vektoranteil $\overrightarrow{F}$ heisst \textbf{Moment} $\overrightarrow{M}_O$ der Kraft bezüglich $O$, wobei $O$ Bezugspunkt des betrachteten Massessystems ist. Ein Moment entspricht immer ein Paar zweier Kräfte. Dabei sind die Kräfte gleich gross, entgegengerichtet und besitzen unterschiedliche Wirklinien. Ein \textbf{reines Moment} hat keine translatorishe Wirkung. Wird ein Körper in seinem Schwerpunkt aufgehängt, so ist da auf den Körper wirkende Moment für alle Orientierungen des Körpers im Raum gleich null.
\begin{equation}
\boxed{\overrightarrow{M}_O=\overrightarrow{r}_{OA}\times \overrightarrow{F}}
\end{equation}
\begin{equation}
\boxed{M_O=\Big\vert\overrightarrow{F}\Big\vert\cdot \Big\vert\overrightarrow{r}_{OA}\Big\vert\cdot \sin\left(\alpha\right)}
\end{equation}
Die Leistung einer Einzelkraft an einem rotierenden starren Massensystem ergibt sich aus dem Skalarprodukt der Rotationsgeschwindigkeit $\overrightarrow{\omega}$ mit dem Moment $\overrightarrow{M}_O$ der Kraft bezüglich eines Bezugspunktes $O$ auf der Rotationsachse.
\newline\newline
Projiziert man den Momentvektor $\overrightarrow{M}_O$ einer Kraft bezüglich $O$ auf eine Achse $Oz$ durch $O$, so erhält man das Moment $M_z$ der Kraft bezüglich der Achse $Oz$.
\begin{equation}
\boxed{M_z:=\overrightarrow{e}_z\bullet \overrightarrow{M}_O}\quad \boxed{\mathcal{P}=\omega\cdot M_z}
\end{equation}
%%%%%%%%%%%%%%%%%%%%%%%%%%%%%%%%%%%%%%%%%%%%%%%%%%%%%%%%%%%%%%%%%%%%%%%%%%%%%%%%%%%%%%%%%%%%
\subsection{Gesamtleistung mehrerer Kräfte}
Die Gesamtleistung mehrerer Kräfte besteht aus der Summe der Leistungen der einzelnen Kräfte.
\begin{equation}
\boxed{\mathcal{P}\left(\overrightarrow{F}_1, \dotso, \overrightarrow{F}_n\right)=\mathcal{P}\left(\overrightarrow{F}_1\right)+\mathcal{P}\left(\overrightarrow{F}_2\right)+\dotso + \mathcal{P}\left(\overrightarrow{F}_n\right)}
\end{equation}
Haben $n$ Einzelkräfte den gleichen Massenangriffspunkt $A$ und ist $\overrightarrow{v}_A$ die Geschwindigkeit dieses Massenpunktes, so ist die Gesamtleistung
\begin{equation}
\boxed{\mathcal{P}=\overrightarrow{F}_1\bullet \overrightarrow{v}_A+\overrightarrow{F}_2\bullet \overrightarrow{v}_A+\dotso+\overrightarrow{F}_n\bullet \overrightarrow{v}_A=\overrightarrow{R}\bullet \overrightarrow{v}_A}
\end{equation}
Haben $n$ Einzelkräfte $n$ verschiedene Massenangriffspunkte $M_i$ und ist $\overrightarrow{v}_{M_i}$ die Geschwindigkeit jedes Massenpunktes, so ist die Gesamtleistung
\begin{equation}
\boxed{\mathcal{P}=\overrightarrow{F}_1\bullet \overrightarrow{v}_{A_1}+\overrightarrow{F}_2\bullet \overrightarrow{v}_{A_2}+\dotso+\overrightarrow{F}_n\bullet \overrightarrow{v}_{A_n}=\displaystyle \sum_{i=1}^n\left(\overrightarrow{F}_i\bullet \overrightarrow{v}_{A_i}\right)}
\end{equation}
%%%%%%%%%%%%%%%%%%%%%%%%%%%%%%%%%%%%%%%%%%%%%%%%%%%%%%%%%%%%%%%%%%%%%%%%%%%%%%%%%%%%%%%%%%%%
\subsection{Gesamtleistung eines starren Massensystems}
Betrachte man ein starres Massensystem $S$ und $n$ Kräfte $\overrightarrow{F}_1$ bis $\overrightarrow{F}_n$ mit Massenangriffspunkte $A_1$ bis $A_n\in S$. Die Bewegung von $S$ sei zum Zeitpunkt $t$ durch die Kinemate $\left\{\overrightarrow{v}_O, \overrightarrow{\omega}\right\}$ in $O\in S$ gegeben.
\begin{equation}
\boxed{\begin{array}{lll}
\mathcal{P}&=&\overrightarrow{F}_1\bullet\overrightarrow{v}_{A_1}+\dotso+\overrightarrow{F}_n\bullet\overrightarrow{v}_{A_n}\\
&=&\overrightarrow{F}_1\bullet\left(\overrightarrow{v}_{O}+\overrightarrow{\omega}\times \overrightarrow{r}_{OA_1}\right)+\dotso+\overrightarrow{F}_n\bullet\left(\overrightarrow{v}_{O}+\overrightarrow{\omega}\times \overrightarrow{r}_{OA_n}\right)\\
&=&\overrightarrow{F}_1\bullet\overrightarrow{v}_O+\overrightarrow{F}_1\bullet \left(\overrightarrow{\omega}\times \overrightarrow{r}_{OA_1}\right)+\dotso+\overrightarrow{F}_n\bullet\overrightarrow{v}_O+\overrightarrow{F}_n\bullet \left(\overrightarrow{\omega}\times \overrightarrow{r}_{OA_n}\right)\\
&=&\overrightarrow{F}_1\bullet\overrightarrow{v}_O+\overrightarrow{\omega}\bullet \left(\overrightarrow{r}_{OA_1}\times \overrightarrow{F}_{1}\right)+\dotso+\overrightarrow{F}_n\bullet\overrightarrow{v}_O+\overrightarrow{\omega}\bullet \left(\overrightarrow{r}_{OA_n}\times \overrightarrow{F}_{n}\right)\\
&=&\displaystyle \sum_{i=1}^n\left(\overrightarrow{F}_i\right)\bullet\overrightarrow{v}_O+\overrightarrow{\omega}\bullet\displaystyle \sum_{i=1}^n\left(\overrightarrow{r}_{OA_i}\times \overrightarrow{F}_{i}\right)\\\\
&=&\overrightarrow{R}\bullet\overrightarrow{v}_O+\overrightarrow{\omega}\bullet\overrightarrow{M}_O\\
\end{array}}
\end{equation}
%\chapter{Funktionen von mehreren Variablen}
\section{Funktionen von mehreren Variablen und ihre Darstellung}
\subsection{Definition einer Funktion von mehreren Variablen}
Unter einer Funktion von zwei unabhängigen Variablen versteht man eine Vorschrift, die jedem geordneten Zahlenpaar $\left(x;y\right)$ aus einer Menge $D$ genau ein Element $z$ aus einer Menge $W$ zuordnet. Symbolische Schreibweise: $z=f\left(x;y\right)$.
\newline\newline
Seien $x,y$ die unabhängige Variable, $z$ die abhängige Variable, $D$ der Definitionsbereich und $W$ der Wertebereich der Funktion. 
\newline\newline
Eine Funktion von $n$ unabhängige Variablen lautet
\begin{equation}
\boxed{y=f\left(x_1; x_2;\dotso; x_n\right)}
\end{equation}
\subsection{Darstellungsformen einer Funktion von zwei Variablen}
\subsubsection{Analytische Darstellung}
Die Funktion wird explizit durch eine Funktionsgleichung dargestellt.
\begin{equation}
\boxed{z=f\left(x;y\right)}
\end{equation}
Die Funktion wird implizit durch eine Funktionsgleichung dargestellt.
\begin{equation}
\boxed{F\left(x;y;z\right)=0}
\end{equation}
\subsubsection{Graphische Darstellung}
Die Variablen $x$, $y$ und $z$ einer Funktion $z=f\left(x;y\right)$ werden als rechtwinklige oder kartesische Koordinaten eines Raumpunktes $P$ gedeutet: $P\left(x;y;z\right)$. Der Funktionswert $z=f\left(x;y\right)$ ist dabei die Höhenkoordinate der zugeordneten Bildpunktes. Man erhält als Bild der Funktion eine über dem Definitionsbereich liegende Fläche.
\newline\newline
Die Schnittkurvendiagramme einer Funktion $z=f\left(x;y\right)$ erhält man durch Schnitte der zugehörigen Bildfläche mit Ebenen, die parallel zu einer der drei koordinatenebenen verlaufen. Die Schnittkurven werden noch in die jeweilige Koordinatenebene projiziert und repräsentieren einparametrige Kurvenscharen. Ihre Gleichungen erhält man aus der Funktionsgleichung $z=f\left(x;y\right)$, indem man der Reihe nach jeweils eine der drei Variablen als Parameter betrachtet. In den naturwissenschaftlich-technischer Anwendungen werden die Schnittkurvendiagramme als Kennlinienfelder bezeichnet.
\newline\newline
Das Höhenliniendiagramm ist ein spezielles Schnittkurvendiagramm mit der Höhenkoordinate $z$ als Kurvenparameter
\begin{equation}
\boxed{f\left(x;y\right)=c}
\end{equation}
\subsection{Ebenen im Raum}
\subsubsection{Ebenen}
Die Bildfläche einer linearen Funktion ist eine Ebene
\begin{equation}
\boxed{ax+by+cz+d=0}
\end{equation}
\subsubsection{Koordinatenebenen}
Die $xy$-Ebene ist $z=0$. Die $xz$-Ebene ist $y=0$. Die $yz$-Ebene ist $x=0$.
\subsubsection{Parallelebenen}
Eine Ebene parallel zur $xy$-Ebene lautet: $z=a$. Eine Ebene parallel zur $xz$-Ebene lautet: $y=a$. Eine Ebene parallel zur $yz$-Ebene lautet: $x=a$
\subsection{Rotationsflächen}
\subsubsection{Gleichung einer Rotationsfläche}
Eine Rotationsfläche entsteht durch Drehung einer ebenen Kurve $z=f\left(x\right)$ um die $z$-Achse für $a\leq r\leq b$. Diese Rotationsfläche lautet in zylindrische und kartesische  Koordinaten
\begin{equation}
\boxed{z=f\left(r\right)=f\left(\sqrt{x^2+y^2}\right)}
\end{equation}
\subsubsection{Spezielle Rotationsfläche}
\textbf{Kugel:} Obere bzw. untere Halbkugel in kartesische  bzw. Zylinderkoordinaten $z=\pm\sqrt{R^2-x^2-y^2}$ bzw. $z=\pm\sqrt{R^2-r^2}$
\begin{equation}
\boxed{x^2+y^2+z^2=R^2}\quad \boxed{r^2+z^2=R^2}
\end{equation}
\textbf{Kreiskegel:} Folgende Gleichungen beschreiben einen Doppelkegel. Für $z\geq 0$ erhält man den Mantel des gezeichneten Kegels mit der Funktionsgleichung $z=\dfrac{H}{R}\sqrt{x^2+y^2}$
\begin{equation}
\boxed{x^2+y^2=\dfrac{R^2}{H^2}z^2}\quad \boxed{z=\dfrac{H}{R}r}
\end{equation}
\textbf{Kreiszylinder:} Sei $z\in \mathbb{R}$ die Höhenkoordinate
\begin{equation}
\boxed{x^2+y^2=R^2}\quad \boxed{r=R}
\end{equation}
\textbf{Ellipsoid:} Durch Auflösen nach $z$ erhält man zwei Funktionen.
\begin{equation}
\boxed{\dfrac{x^2}{a^2}+\dfrac{y^2}{b^2}+\dfrac{z^2}{c^2}=1}
\end{equation}
\subsection{Schnittkurvendiagramme}
Die Struktur einer Funktion $z=f\left(x; y\right)$ kann durch Schnittkurven- pder Schnittliniendiagramme durch ebene Schnitte der zugehörigen Bildfläche dargestellt. Schnittebenen werden parallel zu einer der drei Koordinatenebenen gewählt. Das Schnittliniendiagramm ist das Höhenliniendiagramm, bei dem alle auf der Fläche $z=f\left(x; y\right)$ gelegenen Punkte gleicher Höhe $z=c$ zu einer Flächekurve zusammengefasst werden.
\newline \newline
Diese Kurve lässt sich auch als Schnitt der Fläche $z=f\left(x; y\right)$ mit der zur $x,y$-Ebene parallelen Ebene $z=c$ auffassen. Die Höhenlinien einer Funktion genügen folgender impliziten Kurvengleichung
\begin{equation}
\boxed{z=f\left(x; y\right)=\text{const.}=c}
\end{equation}
Die Höhenlinien repräsentieren eine einparametrige Kurvenschar mit der Höhenkoordinate $z=c$ als Parameter. Zu jedem zulässigen Parameterwert gehört dabei genau eine Höhenlinie. Die Höhenlinie sind die Projektionen der Linien gleicher Höhe in die $x$, $y$-Koordinatenebene. 
\newline\newline
Folgene Schnittkurvendiagramme der Funktion $z=f\left(x; y\right)$ ergeben sich durch Schnitte der zugehörigen Bildfläche mit Ebenen parallel zu einer der drei Koordinatenebenen:
\begin{itemize}
\item Schnitte parallel zur $x$, $y$-Ebene: $z=f\left(x; y\right)=\text{const.}=c$
\item Schnitte parallel zur $y$, $z$-Ebene: $z=f\left(x=c; y\right)$
\item Schnitte parallel zur $x$, $z$-Ebene: $z=f\left(x; y=c\right)$
\end{itemize}
Die Schnittkurvendiagramme repräsentieren somit einparametrige Kurvenscharen. IHre Gleichungen erhält man aus der Funktionsgleichung $z=\left(x; y\right)$, indem man der Reihe nach eine der drei Variablen festhält, d.h. als Parameter betrachtet. Das Höhenliniendiagramm ist ein spezielles Schnittkurvendiagramm mit der Höhenkoordinate $z$ als Kurvenparameter. In den physikalisch-technisch Anwendungen wird das Schnittliniendiagramm einer Funktion meist als \textbf{Kennlinienfeld} bezeichnet.
\section{Grenzwert und Stetigkeit einer Funktion}
Die Begriffe Grenzwert und Stetigkeit einer Funktion lassen sich sinngemäss auch für Funktionen von mehreren Variablen übertragen. Mit dem Grenzwert einer Funktion $z=f\left(x; y\right)$ an der Stelle $\left(x_0; y_0\right)$ lässt sich das Verhalten der Funktion untersuchen, wenn man sich dieser Stelle beliebig nähert. Wr gehen dabei davon aus, dass die Funktion in einer gewissen Umgebung von $\left(x_0; y_0\right)$ definiert ist, eventuell mit Ausnahme dieser Stelle selbst. Diese Funktion hat dann definitionsgemäss an der Stelle $\left(x_0; y_0\right)$ den \textbf{Grenzwert} $g$, wenn sich die Funktionswerte $f\left(x; y\right)$ beim Grenzübergang dem Wert $g$ beliebig nähern.
\begin{equation}
\boxed{\displaystyle \lim_{\left(x; y\right)\rightarrow \left(x_0; y_0\right)}f\left(x; y\right)=g}
\end{equation}
Aus $\left(x; y\right)\rightarrow \left(x_0; y_0\right)$ folgt stets $f\left(x; y\right)\rightarrow g$, und zwar unabhängig vom eingeschlagenen Weg für jede Folge von Zahlenpaaren $\left(x; y\right)$, die sich beliebig der Stelle $\left(x_0; y_0\right)$ nähern. Eine Funktion $f\left(x; y\right)$ kann auch in einer Definitionslücke $\left(x_0; y_0\right)$ einen Grenzwert haben, obwohl sie dort nicht definiert ist. Anschauliche Deutung des Grenzwertes auf der Bildfläche von $z=f\left(x; y\right)$: bewegt man sich auf dieser Fläche in Richtung der Stelle $\left(x_0; y_0\right)$, so unterscheidet sich die erreichte Höhe immer weniger vom Grenzwert $g$.
\newline\newline
Eine in $\left(x_0; y_0\right)$ und einer gewissen Umgebung von $\left(x_0; y_0\right)$ definierte Funktion $z=f\left(x; y\right)$ heisst an der Stelle $\left(x_0; y_0\right)$ \textbf{stetig}, wenn der Grenzwert der Funktion an dieser Stelle vorhanden ist und mit dem dortigen Funktionswert übereinstimmt.
\begin{equation}
\boxed{\displaystyle \lim_{\left(x; y\right)\rightarrow \left(x_0; y_0\right)}f\left(x; y\right)=f\left(x_0; y_0\right)}
\end{equation}
Die Stetigkeit an einer bestimmten Stelle setzt voraus, dass die Funktion dort auch definiert ist. Ferner muss der Grenzwert an dieser Stelle existieren und mit dem Funktionswert übereinstimmen. Eine Funktion $z=f\left(x; y\right)$ heisst dagegen an der Stelle $\left(x_0; y_0\right)$ \textbf{unstetig}, wenn $f\left(x_0; y_0\right)$ nicht vorhanden ist oder $f\left(x_0; y_0\right)$ vom Grenzwert verschieden ist oder dieser nicht existiert. Eine Funktion, die an jeder Stelle ihres Definitionsbereiches stetig ist, wird als stetige Funktion bezeichnet.  
\section{Partielle Differentiation}
\subsection{Partielle Ableitungen 1. Ordnung}
Die Ableitung einer Funktion von einer Variable an der Stelle $x_P$ wird durch den Grenzwert definiert und lässt sich als Steigung $m$ der im Punkt $P=\left(x_P; y_P\right)$ errichteten Kurventangente deuten
\begin{equation}
\boxed{\begin{array}{lll}
m_{\text{Tan},P}=\tan\left(\alpha\right)&=&\displaystyle \lim_{\triangle x\rightarrow 0}\dfrac{\triangle y}{\triangle x}=\displaystyle \lim_{x_Q\rightarrow x_P}\dfrac{f\left(x_Q\right)-f\left(x_P\right)}{x_Q-x_P}\\
&=&\displaystyle \lim_{\triangle x\rightarrow 0}\dfrac{f\left(x_P+\triangle x\right)-f\left(x_P\right)}{\triangle x}\\
&=&\dfrac{\text{d}}{\text{d}x}\Big[f\left(x_P\right)\Big]=f'\left(x_P\right)
\end{array}}
\end{equation}
Analoge Überlegungen führen bei einer Funktion von zwei Variablen, die sich bildlich als Fläche im Raum darstellen lässt, zum Begriff der partiellen Ableitung einer Funktion. Man geht von einem Punkt $P=\left(x_P; y_P; z_P\right)$ auf einer Fläche $z=f\left(x; y\right)$ aus. Durch diesen Flächenpunkt legt man zwei Schnittebenen, die parallel zur $x$, $z$- bzw. $y$, $z$-Koordinatenebene verlaufen. Als Schnittkurven erhält man dann zwei Flächenkurven $K_1$ und $K_2$, mit denen man sich jetzt näher befassen wird.
\newline\newline
Die auf der Schnittkurve $K_1$ gelegenen Punkte stimmen in ihrer $y$-Koordinate miteinander überein: $y=y_P$. Die Höhenkoordinate $z$ dieser Punkte hängt somit nur von der Variablen $x$, d.h. der $x$-Koordinate ab. Diese Funktionsgleichung der Schnittkurve $K_1$ lautet daher
\begin{equation} 
\boxed{K_1:\quad z=f\left(x; y_P\right)=g\left(x\right)}
\end{equation} 
Das Steigungsverhalten dieser Kurve lässt sich besser untersuchen, wenn man die Kurve in die $x$, $z$-Ebene projiziert. Dabei wird die Gestalt der Kurve in kleinster Weise verändert. Für die Steigung $m_x$, der in $P$ errichteten Kurventangente gilt dann definitionsgemäss
\begin{equation}
\boxed{m_x=\tan\left(\alpha\right)=g'\left(x_P\right)=\displaystyle \lim_{\triangle x\rightarrow 0}\dfrac{g\left(x_P+\triangle x\right)-g\left(x_P\right)}{\triangle x}}
\end{equation}
Beachtet man dabei noch, dass $g\left(x\right)=f\left(x; y_P\right)$ ist, so kann man diesen Grenzwert auch wie folgt schreiben
\begin{equation}
\boxed{m_x=\displaystyle \lim_{\triangle x\rightarrow 0}\dfrac{f\left(x_P+\triangle x\right)-f\left(x_P; y_P\right)}{\triangle x}}
\end{equation}
Die erste partielle Ableitung nach $x$ an der Stelle $\left(x_P, y_P\right)$ erfolgt durch Betrachtung der $y$-Variable als Parameter und wird durch das Symbol $f_x\left(x_P, y_P\right)$ oder $z_x\left(x_P, y_P\right)$ oder $\dfrac{\partial z}{\partial x}$ gekennzeichnet.
\newline\newline
Analog für die Schnittkurve $K_2$ gilt 
\begin{equation} 
\boxed{K_2:\quad z=f\left(x_P; y\right)=h\left(y\right)}
\end{equation} 
\begin{equation}
\boxed{m_y=\tan\left(\beta\right)=h'\left(y_P\right)=\displaystyle \lim_{\triangle y\rightarrow 0}\dfrac{h\left(y_P+\triangle y\right)-h\left(y_P\right)}{\triangle y}}
\end{equation}
\begin{equation}
\boxed{m_y=\displaystyle \lim_{\triangle y\rightarrow 0}\dfrac{f\left(x_P; y_P+\triangle y\right)-f\left(x_P; y_P\right)}{\triangle y}}
\end{equation}
Die erste partielle Ableitung nach $y$ an der Stelle $\left(x_P; y_P\right)$ erfolgt durch Betrachtung der $x$-Variable als Parameter und wird durch das Symbol $f_y\left(x_P; y_P\right)$ oder $z_y\left(x_P; y_P\right)$ oder $\dfrac{\partial z}{\partial y}$ gekennzeichnet.
\newline\newline
Unter der partiellen Ableitungen erster Ordnung einer Funktion $z=f\left(x; y\right)$ an der Stelle $\left(x; y\right)$ werden die folgenden Grenzwerte verstanden
\begin{equation}
\boxed{f_x\left(x; y\right)=\displaystyle \lim_{\triangle x\rightarrow 0}\dfrac{f\left(x+\triangle x; y\right)-f\left(x; y\right)}{\triangle x}}\quad \boxed{f_y\left(x; y\right)=\displaystyle \lim_{\triangle y\rightarrow 0}\dfrac{f\left(x; y+\triangle y\right)-f\left(x; y\right)}{\triangle y}}
\end{equation}
Die partielle Ableitung $f_x\left(x_P; y_P\right)$ bzw. $f_y\left(x_P; y_P\right)$ ist der \textbf{Anstieg der Flächentangente im Flächenpunkt} $P$ in der positiven $x$ bzw. $y$-Richtung.
\begin{equation}
\boxed{\dfrac{\partial}{\partial x}\Big[f\left(x; y\right)\Big]=f_x\left(x; y\right)}\quad \boxed{\dfrac{\partial}{\partial y}\Big[f\left(x; y\right)\Big]=f_y\left(x; y\right)}
\end{equation}
\subsection{Partielle Ableitungen höherer Ordnung}
Partielle Ableitungen höherer Ordnung lassen sich auch in Form partieller Differentialquotienten darstellen. So lautet beispielsweose die Schreibweise für partielle Differentialquotienten 2. Ordnung
\begin{equation}
\boxed{f_{xx}=\dfrac{\partial}{\partial x}\Big(\dfrac{\partial}{\partial x}\Big[f\left(x; y\right)\Big]\Big)=\dfrac{\partial^2}{\partial x}\Big[f\left(x; y\right)\Big]}\quad \boxed{f_{yy}=\dfrac{\partial}{\partial y}\Big(\dfrac{\partial}{\partial y}\Big[f\left(x; y\right)\Big]\Big)=\dfrac{\partial^2}{\partial y}\Big[f\left(x; y\right)\Big]}
\end{equation}
\begin{equation}
\boxed{f_{xy}=\dfrac{\partial}{\partial y}\Big(\dfrac{\partial}{\partial x}\Big[f\left(x; y\right)\Big]\Big)=\dfrac{\partial^2}{\partial x \partial y}\Big[f\left(x; y\right)\Big]}\quad \boxed{f_{yx}=\dfrac{\partial}{\partial x}\Big(\dfrac{\partial}{\partial y}\Big[f\left(x; y\right)\Big]\Big)=\dfrac{\partial^2}{\partial y\partial x}\Big[f\left(x; y\right)\Big]}
\end{equation}
Unter bestimmten Voraussetzungen, auf die man im Rahmen dieser Darstellung nur flüchtig eingehen kann, ist bei den gemischten partiellen Ableitungen die Reihenfolge der Differentiationen vertauschbar. Sind nämlich die partiellen Ableitungen $k$-ter Ordnung stetige Funktionen, so gilt der folgende \textbf{Satz von Schwarz}. 
\begin{equation}
\boxed{f_{xy}=f_{yx}}\quad \boxed{f_{xxy}=f_{yxx}=f_{xyx}}\quad \boxed{f_{yxx}=f_{xyy}=f_{yxy}}
\end{equation}
Bei einer gemischten partiellen Ableitung $k$-ter Ordnung darf die Reihenfolge der einzelnen Differentiationsschritte vertauscht werden, wenn die partiellen Ableitungen $k$-ter Ordnung stetige Funktionen sind.
\subsection{Differentiation nach einem Parameter - Verallgemeinerte Kettenregel}
Man betrachte Funktionen von zwei oder mehreren unabhängigen Variablen, die selbst noch von einem Parameter $t_1\leq t\leq t_2$ abhängen. Insbesondere interessiert man sich für die Ableitungen dieser Funktionen nach dem Parameter.
\begin{equation}
\boxed{z=f\Big(x\left(t\right), y\left(t\right)\Big)}
\end{equation}
Diese Funktion wird als eine zusammengesetzte, verkettete oder mittelbare Funktion dieses Parameters bezeichnet. Ihre Ableitung erhält man nach der folgenden, als \textbf{Kettenregel} bezeichnetes Vorschrift.
\begin{equation}
\boxed{\dfrac{\text{d}z}{\text{d}t}=\dfrac{\partial z}{\partial x}\cdot \dfrac{\text{d}x}{\text{d}t}+\dfrac{\partial z}{\partial y}\cdot \dfrac{\text{d}y}{\text{d}t}}
\end{equation}
\subsection{Das totale Differential einer Funktion}
Die Rolle, die die Kurventangente bei einer Funktion von einer Variablen spielt, übernimmt bei einer Funktion $z=f\left(x; y\right)$ von zwei Variablen die sogenannte \textbf{Tangentialebene}. Sie enthält sämtliche im Flächenpunkt $P=\left(x_P; y_P; z_P\right)$ an die Bildfläche von $z=f\left(x; y\right)$ angelegten Tangenten. In der unmittelbaren Umgebung ihres Berührungspunktes $P$ besitzen Fläche und Tangentialebene im Allgemeinen keinen weiteren gemeinsamen Punkt.
\newline\newline
Die Funktionsgleichung dieser Tangentialebene lautet
\begin{equation}
\boxed{z=ax+by+c}
\end{equation}
Die unbekannten Koeffizienten bestimmt man aus den bekannten Eigenschaften der Tangentialebene. So besitzen Fläche und Tangentialebene im Berührungspunkt $P$ den gleichen Anstieg. Dies bedeutet, dass dort die ersten partiellen Ableitungen übereinstimmen müssen. Die partiellen Ableitungen der linearen Funktion sind $f_x\left(x_P; y_P\right)=a$ und $f_y\left(x_P; y_P\right)=b$. Da $P$ ein gemeinsamer Punkt von Fläche und Tangentialebene ist, so erhält man durch Einsetzen der Koordinaten von $P$ in die Gleichung der Tangentialebene den Parameter $c=z_P-ax_P-by_P$.
\newline\newline
Die Gleichung der Tangentialebene an die Fläche $z=f\left(x; y\right)$ im Flächenpunkt $P$ mit $z_P=f\left(x_P; y_P\right)$ lautet in symmetrischer Schreibweise
\begin{equation}
\boxed{z=\underbrace{f_x\left(x_P; y_P\right)}_{a}\cdot x+\underbrace{f_y\left(x_P; y_P\right)}_{b}\cdot y+\underbrace{z_P-\underbrace{f_x\left(x_P; y_P\right)}_{a}\cdot x_P-\underbrace{f_y\left(x_P; y_P\right)}_{b}\cdot y_P}_{c}}
\end{equation}
Die \textbf{Gleichung der Tangentialebene} an die Fläche $z=f\left(x; y\right)$ im Flächenpunkt $P=\left(x_P; y_P; z_P\right)$ mit $z_P=f\left(x_P; y_P\right)$ lautet
\begin{equation}
\boxed{z=\dfrac{\partial}{\partial x}\Big[f\left(x_P; y_P\right)\Big]\cdot \left(x-x_P\right)+\dfrac{\partial}{\partial y}\Big[f\left(x_P; y_P\right)\Big]\cdot \left(y-y_P\right)+z_P}
\end{equation}
Durch das totale Differential kann man Probleme lösen wie die Linearisierung einer Funktion bzw. eines Kennlinienfeldes, implizite Differentiation und Fehlerfortpflanzung lösen. Dabei betrachtet man eine Funktion von zwei unabhängigen Variablen $z=f\left(x; y\right)$ aus auf der sich eine punktförmige Masse $P$ befindet. 
\newline\newline
Die Problemstellung lautet dann welche Änderung erfährt die Höhenkoordinate $z$ des Massenpunktes bei einer Verschiebung auf der Fläche selbst oder auf der zugehörigen Tangentialebene.
\subsubsection{Verschiebung des Massenpunktes auf der Fläche}
Bezogen auf den Ausgangspunkt $P$ bezeichnet man die Reihe nach mit $\triangle x$, $\triangle y$ und $\triangle z$. Die Masse wird nun so auf der Fläche verschoben, dass sich seine beiden unabhängigen Koordinaten $x$ und $y$ um $\triangle x$ bzw. $\triangle y$ ändern. Dabei ändert sich die Höhenkoordinate $z$, d.h. der Funktionswert um
\begin{equation} 
\boxed{\triangle z=f\left(x_P+\triangle x; y_P+\triangle y\right)-f\left(x_P; y_P\right)}
\end{equation} 
Diese Grösse beschreibt den Zuwachs der Höhenkoordinate und damit des Funktionswertes bei einer Verschiebung auf der Fläche. Der Massenpunkt ist dabei vom Punkt $P$ in den Punkt $Q$ gewandert.
\begin{equation}
\boxed{P=\left(x_P; y_P; z_P\right)\Longrightarrow Q=\left(x_P+\triangle x; y_P+\triangle y; z_P+\triangle z\right)}
\end{equation}
\subsubsection{Verschiebung des Massenpunktes auf der Tangtentialebene}
Die mit einer Verschiebung der Masse auf der Tangentialebene verbundenen Koordinatenänderungen bezeichnet man jetzt der Reihe nach mit $\text{d}x$, $\text{d}y$ und $\text{d}z$. Dabei soll der Massenpunkt so auf der Tangentialebene verschoben werden, dass sich seine beiden unabhängigen Koordinaten wiederum um $\triangle x$ bzw. $\triangle y$ ändern.
\newline\newline 
Die Änderung der Höhenkoordinate des Massenpunktes lässt sich dann leicht aus der Funktionsgleichung der Tangentialebene berechnen. So setzt man
\begin{equation}
\boxed{x-x_P=\text{d}x}\quad \boxed{y-y_P=\text{d}y}\quad \boxed{z-z_P=\text{d}z}
\end{equation}
\begin{equation}
\boxed{\text{d}z=\dfrac{\partial}{\partial x}\Big[f\left(x_P; y_P\right)\Big]\cdot \text{d}x+\dfrac{\partial}{\partial y}\Big[f\left(x_P; y_P\right)\Big]\cdot \text{d}y}
\end{equation}
Diese Grösse beschreibt den Zuwachs der Höhenkoordinate $z$ bei einer Verschiebung auf der Tangentialebene. Der Massenpunkt ist dabei vom Ausgangspunkt $P$ in den Punkt $Q'$ gewandert, der zwar auf der Tangentialebene, im Allgemeinen aber nicht auf der Fläche liegt
\begin{equation} 
\boxed{P=\left(x_P; y_P; z_P\right)\Longrightarrow Q'=\left(x_P+\text{d}x; y_P+\text{d}y; z_P+\text{d}z\right)}
\end{equation} 
Es ist somit $\triangle x = \text{d}x$ und $\triangle y = \text{d}y$ aber $\triangle z \neq \text{d}z$. Bei geringfügigen Verschiebungen, d.h. für kleine Werte von $\triangle x = \text{d}x$ und $\triangle y = \text{d}y$ gilt dann näherungsweise
\begin{equation}
\boxed{\triangle z\approx \text{d}z=\dfrac{\partial}{\partial x}\Big[f\left(x_P; y_P\right)\Big]\cdot \triangle x+\dfrac{\partial}{\partial y}\Big[f\left(x_P; y_P\right)\Big]\cdot \triangle y}
\end{equation}
Man darf unter diesen Voraussetzungen die Fläche $z=f\left(x; y\right)$ in der unmittelbaren Umgebung des Berührungspunktes $P$ durch die zugehörige Tangentialebene ersetzen. Diese Näherung wird in der Linearisierung von Funktionen und Kennlinienfeldern und bei der Fehlerfortpflanzung in Gebrauch gemacht.
\newline\newline
Unter dem \textbf{totalen Differential einer Funktion} $z=f\left(x; y\right)$ von zwei unabhängigen Variablen wird folgender lineare Differentialausdruck verstanden.  
\begin{equation}
\boxed{\text{d}z=\dfrac{\partial}{\partial x}\Big[f\left(x; y\right)\Big]\cdot \text{d}x+\dfrac{\partial}{\partial y}\Big[f\left(x; y\right)\Big]\cdot \text{d}y}
\end{equation}
Das totale Differential einer Funktion $z=f\left(x; y\right)$ beschreibt die Änderung der Höhenkoordinate bzw. des Funktionswertes $z$ auf der im Berührungspunkt $P$ errichteten Tangentialebene. Dabei sind die Differentiale $\text{d}x$, $\text{d}y$ und $\text{d}z$ die Koordinaten eines beliebigen Punktes auf der Tangentialebene.
\subsection{Anwendungen}
\subsubsection{Implizite Differentiation}
Hiermit werden ausgehend von impliziten Funktionen der Form $F\left(x; y\right)=0$ ausgegangen und die durch diese Gleichung definierte Kurve als Schnittlinie der Fläche $z=F\left(x; y\right)$ mit der $x$-, $y$-Ebene $z=0$ aufgefasst. Die Kurve $F\left(x; y\right)=0$ ist die \textbf{Schnittkante} der Fläche $z=F\left(x; y\right)$ mit der $x$, $y$-Ebene $z=0$. 
\newline\newline
Unter bestimmten Voraussetzungen ist es dann möglich, den Kurvenanstieg durch die partiellen Ableitungen erster Ordnung von $z=F\left(x; y\right)$ auszudrücken. Zu diesem Zweck bildet man das totale Differential der Funktion $z=F\left(x; y\right)$
\begin{equation}
\boxed{\text{d}z=\dfrac{\partial}{\partial x}\Big[F\left(x; y\right)\Big]\cdot \text{d}x+\dfrac{\partial}{\partial y}\Big[F\left(x; y\right)\Big]\cdot \text{d}y}
\end{equation}
Für die Punkte der \textbf{Schnittpunkte} ist $z=0$ und somit auch $\text{d}z=0$. Dann folgen
\begin{equation}
\boxed{\dfrac{\partial}{\partial x}\Big[F\left(x; y\right)\Big]\cdot \text{d}x+\dfrac{\partial}{\partial y}\Big[F\left(x; y\right)\Big]\cdot \text{d}y=0}
\end{equation}
Teilt man das Polynom durch $\text{d}x$ und durch Auflösen erhält man
\begin{equation}
\boxed{\dfrac{-\dfrac{\partial}{\partial x}\Big[F\left(x; y\right)\Big]}{\dfrac{\partial}{\partial y}\Big[F\left(x; y\right)\Big]}=\dfrac{\text{d}y}{\text{d}x}\quad \left(\dfrac{\partial}{\partial y}\Big[F\left(x; y\right)\Big]\neq 0\right)}
\end{equation}
Der Anstieg einer in der impliziten Form $F\left(x; y\right)=0$ dargestellten Funktionskurve im Kurvenpunkt $P=\left(x_P; y_P\right)$ lässt sich mit Hilfe der partiellen Fifferentiation bestimmen
\begin{equation}
\boxed{\dfrac{\text{d}y}{\text{d}x}\left(x_P; y_P\right)=\dfrac{-\dfrac{\partial}{\partial x}\Big[F\left(x_P; y_P\right)\Big]}{\dfrac{\partial}{\partial y}\Big[F\left(x_P; y_P\right)\Big]}}
\end{equation}
\subsubsection{Linearisierung einer Funktion}
Eine nichtlineare Funktion $y=f\left(x\right)$ lässt sich in einer Umgebung eines Punktes $P=\left(x_P; y_P\right)$ durch eine lineare Funktion bzw. durch eine Kurventangente annähern. Eine Funktion $z=f\left(x; y\right)$ von zwei unabhängigen Variablen lässt sich unter bestimmten Voraussetzungen in der unmittelbaren Umgebung eines Flächenpunktes $P=\left(x_P; y_P; z_P\right)$ linearisieren bzw. durch eine lineare Funktion vom Typ $z=ax+by+cz$ näherungsweise ersetzt werden. Als Ersatzfunktion wählt man die Tangentialebene in $P$. Der Punkt $P$ wird in naturwissenschatlich-technischen Bereich als \textbf{Arbeitspunkt} bekannt.
\newline\newline
Linearisierung einer Funktion $z=f\left(x; y\right)$ bedeutet also, dass man die gekrümmte Bildfläche von $z=f\left(x; y\right)$ in der unmittelbaren Umgebung des Arbeitspunktes $P$ durch die dortige Tangentialebene ersetzt werden kann. Die nichtlineare Funktion $z=f\left(x; y\right)$ wird durch die \textbf{Tangentialebene} bzw. durch das \textbf{totale Differential} angenähert, wobei $\triangle x$, $\triangle y$ und $\triangle z$ die Abweichungen eines beliebigen Flächenpunktes gegenüber dem Arbeitspunkt $P$ sind.
\begin{equation}
\boxed{
\begin{array}{lll}
z-z_P&=&\dfrac{\partial}{\partial x}\left(x_P; y_p\right)\cdot \left(x-x_P\right)+\dfrac{\partial}{\partial y}\left(x_P; y_P\right)\cdot \left(y-y_P\right)\\
\triangle z&=&\dfrac{\partial}{\partial x}\left(x_P; y_p\right)\cdot \triangle x+\dfrac{\partial}{\partial y}\left(x_P; y_P\right)\cdot \triangle y\\
\underbrace{\triangle z}_{w}&=&\underbrace{\left(\dfrac{\partial f}{\partial x}\right)_P}_{a}\cdot \underbrace{\triangle x}_{u}+\underbrace{\left(\dfrac{\partial f}{\partial y}\right)_P}_{b}\cdot \underbrace{\triangle y}_{v}\\
\end{array}
}
\end{equation}
In der Automation sind $u$, $v$ und $w$ die Abweichungen gegenüber dem Arbeitspunkt $P$, d.h. $P$ ist Koordinatenursprung des neuen Koordinatensystems, also die Relativkoordinaten, während $a$ und $b$ die Werte der beiden partiellen Ableitungen erster Ordnung im Arbeitspunkt $P$. 
\newline\newline
Eine Funktion von $n$ unabhängigen Variablen lässt sich linearisieren. In der unmittelbaren Umgebung des Arbeitspunktes $P$ kann die Funktion $y=f\left(x_1; x_2;\dotso; x_n\right)$ näherungsweise durch das \textbf{totale Differential} ersetzt werden. Die Grössen $\triangle x_i$ sind die Änderungen der unabhängigen Variablen bezogen auf den Arbeitspunkt.
\begin{equation}
\boxed{\triangle y=\left(\dfrac{\partial f}{\partial x_1}\right)_P\triangle x_1+\left(\dfrac{\partial f}{\partial x_2}\right)_P\triangle x_2+\dotso+\left(\dfrac{\partial f}{\partial x_n}\right)_P\triangle x_n}
\end{equation}
\subsubsection{Relative oder lokale Extremwerte}
Eine Funktion $z=f\left(x; y\right)$ besitzt an der Stelle $\left(x_P; y_P\right)$ ein \textbf{relatives Maximum} bzw. \textbf{relatives Minimum}, wenn in einer gewissen Umgebung von $\left(x_P; y_P\right)$ stets gilt
\begin{equation}
\boxed{f\left(x_P; y_P\right)>f\left(x; y\right)}\quad \boxed{f\left(x_P; y_P\right)<f\left(x; y\right)}
\end{equation}
Die relativen Maxima und Minima einer Funktion werden unter dem Sammelbegriff ``Relative Extremwerte'' zusammengefasst. Die den Extremwerten entsprechenden Flächenpunkte heissen \textbf{Hoch-} bzw- \textbf{Tiefpunkte}. Ein relativer Extremwert wird auch als lokaler Extremwert bezeichnet, da die extreme Lage meist nur in der unmittelbaren Umgebung, also im lokalen Bereich zutrifft. Ist die Ungleichung an jeder Stelle $\left(x; y\right)$ des Definitionsbereiches von $z=f\left(x; y\right)$ erfüllt, so liegt an der Stelle $\left(x_P; y_P\right)$ ein \textbf{absolutes Maximum} bzw. \textbf{absolutes Minimum} vor. 
\newline\newline
In einem relativen Extremum $\left(x_P; y_P\right)$ der Funktion $z=f\left(x; y\right)$ besitzt die zugehörige Bildfläche stets eine zur $x$, $y$-Ebene parallele Tangentialebene. Folgende Bedingungen sind \textbf{notwendige Voraussetzungen} für die Existenz eines relativen Extremwertes an der Stelle $\left(x_P; y_P\right)$
\begin{equation}
\boxed{\dfrac{\partial f}{\partial x}\left(x_P; y_P\right)=0}\quad \boxed{\dfrac{\partial f}{\partial y}\left(x_P; y_P\right)=0}
\end{equation}
Eine Funktion $z=f\left(x; y\right)$ besitzt an der Stelle $\left(x_P; y_P\right)$ mit Sicherheit einen relativen Extremwert, wenn folgenden \textbf{hinreichenden Voraussetzungen} erfüllt sind
\begin{equation}
\boxed{\triangle =\dfrac{\partial^2 f}{\partial x^2}\left(x_P; y_P\right)\cdot \dfrac{\partial^2 f}{\partial y^2}\left(x_P; y_P\right)-\dfrac{\partial^2 f}{\partial x\partial y}\left(x_P; y_P\right)>0}
\end{equation}
Das Vorzeichen von $\dfrac{\partial^2 f}{\partial x^2}\left(x_P; y_P\right)$ entscheidet dann über die Art des Extremwertes
\begin{equation}
\boxed{\dfrac{\partial^2 f}{\partial x^2}\left(x_P; y_P\right)<0\Longrightarrow \text{Relatives Maximum}}
\end{equation}
\begin{equation}
\boxed{\dfrac{\partial^2 f}{\partial x^2}\left(x_P; y_P\right)>0\Longrightarrow \text{Relatives Minimum}}
\end{equation}
Für $\triangle =0$ liegt kein Extremwert, sondern ein Sattelpunkt liegt vor. Für $\triangle =0$ versagt das Kriterium, d.h. an der Stelle $\left(x_P; y_P\right)$ kann nicht darüber diskutiert werden, ob es sich dabei um einen relativen Extremwert handelt oder nicht.
\subsubsection{Lagrangesches Multiplikatorverfahren zur Lösung einer Extremwertaufgabe mit Nebenbedingungen}
Die Extremwerte einer Funktion $z=f\left(x; y\right)$, deren unabhängige Variable $x$ und $y$ einer Nebenbedingung $\varphi\left(x; y\right)=0$ unterworfen sind, lassen sich mit Hilfe des Lagrangeschen Multiplikatorverfahrens schrittweise wie folgt bestimmen.
\newline \newline
Aus der Funktionsgleichung $z=f\left(x; y\right)$ und der Nebenbedingung $\varphi\left(x; y\right)=0$ wird zunächst die Hilfsfunktion gebildet. Der noch unbekannte Faktor $\lambda$ heisst \textbf{Lagrangescher Multiplikator}.
\begin{equation} 
\boxed{F\left(x; y; \lambda\right)=f\left(x; y\right)+\lambda\cdot \varphi\left(x; y\right)}
\end{equation} 
Dann werden die partiellen Ableitungen erster Ordnung dieser Hilfsfunktion gebildet und gleich Null gesetzt
\begin{equation}
\boxed{
\begin{array}{lllll}
\dfrac{\partial}{\partial x}\Big[F\left(x; y\right)\Big]&=&\dfrac{\partial}{\partial x}\Big[f\left(x; y\right)\Big]+\lambda\cdot \dfrac{\partial}{\partial x}\Big[\varphi\left(x; y\right)\Big]&=&0\\\\
\dfrac{\partial}{\partial y}\Big[F\left(x; y\right)\Big]&=&\dfrac{\partial}{\partial y}\Big[f\left(x; y\right)\Big]+\lambda\cdot \dfrac{\partial}{\partial y}\Big[\varphi\left(x; y\right)\Big]&=&0\\\\
\dfrac{\partial}{\partial \lambda}\Big[F\left(x; y\right)\Big]&=&\varphi\left(x; y\right)&=&0\\\\
\end{array}
}
\end{equation}
Der Lagrangeschen Multiplikator $\lambda$ ist eine Hilfsgrösse und daher meist ohne nähere Bedeutung. Er sollte daher möglichst früh aus den Rechnungen eliminiert werden. Die obigen Bedingungen sind nicht hinreichend für die Existenz eines Extremwertes unter der Nebenbedingung $\varphi\left(x; y\right)=0$.
\newline\newline
Für Funktionen von $n$ Variablen $x_1$, $x_2$, $\dotso$, $x_n$ mit $m$ Nebenbedingungen mit $\left(m<n\right)$ bildet man die Hilfsfunktion
\begin{equation}
\boxed{F\left(x_1; \dotso; x_n; \lambda_1; \dotso; \lambda_m \right)=f\left(x_1; \dotso; x_n\right)+\displaystyle \sum_{i=1}^n\lambda_i\cdot \varphi_i\left(x_1; \dotso; x_n\right)}
\end{equation}
und setzt man die $\left(n+m\right)$ partiellen Ableitungen erster Ordnung dieser Funktion der Reihe nach gleich Null
\begin{equation}
\boxed{\dfrac{\partial}{\partial x_i}\Big[F\left(x_1; \dotso; x_n; \lambda_1; \dotso; \lambda_m \right)\Big]=0}\quad \boxed{\dfrac{\partial}{\partial \lambda_i}\Big[F\left(x_1; \dotso; x_n; \lambda_1; \dotso; \lambda_m \right)\Big]=0}
\end{equation}
Aus diesen $\left(n+m\right)$ Gleichungen lassen sich dann die $\left(n+m\right)$ Unbekannten $x_1$, $x_2$, $\dotso$, $x_n$, $\lambda_1$, $\lambda_2$, $\dotso$, $\lambda_m$ berechnen. 
\subsubsection{Lineare Fortpflanzung - Direkte Messung}
Das Messergebnis einer aus $n$ Meswerten bestehenden Messreihe $x_1$, $x_2$, $\dotso$, $x_n$ wird in folgender Form ausgedrückt
\begin{equation}
\boxed{x=\overline{x}\pm \triangle x}\quad \boxed{\overline{x}=\dfrac{1}{n}\cdot \displaystyle \sum_{i=1}^nx_i}\quad \boxed{\triangle x=s_{\overline{x}}=\sqrt{\dfrac{1}{n\left(n-1\right)}\cdot \displaystyle \sum_{i=1}^n\left(x_i-\overline{x}\right)^2}}
\end{equation}
wobei $\overline{x}$ das arithmetische Mittel der $n$ Einzelwerte, $s_{\overline{x}}$ die Standardabweichung des Mittelwertes und $\triangle x$ die \textbf{Messunsicherheit} der Grösse $x$. Die auftretende Summe $\displaystyle \sum_{i=1}^n\left(x_i-\overline{x}\right)^2$ heisst Summe der \textbf{Abweichungsquadrate}.
\subsubsection{Lineare Fortpflanzung - Indirekte Messung}
Das Messergebnis zweier direkt gemessener Grössen $x$ und $y$ liege in der Form
\begin{equation}
\boxed{x=\overline{x}\pm\triangle x=\overline{x}\pm s_{\overline{x}}}\quad \boxed{y=\overline{y}\pm\triangle y=\overline{y}\pm s_{\overline{y}}}
\end{equation}
Dabei sind $\overline{x}$ und $\overline{y}$ die arithmetischen Mittelwerte und $\triangle x$ und $\triangle y$ die Messunsicherheiten der beiden Grössen, für die man in diesem Zusammenhang meist die Standardabweichung $s_{\overline{x}}$ und $s_{\overline{y}}$ der beide Mittelwerte heranzieht. 
\newline\newline
Die von den direkten Messgrössen $x$ und $y$ abhängige indirekte Messgrösse $z=f\left(x; y\right)$ besitzt dann den Mittelwert
\begin{equation}
\boxed{\overline{z}=f\left(\overline{x}; \overline{y}\right)}
\end{equation}
Mit Hilfe des totalen Differentials der Funktion gelingt es dann, ein sogenannter Fehlerfortpflanzungsgesetz herzuleiten, d.h. eine Beziehung, die darüber Aufschluss gibt, wie sich die Messunsicherheiten $\triangle x$ und $\triangle y$ der beiden unabhängigen Messgrössen $x$ und $y$ auf die Messunsicherheit $\triangle z$ der abhängigen Grösse $z=f\left(x; y\right)$ auswirken. Zu diesem Zweck bildet man das totale Differential der Funktion $z=f\left(x; y\right)$ an der Stelle $x=\overline{x}$ und $y=\overline{y}$
\begin{equation}
\boxed{\text{d}z=\dfrac{\partial f}{\partial x}\left(\overline{x}; \overline{y}\right)\text{}d x+\dfrac{\partial f}{\partial y}\left(\overline{x}; \overline{y}\right)\text{d} y}
\end{equation}
Die Differentiale $\text{d}x$ und $\text{d}y$ deuten als Messunsicherheiten $\triangle x$ und $\triangle y$ der beiden unabhängigem Messgrössen $x$ und $y$. Das totale Differential $\text{d}z$ liefert einen Näherungswert für die Messunsicherheit $\triangle z$.
\begin{equation}
\boxed{\triangle z_{\text{max}}=\Big\vert \dfrac{\partial f}{\partial x}\left(\overline{x}; \overline{y}\right)\triangle x\Big\vert+\Big\vert \dfrac{\partial f}{\partial y}\left(\overline{x}; \overline{y}\right)\triangle y\Big\vert}
\end{equation}
Das Messergebnis für die indirekte Messgrösse $z=f\left(x; y\right)$ wird dann in der Form
\begin{equation}
\boxed{z=\overline{z}\pm \triangle z_{\text{max}}}
\end{equation}
\section{Doppelintegrale}
Ein Mehrfachintegral besteht aush mehreren nacheinander auszuführende gewöhnliche Integrationen. Legt man ein Koordinatensystem zugrunde, das sich der Symmetrie des Problems in besonders günstiger Weise anpasst, so verainfacht sich die Berechnung der Integrale oft erheblich. Bei ebenen Problemen mit Kreissymmetrie etwa wird man daher vorzugsweise \textbf{Polarkoordinaten}, bei rotationssymmetrischen Problemen zweckmässigerweise \textbf{Zylinderkoordinaten} verwenden.
\newline\newline
Der Begriff des Doppelintegrals lässt sich anhand eines geometrischen Problems einführen. Sei $z=f\left(x; y\right)$ im Bereich $\left(A\right)$ eine definierte und stetige Funktion mit $f\left(x; y\right)\geq 0$. Sein Boden besteht aus dem Bereich $\left(A\right)$ und die Mantellinien verlaufen parallel zur $z$-Achse. Man ist hier für das Volumen $V$ interessiert.
\newline\newline
Der Bereich $\left(A\right)$ wird in $n$ Teilbereiche it den Flächeninalten $\triangle A_i$ zerlegt. Der Zylinder selbst erfällt dabei in eine leich grosse Anzahl von Röhren. Betrachte man den $k$-ten Rohr mit flachen Boden $\triangle A_k$ und gekrümmter Deckel als Funktion $z=f\left(x; y\right)$. Das Volumen diesen Rohr stimmt dann mit dem Volumen einer Säule überein, die über der gleichen Grundfläche errichtet wird und deren Höhe durch die Höhenkoordinate $z_k=f\left(x_k; y_k\right)$ des Flächenpunktes $P_k=\left(x_k; y_k\right)$ gegeben ist.
\begin{equation}
\boxed{\triangle V_k\approx \left(\triangle A_k\right)z_k=z_k\triangle A_k=f\left(x_k; y_k\right)\triangle A_k}
\end{equation}
Dieser Näherungswert lässt sich noch verbessern, wenn man in geeigneter Weise die Anzahl der Röhren vergrössert. Die Anzahl der Teilbereiche $n$ wachsen unbegrenzt $\left(n\rightarrow \infty\right)$, wobei gleichzeitig der Durchmesser eines jeden Teilbereiches gegen Null streben soll. Bei diesem Grenzübergang strebt die Summe gegen einen Grenzwert über den Bereich $\left(A\right)$.
\newline\newline
Bei diesem Grenzübergang strebt die Summe gegen einen Bereich $\left(A\right)$ oder kurz \textbf{Doppelintegral}, \textbf{Flächenintegral} oder \textbf{2-dimensionales Gebietsintegral} und darf als Volumen $V$ des Körpers interpretiert werden. Hier sind $x$ und $y$ die Integrationsvariablen, $f\left(x; y\right)$ die Integrandfunktion oder der Integrand, $\text{d}A$ das Flächendifferential oder Flächenelement und $\left(A\right)$ der Integrationsbereich.
\begin{equation}
\boxed{\displaystyle \lim_{\substack{n\rightarrow \infty\\\triangle A_k\rightarrow 0}} \displaystyle \sum_{k=1}^nf\left(x_k; y_K\right)\triangle A_k}
\end{equation}
\subsection{Doppelintegral in kartesische Koordinaten}
Die Berechnung des Doppelintegrals wird in kartesische Koordinaten betrachtet. Ein Integrationsbereich lässt sich durch die Ungleichungen
\begin{equation}  
\boxed{f_u\left(x\right)\leq y\leq f_o\left(x\right),\quad a\leq x\leq b}
\end{equation}
wobei $f_u\left(x\right)$ die untere und $f_o\left(x\right)$ die obere Randkurve ist und die seitlichen Begrenzungen aus zwei Parallelen zur $y$-Achsen mit den Funktionsgleichungen $x=a$ und $x=b$ bestehen. Das Flächenelement $\text{d}A$ besitzt in der kartesischen Darstellung die geometrische Form eines achsenparallelen Rechtecks mit den infinitesimal kleinen Seitenlängen $\text{d}x$ und $\text{d}y$. Somit gilt
\begin{equation}
\boxed{\text{d}A=\text{d}x\,\text{d}y=\text{d}y\,\text{d}x}
\end{equation}
Über den Flächenelement liegt eine quaderförmige Säule mit dem infinitesimal kleinen Rauminhalt. Das Volumen $V$ des Zylinders berechnet man schrittweise durch Summation der Säulenvolumina.
\begin{equation}
\boxed{\text{d}V=z\,\text{d}A=f\left(x; y\right)\,\text{d}x\,\text{d}y=f\left(x; y\right)\,\text{d}y\,\text{d}x}
\end{equation}
Die Berechnung des Doppelintegrals erfolgt durch zwei nacheinander auszuführende gewöhnliche Integrationen. 
\newline\newline
\textbf{Innere Integration:} Die Variable $x$ wird zunächst als eine Art Konstante betrachtet und die Funktion $f\left(x; y\right)$ unter Verwendung der für gewöhnliche Integrale gültigen Regeln nach der Variablen $y$ integriert. In die ermittelte Stammfunktion setzt man dann für $y$ die Integrationsgrenzen $f_o\left(x\right)$ bzw. $f_u\left(x\right)$ ein und bildet die entsprechende Differenz. 
\newline\newline
\textbf{Äussere Integration:} Die als Ergebnis der inneren Integration erhaltene, nur noch von der Variablen $x$ abhängige Funktion wird nun in den Grenzen von $x=a$ bis $x=b$ integriert.
\begin{equation}
\boxed{V=\displaystyle \iint_{\left(A\right)}f\left(x; y\right)\,\text{d}A=\displaystyle \int_{x=a}^{b}\text{d}V_{\text{Scheibe}}=\underbrace{\displaystyle \int_{x=a}^b\underbrace{\left(\displaystyle \int_{y=f_a\left(x\right)}^{f_o\left(x\right)}f\left(x; y\right)\,\text{d}y\right)}_{\text{inneres Integral}}\,\text{d}x}_{\text{äusseres Integral}}}
\end{equation}
im Allgemeinen gilt: bei einer Vertauschung der Integrationsreihenfolge müssen die Integrationsgrenzen jeweils neu bestimmt werden.
\subsection{Doppelintegral in Polarkoordinaten}
Die Berechnung des Doppelintegrals vereinfacht sich, wenn man an Stelle der kartesischen Koordinaten $x$ und $y$ die Polarkoordinaten $r$ und $\varphi$ verwendet. Zwischen ihnen besteht dabei der folgende Zusammenhang
\begin{equation} 
\boxed{x=r\cdot \cos\left(\varphi\right)}\quad \boxed{y=r\cdot \sin\left(\varphi\right)}\quad \boxed{r\geq 0}\quad \boxed{0\leq \varphi\leq 2\pi}
\end{equation}
Die Gleichung einer Kurve lautet in Polarkoordinaten $r=f\left(\varphi\right)$ oder $r=r\left(\varphi\right)$. Eine von zwei Variablen $x$ und $y$ abhängige Funktion $z=f\left(x; y\right)$ geht bei der \textbf{Koordinatentransformation} in die von $r$ und $\varphi$ abhängige Funktion über
\begin{equation}
\boxed{z=f\left(x; y\right)=f\left(r\cos\left(\varphi\right); r\sin\left(\varphi\right)\right)=F\left(r; \varphi\right)}
\end{equation}
Die bei Doppelintegralen in Polarkoordinatendarstellung auftretenden Integrationsbereiche $\left(A\right)$ besitzt die Gestalt zwei Strahlen $\varphi=\varphi_1$ und $\varphi=\varphi_2$ sowie einer inneren Kurve $r=r_i\left(\varphi\right)$ und einer äusseren Kurve $r=r_a\left(\varphi\right)$ begrenzt und lassen sich durch die Ungleichungen beschreiben
\begin{equation}
\boxed{r_i\left(\varphi\right)\leq r\leq r_a\left(\varphi\right)}\quad \boxed{\varphi_1\leq \varphi\leq \varphi_2} 
\end{equation}
Das Flächenelement $\text{d}A$ wird in der Polarkoordinaten von zwei infinitesimal benachbarten Kreisen mit den Radien $r$ und $r+\text{d}r$ und zwei infinitesimal benachbarten Strahlen mit den Polarwinkeln $\varphi$ und $\varphi+\text{d}\varphi$ berandet. Dabei gilt
\begin{equation}
\boxed{\text{d}A=\left(r\,\text{d}\varphi\right)\text{d}r=r\,\text{d}r\,\text{d}\varphi}
\end{equation}
Das Doppelintegral besitzt somit in Polarkoordinaten das folgende Aussehen und die Berechnung erfolgt wiederum von innen nach aussen, zuerst radial zwischen den Randkurven $r=r_i\left(\varphi\right)$ und $r=r_a\left(\varphi\right)$ integriert und anschliessend in Winkelrichtung $\varphi=\varphi_1$ und $\varphi=\varphi_2$. Wird die Reihenfolge der Integrationen geändert, dann müssen auch die Integrationsgrenzen neu bestimmt werden.
\begin{equation} 
\boxed{\displaystyle \iint_{\left(A\right)}f\left(x; y\right)\,\text{d}A=\displaystyle \int_{\varphi=\varphi_1}^{\varphi=\varphi_2}\displaystyle \int_{r=r_i\left(\varphi\right)}^{r_a\left(\varphi\right)}f\left(r\cdot \cos\left(\varphi\right); r\cdot \sin\left(\varphi\right)\right)\cdot \underbrace{r\,\text{d}r\,\text{d}\varphi}_{\text{d}A}}
\end{equation} 
\subsection{Anwendungen der Doppelintegrale}
\subsubsection{Flächeninhalt}
Der Flächeninhalt $A$ eines Normalbereichs $\left(A\right)$ lässtsich nach dem Baukastenprinzip aus infinitesimal kleinen rechteckigen Flächenelementen vom Flächeninhalt $\text{d}A=\text{d}y\,\text{d}x$ zusammensetzen. Man betrachte einen in der Fläche liegenden und zur $y$-Achse parallelen Streifen der Breite $\text{d}x$. Der Flächeninhalt eines Streifens erhält man, indem man den Flächeninhalt sämtlicher im Streifen gelegener Flächenelemente aufsummiert. Die Summation der Flächenelemente bedeutet eine Integration in der $y$-Richtung zwischen der unteren Grenze $y=f_u\left(x\right)$ und der oberen Grneze $y=f_o\left(x\right)$. Der Flächeninhalt eines solchen infinitesimal schmalen Streifens beträgt
\begin{equation}
\boxed{\text{d}A_{\text{Streifen}}=\displaystyle \int_{y=f_u\left(x\right)}^{f_o\left(x\right)}\text{d}A=\left(\displaystyle \int_{y=f_u\left(x\right)}^{f_o\left(x\right)}\text{d}y\right)\text{d}x}
\end{equation}
Jetzt summieren über sämtliche Streifenelemente $x=a$ und $x=b$. Der Flächeninhalt in \textbf{kartesische Koordinaten} $A$ lautet
\begin{equation}
\boxed{A=\displaystyle \iint_{\left(A\right)}\text{d}A=\displaystyle \int_{x=a}^{b}\text{d}A_{\text{Streifen}}=\displaystyle \int_{x=a}^b\displaystyle \int_{y=f_u\left(x\right)}^{f_o\left(x\right)}1\,\text{d}y\,\text{d}x}
\end{equation}
Bei Verwendung von \textbf{Polarkoordinaten} lautet 
\begin{equation}
\boxed{A=\displaystyle \int_{\varphi=\varphi_1}^{\varphi=\varphi_2}\displaystyle \int_{r=r_i\left(\varphi\right)}^{r=r_a\left(\varphi\right)}r\,\text{d}r\,\text{d}\varphi}
\end{equation}
\subsubsection{Schwerpunkt einer homogenen Fläche}
Für die Schwerpunktskoordinaten $x_S$ und $y_S$ einer homogenen ebenen Fläche vom Flächenihalt $A$ bei Verwendung kartesischer Koordinaten ist für das Flächenelement $\text{d}A=\text{dy}\,\text{d}x$ zu setzen, bei Verwendung von Polarkoordinaten setzt man $x=r\cdot \cos\left(\varphi\right)$, $y=r\cdot \sin\left(\varphi\right)$ und $\text{d}A=r\,\text{d}r\,\text{d}\varphi$
\begin{equation}
\boxed{x_S=\dfrac{1}{A}\displaystyle \iint_{\left(A\right)}x\,\text{d}A}\quad \boxed{y_S=\dfrac{1}{A}\displaystyle \iint_{\left(A\right)}y\,\text{d}A}
\end{equation}
Für \textbf{kartesische Koordinaten} lauten die Koordinaten des Schwerpunktes
\begin{equation}
\boxed{x_S=\dfrac{1}{A}\displaystyle \int_{x=a}^{b}\displaystyle \int_{y=f_u\left(x\right)}^{f_o\left(x\right)}x\,\text{d}y\,\text{d}x}\quad \boxed{y_S=\dfrac{1}{A}\displaystyle \int_{x=a}^{b}\displaystyle \int_{y=f_u\left(x\right)}^{f_o\left(x\right)}y\,\text{d}y\,\text{d}x}
\end{equation}
Für \textbf{Polarkoordinaten} lauten die Koordinaten des Schwerpunktes
\begin{equation}
\boxed{x_S=\dfrac{1}{A}\displaystyle \int_{\varphi=\varphi_1}^{\varphi_2}\displaystyle \int_{r=r_i\left(\varphi\right)}^{r_a\left(\varphi\right)}r^2\cdot \cos\left(\varphi\right)\,\text{d}r\,\text{d}\varphi}\quad \boxed{y_S=\dfrac{1}{A}\displaystyle \int_{\varphi=\varphi_1}^{\varphi_2}\displaystyle \int_{r=r_i\left(\varphi\right)}^{r_a\left(\varphi\right)}r^2\cdot \sin\left(\varphi\right)\,\text{d}r\,\text{d}\varphi}
\end{equation}
\subsubsection{Flächenmomente}
Flächenmomente sind Grössen, die im Zusammenhang mit Biegeproblemen bei Balken und Trägern auftreten. Dabei wird noch zwischen axialen oder äquatorialen und polaren Flächenmomenten unterschieden. Bei einem axialen Flächenmoment liegt die Bezugsachse in der Flächenebene, während sie bei einem polaren Flächenmoment senkrecht zur Flächenebene orientiert ist.
\begin{equation} 
\boxed{\text{d}I_x=y^2\,\text{d}A}
\end{equation}
Man behandle zunächst die auf die Koordinatenachsen bezogenen Flächenmomente. Definitionsgemäss wird dabei die infinitesiml kleine Grösse als axiales Flächenelement von $\text{d}A$ bezüglich der $x$-Achse bezeichnet. Sie ist das Produkt aus dem Flächenelement $\text{d}A$ und dem Quadrat des Abstandes, den dieses Flächenelemtn von der $x$-Achse besitzt.
\newline\newline
Durch Integration über die Gesamtfläche $A$ erhält man hieraus das axiale Flächenmoment $I_x$ der Fläche $A$ bezüglich der $x$-Achse
\begin{equation}
\boxed{I_x=\displaystyle \int_{\left(A\right)}\text{d}I_x=\displaystyle \iint_{\left(A\right)}y^2\,\text{d}A}
\end{equation}
Analog wird das axiale Flächenmoment $I_y$ der Fläche $A$ bezüglich der $y$-Achse als Bezugsachse definiert. Ausgehend von dem Beitrag $\text{d}I_y=x^2\,\text{d}A$ eines Flächenelements $\text{d}A$ erhält man durch Integration das Flächenmoment der Gesamtfläche
\begin{equation}
\boxed{I_y=\displaystyle \iint_{\left(A\right)}\,\text{d}I_y=\displaystyle \iint_{\left(A\right)}x^2\,\text{d}A}
\end{equation}
Unter dem \textbf{polaren Flächenmoment} $I_p$ einer Fläche $A$, bezogen auf eine durch den Koordinatenursprung senkrecht zur Flächenebene verlaufende Bezugsachse ($z$-Achse), wird die wie folgt definierte Grösse verstanden wobei $\text{d}I_p=r^2\,\text{d}A$
\begin{equation}
\boxed{I_p=\displaystyle \iint_{\left(A\right)}\text{d}I_p=\displaystyle \iint_{\left(A\right)}r^2\,\text{d}A}
\end{equation}
Wegen des Satzes von Pythagoras besteht zwischen den beiden axialen und dem polaren Flächenmoment stets die Beziehung
\begin{equation}
\boxed{I_p=I_x+I_y}
\end{equation}
So gelten in \textbf{kartesische Koordinaten} 
\begin{equation}
\boxed{I_x=\displaystyle \int_{x=a}^{b}\displaystyle \int_{y=f_u\left(x\right)}^{f_o\left(x\right)}y^2\,\text{d}y\,\text{d}x}\quad \boxed{I_y=\displaystyle \int_{x=a}^{b}\displaystyle \int_{y=f_u\left(x\right)}^{f_o\left(x\right)}x^2\,\text{d}y\,\text{d}x}
\end{equation}
\begin{equation}
\boxed{I_p=\displaystyle \int_{x=a}^{b}\displaystyle \int_{y=f_u\left(x\right)}^{f_o\left(x\right)}\left(x^2+y^2\right)\,\text{d}y\,\text{d}x}
\end{equation}
und in \textbf{Polarkoordinaten}
\begin{equation}
\boxed{I_x=\displaystyle \int_{\varphi=\varphi_1}^{\varphi_2}\displaystyle \int_{r=r_i\left(x\right)}^{r_a\left(\varphi\right)}r^3\cdot \sin^2\left(\varphi\right)\,\text{d}r\,\text{d}\varphi}\quad \boxed{I_y=\displaystyle \int_{\varphi=\varphi_1}^{\varphi_2}\displaystyle \int_{r=r_i\left(x\right)}^{r_a\left(\varphi\right)}r^3\cdot \cos^2\left(\varphi\right)\,\text{d}r\,\text{d}\varphi}
\end{equation}
\begin{equation}
\boxed{I_p=\displaystyle \int_{\varphi=\varphi_1}^{\varphi_2}\displaystyle \int_{r=r_i\left(x\right)}^{r_a\left(\varphi\right)}r^3\,\text{d}r\,\text{d}\varphi}
\end{equation}
\section{Dreifachintegrale}
Die Integration einer Funtion von drei unabhängigen Variablen hat keine geometrische Interpretation.Sei $u=f\left(x; y; z\right)$ eine im räumlichen Bereich $\left(V\right)$ definierte und stetige Funktion. Den Bereich unterteilt man zunächst in $n$ Teilbereiche. Mit dem $k$-ten Teilbereich vom Volumen $\triangle V_k$ wird man sich eingehender befassen. In diesem Teilbereich wählt man einen beliebigen Punkt $P_k=\left(x_k; y_K; z_k\right)$ aus, berechnet an dieser Stelle den Funktionswert $u_k=f\left(x_k; y_k; z_k\right)$ und bildet schliesslich das Produkt aus Funktionswert und Volumen: $f\left(x_k; y_k; z_k\right)\cdot \triangle V_k$.
\newline\newline
Die Summe aller dieser Produkte wird Zwischensumme $Z_n$ genannt
\begin{equation}
\boxed{Z_n=\displaystyle \sum_{k=1}^n f\left(x_k; y_k; z_k\right)\triangle V_k}
\end{equation}
Man lässt die Anzahl $n$ der Teilbereiche unbegrenzt wachsen $\left(n\rightarrow \infty\right)$, wobei gleichzeitig der Durchmesser eines jeden Teilbereiches gegen Null gehen soll. 
\begin{equation}
\boxed{\displaystyle \iiint_{\left(V\right)}f\left(x; y; z\right)\,\text{d}V=\displaystyle \lim_{n\rightarrow \infty}Z_n=\displaystyle \lim_{n\rightarrow \infty}\displaystyle \sum_{k=1}^n f\left(x_k; y_k; z_k\right)\triangle V_k}
\end{equation}
Bei diesem Grenzübergang strebt die Zwischensumme $Z_n$ gegen einen Grenzwert, der als \textbf{3-dimensionales Bereichsintegral} von $f\left(x; y; z\right)$ über $\left(V\right)$ oder \textbf{Dreifachintegral} bezeichnet wird, wobei $x$, $y$ und $z$ die Integrationsvariablen, $f\left(x; y; z\right)$ die Integrandfunktion, $\text{d}V$ das Volumendifferential oder Volumenelement und $\left(V\right)$ der räumlicher Integrationsbereich oder Körper.
\subsection{Dreifachintegral in kartesische Koordinaten}
Der Berechnung eines Dreifachintegrals legt man zunächst ein kartesisches Koordinatensystem und einen zylinderischen Integrationsbereich $\left(V\right)$ zugrunde, der unten durch eine Fläche $z=z_u\left(x; y\right)$ und oben durch eine Fläche $z=z_o\left(x; y\right)$ begrenzt wird. Die Projektion dieser Begrenzungsflächen in die $x$, $y$-Ebene führt zu einem Bereich $\left(A\right)$, der durch die Kurven $y=f_u\left(x\right)$ und $y=f_o\left(x\right)$ sowie die Parallelen $x=a$ und $x=b$ berandet wird.
\newline\newline
Der zylindrische Integrationsbereich $\left(V\right)$ kann somit durch die Ungleichungen beschrieben werden
\begin{equation}
\boxed{z_u\left(x; y\right)\leq z\leq z_o\left(x; y\right)}\quad \boxed{f_u\left(x\right)\leq y\leq f_o\left(x\right)}\quad \boxed{a\leq x\leq b}
\end{equation}
Das Volumenelement $\text{d}V$ besitzt in der kartesischen Darstellung die geometrische Form eines Quaders mit den infinitesimal kleinen Seitenlängen $\text{d}x$, $\text{d}y$ und $\text{d}z$. Ein Dreifachintegral lässt sich durch drei nacheinander auszuführende gewöhnliche Integrationen durchführen
\begin{equation}
\boxed{\displaystyle \iiint_{\left(V\right)}f\left(x; y; z\right)\,\text{d}V=\underbrace{\displaystyle \int_{x=a}^b\underbrace{\left(\displaystyle \int_{y=f_a\left(x\right)}^{f_o\left(x\right)}\underbrace{\left(\displaystyle \int_{y=f_a\left(x\right)}^{f_o\left(x\right)}\,\text{d}y\right)}_{\text{Integration 1}}\,\text{d}y\right)}_{\text{Integration 2}}\,\text{d}x}_{\text{Integration 3}}}
\end{equation}
Dabei wird wie bei den Doppelintegralen von innen nach aussen integriert. Die Integrationsreihenfolge ist nur dann vertauschbar, wenn sämtliche Integrationsgrenzen konstant sind. Bei der Vertauschung der Integrationsreihenfolge müssen die Integrationsgrenzen neu bestimmt werden.
\subsection{Dreifachintegral in Zylinderkoordinaten}
In technischen Anwendungen treten häufig Körper mit Rotationssymmetrie auf. Zu ihrer Beschreibung verwendet man zweckmässigerweise Zylinderkoordinaten $\left(r; \varphi; z\right)$, die sich der Symmetrie des Körpers in besonderem Masse anpassen. Man spricht in diesem Zusammenhang auch von symmetriegerechten Koordinaten. Auch die Berechnung eines Dreifachintegrals lässt sich in zahlreichen Fällen bei Verwendung von Zylinderkoordinaten erheblich vereinfachen.
\newline\newline
Sei $P=\left(x; y; z\right)$ ein beliebiger Punkt des kartesischen Raumes, $P'=\left(x; y\right)$ der durch senkrechte Projektion von $P$ in die $x$, $y$-Ebene erhaltene Bildpunkt. Die Lage von $P'$ kann man auch durch die Polarkoordinaten $r$ und $\varphi$ beschreiben. Sie bestimmen zugleich zusammen mit der Höhenkoordinate $z$ in eindeutiger Weise die Lage des Raumpunktes $P$. Die drei Koordinaten $r$, $\varphi$ und $z$ werden als Zylinderkoordinaten von $P$ bezeichnet.
\newline\newline
Zwischen den kartesischen Koordinate und den Zylinderkoordinaten bestehen dabei die folgenden Umrechnungen
\begin{equation} 
\boxed{x=r\cdot \cos\left(\varphi\right)}\quad \boxed{y=r\cdot \sin\left(\varphi\right)}\quad \boxed{z=z}
\end{equation}
\begin{equation}
\boxed{r=\sqrt{x^2+y^2}}\quad \boxed{\tan\left(\varphi\right)=\dfrac{y}{x}}\quad \boxed{z=z}
\end{equation}
Das Volumenelement $\text{d}V$ besitzt in Zylinderkoordinaten die Form
\begin{equation}
\boxed{\text{d}V=\left(\text{d}A\right)\text{d}z=\left(r\,\text{d}r\,\text{d}\varphi\right)\text{d}z=r\,\text{d}z\,\text{d}r\,\text{d}\varphi}
\end{equation}
Für ein Dreifachintegral erhält man dann in Zylinderkoordinaten die Darstellung
\begin{equation}
\boxed{\displaystyle \iiint_{\left(V\right)}f\left(x; y; z\right)\,\text{d}V=\displaystyle \iiint_{\left(V\right)}f\left(r\cdot \cos\left(\varphi\right); r\cdot \sin\left(\varphi\right); z\right)\cdot \underbrace{r\,\text{d}z\,\text{d}r\,\text{d}\varphi}_{\text{d}V}}
\end{equation}
Die Mantelfläche eines rotationssymmetrischen Körpers entsteht durch Drehung einer Kurve $z=f\left(x\right)$ und die $z$-Achse, die damitauch zugleich Symmetrieachse ist. Bei der Rotation wird aus der kartesischen Koordinate $x$ die Zylinderkoordinate $r$ und die Kurvengleichung $z=f\left(x\right)$ geht dabei in die Funktionsgleichung $z=f\left(r\right)$ der Rotationsfläche $\left(x\rightarrow r\right)$ über.
\subsection{Anwendungen des Dreifachintegrals}
\subsubsection{Volumen und Masse eines Körpers}
Das Volumen eines zylindrischen Körpers mit Grundfläche $z=z_u\left(x; y\right)$ und Deckelfläche $z=z_o\left(x\right)$ kann man durch ein Dreifachintegral berechnen. Durch Projektion des Zylinders in die $x$, $y$-Ebene erhält man den Normalbereich $\left(A\right)$ durch die Kurven $y_u=f_u\left(x\right)$ und $y=f_o\left(x\right)$ sowie die Parallelen $x=a$ und $x=b$.
\begin{equation}
\boxed{V=\displaystyle \iiint_{\left(V\right)}\text{d}V=\displaystyle \iiint_{\left(V\right)}\text{d}z\,\text{d}y\,\text{d}x}
\end{equation}
Betrachte man nun ein infinitesimal kleines, im Körper gelegenes Volumenelement $\text{d}V$. Es besitzt in einem karteischen Koordinatensystem bekanntlich die Gestalt eines achsenparallelen Quaders mit den Kantenlängen $\text{d}x$, $\text{d}y$ und $\text{d}z$. Sein Volumen beträgt $\text{d}V=\text{d}x\,\text{d}y\,\text{d}z=\text{d}z\,\text{d}y\,\text{d}x$
\newline\newline
\textbf{Volumen einer Säule:} Betrachte man eine zur $z$-Achse parallele Säule mit der infinitesimal kleiinen Querschnittsfläche $\text{d}A=\text{d}x\,\text{d}y=\text{d}y\,\text{d}x$, indem Volumenelement über Volumenelement gesetzt wird, bis man an die beiden Begrenzungsflächen des Körpers stösst. Das Volumen dieser Säule erhält man durch Summation sämtlicher in der Säule gelegener Volumenelemente, d.h. durch Integration der Volumenelemente $\text{d}V$ in der $z$-Richtung zwischen den Grenzen $z=z_u\left(x; y\right)$ und $z=z_o\left(x; y\right)$. Das infinitesimal kleine Säulenvolumen beträgt
\begin{equation}
\boxed{\text{d}V_{\text{Säule}}=\displaystyle \int_{z=z_u\left(\right)}^{z=z_o\left(\right)}\text{d}V=\left(\displaystyle \int_{z=z_u\left(x; y\right)}^{z=z_o\left(x; y\right)}\text{d}z\right)\text{d}y\,\text{d}x}
\end{equation}
\textbf{Volumen einer Scheibe:} Legt man zur $y$-Richtung Säule an Säule, bis man an die Randkurven $y=f_u\left(x\right)$ bzw. $y=f_o\left(x\right)$ des Bereiches $\left(A\right)$ in der $x$, $y$-Ebene stossen und erhält eine Volumenschicht der Breite oder Dicke $\text{d}x$. Das Volumen $\text{d}V_{\text{Scheibe}}$ dieser infinitesimal dünnen Scheibe ergibt sich dann durch Summation der Säulenvolumina, d.h. durch INtegration der Säulenvolumina $\text{d}V_{\text{Säule}}$ in der $y$-Richtung zwischen den Grenzen $y=f_u\left(x\right)$ und $y=f_o\left(x\right)$.  
\begin{equation}
\boxed{\text{d}V_{\text{Säule}}=\displaystyle \int_{y=f_u\left(x\right)}^{f_o\left(x\right)}\text{d}V_{\text{Säule}}=\left(\displaystyle \int_{y=f_u\left(x\right)}^{f_o\left(x\right)}\left(\displaystyle \int_{z=z_u\left(x; y\right)}^{z_o\left(x; y\right)}\text{d}z\right)\text{d}y\right)\text{d}x}
\end{equation}
\textbf{Volumen des Zylinders:} Es wird in der $x$-Richtung Scheibe an Scheibe gelegt, bis der zylindrische Körper vollständig ausgefüllt ist. Das Zylindervolumen $V$ erhält man dann durch Summation über die Volumina sämtlicher Scheiben, d.h. durch Integration in der $x$-Richtung zwischen den Grenzen $x=a$ und $x=b$. Es gilt
\begin{equation}
\boxed{V=\displaystyle \iiint_{\left(V\right)}\text{d}V=\displaystyle \int_{x=a}^{b}\text{d}V_{\text{Scheibe}}=\displaystyle \int_{x=a}^b\displaystyle \int_{y=f_u\left(x\right)}^{f_o\left(x\right)}\displaystyle \int_{z=z_u\left(x; y\right)}^{z_o\left(x; y\right)}\text{d}z\,\text{d}y\,\text{d}x}
\end{equation}
\subsubsection{Schwerpunkt eines Körpers}
Die Berechnung des Schwerpunktes eines homogenen Körpers für die Schwerpunktkoordinaten $x_S$, $y_S$ und $z_S$ berechnet sich wie folgt
\begin{equation}
\boxed{x_S=\dfrac{1}{V}\cdot \displaystyle \iiint_{\left(V\right)}x\,\text{d}V}\quad \boxed{y_S=\dfrac{1}{V}\cdot \displaystyle \iiint_{\left(V\right)}y\,\text{d}V}\quad \boxed{z_S=\dfrac{1}{V}\cdot \displaystyle \iiint_{\left(V\right)}z\,\text{d}V}
\end{equation}
Somit handelt es sich um Dreifachintegrale. Die Schwerpunktkoordinaten in \textbf{kartesischen Koordinaten} sind
\begin{equation}
\boxed{x_S=\dfrac{1}{V}\cdot \displaystyle \int_{x=a}^{b}\displaystyle \int_{y=f_u\left(x\right)}^{f_o\left(x\right)}\displaystyle \int_{z=z_u\left(x; y\right)}^{z_o\left(x;y\right)}x\,\text{d}z\,\text{d}y\,\text{d}x}
\end{equation}
\begin{equation}
\boxed{y_S=\dfrac{1}{V}\cdot \displaystyle \int_{x=a}^{b}\displaystyle \int_{y=f_u\left(x\right)}^{f_o\left(x\right)}\displaystyle \int_{z=z_u\left(x; y\right)}^{z_o\left(x;y\right)}y\,\text{d}z\,\text{d}y\,\text{d}x}
\end{equation}
\begin{equation}
\boxed{z_S=\dfrac{1}{V}\cdot \displaystyle \int_{x=a}^{b}\displaystyle \int_{y=f_u\left(x\right)}^{f_o\left(x\right)}\displaystyle \int_{z=z_u\left(x; y\right)}^{z_o\left(x;y\right)}z\,\text{d}z\,\text{d}y\,\text{d}x}
\end{equation}
Bei einem rotationssymmetrischen Körper liegt der Schwerpunkt $S$ auf der Rotationsachse. Legt man das Koordinatensystem so, dass die Rotationsachse in die $z$-Achse fällt, dann  gilt in der \textbf{Zylinderkoordinaten}
\begin{equation}
\boxed{x_S=0}\quad \boxed{y_S=0}\quad \boxed{z_S=\dfrac{1}{V}\cdot \displaystyle \iiint_{\left(V\right)}z\,\text{d}z\,\text{d}r\,\text{d}\varphi}
\end{equation}
\subsubsection{Massenträgheitsmomente}
Das Massenträgheitsmoment ist eine von der Masse und der räumliche Verteilung um die Drehachse physikalische Grösse. Definitionsgemäss liefert ein Massenelement $\text{d}m$ des Körpers den infinitesimalen kleinen Beitrag zum Massenträgheitsmoment $J$ des Gesamtkörpers, bezogen auf eine bestimmte Achse $A$, wobei $r_A$ der senkrechte Abstand des Massen- bzw. Volumenelementes von der Bezugsachse $A$, $\rho$ die konstante Dichte des Körpers mit $\text{d}m=\rho\,\text{d}V$ 
\begin{equation}
\boxed{\text{d}J=r_A^2\,\text{d}m=r_A^2\left(\rho\,\text{d}V\right)=\rho \,r_A^2\,\text{d}V}
\end{equation}
Für das Massenträgheitsmoment eines homogenen Körpers erhält man dann durch Aufsummieren, d.h. Integration der Gleichung folgende Integralformel
\begin{equation}
\boxed{J=\rho\cdot \displaystyle \iiint_{\left(V\right)}r_A^2\,\text{d}V}
\end{equation}
In \textbf{kartesischen Koordinaten}
\begin{equation}
\boxed{J=\rho\cdot \displaystyle \int_{x=a}^{b}\displaystyle \int_{y=f_u\left(x\right)}^{f_o\left(x\right)}\displaystyle \int_{z=z_u\left(x; y\right)}^{z_o\left(x; y\right)}\left(x^2+y^2\right)\,\text{d}z\,\text{d}y\,\text{d}x}
\end{equation}
In \textbf{Zylinderkoordinaten} bezogen auf die Rotationsachse bzw. $z$-Achse
\begin{equation}
\boxed{J_z=\rho\cdot \displaystyle \iiint_{\left(V\right)}r^3\,\text{d}z\,\text{d}r\,\text{d}\varphi}
\end{equation}
\chapter{Gewöhnliche Differentialgleichungen}
\section{Grundbegriffe}
\subsection{Ein einführendes Beispiel}
Ein Körper unter Einfluss der Schwerkraft erfährt in der Nähe der Erdoberfläche die konstante Erdbeschleunigung $a=-g$. Die Geschwindigkeit und Beschleunigung einer Bewegung lautet
\begin{equation}
\boxed{a\left(t\right)=\dot{v}\left(t\right)=\ddot{s}\left(t\right)-g}
\end{equation}
Diese Gleichung enthält die 2. Ableitung einer unbekannten Weg-Zeit-Funktion $s=s\left(t\right)$. Gleichungen dieser Art werden in der Mathematik als Differentialgleichungen bezeichnet. Die Lösung der Differentialgleichung ist eine Funktion, nämlich die Weg-Zeit-Funktion der Fallbewegung und entsteht durch zweimal integrieren mit zwei unbekannten Parameter, welche mit einer physikalischen Nebenbedingung wie Anfangshöhe bzw. Anfangsgeschwindigkeit bestimmt werden können.
\begin{equation}
\boxed{v\left(t\right)=\displaystyle \int a\left(t\right)\,\text{d}t=-\displaystyle \int g\,\text{d}t=-gt+C_1}\quad \boxed{v\left(0\right)=C_1=v_0}
\end{equation}
\begin{equation}
\boxed{s\left(t\right)=\displaystyle \int v\left(t\right)\,\text{d}t=\displaystyle \int \left(-gt+C_1\right)\,\text{d}t=-\dfrac{1}{2}gt^2+C_1t+C_2}\quad \boxed{s\left(0\right)=C_2=s_0}
\end{equation}
Die \textbf{allgemeine Löeung} geht dann in die spezielle, den physikalischen ANfangsbedingungen angepasste Lösung über, die auch als \textbf{partikuläre Lösung} der Differentialgleichung bezeichnet wird.
\begin{equation}
\boxed{s\left(t\right)=-\dfrac{1}{2}gt^2+v_ot+s_0\quad \left(t\geq 0\right)}
\end{equation}
\subsection{Definition einer gewöhnliche Differentialgleichung}
Eine \textbf{gewöhnliche Differentialgleichung} $n$-ter Ordnung enthält als höchste Ableitung die $n$-te Ableitung der unbekannten Funktion $y=y\left(x\right)$, kann aber auch Ableitungen niedrigerer Ordnung sowie die Funktion $y=y\left(x\right)$ und deren unabhängige Variable $x$ enthalten. Sie ist in der impliziten Form oder falls diese Gleichung nach der höchsten Ableitung auflösbar ist in der expliziren Form
\begin{equation} 
\boxed{F\left(x; y; y'; y''; \dotso\right)=0}\quad \boxed{y^{\left(n\right)}=f\left(x; y; y'; y'';\dotso\right)}
\end{equation} 
Neben den gewöhnlichen Differentialgleichungen gibt es noch die partiellen Differentialgleichungen. Sie enthalten \textbf{partiellen Ableitungen} einer unbekannten Funktion von mehreren Variablen. 
\subsection{Lösungen einer Differentialgleichung}
Eine Funktion $y=y\left(x\right)$ heisst eine Lösung der Differentialgleichung, wenn sie mit ihren Ableitungen der Differentialgleichung identisch erfüllt. Man unterscheidet zwischen der allgemeinen Lösung und der speziellen oder partikulären Lösung. Die allgemeine Lösung einer Differentialgleichung $n$-ter Ordnung enthält noch $n$ voneinander unabhängige Parameter. Eine partikuläre Lösung wird aus der allgemeinen Lösung gewonnen, indem man aufgrund zusätzlicher Bedingungen den $n$ Parametern feste Werte zuweist. Dies kann durch Anfangsbedingungen oder Randbedingungen geschehen. 
\newline \newline
Die Anzahl der unabhängigen Parameter in der allgemeinen Lösung einer Differentialgleichung ist durch die Ordnung der Differentialgleichung bestimmt. Die allgemeine Lösung einer Differentialgleichung 1. Ordnung enthält somit einen Parameter, die allgemeine Lösung einer Differentialgleichung 2. Ordnung genau zwei unabhängige Parameter.
\newline\newline
Die allgemeine Lösung einer Differentialgleichung $n$-ter Ordnung repräsentiert eine Kurvenschar mit $n$ Parametern. Für jede spezielle Parameterwahl erhält man eine Lösungskurve. Die Lösungen einer Differentialgleichung werden als Integrale bezeichnet.
\section{Differentialgleichungen 1. Ordnung}
\subsection{Geometrische Betrachtung}
Die Differentialgleichung $y'=f\left(x; y\right)$ besitze die Eigenschaft, dass durch jeden Punkt des Definitionsbereiches von $f\left(x; y\right)$ genau eine Lösungskurve verlaufe. $P_0=\left(x_0; y_0\right)$ ist ein solcher Punkt und $y=y\left(x\right)$ die durch den Punkt $P_0$ gehende Lösungskurve.
\newline\newline
Die Steigung $m=\tan\left(\alpha\right)$ der Kurventangejte $P_0$ kann auf zwei verschiedene Arten berechnet werden: Aus der Funktionsgleichung $y=y\left(x\right)$ der Lösungskurve durch Differentiation nach der Variablen $x$: $m=y'\left(x_0\right)$ und aus der Differentialgleichung $y'=f\left(x; y\right)$ selbst, indem man in diese Gleichung die Koordinaten des Punktes $P_0$ einsetzt: $m=f\left(x_0; y_0\right)$. Somit gilt
\begin{equation}
\boxed{m=y'\left(x_0\right)=f\left(x_0; y_0\right)}
\end{equation}
Der Anstieg der Lösungskurve durch den Punkt $P_0$ kann somit direkt aus der Differentialgleichung berechnet werden, die Funktionsgleichung der Lösungskurve wird dabei überhaupt nicht benötigt. Durch die Differentiagleichung der Funktion $f\left(x; y\right)$ wird nämlich jedem Punkt $P=\left(x; y\right)$ aus dem Definitionsbereich der Funktion $f\left(x; y\right)$  ein Richtungs- oder Steigungswert zugeordnet. Er gibt den Anstieg der durch $P$ gehenden Lösungskurve in diesem Punkt an.
\newline\newline
Die Richtung der Kurventangente in $P$ kennzeichnet man graphisch durch eine kleine, in der Tangente liegende Strecke, die als Linien- oder Richtungselement bezeichnet wird. Das dem Punkt $P=\left(x; y\right)$ zugeordnete Linienelement ist demnach durch die Angabe der beiden Koordinaten $x$, $y$ und des Steigungsswertes $m=f\left(x; y\right)$ eindeutig bestimmt. Die Gesamtheit der Linienelemente bildet das Richtungsfeld der Differentialgleichung, aus dem sich ein erster, grober Überblick über den Verlauf der Lösungskurven gewinnen lässt. Eine Lösungskurve muss dabei in jedem ihrer Punkte die durch das Richtungsfeld vorgegebene Steigung aufweisen.
\newline\newline
Bei der Konstruktion von Näherungskurven erweisen sich die sogenannten Isoklinen als sehr hilfreich. Unter einer Isokline versteht man dabei die Verbindungslinie aller Punkte, deren zugehörige Linienelemente in die gleiche Richtung zeigen, d.h. zueinande rparallel sind. Die Isoklinen der Differentialgleichung $y'=f\left(x; y\right)$ sind daher durch folgende Gleichung definiert
\begin{equation}
\boxed{f\left(x; y\right)=\text{const.}}
\end{equation}
Im Richtungsfeld der Differentialgleichung konstruiert man nun Kurven, die in ihren Schnittpunkten mit den Isoklinen den gleichen Anstieg besitzen wie die dortigen Linienelemente. In einem Schnittpunkt verlaufen somit Kurventangente und Linienelement parallel, d.h. das Linienelement fällt in die dortige Kurventangente. Kurven mit dieser Eigenschaft sind dann Näherungen für die tatsächlichen Lösungskurven.
\subsection{Differentialgleichungen mit trennbaren Variablen}
Eine Differentialgleichung 1. Ordnung vom Typ 
\begin{equation}
\boxed{\dfrac{\text{d}y}{\text{d}x}=f\left(x\right)\cdot g\left(x\right)}
\end{equation}
heisst separabel und lässt sich durch Trennung der Variablen lösen. Dabei wird die Differentialgleichung zunächst wie folgt, wobei $g\left(y\right)\neq 0$ umgestellt
\begin{equation}
\boxed{\dfrac{\text{d}y}{\text{d}x}=f\left(x\right)\cdot g\left(y\right)\Longrightarrow \dfrac{\text{d}y}{g\left(y\right)}=f\left(x\right)\,\text{d}x}
\end{equation}
Die linke Seite der Gleichung enthält nur noch die Variable $y$und deren Differential $\text{d}y$, die rechte Seite dagegen nur noch die Variable $x$ und deren Differential $\text{d}x$. Die Variablen wurden somit getrennt und beide Seiten integriert.
\begin{equation}
\boxed{\displaystyle \int \dfrac{\text{d}y}{g\left(y\right)}=\displaystyle \int f\left(x\right)\,\text{d}x}
\end{equation}
Die dann in Form einer impliziten Gleichung vom Typ $F_1\left(y\right)=F_2\left(x\right)$ vorliegende Lösung wird nach der Variablen $y$ aufgelöst, was in den meisten Fällen möglich ist und man erhält die allgemeine Lösung der Differentialgleichung $y'=f\left(x\right)\cdot g\left(y\right)$ in der expliziten Form $y=y\left(x\right)$. Die Lösungen der Gleichung $g\left(y\right)=0$ sind vom Typ $y=\text{const.}=a$ und zugleich dauch Lösungen der Differentialgleichung $y'=f\left(x\right)\cdot g\left(y\right)$.
\subsection{Differentialgleichungen durch Substitution}
In einigen Fällen ist es möglich, eine expliziten Differentialgleichung 1. Ordnung $y'=f\left(x; y\right)$ mit Hilfe einer geeigneten Substitution auf eine separable Differentialgleichung 1. Ordnung zurückzuführen, die dann durch Trennung der Variablen gelöst werden kann. 
\subsubsection{Differentialgleichungen vom Typ $y'=f\left(ax+by+c\right)$}
Eine Differentialgleichung von diesem Typ lässt sich durch die lineare Substitution lösen
\begin{equation}
\boxed{u=ax+by+c}
\end{equation}
Dabei sind $y$ und $u$ als Funktionen von $x$ zu betrachten. Berücksichtigt man noch, dass $y'=f\left(u\right)$ ist, so folgt hieraus die Differentialgleichung
\begin{equation}
\boxed{\dfrac{\text{d}u}{\text{d}x}=a+b\dfrac{\text{d}y}{\text{d}x}=a+b\cdot f\left(u\right)}
\end{equation}
\subsubsection{Differentialgleichungen vom Typ $y'=f\left(\dfrac{y}{x}\right)$}
Eine Differentialgleichung von diesem Typ wird durch die Substitution gelöst
\begin{equation}
\boxed{u=\dfrac{y}{x}\Longleftrightarrow y=x\cdot u}
\end{equation}
Man differenziert diese Gleichung nach $x$ und erhält
\begin{equation}
\boxed{\dfrac{\text{d}y}{\text{d}x}=u+x\cdot \dfrac{\text{d}u}{\text{d}x}}
\end{equation}
wobei $y$ und $u$ Funktionen von $x$ sind. Da $\dfrac{\text{d}y}{\text{d}x}=f\left(x\right)$ ist, geht die Differentialgleichung schliesslich in die separable Differentialgleichung über, die ebenfalls durch Trennung der Variablen gelöst werden kann. Anschluessen folgt die Rücksubstitution und Auflösen nach $y$.
\begin{equation} 
\boxed{u+x\cdot \dfrac{\text{d}u}{\text{d}x}=f\left(u\right)\Longleftrightarrow \dfrac{\text{d}u}{\text{d}x}=\dfrac{f\left(u\right)-u}{x}}
\end{equation} 
\subsection{Exakte Differentialgleichungen}
Eine Differentialgleichung 1. Ordnung vom Typ
\begin{equation}
\boxed{\dfrac{\text{d}y}{\text{d}x}=-\dfrac{g\left(x; y\right)}{h\left(x; y\right)}}
\end{equation}
heisst \textbf{exakt} oder vollständig, wenn sie folgende Bedingung erfüllt
\begin{equation}
\boxed{\dfrac{\partial g\left(x; y\right)}{\partial y}=\dfrac{\partial h\left(x; y\right)}{\partial x}}
\end{equation}
Die linke Seite der Gleichung ist dann das \textbf{totale Differential} einer unbekannten Funktion $u\left(x; y\right)$. Es gilt
\begin{equation}
\boxed{\text{d}u=\dfrac{\partial u}{\partial x}\text{d}x+\dfrac{\partial u}{\partial y}\text{d}y=h\left(x; y\right)\text{d}x+g\left(x; y\right)\text{d}y=0}
\end{equation}
Die Faktorfunktionen $g\left(x; y\right)$ und $h\left(x; y\right)$ in der exakten Differentialgleichung sind also die partiellen Ableitungen 1. Ordnung von $u\left(x; y\right)$
\begin{equation}
\boxed{\dfrac{\partial u}{\partial x}=g\left(x; y\right)}\quad \boxed{\dfrac{\partial u}{\partial y}=h\left(x; y\right)}
\end{equation}
Die allgemeine Lösung der Differentialgleichung lautet dann in impliziter Form $u\left(x; y\right)=\text{const}=C$. Die Funktion $u\left(x; y\right)$ lässt sich aus den Gleichungen bestimmen. Die erste der beiden Gleichungen wird bezüglich der Variablen $x$ integriert, wobei zu beachten ist, dass die Integrationskonstante $K$ noch von $y$ abhängen
\begin{equation}
\boxed{u=\displaystyle \int \dfrac{\partial u}{\partial x}\,\text{d}x=\displaystyle \int g\left(x; y\right)\,\text{d}x+K\left(y\right)}
\end{equation}
Wenn man diese Funktion nach der Variable $y$ partiell ableitet, erhält man die Faktorfunktion $h\left(x; y\right)$
\begin{equation}
\boxed{
\begin{array}{lll}
\dfrac{\partial u}{\partial y}&=&\dfrac{\partial}{\partial y}\Big[\displaystyle \int g\left(x; y\right)\,\text{d}x+K\left(y\right)\Big]=\dfrac{\partial}{\partial y}\displaystyle \int g\left(x; y\right)\,\text{d}x+\dfrac{\partial}{\partial y}K\left(y\right)\\\\
&=&\displaystyle \int \dfrac{\partial g\left(x; y\right)}{\partial y}\,\text{d}x+K'\left(y\right)=h\left(x; y\right)
\end{array}
}
\end{equation}
Aufgelöst nach $K'\left(y\right)$ und durch Integration erhält man die gesuchte Funktion $K\left(y\right)$. Damit ist auch $u\left(x; y\right)$ und die allgemeine Lösung der exakten Differentialgleichung bekannt.
\subsection{Lineare Differentialgleichungen 1. Ordnung}
\subsubsection{Definition}
Eine Differentialgleichung 1. Ordnung heisst \textbf{linear}, wenn sie in folgender Form darstellbar ist
\begin{equation}
\boxed{\dfrac{\text{d}y}{\text{d}x}+f\left(x\right)\cdot y=g\left(x\right)}
\end{equation}
Die Funktion $g\left(x\right)$ wird als \textbf{Störfunktion} bezeichnet. Ist $g\left(x\right)=0$, so heisst die lineare Fifferentialgleichung \textbf{homogen}, ansonsten \textbf{inhomogen}.
\subsubsection{Integration der homogenen linearen Differentialgleichung}
Eine homogene lineare Differentialgleichung 1. Ordnung
\begin{equation}
\boxed{\dfrac{\text{d}y}{\text{d}x}+f\left(x\right)\cdot y=0}
\end{equation}
lässt sich durch Trennung der Variablen wie folgt lösen. Zunächst trennt man die beiden Variablen
\begin{equation}
\boxed{
\begin{array}{lll}
\displaystyle \int\dfrac{\text{d}y}{y}&=&-\displaystyle \int f\left(x\right)\, \text{d}x\\
\ln\Big\vert y\Big\vert&=&-\displaystyle \int f\left(x\right)\, \text{d}x+\ln\Big\vert C\Big\vert\\
y&=&e^C\cdot e^{-\displaystyle \int f\left(x\right)\,\text{d}x}\quad \left(C\in \mathbb{R}\right)
\end{array}
}
\end{equation}
\subsubsection{Integration der inhomogenen linearen Differentialgleichung durch Variation der Konstanten}
Eine inhomogene lineare Differentialgleichung 1. Ordnung 
\begin{equation}
\boxed{\dfrac{\text{d}y}{\text{d}x}+f\left(x\right)\cdot y=g\left(x\right)}
\end{equation}
lässt sich wie folgt durch Variation der Konstanten lösen. Zunächst wird die zugehörige homogene Differentialgleichung durch Trennung der Variablen gelöst. Dies führt zu der allgemeinen Lösung. Die Integrationskonstante $K$ wird durch eine noch unbekannte Funktion $K\left(x\right)$ ersetzt. 
\begin{equation} 
\boxed{y_0=K\cdot e^{-\displaystyle \int f\left(x\right)\,\text{d}x}}\quad \boxed{y=K\left(x\right)\cdot e^{-\displaystyle \int f\left(x\right)\,\text{d}x}}
\end{equation} 
Die inhomogene Differentialgleichung wird durch die 1. Ableitung unter Verwendung von Produkt- und Kettenregel gelöst
\begin{equation}
\boxed{\dfrac{\text{d}y}{\text{d}x}=\dfrac{\text{d}}{\text{d}x}\Big[K\left(x\right)\Big]\cdot e^{-\displaystyle \int f\left(x\right)\,\text{d}x}-K\left(x\right)\cdot f\left(x\right)\cdot e^{-\displaystyle \int f\left(x\right)\,\text{d}x}}
\end{equation}
Man setzt für die für $y$ und $\dfrac{\text{d}y}{\text{d}x}$ gefundenen Funktionsterme in die inhomogene Differentialgleichung ein
\begin{equation}
\boxed{\underbrace{\dfrac{\text{d}}{\text{d}x}\Big[K\left(x\right)\Big]\cdot e^{-\displaystyle \int f\left(x\right)\,\text{d}x}-K\left(x\right)\cdot f\left(x\right)\cdot e^{-\displaystyle \int f\left(x\right)\,\text{d}x}}_{\dfrac{\text{d}y}{\text{d}x}}+f\left(x\right)\cdot \underbrace{K\left(x\right)\cdot e^{-\displaystyle \int f\left(x\right)\,\text{d}x}}_{y}=g\left(x\right)}
\end{equation}
Somit erhält man
\begin{equation}
\boxed{\dfrac{\text{d}}{\text{d}x}\Big[K\left(x\right)\Big]\cdot e^{-\displaystyle \int f\left(x\right)\,\text{d}x}=g\left(x\right)\Longrightarrow \dfrac{\text{d}}{\text{d}x}\Big[K\left(x\right)\Big]=g\left(x\right)\cdot e^{\displaystyle \int f\left(x\right)\,\text{d}x}}
\end{equation}

Durch Integration erfolgt
\begin{equation}
\boxed{K\left(x\right)=\displaystyle \int g\left(x\right)\cdot e^{\displaystyle \int f\left(x\right)\,\text{d}x}\,\text{d}x+C}
\end{equation}
Diesen Ausdruck setzt man für die Faktorfunktion $K\left(x\right)$ des Lösungsansatzes ein und erhält dann die allgemeine Lösung der inhomogenen Differentialgleichung
\begin{equation}
\boxed{y=\underbrace{\Big[\displaystyle \int g\left(x\right)\cdot e^{\displaystyle \int f\left(x\right)\,\text{d}x}\,\text{d}x+C\Big]}_{K\left(x\right)}\cdot e^{-\displaystyle \int f\left(x\right)\,\text{d}x}}
\end{equation}
Durch die Bezeichnung "Variation der Konstanten" soll zum Ausdruck gebracht werden, dass die Integrationskonstante $K$ "variiert", d.h. durch eine Funktion $K\left(x\right)$ ersetzt wird.
\subsubsection{Integration der inhomogenen Differentialgleichung durch Aufsuchen einer partikulären Lösung}
Die allgemeine Lösung einer inhomogenen linearen Differentialgleichung 1. Ordnung vom Typ
\begin{equation}
\boxed{\dfrac{\text{d}y}{\text{d}x}+f\left(x\right)\cdot y=g\left(x\right)}
\end{equation}
ist als Summe aus der allgemeinen Lösung $y_0=y_0\left(x\right)$ der zugehörigen homogenen linearen Differentialgleichung
\begin{equation}
\boxed{\dfrac{\text{d}y}{\text{d}x}+f\left(x\right)\cdot y=0}
\end{equation}
und einer beliebigen partikulären Lösung $y_p=y_p\left(x\right)$ der inhomogenen linearen Differentialgleichung darstellbar
\begin{equation}
\boxed{y\left(x\right)=y_0\left(x\right)+y_p\left(x\right)}\quad \boxed{y_0=C\cdot e^{-\displaystyle \int f\left(x\right)\,\text{d}x}\quad \left(C\in \mathbb{R}\right)}
\end{equation}
Auch lineare Differentialgleichung 2. und höherer Ordnung besitzen diese Eigenschaft. Der Lösungsansatz für eine partikuläre Lösung $y_p$ hängt noch sowohl vom Typ der Koeffizientenfunktion $f\left(x\right)$ als auch vom Typ der Störfunktion $g\left(x\right)$ ab. Man muss sich für einen speziellen Funktionstyp entscheiden und dann versuchen, die im Ansatz $y_p$ enthaltenen Parameter so zu bestimmen, dass diese Funktion der inhomogenen Differentialgleichung genügt. Der partikuläre Lösungsansatz $y_p$ wird in die ursprüngliche lineare Differentialgleichung eingesetzt und die Parameter ausgerechnet. 
\subsection{Lineare Differentialgleichung 1. Ordnung mit konstanten Koeffizienten} 
In den Anwendungen spielen lineare Differentialgleichungen 1. Ordnung mit konstzanten Koeffizienten iene besondere Rolle. Sie sind vom Typ
\begin{equation}
\boxed{\dfrac{\text{d}y}{\text{d}x}+ay=g\left(x\right)}
\end{equation}
Die zugehörige homogene Gleichung enthält nur konstante Koeffizienten und wird durch Trennung der Variablen oder durch den Exponentialansatz gelöst.
\begin{equation}
\boxed{\dfrac{\text{d}y}{\text{d}x}+ay=0}\quad \boxed{y_0=C\cdot e^{\lambda x}}\quad \boxed{y_0'=\lambda \cdot C \cdot e^{\lambda x}}
\end{equation}
Mit diesem Ansatz geht man in die homogene Differentialgleichung ein und erhält eine Bestimmungsgleichung für den Parameter $\lambda$
\begin{equation}
\boxed{y_0'+ay_0=\lambda \cdot C\cdot e^{\lambda x}+a\cdot C\cdot e^{\lambda x}=\underbrace{\left(\lambda + a\right)}_{0}\cdot C\cdot e^{\lambda x}=0\Longrightarrow \lambda = -a}
\end{equation}
Die homogene Differentialgleichung $y'+ay=0$ besitzt also die allgemeine Lösung
\begin{equation}  
\boxed{y_0=C\cdot e^{-ax}\quad \left(C\in \mathbb{R}\right)}
\end{equation}  
Folgende Tabelle zeigt die Lösungsansätze $y_p$ für einige in den Anwendungen besonders häufig auftretende Störfunktionen
\begin{table}[H]
\centering
\begin{tabular}{|l|l|}
\hline
\textbf{Störfunktion $g\left(x\right)$}& \textbf{Lösungsansatz $y_p\left(x\right)$}\\\hline
Konstante Funktion& $y_p=c_0$\\\hline
Lineare Funktion& $y_p=c_1x+c_0$\\\hline
Quadratische Funktion& $y_p=c_2x^2+c_1x+c_0$\\\hline
Polynomfunktion vom Grade $n$& $y_p=c_nx^n+\dotso+c_1x+c_0$\\\hline
$\begin{array}{l}g\left(x\right)=A\cdot \sin\left(\omega x\right)\\g\left(x\right)=B\cdot \cos\left(\omega x\right)\\g\left(x\right)=A\cdot \sin\left(\omega x\right)+B\cdot \cos\left(\omega x\right)\end{array}$& $\begin{array}{l}y_p=C_1\cdot \sin\left(\omega x\right)+C_2\cdot \cos\left(\omega x\right)\\\text{oder}\\y_p=C\cdot \sin\left(\omega x+\varphi\right)\end{array}$\\\hline
$g\left(x\right)=A\cdot e^{bx}$ & $y_p=\Big\{\begin{array}{l}C\cdot e^{bx}\text{ für } b\neq -a\\Cx\cdot e^{bx}\text{ für } b=-a\end{array}$\\\hline
\end{tabular}
\caption{Lösungsansatz für die partikuläre Lösung $y_p\left(x\right)$ der inhomogenen Differentialgleichung 1. Ordnung mit konstanten Koeffizienten}
\end{table}
\noindent Die im Lösungsansatz $y_p$ enthaltenen Parameter sind so zu bestimmen, dass die Funktion eine partikuläre Lösung der vorgegebenen inhomogenen Differentialgleichung darstellt. Bei einem richtig gewählten Ansatz stösst man stets auf ein eindeutig lösbares Gleichungssystem für die im Lösungsansatz  enthaltenen Stellparameter.
\newline\newline
Die Störfunktion $g\left(x\right)$ ist eine Summe aus mehreren Störgliedern, somit werden die Lösungsansätze für die einzelnen Glieder addiert. Die Störfunktion $g\left(x\right)$ ist ein Produkt aus mehreren Störgliedern, somit werden die einzelnen Gliedern multipliziert.
\section{Lineare Differentialgleichung 2. Ordnung mit konstanten Koeffizienten}
\subsection{Definition einer linearen Differentialgleichung 2. Ordnung}
Eine Differentialgleichung vom Typ 
\begin{equation}
\boxed{\dfrac{\text{d}^2y}{\text{d}x}+a\dfrac{\text{d}y}{\text{d}x}+by=g\left(x\right)}
\end{equation}
heisst lineare Differentialgleichung 2. Ordnung mit konstanten Koeffizienten $\left(a, b\in \mathbb{R}\right)$. Die Funktion $g\left(x\right)$ wird als Störfunktion oder Störglied bezeichnet. Fehlt das Störglied, so heisst die lineare Differentialgleichung homogen, sonst inhomogen.
\subsection{Allgemeine Eigenschaft der homogenen linearen Differentialgleichung}
Eine homogene lineare Differentialgleichung vom Typ 
\begin{equation}
\boxed{\dfrac{\text{d}^2y}{\text{d}x}+a\dfrac{\text{d}y}{\text{d}x}+by=0}
\end{equation}
besitzt folgende Eigenschaften
\begin{itemize}
\item Ist $y_1\left(x\right)$ eine Lösung der Differentialgleichung, so ist auch die mit einer beliebigen Konstanten $C$ multiplizierte Funktion eine Lösung der Differentialgleichung $\left(C\in \mathbb{R}\right)$
\begin{equation}
\boxed{y\left(x\right)=C\cdot y_1\left(x\right)}\quad \boxed{y'\left(x\right)=C\cdot y_1'\left(x\right)}\quad \boxed{y''\left(x\right)=C\cdot y_1''\left(x\right)}
\end{equation}
\item Sind $y_1\left(x\right)$ und $y_2\left(x\right)$ zwei Lösungen der Differentialgleichung, so ist auch die aus ihnen gebildete Linearkombination eine Lösung der Differentialgleichung $\left(C_1, C_2\in \mathbb{R}\right)$
\begin{equation}
\boxed{y\left(x\right)=C_1\cdot y_1\left(x\right)+C_2\cdot y_2\left(x\right)}\quad \boxed{y'\left(x\right)=C_1\cdot y_1'\left(x\right)+C_2\cdot y_2'\left(x\right)}
\end{equation}
\begin{equation}
\boxed{y''\left(x\right)=C_1\cdot y_1''\left(x\right)+C_2\cdot y_2''\left(x\right)}
\end{equation}
\item Ist $y\left(x\right)$ eine komplexwertige Lösung der Differentialgleichung, so sind auch Realteil $u\left(x\right)$ und Imaginärteil $v\left(x\right)$ reelle Lösungen der Differentialgleichung
\begin{equation}
\boxed{y\left(x\right)=u\left(x\right)+\text{j}\cdot v\left(x\right)}\quad \boxed{y'\left(x\right)=u'\left(x\right)+\text{j}\cdot v'\left(x\right)}\quad \boxed{y''\left(x\right)=u''\left(x\right)+\text{j}\cdot v''\left(x\right)}
\end{equation}
\end{itemize}
Zwei Lösungen $y_1=y_1\left(x\right)$ und $y_2=y_2\left(x\right)$ einer homogenen linearen Differentialgleichung 2. Ordnung mit konstanten Koeffizienten vom Typ
\begin{equation}
\boxed{\dfrac{\text{d}^2y}{\text{d}x}+a\dfrac{\text{d}y}{\text{d}x}+by=0}
\end{equation}
werden als Basisfunktionen oder Basislösungen der Differentialgleichung bezeichnet, wenn die aus ihnen gebildete sog. \textbf{Wronski-Determinante} von Null verschieden ist
\begin{equation}
\boxed{W\left(y_1; y_2\right)=\begin{vmatrix}y_1\left(x\right)&y_2\left(x\right)\\y_1'\left(x\right)&y_2'\left(x\right)\end{vmatrix}}
\end{equation}
Die Wronski-Determinante ist eine 2-reihige Determinante. Sie enthält in der 1. Zeile die beiden Lösungsfunktionen $y_1$ und $y_2$ und in der 2. Zeile deren Ableitungen $y_1'$ und $y_2'$. Man beachte, dass der Wert der Wronski-Determinante noch von der Variablen $x$ abhängt. Es genügt zu zeigen, dass die Wronski-Determinante an einer Stelle $x_0$ vom Null verschieden ist.
\newline\newline
Zwei Basislösungen $y_1\left(x\right)$ und $y_2\left(x\right)$ der homogenen Differentialgleichung werden auch als linear unabhängige Lösungen bezeichnet. Verschwindet dagegen die Wronski-Determinante zweier Lösungen $y_1$ und $y_2$, so werden die Lösungen als linear abhängig bezeichnet. 
\newline\newline
Die Konstanten im Lösungsansatz müssen eindeutig aus Anfangsbedingungen bestimmbar sein. Mit den Anfangsbedingungen erhält man ein lineares Gleichungssystem. Das System hat genau eine Lösung, wenn die Wronski-Determinante an der Stelle $x_0$ von Null verschieden ist.
\subsection{Integration der homogenen linearen Differentialgleichung}
Eine Fundamentalbasis der homogenen linearen Differentialgleichung 2. Ordnung mit konstanten Koeffizienten vom Typ
\begin{equation}
\boxed{\dfrac{\text{d}^2y}{\text{d}x}+a\dfrac{\text{d}y}{\text{d}x}+by=0}
\end{equation}
lässt sich durch einen Lösungsansatz in Form einer Exponentialfunktion mit Parameter $\lambda$ vom Typ
\begin{equation}
\boxed{y=e^{\lambda x}}\quad \boxed{\dfrac{\text{d}y}{\text{d}x}=\lambda \cdot e^{\lambda x}}\quad \boxed{\dfrac{\text{d}^2y}{\text{d}x}=\lambda^2\cdot e^{\lambda x}}
\end{equation}
Eingesetzt in die lineare Differentialgleichung erhält man die charakteristische Gleichung der homogenen Gleichung
\begin{equation}
\boxed{\dfrac{\text{d}^2y}{\text{d}x}+a\dfrac{\text{d}y}{\text{d}x}+by=\lambda^2\cdot e^{\lambda x}+a\lambda\cdot e^{\lambda x}+b\cdot e^{\lambda x}=e^{\lambda x}\left(\lambda^2+a\lambda+b\right)=0}
\end{equation}
\begin{equation}
\boxed{\lambda^2+a\lambda+b=0}\quad \boxed{\lambda_{1,2}=-\dfrac{a}{2}\pm \sqrt{\dfrac{a^2}{4}-b}=-\dfrac{a}{2}\pm \dfrac{\sqrt{a^2-4b}}{2}}
\end{equation}
Die Diskriminante $a^2-4b$ entscheidet dabei über die Art der Lösungen
\begin{itemize}
\item Fall $a^2-4b<0$: $\lambda_1\neq \lambda_2$
\begin{equation}
\boxed{y_1=e^{\lambda_1 x}}\quad \boxed{y_2=e^{\lambda_2 x}}\quad \boxed{y=C_1\cdot e^{\lambda_1 x}+C_2\cdot e^{\lambda_2 x}}
\end{equation}
\item Fall $a^2-4b<0$: $\lambda_1=\lambda_2=c$
\begin{equation}
\boxed{y_1=e^{c x}}\quad \boxed{y_2=x\cdot e^{c x}}\quad \boxed{y=\left(C_1+C_2\cdot x\right)\cdot e^{c x}}
\end{equation}
\item Fall $a^2-4b>0$: $\lambda_{1,2}=\alpha\pm\text{j}\omega$
\begin{equation}
\boxed{y_1=e^{\alpha x}\cdot \sin\left(\omega x\right)}\quad \boxed{y_2=e^{\alpha x}\cdot \cos\left(\omega x\right)}\quad \boxed{y=e^{\alpha x}\cdot \Big[C_1\cdot \sin\left(\omega x\right)+C_2\cdot \cos\left(\omega x\right)\Big]}
\end{equation}
\end{itemize}
Die charakteristische Gleichung hat dieselben Koeffizienten wie die homogene Differentialgleichung.
\subsection{Integration der inhomogenen linearen Differentialgleichung}
Die allgemeine Lösung $y=y\left(x\right)$ einer inhomogenen linearen Differentialgleichung 2. Ordnung mit konstanten Koeffizienten vom Typ
\begin{equation}
\boxed{\dfrac{\text{d}^2y}{\text{d}x}+a\dfrac{\text{d}y}{\text{d}x}+by=g\left(x\right)}
\end{equation}
ist als Summe aus der allgemeinen Lösung $y_0=y_0\left(x\right)$ der zugehörigen homogenen linearen Differentialgleichung
\begin{equation}
\boxed{\dfrac{\text{d}^2y}{\text{d}x}+a\dfrac{\text{d}y}{\text{d}x}+by=0}
\end{equation}
und einer beliebigen partikulären Lösung $y_p=y_p\left(x\right)$ der inhomogenen linearen Differentialgleichung 
\begin{equation}
\boxed{y\left(x\right)=y_0\left(x\right)+y_p\left(x\right)}
\end{equation}
\begin{table}[H]
\centering
\begin{tabular}{|l|l|}
\hline
\textbf{Störfunktion $g\left(x\right)$}& \textbf{Lösungsansatz $y_p\left(x\right)$}\\\hline
$g\left(x\right)=P_n\left(x\right)$& $y_p=\Big\{\begin{array}{l}Q_n\left(x\right)\text{ für } b\neq 0\\x\cdot Q_n\left(x\right)\text{ für } a\neq 0 \text{ und } b=0\\x^2\cdot Q_n\left(x\right)\text{ für } a=b=0\end{array}$\\\hline
$g\left(x\right)=e^{cx}$&$y_p=\Big\{\begin{array}{l}A\cdot e^{cx}\text{ für } c \text{ keine Lösung der ch. Gleichung}\\Ax\cdot e^{cx}\text{ für } c \text{ eindeutige Lösung der ch. Gleichung}\\Ax^2\cdot e^{cx}\text{ für } c \text{ doppelte Lösung der ch. Gleichung}\end{array}$\\\hline
$g\left(x\right)=\sin\left(\beta x\right)+\cos\left(\beta x\right)$& $y_p=\Big\{\begin{array}{l}A\cdot \sin\left(\beta x\right)+B\cdot \cos\left(\beta x\right)\text{ für }\text{j}\beta \text{ keine Lösung}\\C\cdot \sin\left(\beta x+ \varphi\right)\text{ für }\text{j}\beta \text{ keine Lösung}\end{array}$\\
& $y_p=\Big\{\begin{array}{l}x\cdot \Big[A\cdot \sin\left(\beta x\right)+B\cdot \cos\left(\beta x\right)\Big]\text{ für }\text{j}\beta \text{ eindeutige Lösung}\\C\cdot x\cdot \sin\left(\beta x+ \varphi\right)\text{ für }\text{j}\beta \text{ eindeutige Lösung}\end{array}$\\\hline
$g\left(x\right)=P_n\left(x\right)\cdot e^{cx}\cdot \sin\left(\beta x\right)$&$\text{Für }\text{j}\beta \text{ keine Lösung}$\\
$g\left(x\right)=P_n\left(x\right)\cdot e^{cx}\cdot \cos\left(\beta x\right)$& $y_p=e^{cx}\cdot \Big[Q_n\left(x\right)\cdot \sin\left(\beta x\right)+R_n\left(x\right)\cdot \cos\left(\beta x\right)\Big]$\\
&$\text{Für }\text{j}\beta \text{ eindeutige Lösung}$\\
&$y_p=x\cdot e^{cx}\cdot \Big[Q_n\left(x\right)\cdot \sin\left(\beta x\right)+R_n\left(x\right)\cdot \cos\left(\beta x\right)\Big]$\\\hline
\end{tabular}
\caption{Lösungsansatz für die partikuläre Lösung $y_p\left(x\right)$ der inhomogenen Differentialgleichung 1. Ordnung mit konstanten Koeffizienten}
\end{table}
\end{multicols}
\end{document}
    