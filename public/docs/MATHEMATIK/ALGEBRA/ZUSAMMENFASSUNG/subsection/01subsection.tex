\section{Darstellungsformen einer komplexe Zahlen}
\subsection{Die Kartesische Form}
Eine komplexe Zahl $z$ lässt sich in der Gaussschen Zahlenebene durch einen Bildpunkt $P\left(z\right)$ oder durch einen vom Koordinatenursprung $O$ zum Bildpunkt $P\left(z\right)$ gerichteten Zeiger bildlich darstellen. Die komplexe Zahl besteht aus einem reellen $\text{Re}\left(z\right)=a$, einem imaginären Anteil $\text{Im}\left(z\right)=b$ und eine imaginäre Einheit $\text{j}^2=-1$. Die Menge ist $\mathbb{C}=\left\{z\,\vert\, z=a+\text{j}b\quad\,\text{ mit }a,b\in \mathbb{R}\right\}$
\begin{equation}
\boxed{z=a+\text{j}\,b}
\end{equation}
Die Länge des Zeigers heisst Betrag $\Big\vert z\Big\vert$ der komplexen Zahl $z$. 
\begin{equation}
\boxed{\Big\vert z\Big\vert=\sqrt{a^2+b^2}}
\end{equation}
Zwei komplexe Zahlen $z_1$ und $z_2$ sind genau gleich, $z_1=z_2$, wenn ihre Bildpunkte zusammenfallen, d.h. $a_1=a_2$ und $b_1=b_2$ ist.
\newline\newline
Die zu $z$ konjugiert komplexe Zahl $\overline{z}$ liegt spiegelsymmetrisch zur rellen Achse. Die komplexe Zahl $z$ und ihre konjugiert $\overline{z}$ unterscheiden sich in ihrem Imaginärteil durch das Vorzeichen.
\begin{equation}
\boxed{\overline{z}=\overline{a+\text{j}b}=a-\text{j}b}
\end{equation}
Somit gelten folgende Ausdrücke
\begin{enumerate}[$(i)$]
\item $\text{Re}\left(\overline{z}\right)=\text{Re}\left(z\right)=a$
\item $\Big\vert \overline{z}\Big\vert=\Big\vert z\Big\vert$
\item $\overline{\left(\overline{z}\right)}=z$
\item Gilt $\overline{z}=z$, so ist $z$ reell
\end{enumerate}
%%%%%%%%%%%%%%%%%%%%%%%%%%%%%%%%%%%%%%%%%%%%%%%%%%%%%%%%%%%%%%%%%%%%%%%%%%%%%%%%%
\subsection{Die Polarform}
In der Polarform erfolgt die Darstellung einer komplexen Zahl durch die Polarkoordinaten $r$ und $\varphi$. Man beschränkt sich auf $\left[0,2\pi\right)$.
\begin{equation}
\boxed{r=\sqrt{a^2+b^2}}\quad \boxed{\varphi=\arctan\left(\dfrac{b}{a}\right)+\left\{\begin{array}{l}+0,\quad \text{(I)}\\+\pi,\quad \text{(II, III)}\\+2\pi,\quad \text{(IV)}\end{array}\right\}}
\end{equation}
Die \textbf{goniometrische Form} besteht aus dem Betrag und dem Argument von $z$. Die entsprechende konjugiert komplexe Zahl ist
\begin{equation}
\boxed{z=r\cdot \Big(\cos\left(\varphi\right)+\text{j}\sin\left(\varphi\right)\Big)}\quad 
\boxed{\overline{z}=r\cdot \Big(\cos\left(\varphi\right)-\text{j}\sin\left(\varphi\right)\Big)}
\end{equation}
Die \textbf{Exponentialform} besteht aus dem Betrag und dem Argument von $z$. Die entsprechende konjugierte komplexe Zahl ist
\begin{equation}
\boxed{z=r\cdot e^{\text{j}\varphi}}\quad
\boxed{\overline{z}=r\cdot e^{-\text{j}\varphi}}
\end{equation}
Somit ergeben sich folgende Beziehungen
\begin{equation}
\boxed{e^{\text{j}\varphi}=\cos\left(\varphi\right)+\text{j}\sin\left(\varphi\right)}\quad \boxed{e^{-\text{j}\varphi}=\cos\left(\varphi\right)-\text{j}\sin\left(\varphi\right)}
\end{equation}
\section{Grundrechenarten für komplexe Zahlen}
\subsection{Addition und Subtraktion komplexer Zahlen}
Zwei komplexe Zahlen werden addiert bzw. subtrahiert, indem man ihre Real- und Imaginärteil (jeweils für sich getrennt) addiert bzw. subtrahiert. Addition und Subtraktion sind nur in der karteischen Form durchführbar. Geometrisch entspricht dem Parallelogramregel der Vektorrechnung.  
\begin{equation}
\boxed{\begin{array}{lll}
z_1\pm z_2&=&\left(a_1+ \text{j}b_1\right)\pm \left(a_2+ \text{j}b_2\right)\\
&=&\left(a_1\pm a_2\right)+\text{j}\left(b_1\pm b_2\right)
\end{array}}
\end{equation}
\begin{enumerate}[$(i)$]
\item $z_1+z_2=z_2+z_1$
\item $z_1+\left(z_2+z_3\right)=\left(z_1+z_2\right)+z_3$
\end{enumerate}
\subsection{Multiplikation komplexer Zahlen}
Die Multiplikation in \textbf{kartesische Form} erfolgt, indem jeder Summand der ersten Klammer mit jedem Summand der zweiter Klammer unter Beachtung von $\text{j}^2=-1$ multipliziert. Geometrisch erfolgt eine Drehung im Gegenuhrzeigersinn falls $\varphi_2>0$ und im Uhrzeigersinn falls $\varphi_2<0$ und eine Streckung um den Faktor $r_2$. 
\begin{equation}
\boxed{\begin{array}{lll}
z_1\cdot z_2&=&\left(a_1+\text{j}b_1\right)\cdot \left(a_2+\text{j}b_2\right)\\
&=&\left(a_1a_2-b_1b_2\right)+\text{j}\left(a_1b_2+a_2b_1\right)
\end{array}}
\end{equation}
Die Multiplikation in \textbf{Polarform} erfolgt, indem man ihre Beträge multipliziert und die Argumente addiert.
\begin{equation}
\boxed{\begin{array}{lll}
z_1\cdot z_2&=&\Big[r_1\Big(\cos\left(\varphi_1\right)+\text{j}\sin\left(\varphi_1\right)\Big)\Big]\cdot r_2\Big(\cos\left(\varphi_2\right)+\text{j}\sin\left(\varphi_2\right)\Big)\Big]\\
&=&\left(r_1r_2\right)\cdot \Big[\cos\left(\varphi_1+\varphi_2\right)+\text{j}\sin\left(\varphi_1+\varphi_2\right)\Big]
\end{array}}
\end{equation}
\begin{equation}
\boxed{\begin{array}{lll}
z_1\cdot z_2&=&\Big(r_1\cdot e^{\text{j}\varphi_1}\Big)\cdot \Big(r_2\cdot e^{\text{j}\varphi_2}\Big)\\
&=&\left(r_1r_2\right)\cdot e^{\text{j}\left(\varphi_1+\varphi_2\right)}
\end{array}}
\end{equation}
\begin{enumerate}[$(i)$]
\item $z_1z_2=z_2z_1$
\item $z_1\left(z_2z_3\right)=\left(z_1z_2\right)z_3$
\item $z_1\left(z_2+z_3\right)=z_1z_2+z_1z_3$
\item $z\cdot \overline{z}=a^2+b^2=\Big\vert z\Big\vert^2\Longrightarrow \Big\vert z\Big\vert=\sqrt{z\cdot \overline{z}}$
\item $\text{j}^{4n}=1,\quad \text{j}^{4n+1}=\text{j},\quad \text{j}^{4n+2}=-1,\quad \text{j}^{4n+3}=-\text{j}\quad \left(n\in \mathbb{Z}\right)$
\end{enumerate}
\subsection{Division komplexer Zahlen}
Zähler und Nenner des Quotienten  werden zunächst mit dem konjugiert komplexen Nenner, d.h. der Zahl $\overline{z}_2$ multipliziert, dadurch wird der Nenner reell. Geometrisch erfährt $z_1$ eine Zurückdrehung im Uhrzeigersinn falls $\varphi_2>0$ und im Gegenuhrzeigersinn falls $\varphi_2<0$ und eine Streckung um den Faktor $1/r_2$.
\begin{equation}
\boxed{
\begin{array}{lll}
\dfrac{z_1}{z_2}&=&\dfrac{a_1+\text{j}b_1}{a_2+\text{j}b_2}=\dfrac{\left(a_1+\text{j}b_1\right)\cdot \left(a_2+\text{j}b_2\right)}{\left(a_2+\text{j}b_2\right)\cdot \left(a_2+\text{j}b_2\right)}\\\\
&=&\dfrac{a_1a_2+b_1b_2}{a_2^2+b_2^2}+\text{j}\dfrac{a_2b_1-a_1b_2}{a_2^2+b_2^2}
\end{array}
}
\end{equation}
\begin{equation}
\boxed{
\begin{array}{lll}
\dfrac{z_1}{z_2}&=&\dfrac{r_1\Big[\cos\left(\varphi_1\right)+\text{j}\sin\left(\varphi_1\right)\Big]}{r_2\Big[\cos\left(\varphi_2\right)+\text{j}\sin\left(\varphi_2\right)\Big]}\\
&=&\left(\dfrac{r_1}{r_2}\right)\cdot \Big[\cos\left(\varphi_1-\varphi_2\right)+\text{j}\sin\left(\varphi_1-\varphi_2\right)\Big]
\end{array}
}
\end{equation}
\begin{equation}
\boxed{\begin{array}{lll}
\dfrac{z_1}{z_2}&=&\dfrac{r_1\cdot e^{\text{j}\varphi_1}}{r_2\cdot e^{\text{j}\varphi_2}}=\left(\dfrac{r_1}{r_2}\right)\cdot e^{\text{j}\left(\varphi_1-\varphi_2\right)}\end{array}}
\end{equation}
\begin{enumerate}[$(i)$]
\item $\dfrac{1}{z}=\dfrac{1}{r\cdot e^{\text{j}\varphi}}=\left(\dfrac{1}{r}\right)\cdot e^{-\text{j}\varphi}$
\item $\dfrac{1}{z}=\dfrac{1}{a+\text{j}b}=\dfrac{a}{a^2+b^2}-\text{j}\dfrac{b}{a^2+b^2}$
\item $\dfrac{1}{\text{j}}=-\text{j}$
\end{enumerate}
\section{Potenzieren komplexer Zahlen}
\subsection{Die kartesische Form}
In \textbf{kartesischer Form} ist das Potenzieren komplexer Zahlen nach dem binomischen Lehrsatz  
\begin{equation}
\boxed{z^n=\left(a+\text{j}b\right)^n=a^n+\text{j}\displaystyle \binom{n}{1}a^{n-1}b+\text{j}^2\displaystyle\binom{n}{2}a^{n-2}b^2+\dotso+\text{j}^nb^n}
\end{equation}
\subsection{Die Polarform}
In \textbf{Polarform} wird der Betrag in die $n$-te Potenz erhebt und ihr Argument mit dem Exponenten $n$ multipliziert
\begin{equation} 
\boxed{
\begin{array}{lll}
z^n&=&\Big[r\cdot \Big(\cos\left(\varphi\right)+\text{j}\sin\left(\varphi\right)\Big)\Big]^n\\\\
&=&r^n\cdot \Big[\cos\left(n\varphi\right)+\text{j}\sin\left(n\varphi\right)\Big]
\end{array}
}
\end{equation} 
\begin{equation}
\boxed{
\begin{array}{lll}
z^n&=&\Big[r\cdot e^{\text{j}\varphi}\Big]^n=r^n\cdot e^{\text{j}n\varphi}
\end{array}
} 
\end{equation} 
\section{Radizieren komplexer Zahlen}
Eine komplexe Zahl $z$ heisst eine $n$-te Wurzel aus $a$, wenn sie der algebraischen Gleichung $z^n=a$ genügt $\left(a\in\mathbb{C};\quad n\in \mathbb{N}^*\right)$.
\newline\newline
Eine algebraische Gleichung $n$-ten Grades von folgendem Typ besitzt in der Menge $\mathbb{C}$ der komplexen Zahlen stets genau $n$ Lösungen. Bei ausschliesslich reellen koeffizienten $a_i$ treten komplexe Lösungen immer paarweise in Form konjugiert komplexer Zahlen auf.
\begin{equation}
\boxed{a_nz^n+a_{n-1}z^{n-1}+\dotso + a_1z+a_0=0,\quad \left(a_i:\,\text{reell oder komplex}\right)}
\end{equation}
Die $n$ Wurzeln der Gleichung $z^n=a=a_0\cdot e^{\text{j}\varphi}$ mit $a_0>0$ und $n\in \mathbb{N}^*$ lauten
\begin{equation}
\boxed{z_k=\sqrt[n]{a_0}\cdot \Big[\cos\left(\dfrac{\alpha+k\,2\pi}{n}\right)+\text{j}\sin\left(\dfrac{\alpha+k\,2\pi}{n}\right)\Big],\quad \left(k=0,\,1,\dotso, n-1\right)}
\end{equation}
Der Hauptwert ist bei $k=0$ und für $k=1, 2, \dotso, n-1$ erhält man Nebenwerte. Die Winkel können auch mîm Gradmass angegeben werden.
\newline\newline
geometrisch liegen die zugehörige Bildpunkte aufdem Mittelpunktskreis mit dem Radius $R=\sqrt[n]{a_0}$ und bilden die Ecken eines regelmässigen $n$-Ecks.
\newline\newline
Die $n$ Lösungen der Gleichung $z^n=1$ heissen $n$-te Einheitswurzeln und lauten
\begin{equation}
\boxed{z^n=1\Longrightarrow z_k=\cos\left(\dfrac{k\,2\pi}{n}\right)+\text{j}\sin\left(\dfrac{k\,2\pi}{n}\right)=e^{\text{j}\dfrac{k\,2\pi}{n}}}
\end{equation}
\section{Logarithmieren komplexer Zahlen}
Der natürliche Logarithmus einer komplexen Zahl 
\begin{equation}
\boxed{z=r\cdot e^{\text{j}\varphi}=r\cdot e^{\text{j}\left(\varphi+k\,2\pi\right)},\quad \left(0\leq \varphi< 2\pi;\,k\in \mathbb{Z}\right)}
\end{equation}
ist unendlich vieldeutig, denn der Hauptwert des Winkels wird häufig auch im Intervall $-\pi<\varphi < \pi$
\begin{equation}
\boxed{\ln\left(z\right)=\ln\left(r\right)+\text{j}\left(\varphi+k\,2\pi\right),\quad \left(k\in \mathbb{Z}\right)}
\end{equation}
\section{Ortskurven}
\subsection{Komplexwertige Funktion einer reellen Variablen}
Die von einem reellen Parameter $t$ abhängige komplexe Zahl heisst komplexwertige Funktion $z\left(t\right)$ der reellen Variablen $t$.
\begin{equation}
\boxed{z=z\left(t\right)=x\left(t\right)+\text{j}y\left(t\right),\quad \left(a\leq z\leq b\right)}
\end{equation}
\subsection{Ortskurve einer parameterabhängigen komplexen Zahl}
Die von einem parameterabhängigen komplexen Zeiger $\underline{z}=\underline{z}\left(t\right)$ in der Gaussschen Zahlenebene beschriebene Bahn heisst Ortskurve
\begin{equation}
\boxed{\underline{z}\left(t\right)=x\left(t\right)+\text{j}y\left(t\right)}
\end{equation}
\subsection{Inversion einer Ortskurve}
Der Übergang von einer komplexen Zahl $z\neq 0$ zu ihrem Kehrwert $w=1/z$ heisst Inversion. Vorzeichenwechsel im Argument, Kehrwertbildung des Betrages von $z$. Geometrisch wird der Zeiger an der reellen Achse gespiegelt und dann gestreckt mit Faktor $1/r^2$.
\begin{equation}
\boxed{z=r\cdot e^{\text{j}\varphi}\rightarrow w=\dfrac{1}{z}=\left(\dfrac{1}{r}\right)\cdot e^{-\text{j}\varphi}}
\end{equation}
\begin{enumerate}[$(i)$]
\item Der Punkt mit dem kleinsten Abstand vom Nullpunkt führt zu dem Bildpunkt mit dem grössten Abstand und umgekehrt.
\item Ein Punkt oberhalb der reellen Achse führt zu einem Bildpunkt unterhalb der reellen Achse und umgekehrt.
\item Eine Gerade durch den Nullpunkt auf der $z$-Ebene erzeugt eine Gerade durch den Nullpunkt auf der $w$-Ebene.
\item Eine Gerade, die nicht durch den Nullpunkt auf der $z$-Ebene geht, erzeugt einen Kreis durch den Nullpunkt auf der $w$-Ebene.
\item Ein Mittelpunktskreis auf der $z$-Ebene erzeugt einen Mittelpunktskreis auf der $w$-Ebene.
\item Ein Kreis durch den Nullpunkt auf der $z$-Ebene erzeugt eine Gerade, die nicht durch den Nullpunkt verläuft auf der $w$-Ebene.
\item Ein Kreis, der nicht durch den Nullpunkt auf der $z$-Ebene verläuft erzeugt einen Kreis, der nicht durch den Nullpunkt verläuft, auf der $w$-Ebene.
\end{enumerate}
\section{Komplexe Funktionen}
\subsection{Definition einer komplexen Funktion}
Unter der komplexen Funktion versteht man eine Vorschrift, de jeder komplexen Zahl $z\in D$ genau eine komplexe Zahl $w\in W$ zuordnet. Symbolische Schreibweise sind $w=f\left(z\right)$. $D$ und $W$ sind Teilmengen von $\mathbb{C}$.
\subsection{Definitionsgleichungen elementarer Funktionen}
\subsubsection{Trigonometrische Funktionen}
\begin{equation}
\boxed{\sin\left(z\right)=z-\dfrac{z^3}{3!}+\dfrac{z^5}{5!}-\dotso}
\quad
\boxed{\cos\left(z\right)=1-\dfrac{z^2}{4!}+\dfrac{z^4}{4!}-\dotso}
\end{equation}
\begin{equation}
\boxed{\tan\left(z\right)=\dfrac{\sin\left(z\right)}{\cos\left(z\right)}}
\quad
\boxed{\cot\left(z\right)=\dfrac{\cos\left(z\right)}{\sin\left(z\right)}=\dfrac{1}{\tan\left(z\right)}}
\end{equation}
\subsubsection{Hyperbelfunktionen}
\begin{equation}
\boxed{\sinh\left(z\right)=z+\dfrac{z^3}{3!}+\dfrac{z^5}{5!}-\dotso}
\quad
\boxed{\cosh\left(z\right)=1+\dfrac{z^2}{4!}+\dfrac{z^4}{4!}-\dotso}
\end{equation}
\begin{equation}
\boxed{\tanh\left(z\right)=\dfrac{\sinh\left(z\right)}{\cosh\left(z\right)}}
\quad
\boxed{\coth\left(z\right)=\dfrac{\cosh\left(z\right)}{\sinh\left(z\right)}=\dfrac{1}{\tanh\left(z\right)}}
\end{equation}
\subsubsection{Exponentialfunktion}
\begin{equation}
\boxed{e^z=1+\dfrac{z}{1!}+\dfrac{z^2}{2!}+\dfrac{z^3}{3!}+\dotso}
\end{equation}
\subsection{Beziehungen}
\subsubsection{Eulersche Formeln}
\begin{equation}
\boxed{e^{\text{j}x}=\cos\left(x\right)+\text{j}\sin\left(x\right)}
\quad
\boxed{e^{-\text{j}x}=\cos\left(x\right)-\text{j}\sin\left(x\right)}
\end{equation}
\subsubsection{Beziehungen trigonometrischen und komplexen $e$-Funktion}
\begin{equation}
\boxed{\sin\left(x\right)=\dfrac{1}{2\text{j}}\left(e^{\text{j}x}-e^{-\text{j}x}\right)}
\quad
\boxed{\cos\left(x\right)=\dfrac{1}{2}\left(e^{\text{j}x}+e^{-\text{j}x}\right)}
\end{equation}
\begin{equation}
\boxed{\tan\left(x\right)=-\text{j}\dfrac{\left(e^{\text{j}x}-e^{-\text{j}x}\right)}{\left(e^{\text{j}x}+e^{-\text{j}x}\right)}}
\quad
\boxed{\cot\left(x\right)=\text{j}\dfrac{\left(e^{\text{j}x}+e^{-\text{j}x}\right)}{\left(e^{\text{j}x}-e^{-\text{j}x}\right)}}
\end{equation}
\subsubsection{Beziehungen trigonometrischen und Hyperbelfunktionen}
\begin{equation}
\boxed{\sin\left(\text{j}x\right)=\text{j}\cdot \sinh\left(x\right)}\quad \boxed{\cos\left(\text{j}x\right)=\cosh\left(x\right)}\quad \boxed{\tan\left(\text{j}x\right)=\text{j}\cdot \tanh\left (x\right)}
\end{equation}
\begin{equation}
\boxed{\sinh\left(\text{j}x\right)=\text{j}\cdot \sin\left(x\right)}\quad \boxed{\cosh\left(\text{j}x\right)=\cos\left(x\right)}\quad \boxed{\tanh\left(\text{j}x\right)=\text{j}\cdot \tan\left (x\right)}
\end{equation}
\subsubsection{Additionstheoreme der trigonometrischen und Hyperbelfunktionen}
\begin{equation}
\boxed{\begin{array}{lll}
\sin\left(x\pm \text{j}y\right)&=&\sin\left(x\right)\cdot \cosh\left(y\right)\pm \text{j}\cdot \cos\left(x\right)\cdot \sinh\left(y\right)\\
\cos\left(x\pm \text{j}y\right)&=&\cos\left(x\right)\cdot \cosh\left(y\right)\mp \text{j}\cdot \sin\left(x\right)\cdot \sinh\left(y\right)\\
\tan\left(x\pm \text{j}y\right)&=&\dfrac{\sin\left(2x\right)\pm \text{j}\cdot \sinh\left(2x\right)}{\cos\left(2x\right)+\cosh\left(2x\right)}\\
\end{array}}
\end{equation}
\begin{equation}
\boxed{\begin{array}{lll}
\sinh\left(x\pm \text{j}y\right)&=&\sinh\left(x\right)\cdot \cos\left(y\right)\pm \text{j}\cdot \cosh\left(x\right)\cdot \sin\left(y\right)\\
\cosh\left(x\pm \text{j}y\right)&=&\cosh\left(x\right)\cdot \cos\left(y\right)\pm \text{j}\cdot \sinh\left(x\right)\cdot \sin\left(y\right)\\
\tanh\left(x\pm \text{j}y\right)&=&\dfrac{\sinh\left(2x\right)\pm \text{j}\cdot \sin\left(2x\right)}{\cosh\left(2x\right)+\cos\left(2x\right)}\\
\end{array}}
\end{equation}
\subsubsection{Arkus- und Areafunktionen}
\begin{equation}
\boxed{\arcsin\left(\text{j}x\right)=\text{j}\cdot \arsinh\left(x\right)}\quad \boxed{\arccos\left(\text{j}x\right)=\text{j}\cdot \text{Arcosh}\left(x\right)}
\end{equation}
\begin{equation}
\boxed{\arsinh\left(\text{j}x\right)=\text{j}\cdot \arcsin\left(x\right)}\quad \boxed{\text{Arcosh}\left(\text{j}x\right)=\text{j}\cdot \arccos\left(x\right)}
\end{equation}
\begin{equation}
\boxed{\arctan\left(\text{j}x\right)=\text{j}\cdot \text{Artanh}\left(x\right)}\quad \boxed{\text{Artanh}\left(\text{j}x\right)=\text{j}\cdot \arctan\left(x\right)}
\end{equation}
\section{Komplexe Zahlen in MATLAB}
\subsection{Definition der imaginäre Einheit}
Folgende Befehle sind Berechnungen der komplexen Zahlen
\begin{enumerate}[$\texttt{>}\texttt{>}$]
\item {\color{red}\texttt{sqrt(-1) => 0.0000 + 1.0000i}}
\item {\color{red}\texttt{i\^\,5 => 0.0000 + 1.0000i}}
\item {\color{red}\texttt{i\^\,10 => -1}}
\end{enumerate}
\subsection{Operationen mit komplexen Zahlen}
\begin{enumerate}[$\texttt{>}\texttt{>}$]
\item {\color{red}\texttt{double(solve('x\^\,2+x+1')) => -0.5000 + 0.8660i, -0.5000-0.8660i}}
\item {\color{red}\texttt{z1=3+4*i => 3.0000 + 4.0000i}}
\item {\color{red}\texttt{z2=complex(3,4) => 3.0000 + 4.0000i}} 
\item {\color{red}\texttt{real(z1) => 3}}
\item {\color{red}\texttt{imag(z1) => 4}}
\item {\color{red}\texttt{(3+4*i)+(1+2*i)/(-4+3*i) => 3.0800 + 3.5600i}}
\item {\color{red}\texttt{conj(1+2*i) => 1.0000 - 2.0000i}}
\item {\color{red}\texttt{(1+2*i)*conj(1+2*i) => 5}}
\end{enumerate}
\subsection{Gauss'sche Zahlenebene}
\begin{enumerate}[$\texttt{>}\texttt{>}$]
\item {\color{red}\texttt{plot([1+2*i,2-3*i,4-5*i,-3+i,i, -6, -3-2*i], 'r*')}}
\item {\color{red}\texttt{z=-2+i = -2.0000 + 1.0000i}}
\item {\color{red}\texttt{betrag=abs(z) => betrag =  2.2361}}
\item {\color{red}\texttt{sym=abs(z) => sym =  2.2361}}
\item {\color{red}\texttt{winkel=angle(z) => winkel =  2.6779}}
\item {\color{red}\texttt{winkel\_grad=winkel/pi*180 => winkel\_grad = 153.4349}}
\end{enumerate}
\subsection{Satz von Moivre}
\begin{enumerate}[$\texttt{>}\texttt{>}$]
\item {\color{red}\texttt{(-sqrt(3)-i)\^\,10 => 5.1200e+02 - 8.8681e+02i}}
\item {\color{red}\texttt{abs((-sqrt(3)-i)\^\,10) => 1.0240e+03}}
\item {\color{red}\texttt{angle((-sqrt(3)-i)\^\,10) => -1.0472}}
\item {\color{red}\texttt{(-sqrt(3)-i)\^\,(1/3) => 0.8099 - 0.9652i}}
\item {\color{red}\texttt{double(solve('z\^\,3-(-sqrt(3)-i)')) => 0.8099 - 0.9652i, -1.2408 - 0.2188i, 0.4309 + 1.1839i}}
\item {\color{red}\texttt{compass(double(solve('z\^\,3-(-sqrt(3)-i)')))}}
\end{enumerate}
\subsection{Eulersche Formel}
\begin{enumerate}[$\texttt{>}\texttt{>}$]
\item {\color{red}\texttt{log(-1) = 0.0000 + 3.1416i}}
\item {\color{red}\texttt{log(-10*i) = 2.3026 - 1.5708i}}
\item {\color{red}\texttt{log2(sqrt(2)-sqrt(2)*i) = 1.0000 - 1.1331i}}
\item {\color{red}\texttt{exp(1+i*pi) = -2.7183 + 0.0000i}}
\item {\color{red}\texttt{2\^i = 0.7692 + 0.6390i}}
\item {\color{red}\texttt{(-i)\^\,(-i) = 0.2079}}
\end{enumerate}