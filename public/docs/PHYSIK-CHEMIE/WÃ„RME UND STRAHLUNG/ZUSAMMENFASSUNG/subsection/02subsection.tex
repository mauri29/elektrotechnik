\section{Elektromagnetische Strahlung}
\subsection{Grundlegende Eigenschaften von Wellen}
Wellen sind wandernde Schwingungen. Ein einzelner Punkt auf dem Seil wandert jedoch nicht mit, sondern schwingt einfach nur auf und ab. Mit der Welle. Mit der Welle wird keine Masse sondern Energie transportiert.
\\\\
In \textbf{Transversalwellen} schwingen die einzelne Elemente der Welle senkrecht zur Ausbreitungsrichtung. Bei \textbf{Longitudinalwellen} schwingen die Elemente der Welle in der Ausbreitungsrichtung. Man unterscheidet auch zwischen \textbf{mechanische} und \textbf{elektromagnetische Wellen}.
\\\\
Alltäglich kommen Wellen als Oberflächenwellen, seismische Wellen, Schallwellen und transversale Seilwellen. Mechanische Wellen brauchen immer in Medium zur Ausbreitung und können sich daher nicht im Vakuum ausbreiten. 
\\\\
Eine Welle wird durch drei Hauptgrössen beschrieben: die \textbf{Amplitude} $A$, die maximale Auslenkung einer lokalen Schwingung; die \textbf{Periode} $T$, die Zeit für eine ganze Schwingung eines einzelnen Punktes; und die \textbf{Wellenlänge} $\lambda$ der räumlicher Abstand zwischen zwei Wellenmaxima. Statt mit der Periode arbeitet man mit der Frequenz $[f]=\text{s}^{-1}=\text{Hz}$, also Anzahl Schwingungen pro Sekunde.
\begin{equation}
\boxed{f=\dfrac{1}{T}}
\end{equation}
Für die Fortpflanzungsgeschwindigkeit $c$ gilt
\begin{equation}
\boxed{c=\dfrac{\lambda}{T}=\lambda\cdot f}
\end{equation}
\subsection{Elektromagnetische Wellen und Licht}
Elektromagnetische Strahlung ist eine Welle, die aus einem gekoppelten elektrischen und magnetischen Feld besteht, dabei steht die schwingende elektrische Komponente stets senkrecht auf der schwingenden magnetischen Komponente. Beide Komponenten stehen auch senkrecht auf der Fortpflanzungsrichtung der Welle. Elektromagnetische Wellen sind also \textbf{Transversalwellen}.
\subsubsection{Polarisation}
Die gekoppelte Schwingung des elektrischen und magnetischen Feldes hat zur Folge, dass die Richtung der resultierenden Schwingung im allgemeinen von Ort und Zeit abhängt. Diese Eigenschaft von Licht wird als \textbf{Polarisation} beschrieben. Bei \textbf{linearer polarisiertem Licht} ist die Richtung der resultierenden Schwingung zeitlich und örtlich konstant, während \textbf{zirkular polarisiertes Licht} sich dadurch auszeichnet, dass die Richtung der Schwingung sich im Kreis dreht und örtlich eine kreisförmige Spirale bildet.
\subsubsection{Vakuumlichtgeschwindigkeit}
Die schwingenden Felder können sich im Vakuum ausbreiten und transportieren dabei die in den Feldern gespeicherte elektromagnetische Energie. Ohne Fort-pflanzungsfähigkerit im Vakuum würde das Sonnenlicht die Erde nicht erreichen. Elektromagnetische Wellen können sich auch in Medien ausbreiten. Ihre Fortpflanzungsgeschwindigkeit hängt dann vom Medium ab. Sie ist jedoch niemals grösser als im Vakuum. Die Ausbreitungsgeschwindigkeit elektromagnetischer Wellen im Vakuum heisst \textbf{Lichtgeschwindigkeit} und beträgt
\begin{equation}
\boxed{c=2.99795\cdot 10^{8}\,\text{ms}^{-1}}
\end{equation}
\subsubsection{Das elektromagnetische Spektrum}
Elektromagnetische Wellen sind nach Frequenz und Wellenlänge kategorisiert. Die \textbf{Gammastrahlung} entsteht beim radioaktiven Zerfall von Atomkernen und hat eine Wellenlänge von $10^{-3}\text{nm}$ und besitzt eine sehr hohe Energie. Gammastrahlung verursacht Krebs und wird daher auch als Strahlentherapie eingesetzt, um Tumore im menschlichen Körper zu zerstören.
\\\\
Als \textbf{Röntgenstrahlung} werden elektromagnetische Wellen bezeichnet, deren Wellenlänge zwischen $10^{-3}\text{nm}$ und $1\text{nm}$ liegt. Röntgenstrahlung entsteht wenn Elektronen mit sehr hoher Geschwindigkeit von einem Hindernis oder einem Magnetfeld abgebremst werden. Radioaktive Strahlung kann aber auch durch Luminiszenz bei hochenergetischen Energieübergängen von Elektronen in Materie entstehen. Eine hohe Dosis von Röntgenstrahlung ist krebserregend. Röntgenstrahlung eignet sich sehr gut für die chemische und mikrostrukturelle Analyse von organischen und anorganischen Materialien. 
\\\\
Die Strahlung im Wellenlängenbereich zwischen $1\text{nm}$ und $380\text{nm}$ wird als ultra-violett oder \textbf{UV-Strahlung} bezeichnet. Elektromagnetische Strahlung im Wellenlängenbereich von $380\text{nm}$ bis $720\text{nm}$ ist für das Auge \textbf{sichtbare Licht}. Verschiedene Wellenlängenbereiche nehmen Menschen als Farben wahr: $380\text{nm}$ bis $490\text{nm}$: violett/blau, $490\text{nm}$ bis $550\text{nm}$: grün, $550\text{nm}$ bis $600\text{nm}$: gelb/orange, $600\text{nm}$ bis $720\text{nm}$: rot.
\\\\
Strahlung im Wellenlängenbereich zwischen $720\text{nm}$ bis $1\text{mm}$ wird \textbf{Infrarotstrahlung}, oder IR-Strahlung genannt. Der Bereich zwischen $720\text{nm}$ und $3\mu\text{m}$ wird als \textbf{nahes Infrarot} oder NIR genannt. Als mittleres Infrarot MIR liegt zwischen $3\mu\text{m}$ und $100\mu\text{m}$. Das ferne Infrarot FIR liegt zwischen $100\mu\text{m}$ und $1\text{mm}$.
\\\\
Oberhalb des infraroten Bereichs sind die Kategorien \textbf{Mikrowellen}, welche zwischen $1\text{mm}$ und $30\text{cm}$ liegen. \textbf{Radiowellen} liegen zwischen $30\text{cm}$ und $1\text{km}$ und \textbf{Längenwellen} kleiner als $1\textbf{km}$.
\\\\
Je grösser die Wellenlänge, desto länger ist die Distanz, über welche kabellos kommuniziert werden kann. Da Wasser Strahlung mit $10\text{cm}$ Wellenlänge sehr gut absorbiert, werden die Mikrowellen mit $12\text{cm}$ Wellenlänge in Mikrowellenöfen für das Aufheizen von Lebensmitteln verwendet.
\subsection{Interferenzphänomene}
\subsubsection{Interferenz}
Interferenz ist die Überlagerung mehrerer Wellen an einem Punkt im Raum. Werden mit gleicher \textbf{Frequenz} und \textbf{konstanter Phasenbeziehung} überlagert, so  treten einfach zu beobachtende Interferenzmuster auf. Interferenz tritt als Verstärkung oder Abschwächung, wobei es tritt \textbf{konstruktive} (Amplitudevergrösserung) und \textbf{destruktive Interferenz} (Amplitudeerniedrigung) auf.
\\\\
Emittieren zwei Quellen Wellen mit gleicher Wellenlänge $\lambda$ und Amplutide, so interferieren ihre Wellen konstruktiv bzw. destruktive an den Orten im Raum, für welche der Gangunterschied $\triangle$ der Wellen folgende Werte annimmt.
\begin{equation} 
\boxed{\text{Konstruktiv: }\triangle=2m\cdot\dfrac{\lambda}{2},\quad m\in\mathbb{Z}}
\end{equation} 
\begin{equation} 
\boxed{\text{Destruktiv: }\triangle=\left(2m+1\right)\cdot\dfrac{\lambda}{2},\quad m\in\mathbb{Z}}
\end{equation} 
\subsubsection{Interferenz an dünnen Schichten}
Interferenz von Licht wird im Alltag oft beobachtet, wenn reflektierende Oberflächen mit dünnen transparenten Schichten bedeckt sind. Treffen Wellen auf eine Mediumsgrenze, so tritt \textbf{Reflexion} auf. An einer dünnen Schicht findet diese Reflexion zweimal statt, beim Eintritt in die Schicht und beim Austritt. Die beiden reflektierten Strahlen interferieren. Je nach Gangunterschied kann es zu Verstärkung oder Auslöschung kommen. Die Art der Interferenz hängt von der Wellenlänge, also von der Farbe, des Lichts ab.
\subsubsection{Beugung}
Keine Welle läuft ungestört durch den Raum; Hindernisse stehen ihr im Weg. Sind diese gross im Vergleich mit der Wellenlänge, so kommt es zu Schattenwurf, Reflexion und Brechung. Sind diese von vergleichbarer Grössenordnung wie die Wellenlänge, so kommt es zu Beugung.
\\\\
Bei Schallwellen äussert sich die Beugung dadurch. Trifft eine \textbf{Lichtwelle} auf ein Hindernis, so beobachtet man hinter dem Hindernis helle bzw. dunkle Gebiete, welche nicht mittels geometrischer Strahlenausbreitung erklärt werden können. Die Welle wird von der geradlinigen Ausbreitung abgelenkt.
\\\\
Die Ablenkung von Wellen an Hindernissen lässt sich mit dem \textbf{Huygen'schen Prinzip} begründen. Dieses besagt, dass sich eine ebene Wellenfront aus der Überlagerung vieler kleiner Kreis- bzw. Kugelwellen, sog, \textbf{Elementarwellen} ergibt, wobei jeder Punkt der Wellenfront wiederum Ausgangspunkt neuer Elementarwellen ist.
\subsubsection{Beugung von Licht an einem Gitter}
Werden auf einer Glasplatte maschinell zahlreiche Spalten mit dem gleichen Abstand eingereizt, dann spricht man von einem \textbf{optischen Gitter}. Das Gitter wird durch den Abstand der Gitterlinien, die sogenannte Gitterkonstante $g$ charakterisiert. Trifft eine ebene Lichtwelle auf ein optisches Gitter, entsteht für jeden Spalt ein Beugungsmuster. Für die Richtung ist der Wegunterschied $\triangle s$ zwischen zwei benachbarten Wellen, wobei $\varphi$ der Abstrahlwinkel ist, gegeben durch
\begin{equation}
\boxed{\triangle s=g\cdot \sin\left(\varphi\right)=m\cdot \lambda,\quad m=\pm 1, \pm 2, \dotso}
\end{equation}
Hinter dem Gitter entsteht wiederum ein Strichmuster mit einer Linie in der Mitte $m=0$ und mehreren Seitenlinein zu den Ordnungen $m=\pm 1, \pm 2, \dotso$. Die Zahl $m$ entspricht dem \textbf{Ordnungszahl} der Beugung. Das Licht unterschiedlicher Wellenlängen wird verschieden stark gebeugt. Mit einem optischen Gitter kann das Wellenlängenspektrum analysiert werden.
\subsection{Der Begriff des Spektrums}
Beugung wird genutzt, um Strahlung in seine Wellenlänge zu zerlegen. Das quantitative Resultat dieser Zerlegung wird \textbf{Spektrum} genannt. Der einfallende Strahlung wird von einer Blende auf einen schmalen Streifen eingeschränkt, welcher dann über einen fokussierenden Hohlspiegel auf ein reflektierendes Liniengitter abgebildet wird. Am Gitter erfolgt mittels Beugung die Aufteilung der Strahlung der Strahlung in seine Wellenlängen. Dabei werden Gitterkonstante $g$ und Einstrahlwinkel $\varphi$ so gewählt, dass es für jede Wellenlänge genau einen Reflexionswinkel mit Intensität ungleich null gibt. Die reflektierten Strahlen werden über einen zweiten Spiegel auf einen Detektor, hier einen CCD-Sensor, abgebildet. Der Ort, an welchem ein Teilstrahl mit bestimmter Wellenlänge auf dem Sensor auftritt, gibt die Wellenlänge des Strahls an. Der CCD misst die Intensität jedes Teilstrahls in Abhängigkeit der Wellenlänge.
\\\\
Je nach Mechanismus der Lichterzeugung treten zwei grundsätzlich verschiedene Typen von Spektren auf: Das \textbf{kontinuerliche Spektrum} (Sonnenlicht, alle Farben vorkommen) und das \textbf{Linienspektrum} (Quellen mit schmalen Linien).
\subsubsection{Farbwahrnehmung}
Das Farbsehens des menschlichen Auges hat drei verschiedene Rezeptoren: Typ-S, Typ-M und Typ-L die für je einen anderen Wellenlängenbereich ('rot', `grün', blau') und für das Sehen bei Tageslicht verantwortlich sind. Das Typ-R ermöglicht das Sehen bei geringer Lichtintensität, decken aber nur einen limitierten Wellenlängenbereich ab, weshalb man im DUnkeln Farben schlecht erkennt.
\subsubsection{Additive Farbmischung}
Licht, das nur einen Typ erregt, erzeugt den Eindruck der drei Grundfarben rot, grün oder blau. Werden zwei oder drei Typen gleichzeitig angeregt entstehen Mischfarben. Dies nennt man die \textbf{additive Farbmischung}. Die Werte auf jeder Achse rot, grün und blau sind ganzzahlig und liegen im Intervall $[0, 255]$. 
\subsubsection{Körperfarben - subtraktive Farbmischung}
Wird eine farbige Oberfläche mit weissem Licht bestrahlt, so werden gewisse Wellenlängenbereiche von der Oberfläche absorbiert und andere werden reflektiert. Bei Körperfarben findet die Mischung von Farben durch Subtraktion statt durch Addition statt. Werden Farben gemischt, so werden zusätzliche Wellenlängenbereiche absorbiert.
\\\\
Die Basis für die subtraktive Farbmischung bilden die Farben gelb (weiss minus blau), magenta (weiss minus grün) und cyan (weiss minus rot) und bilden das Koordinatensystem des CMY-Farbschemas, welcher von Farbdruckern verwendet wird. Meistens wird noch schwarz als vierte Farbe ergänzt, was dem CMYK-Farbraum ergibt.
\section{Radiometrie und Photometrie}





