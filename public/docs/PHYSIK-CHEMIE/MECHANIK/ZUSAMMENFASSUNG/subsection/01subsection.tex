Ein \textbf{Massensystem} $S\{M\}$ besteht aus \textbf{Massenpunkten} $M$. Jeder Massenpunkt $M$ hat eine bestimmte geometrische Lage in Abhängigkeit der Zeit im Euklidischem Vektorraum. Zur Beschreibung der Lage von Massensystemen $S$ braucht man Bezugspunkten und ein geeignetes \textbf{Koordinatensystem}.
%%%%%%%%%%%%%%%%%%%%%%%%%%%%%%%%%%%%%%%%%%%%%%%%%%%%%%%%%%%%%%%%%%%%%%%%%%%%%%%%%%%%%%%%%%%%
\subsection{Koordinaten bezüglich eines Bezugspunktes}
Ein Bezugspunkt $O$ ist ein starrer Körper, der für alle Zeiten eine feste konstante Lage besitzt. Der Bezugspunkt ist durch ein Koordinatensystem, durch drei unabhängige Grössen mit drei fest orthogonal gerichtete Achsen mit unendlicher Ausdehnung ohne Behaftung von Materie, orientiert. 
\newline\newline
Die Lage vom Massensystem $S\{M\}$ aus Massenpunkten $M$ kann demzufolge bezüglich einen Bezugspunkt $O$ beschrieben werden. Die Koordinaten jedem Massenpunkt $M$ können durch Abstände und Winkel bezüglich des Bezugspunktes $O$ gebildet werden. Die drei hauptsächliche Koordinaten sind die kartesische, die zylindrische und die sphärische Koordinaten.
\newline\newline
%%%%%%%%%%%%%%%%%%%%%%%%%%%%%%%%%%%%%%%%%%%%%%%%%%%%%%%%%%%%%%%%%%%%%%%%%%%%%%%%%%%%%%%%%%%%
\subsection{Kartesische Koordinaten}
Die Lage des Massenpunktes $M$ bezügich $O$ zur Zeit $t$ im Euklidischem Vektorraum ist durch drei Projektionen auf die kartesische Achsen orientiert. Die drei kartesische Koordinaten eines Massenpunktes $M$ sind $x:=\overline{OM_x}$, $y:=\overline{OM_y}$ und $z:=\overline{OM_z}$ mit dem Definitionsintervall $\left(-\infty, \infty\right)$.
\newline\newline
Man spricht von einer Lageänderung eines Massenpunktes, wenn zu Zeiten $t_1$ und $t_2$ zwei verschiedene Lageorten durch die kartesische Koordinaten erzeugt werden. Die kartesische Koordinaten erfolgen durch drei kartesische Funktionen in Abhängigkeit von der Zeit. Diese Funktionen stellen eine Bahnkurve durch parametrische Gleichungen dar.
\begin{equation}
\boxed{x(t)=f_x(t)}\quad \boxed{y(t)=f_y(t)} \quad \boxed{z(t)=f_z(t)}
\end{equation}
Werden zwei der drei Grössen konstant gehalten und die andere wird verändert, so beschreibt den Massenpunkt je eine Gerade. Somit entstehen drei Geraden, welche zur Achse der jeweils veränderten Koordinate parallel sind und heissen \textbf{kartesische Koordinatenlinien}. 
\newline\newline
Wird nur eine Koordinate konstant gehalten und die zwei anderen werden verändert, so beschreibt den Massenpunkt eine Ebene. Somit entstehen drei Ebenen, welche zu den Ebenen $O_{xy}$, $O_{yz}$ und $O_{xz}$ parallel sind und heissen \textbf{kartesische Koordinatenflächen}. Die kartesischen Koordinatenlinien sind geradlinig orthogonal.
%%%%%%%%%%%%%%%%%%%%%%%%%%%%%%%%%%%%%%%%%%%%%%%%%%%%%%%%%%%%%%%%%%%%%%%%%%%%%%%%%%%%%%%%%%%%
\subsection{Zylindrische Koordinaten}
Drei Funktionen der Zeit im Zeitintervall ergeben die Bewegung des Massenpunktes $M$ in zylindrischen Koordinaten.
\begin{equation}
\boxed{\rho\left(t\right)=f_{\rho}\left(t\right)}\quad \boxed{\varphi\left(t\right)=f_{\varphi}\left(t\right)} \quad \boxed{z\left(t\right)=f_z(t)}
\end{equation}
Sei $M_{xy}$ die Projektion des Massenpunktes in die Ebene $O_{xy}$. Der Abstand $\rho=\overline{OM_{xy}}$, der Winkel $\varphi:=\measuredangle\left(Ox,OM_{xy}\right)$ zwischen der $x$-Achse und die Projektion in die $xy$-Ebene sowie die $z$-Komponente des Massenpunktes sind die zylindrischen Koordinaten von $M$ zur Zeit $t$. Der Winkel $\varphi$ ist in Bogenmass angegeben und im gegenuhrzeigersinn positiv gemessen. Den Definitionsintervalle gehören $\rho\in\left[0\infty\right)$, $\varphi\in\left(\infty, \infty\right)$ und $z\in\left(\infty, \infty\right)$ 
\newline\newline
Die zugehörige Umwandlungen von der kartesischen in die zylindrische Koordinaten sind
\begin{equation}
\boxed{\rho=\sqrt{x^2+y^2}}\quad \boxed{\varphi=\arctan\left(\dfrac{y}{x}\right)}\quad\boxed{z=z}
\end{equation}
Von der zylindrischen in die kartesischen Koordinaten lauten die Umwandlungen
\begin{equation}
\boxed{x=\rho\cdot\cos\left(\varphi\right)}\quad \boxed{y=\rho\cdot\sin\left(\varphi\right)}\quad \boxed{z=z}
\end{equation}
Werden zwei der drei Grössen konstant gehalten und die andere wird verändert, so entsteht im betrachteten Massenpunkt $M$ die $\rho$-Koordinatenlinie $\overline{\rho}$ als zylindrisch-radiale Halbgerade, die $\varphi$-Koordinatenlinie $\overline{\varphi}$ als Parallelkreis und die $z$-Koordinatenlinie $\overline{z}$ als axiale Gerade parallel zur $z$-Achse. Die drei Koordinatenlinien sind orthogonal zueinander. Die zylindrischen Koordinatenlinien sind \textbf{krummlinig orthogonal}.
\newline\newline
Wird nur eine Koordinate konstant gehalten, so entstehen Kreiszylinderflächen um die $z$-Achse ($\rho$ konstant), Halbebenen durch $O_z$ ($\varphi$ fest) und Ebenen parallel zu $O_{xy}$ ($z$ fest). Diese \textbf{zylindrische Koordinatenflächen} sind zueinander orthogonal.
\newline\newline
Ist die $z$-Koordinate konstant, so wird die Bewegung des Massenpunktes $M$ durch $\rho=f_{\rho}(t)$ und $\varphi=f_{\varphi}(t)$ beschrieben und heissen in diesem Fall die zylindrischen Koordinaten $\rho$ und $\varphi$ \textbf{Polarkoordinaten} und wird in der \textbf{Kreisbewegung} benutzt. 
%%%%%%%%%%%%%%%%%%%%%%%%%%%%%%%%%%%%%%%%%%%%%%%%%%%%%%%%%%%%%%%%%%%%%%%%%%%%%%%%%%%%%%%%%%%%
\subsection{Sphärische Koordinaten}
Drei Funktionen der Zeit im zeitintervall ergeben die Bewegung des Massenpunktes $M$ in sphärischen Koordinaten.
\begin{equation}
\boxed{r=f_r\left(t\right)}\quad \boxed{\theta=f_{\theta}\left(t\right)}\quad \boxed{\psi=f_{\psi}\left(t\right)}
\end{equation}
Der Abstand $r:=\overline{OM}$ sowie die Winkel $\theta:=\measuredangle\left(O_z, OM\right)$ und $\psi=\measuredangle\left(Ox, OM_{xy}\right)$ sind die sphärischen Koordinaten von $M$ in $O_{xyz}$ mit den Definitionsintervallen $r\in\left[0,\infty\right)$, $\theta\in\left[-\infty, \infty\right]$, $\psi\in\left(-\infty, \infty\right)$. Der Winkel $\psi$ übereinstimmt den Winkel $\varphi$ aus den zylindrischen Koordinaten.
\newline\newline
Die zugehörige Umwandlungen von der kartesischen in die sphärischen Koordinaten sind
\begin{equation} 
\boxed{r=\sqrt{x^2+y^2+z^2}}\quad \boxed{\theta=\arctan\left(\dfrac{\sqrt{x^2+y^2}}{z}\right)}\quad \boxed{\psi=\arctan\left(\dfrac{y}{x}\right)}
\end{equation} 
Von den sphärischen in die kartesische Koordinaten lauten die Umwandlungen
\begin{equation}
\boxed{x=r\sin\left(\theta\right)\cos\left(\psi\right)}\quad \boxed{y=r\sin\left(\theta\right)\sin\left(\psi\right)}\quad \boxed{z=r\cos\left(\theta\right)}
\end{equation}
Die \textbf{sphärischen Koordinatenlinien} werden analog zu den zylindrischen Koordinatenlinien konstruiert. Im betrachteten Massenpunkt $M$ entsteht als $\psi$-Koordinatenlinie ein Parallelkreis, als $\theta$-Koordinatenlinie ein Meridiankreis und als $r$-Koordinaten-linie eine radiale Halbgerade. Die sphärischen Koordinatenlinien sind \textbf{krummlinig orthogonal}. Die \textbf{sphärische Koordinatenfläche} für festgehaltenes $r$ ist eine Kugel.