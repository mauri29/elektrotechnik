%%%%%%%%%%%%%%%%%%%%%%%%%%%%%%%%%%%%%%%%%%%%%%%%%%%%%%%%%%%%%%%%%%%%%%%%%%%%%%%%%%%%%%%%%%%%
\subsection{Definition der Kraft}
Kräfte beeinflussen Körperzustände. Sie können den Bewegungszustand eines Körpers, die Geschwindigkeit, die Arbeit und die Energie eines Körpers beeinflussen und somit den Körper deformieren. Die Einheit der Kraft ist das Newton.
\newline\newline
Wirkt eine Kraft auf ein Massensystem, so wird die Wirkung abhängen von dem Betrag $\Big\vert \overrightarrow{F}\Big\vert$, der Richtung $\dfrac{\overrightarrow{F}}{\Big\vert\overrightarrow{F}\Big\vert}$ und dem Angriffspunkt und die Wirklinie der Kraft. Die Kräfte sind punktgebundene Vektoren. Eine Kräftegruppe ist eine Gruppe von Massenangriffspunkte mit den zugehörigen Angriffskräfte.
\begin{equation}
\boxed{\left\{\left\{A_1 | \overrightarrow{F}_1\right\}, \,\left\{A_2 | \overrightarrow{F}_2\right\},\,\dotso\right\}}
\end{equation}
Die Vektoranteile der Kräfte mit verschiedenen Massenangriffspunkte kann man durch das Parallelogramregel addieren. Die Addition heisst die resultierende Kraft.
\begin{equation}
\boxed{\overrightarrow{R}=\displaystyle \sum_{i=1}^n\overrightarrow{F}_i}
\end{equation}
Die Stärke der Wechselwirkung der Betrag der Kraft wird gemessen in \textbf{Newton}, also die Kraft welche einem Kilogramm Masse 1kg m s$^{-2}$ die Beschleunigung erteilen kann. Richtung und Ort können durch die Komponenten des Kraftvektors durch die Koordinaten des Angriffpunktes bezüglich eines passend gewählten \textbf{Bezugskörpers} ermittelt werden.
\newline\newline
Kräfte dürfen entlang ihrer Wirklinie beliebig verschoben werden. Kräfte dürfen nicht parallel verschoben werden. 
\newline\newline
Man spricht von \textbf{Fernkräfte}, wenn die Wechselwirkung zwischen den Massensystemen ohne Berührung stattfinden. Beispiele sind Gravitationskräfte zwischen der Sonne und den Planeten, die elektromagnetische Kräfte zwischen geladenen Partikeln, die induktiven Wechselwirkungen zwischen Stator und Rotor in einem Elektromotor oder Generator. Die Gewichtskraft ist eine Fernkraft, welche die Erde auf einen Körper das Gewicht verursacht.
\newline\newline
Man spricht von \textbf{Kontaktkräfte}, wenn die Wechselwirkung zwischen den Massensystemen mit Berührung auf einem gemeinsamen Massenangriffspunkt stattfinden. Beispiele sind Kräfte zwischen Kugel und Ring in einem Kugellager, die mechanische Wechselwirkung zwischen Dampf und Turbinenschaufel in einer Dampfturbine, die Kräfte zwischen dem Lastkörper und dem Träger sowie zwischen Träger und den Stützen. 
%%%%%%%%%%%%%%%%%%%%%%%%%%%%%%%%%%%%%%%%%%%%%%%%%%%%%%%%%%%%%%%%%%%%%%%%%%%%%%%%%%%%%%%%%%%%
\subsection{Das Reaktionsprinzip}
Übt ein Massenangriffspunkt $A_1$ auf einen Massenangriffspunkt $A_2$ die Kraft $\left\{A_2 | \overrightarrow{F}\right\}$ aus, so wirkt seinerseits der Massenangriffspunkt $A_2$ auf $A_1$ mit der Gegenkraft $\left\{A_1 | -\overrightarrow{F}\right\}$.  
\newline\newline
Der Vektoranteil der Gegenkraft wird als \textbf{Reaktion} genannt und entspricht dem negativen Vektoranteil der Kraft. Eine Kraft existiert nicht ohne ihre Reaktion mit negativen Vektoranteil. Kraft und Gegenkraft sind entgegengesetzt, sind aber auf gleiche \textbf{Wirkungslinie} durch $A_1$ und $A_2$.
%%%%%%%%%%%%%%%%%%%%%%%%%%%%%%%%%%%%%%%%%%%%%%%%%%%%%%%%%%%%%%%%%%%%%%%%%%%%%%%%%%%%%%%%%%%%
\subsection{Innere und äussere Kräfte}
Das Massensystem ist das betrachtete System, das freigeschnitten oder abgegrenzt wird. Dabei treten Wechselwirkungen in Form von Kräften beim Auftrennen von Kontakten. Eine \textbf{innere} oder \textbf{äussere Kraft} bezeichnet je nachdem ob der Massenangriffspunkt der Reaktion innerhalb oder ausserhalb des Massensystems liegt. Der Unterschied liegt in der \textbf{Systemabgrenzung} des Massensystems $S$. 
%%%%%%%%%%%%%%%%%%%%%%%%%%%%%%%%%%%%%%%%%%%%%%%%%%%%%%%%%%%%%%%%%%%%%%%%%%%%%%%%%%%%%%%%%%%%
\subsection{Verteilte Kräfte und Kraftdichte} 
Die Kontaktkräfte sind auf einer endlichen Berührungsfläche verteilt. Solche Kräfteverteilungen auf einer Fläche werden \textbf{Flächenkräfte} bezeichnet. Analog sind die auf einen endlichen Raumteil verteilten Fernkräfte \textbf{Raumkräfte}.
\newline\newline
Die Flächenkraftdichte $\overrightarrow{s}\left(Q\right)$ ist eine spezifische Kontaktkraft je Flächeninhalt mit dem Massenangriffspunkt $Q$ und dem Vektoranteil $\overrightarrow{s}$. Der Vektoranteil der totalen Kraft auf ein Flächenstück $\triangle A$ sei $\triangle \overrightarrow{F}$. Der Betrag $\Big\vert\overrightarrow{s}\Big\vert$ hat die DImension Kraft je Flächeninhalt. Die Flächenkraftdichte wird auch \textbf{Spannungsvektor} genannt.
\begin{equation}
\boxed{\overrightarrow{s}:=\displaystyle \lim_{\triangle A\rightarrow 0} \dfrac{\triangle \overrightarrow{F}}{\triangle A}}
\end{equation}
Die \textbf{Raumkraftdichte} $\overrightarrow{f}\left(Q\right)$ ist eine spezifische Fernkraft je Volumeneinheit mit dem Massenangriffspunkt $Q$ und dem Vektoranteil $\overrightarrow{f}$. Der Vektoranteil der totalen Kraft auf ein Körperstück $\triangle V$ sei $\triangle \overrightarrow{F}$. Der Betrag $\Big\vert\overrightarrow{f}\Big\vert$ hat die DImension einer Kraft je Volumeneinheit.
\begin{equation}
\boxed{\overrightarrow{f}:=\displaystyle \lim_{\triangle V\rightarrow 0} \dfrac{\triangle \overrightarrow{F}}{\triangle V}}
\end{equation}
Die \textbf{Linienkraftdichte} $\overrightarrow{q}\left(Q\right)$ ist die spezifische Kraft je Längeneinhet mit dem Massenangriffspunkt $Q$ und dem Vektoranteil $\overrightarrow{q}$, welche auf einen linienförmigen Massensystem wirkt. Der Vektoranteil der totalen Kraft auf ein Körperstück $\triangle s$ sei $\triangle \overrightarrow{F}$. Der Betrag $\Big\vert\overrightarrow{q}\Big\vert$ hat die Dimension einer Kraft je Längeneinheit.  
\begin{equation}
\boxed{\overrightarrow{q}:=\displaystyle \lim_{\triangle s\rightarrow 0} \dfrac{\triangle \overrightarrow{F}}{\triangle s}}
\end{equation}