Hiermit wird die Beziehung zwischen dem kinematischen Zustand \textbf{Ruhe} eines materiellen Massensystems und das Gleichgewicht der auf das materiellen Systems wirkende Kräfte diskutiert.
%%%%%%%%%%%%%%%%%%%%%%%%%%%%%%%%%%%%%%%%%%%%%%%%%%%%%%%%%%%%%%%%%%%%%%%%%%%%%%%%%%%%%%%%%%%%
\subsection{Definitionen}
Ein materielles Massensystem $S$ ist in \textbf{Ruhe}, falls die Geschwindigkeiten null sind. Das System $S$ ist in \textbf{momrntaner Ruhe}, falls die obige Bedingung zum betrachteten Zeitpiunkt $t_0$ erfüllt ist
\begin{equation}
\boxed{\overrightarrow{v}_M=\overrightarrow{0},\quad \forall M\in S}
\end{equation}
Die \textbf{Ruhelage} eines materiellen Massensystems existiert, sobald dieses zu einem beliebigen Zeitpunkt in Ruhe ist, so bleibt es für alle Zeiten in dieser Lage in Ruhe. Für die Ruhelage wird auch den Begriff \textbf{Gleichgewichtslage} verwendet. 
\newline\newline 
Materielle Massensysteme sind oft \textbf{Bindungen} und schränken die Bewegungs-möglichkeites des Systems ein und sind mit Kräfte und Momente verknüpft, welche \textbf{Bindungskräfte} bzw. \textbf{Bindungsmomente} heissen.  Jeder Komponente der Bindungskräfte und -momente nennt man \textbf{Bindungskomponente} und entspricht eine linear unabhängige Gleichung zwischen den Bewegungszuständen der verbundenen Körper.
\newline\newline
\textbf{Innere Bindungen} sind Einschränkungen der Bewegungsfreiheit von Bestand-teilen des materiellen Massensystems $S$ relativ zueinander und die zugehörige Kräfte sind die \textbf{innere Bindungskräfte}.
\newline\newline
\textbf{Äussere Bindungen} sind Einschränkungen der Bewegungsfreiheit von Randpunkten des materiellen Massensystems $S$ und die zugehörige Kräfte sind die \textbf{innere Bindungskräfte}.
\newline\newline
Ein \textbf{virtueller Bewegungszustand} besteht aus einer Menge von willkürlich wählbaren gedachten Geschwindigkeiten der materiellen Massenpunkte eines Massensystems. Der virtuelle Bewegungszustand braucht mit der willkürlich Bewegung des Massensystems in keiner Beziehung zu stehen. Virtuelle Geschwin-digkeiten $\tilde{\overrightarrow{v}}$ werden mit einer Tilde bezeichnet.
\newline\newline
Ein \textbf{zulässiger virtueller Bewegungszustand eines Massensystems} ist eine Verteilung von gedachten Geschwindigkeiten, die mit den inneren und äusseren Bindungen des Systems verträglich sind, sonst aber beliebig wählbar bleiben. Der Satz der projizierten Geschwindigkeiten muss gelten.
\newline\newline
Eine Bindung heisst \textbf{einseitig}, wenn sie durch eine wirkliche BEwegung gelöst oder aufgehoben werden kann, sonst heisst sie \textbf{vollständig}.
\newline\newline
In einer Bindung zwischen starren Systemen, welche unabhängig und nicht durch Antrieb beeinflusst ist, heissen Bindungskräfte parallel zu zulässigen virtuellen Geschwindigkeiten in der Bindung \textbf{Reibungskräfte}. Bindungsmomente parallel zu zulässigen virtuellen Rotationsgeschwindigkeiten heissen \textbf{Reibungsmomente}. Eine unnachgiebige Bindung zwischen starren Systemen heisst \textbf{reibungsfrei}, falls in ihr alle Reibungskräfte und -momente verschwinden.
%%%%%%%%%%%%%%%%%%%%%%%%%%%%%%%%%%%%%%%%%%%%%%%%%%%%%%%%%%%%%%%%%%%%%%%%%%%%%%%%%%%%%%%%%%%%
\subsection{Berechnung von virtuellen Leistungen}
Für einen virtuellen Bewegungszustand kann die \textbf{virtuelle Leistung} $\tilde{\mathcal{P}}$ einer Kraft $\overrightarrow{F}$ berechnet werden, deren Massenangriffspunkt sich mit der virtuellen Geschwindigkeit $\tilde{\overrightarrow{v}}_M$ bewegt
\begin{equation}
\boxed{\tilde{\mathcal{P}}=\overrightarrow{F}\bullet \tilde{\overrightarrow{v}}_M}
\end{equation}
Die \textbf{virtuelle Gesamtleistung} einer Kräftegruppe ist die Summe der virtuellen Leistungen der Einzelkräfte.
%%%%%%%%%%%%%%%%%%%%%%%%%%%%%%%%%%%%%%%%%%%%%%%%%%%%%%%%%%%%%%%%%%%%%%%%%%%%%%%%%%%%%%%%%%%%
\subsection{Das Grundprinzip der Statik}
Ein beliebig abgegrenztes Massensystem befindet sich dann und nur dann in einer Ruhelage, wenn in dieser Lage die virtuelle Gesamtleistung aller inneren und äusseren Kräfte, einschliesslich der inneren und äusseren Bindungskräfte, bei jedem virtuellen Bewegungszustand des Systems null ist. Dies ist fie Grunddefinition des \textbf{Prinzips der virtuellen Leistungen}, kurz PdvL.
\newline\newline
Aus dem PdvL lassen sich die inneren und äusseren Kräfte in einer Ruhelage berechnen. Zusatzbedingungen können unter anderen die richtige Richtung der Normalkraft in einer einseitigen Bindung, die Standfestigkeit oder die Haftreibung sein. Bei der Anwendung des PdvL wählt man spezielle virtuelle Bewegungszustände und leitet mit ihrer Hilfe aus 
\begin{equation}
\boxed{\tilde{P}=0} 
\end{equation}
notwendige Bedingungen für die Ruhelage her. Damit kann man Bindungskräfte bestimmen.
%%%%%%%%%%%%%%%%%%%%%%%%%%%%%%%%%%%%%%%%%%%%%%%%%%%%%%%%%%%%%%%%%%%%%%%%%%%%%%%%%%%%%%%%%%%%
\subsection{Der Hauptsatz der Statik}
Alle äussere Kräfte, einschliesslich der äusseren Bindungskräfte, welche auf ein Massensystem in einer \textbf{Ruhelage} wirken, müssen notwendigerweise im \textbf{Gleichgewicht} sein. Die Gleichgewichtsbedingungen für die Kräftegruppe von äusseren Kräften sind also notwendige Bedingungen, welche in einer Ruhelage eines beliebigen Massensystems mit frei wählbarer Abgrenzung erfüllt sein müssen.
\begin{equation} 
\boxed{\overrightarrow{R}^{\left(a\right)}=\overrightarrow{0}}\quad \boxed{\overrightarrow{M}_O^{\left(a\right)}=\overrightarrow{0}}
\end{equation} 
Man erteile dem Massensystem mit beliebiger Abgrenzung, das sich in einer Ruhelage befindet, eine beliebige moemntane Starrkörperbewegung. Die inneren Kräfte, einschliesslich der inneren Bindungskräfte, bilden gemäss Reaktionsprinzip eine Nullgruppe, da in der Kräftegruppe der inneren Kräfte am Massensystem mit jeder inneren Kraft auch ihre Reaktion enthalten ist. Die Resultierende und das Moment der Kräftegruppe der inneren Kräfte erfülen die Gleichungen $\overrightarrow{R}^{\left(i\right)}=\overrightarrow{0}$ und $\overrightarrow{M}_O^{\left(i\right)}=\overrightarrow{0}$.
\newline\newline
Mit den Bezeichnungen $\overrightarrow{R}^{\left(a\right)}$ und $\overrightarrow{M}_O^{\left(a\right)}$ für die Resultierende und das Moment der äussere Kräfte bezüglich eines beliebigen Bezugspunktes $O$ beträgt dann die virtuelle Gesamtleistung am Massensystem
\begin{equation} 
\boxed{\tilde{\mathcal{P}}=\tilde{\overrightarrow{v}}_O\bullet \overrightarrow{R}^{\left(a\right)}+\tilde{\overrightarrow{\omega}}\bullet \overrightarrow{M}_O^{\left(a\right)}=0}
\end{equation} 
wobei $\left\{\tilde{\overrightarrow{v}}_O,\tilde{\overrightarrow{\omega}}\right\}$ die willkürlich wählbare Kinemate in $O$ der momentanen virtuellen Starrkörperbewegung ist. Das Massensystem ist in einer Ruhelage, so dass das PdvL $\tilde{\mathcal{P}}=0$
\newline\newline
Man beachte, dass die Gleichgewichtsbedingungen nur nptwendige Bedingungen für bleibende Ruhe sind und im Allgemeinen zur vpollständigen Charakterisierung der Ruhelage nicht ausreichen. Ein deformierbars materielles System braucht nicht in Ruhe zu bleiben.
\newline\newline
Es entstehen Differentialgleichungen des Gleichgewichts wenn die Gleichgewichtsbedingungen für jeden infinitesimalen Bestandteil eines beliebigen deformierbaren Massensystems formuliert werden. Diese Gleichungen, die zugehörigen Randbedingungen, die durch die Bindungen entstehen kinematischen Beziehungen und die Bedingungen für die zugehörigen inneren und äusseren Bindungskräfte ergeben einen vollständigen Satz von notwendigen und hinreichenden Bindungen für etwaige Ruhelagen eines deformierbaren Massenkörpers.
\newline\newline
Aus der Formulierung des PdvL für bewegte Massensysteme folgt, dass sich die Kinemate eines starren Massenkörpers bezüglich seines Massenmittelpunktes genau dann nicht verändert, wenn die Gleichgewichtsbedingungen erfüllt sind. Dies liegt bei einer gleichförmigen Translationsbewegung und bei einer gleichförmigen Rotation um deine Achse durch den Massenmittelpunkt vor, sowie bei allen Überlagerungen dieser Bewegungszustände.   