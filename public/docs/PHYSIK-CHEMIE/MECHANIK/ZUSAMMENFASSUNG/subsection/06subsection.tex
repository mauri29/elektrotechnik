Das Skalarprodukt aus der Kraft mit der Geschwindigkeit ergibt eine skalare Grösse, nämlich die Leistung, welche die Wirkung der Kräfte an bewegten Massensystemen charakterisiert. Die Leistung ist eine zeit- und bewegungsabhängige Grösse. Die Dimension der Leistung ist der \textbf{Watt}, also die Kraft welche 1N m s$^{-1}$
%%%%%%%%%%%%%%%%%%%%%%%%%%%%%%%%%%%%%%%%%%%%%%%%%%%%%%%%%%%%%%%%%%%%%%%%%%%%%%%%%%%%%%%%%%%%
\subsection{Leistung einer Einzelkraft}
Die Leistung einer Einzelkraft mit dem Massenangriffspunkt $M$ ist
\begin{equation}
\boxed{\mathcal{P}=\overrightarrow{F}\bullet \overrightarrow{v}_A}
\end{equation}
\begin{equation}
\boxed{\mathcal{P}=\Big\vert\overrightarrow{F}\Big\vert\cdot \Big\vert\overrightarrow{v}_A\Big\vert\cdot \cos\left(\alpha\right)}
\end{equation}
Für $\alpha<90^{\circ}$ ist $\mathcal{P}>0$ und die Kraft $\overrightarrow{F}$ wirkt als \textbf{Antriebskraft}. Ist $\alpha>90^{\circ}$ ist $\mathcal{P}<0$ und die Kraft $\overrightarrow{F}$ wirkt als \textbf{Widerstandskraft}. Ist $\alpha=90^{\circ}$, so verschwindet die Leistung $\mathcal{P}=0$ und die Kraft $\overrightarrow{F}$ ist \textbf{momentan leistungslos}.
\begin{equation}
\boxed{\mathcal{P}=\overrightarrow{F}\bullet \overrightarrow{v}_A=\overrightarrow{F}\bullet \left(\overrightarrow{\omega}\times\overrightarrow{r}_{OA}\right)=\overrightarrow{\omega}\bullet \underbrace{\left(\overrightarrow{r}_{OA}\times \overrightarrow{F}\right)}_{\overrightarrow{M}_O}=\overrightarrow{\omega}\bullet \overrightarrow{M}_O}
\end{equation}
Das Vektorprodukt zwischen dem Ortsvektor $\overrightarrow{r}_{OA}$ des Massenangriffspunkts $A$ einer Kraft und dem Vektoranteil $\overrightarrow{F}$ heisst \textbf{Moment} $\overrightarrow{M}_O$ der Kraft bezüglich $O$, wobei $O$ Bezugspunkt des betrachteten Massessystems ist. Ein Moment entspricht immer ein Paar zweier Kräfte. Dabei sind die Kräfte gleich gross, entgegengerichtet und besitzen unterschiedliche Wirklinien. Ein \textbf{reines Moment} hat keine translatorishe Wirkung. Wird ein Körper in seinem Schwerpunkt aufgehängt, so ist da auf den Körper wirkende Moment für alle Orientierungen des Körpers im Raum gleich null.
\begin{equation}
\boxed{\overrightarrow{M}_O=\overrightarrow{r}_{OA}\times \overrightarrow{F}}
\end{equation}
\begin{equation}
\boxed{M_O=\Big\vert\overrightarrow{F}\Big\vert\cdot \Big\vert\overrightarrow{r}_{OA}\Big\vert\cdot \sin\left(\alpha\right)}
\end{equation}
Die Leistung einer Einzelkraft an einem rotierenden starren Massensystem ergibt sich aus dem Skalarprodukt der Rotationsgeschwindigkeit $\overrightarrow{\omega}$ mit dem Moment $\overrightarrow{M}_O$ der Kraft bezüglich eines Bezugspunktes $O$ auf der Rotationsachse.
\newline\newline
Projiziert man den Momentvektor $\overrightarrow{M}_O$ einer Kraft bezüglich $O$ auf eine Achse $Oz$ durch $O$, so erhält man das Moment $M_z$ der Kraft bezüglich der Achse $Oz$.
\begin{equation}
\boxed{M_z:=\overrightarrow{e}_z\bullet \overrightarrow{M}_O}\quad \boxed{\mathcal{P}=\omega\cdot M_z}
\end{equation}
%%%%%%%%%%%%%%%%%%%%%%%%%%%%%%%%%%%%%%%%%%%%%%%%%%%%%%%%%%%%%%%%%%%%%%%%%%%%%%%%%%%%%%%%%%%%
\subsection{Gesamtleistung mehrerer Kräfte}
Die Gesamtleistung mehrerer Kräfte besteht aus der Summe der Leistungen der einzelnen Kräfte.
\begin{equation}
\boxed{\mathcal{P}\left(\overrightarrow{F}_1, \dotso, \overrightarrow{F}_n\right)=\mathcal{P}\left(\overrightarrow{F}_1\right)+\mathcal{P}\left(\overrightarrow{F}_2\right)+\dotso + \mathcal{P}\left(\overrightarrow{F}_n\right)}
\end{equation}
Haben $n$ Einzelkräfte den gleichen Massenangriffspunkt $A$ und ist $\overrightarrow{v}_A$ die Geschwindigkeit dieses Massenpunktes, so ist die Gesamtleistung
\begin{equation}
\boxed{\mathcal{P}=\overrightarrow{F}_1\bullet \overrightarrow{v}_A+\overrightarrow{F}_2\bullet \overrightarrow{v}_A+\dotso+\overrightarrow{F}_n\bullet \overrightarrow{v}_A=\overrightarrow{R}\bullet \overrightarrow{v}_A}
\end{equation}
Haben $n$ Einzelkräfte $n$ verschiedene Massenangriffspunkte $M_i$ und ist $\overrightarrow{v}_{M_i}$ die Geschwindigkeit jedes Massenpunktes, so ist die Gesamtleistung
\begin{equation}
\boxed{\mathcal{P}=\overrightarrow{F}_1\bullet \overrightarrow{v}_{A_1}+\overrightarrow{F}_2\bullet \overrightarrow{v}_{A_2}+\dotso+\overrightarrow{F}_n\bullet \overrightarrow{v}_{A_n}=\displaystyle \sum_{i=1}^n\left(\overrightarrow{F}_i\bullet \overrightarrow{v}_{A_i}\right)}
\end{equation}
%%%%%%%%%%%%%%%%%%%%%%%%%%%%%%%%%%%%%%%%%%%%%%%%%%%%%%%%%%%%%%%%%%%%%%%%%%%%%%%%%%%%%%%%%%%%
\subsection{Gesamtleistung eines starren Massensystems}
Betrachte man ein starres Massensystem $S$ und $n$ Kräfte $\overrightarrow{F}_1$ bis $\overrightarrow{F}_n$ mit Massenangriffspunkte $A_1$ bis $A_n\in S$. Die Bewegung von $S$ sei zum Zeitpunkt $t$ durch die Kinemate $\left\{\overrightarrow{v}_O, \overrightarrow{\omega}\right\}$ in $O\in S$ gegeben.
\begin{equation}
\boxed{\begin{array}{lll}
\mathcal{P}&=&\overrightarrow{F}_1\bullet\overrightarrow{v}_{A_1}+\dotso+\overrightarrow{F}_n\bullet\overrightarrow{v}_{A_n}\\
&=&\overrightarrow{F}_1\bullet\left(\overrightarrow{v}_{O}+\overrightarrow{\omega}\times \overrightarrow{r}_{OA_1}\right)+\dotso+\overrightarrow{F}_n\bullet\left(\overrightarrow{v}_{O}+\overrightarrow{\omega}\times \overrightarrow{r}_{OA_n}\right)\\
&=&\overrightarrow{F}_1\bullet\overrightarrow{v}_O+\overrightarrow{F}_1\bullet \left(\overrightarrow{\omega}\times \overrightarrow{r}_{OA_1}\right)+\dotso+\overrightarrow{F}_n\bullet\overrightarrow{v}_O+\overrightarrow{F}_n\bullet \left(\overrightarrow{\omega}\times \overrightarrow{r}_{OA_n}\right)\\
&=&\overrightarrow{F}_1\bullet\overrightarrow{v}_O+\overrightarrow{\omega}\bullet \left(\overrightarrow{r}_{OA_1}\times \overrightarrow{F}_{1}\right)+\dotso+\overrightarrow{F}_n\bullet\overrightarrow{v}_O+\overrightarrow{\omega}\bullet \left(\overrightarrow{r}_{OA_n}\times \overrightarrow{F}_{n}\right)\\
&=&\displaystyle \sum_{i=1}^n\left(\overrightarrow{F}_i\right)\bullet\overrightarrow{v}_O+\overrightarrow{\omega}\bullet\displaystyle \sum_{i=1}^n\left(\overrightarrow{r}_{OA_i}\times \overrightarrow{F}_{i}\right)\\\\
&=&\overrightarrow{R}\bullet\overrightarrow{v}_O+\overrightarrow{\omega}\bullet\overrightarrow{M}_O\\
\end{array}}
\end{equation}