\section{Zweidimensionale Graphik}
\subsection{Elementare zweidimensionale Graphik}
Der Befehl \boxed{\textbf{\texttt{plot}}} öffnet ein Graphikfenster namens "figure" mit einer Nummer, in das eine Graphik engebettet werden kann. Falls für die Abszisse und für die Ordinate keine Schranken gesetzt werden, passt sich die Skalierung des Koordinatensystems den Daten automatisch an. Für jedes weitere Bild mus mit dem Befehl \boxed{\textbf{\texttt{figure}}} ein neues Graphikfenster geöffnet werden, es erhält eine neue Nummer. Andernfalls wird das alte Bild im Graphikfenster durch das neue Bild überschrieben. 
\newline\newline
Bekanntlich basiert die Grundstruktur von MATLAB auf einer \texttt{n$\times$m}-Matrix aus reellen oder komplexen Elementen. Auch Daten werden ja in Matrizen abgelegt. Für ein zweidimensaionales Bild benötigt MATLAB also mindestens zwei Kolonnenvektoren gleicher Länge.
\newline\newline
Der Befehl {\color{red}\texttt{plot(x,y)}} zeichnet den Datensatz \texttt{y} in Funktion von Datensatz \texttt{x} auf. Wie üblich wird jedem Wert von \texttt{x} ein Wert von \texttt{y} zugeordnet. \texttt{x} sind die Werte der Abszisse und \texttt{y} diejenigen auf der Ordinate. Die daraus resultierende Punkte werden mit geraden Linien verbunden (lineare Interpolation). Beide Achsen haben eine lineare Skala.
\newline\newline
Mit {\color{red}\texttt{plot(x,y,s)}} werden im String \texttt{s} der Linientyp und die Farbe der Kurve definiert. Der Befehl {\color{red}\texttt{plot(x,y,'c+:')}} plottet eine rot punktierte Linie, die jedem Datenpunkt ein rotes Plus-Zeichen hat. Der Befehl {\color{red}\texttt{plot(y)}} enthält nur einen Kolonnenvektor \texttt{y} Element von \texttt{$\mathbb{R}^{n\times 1}$}. In diesem Fall generiert MATLAB für die \texttt{x}-Achse automatisch Werte, nämlich 1 bis \texttt{n}, die Indizes der \texttt{n} Kolonnenwerte. 
\newline\newline
Handelt es sich jedoch bei \texttt{y} um einen Vektor mit komplexen Zahlen, so werden beim Befehl {\color{red}\texttt{plot(y)}} die Realteile auf der \texttt{x}-Achse und die Imaginärteile auf der \texttt{y}-Achse aufgetragen. 
\newline\newline
Der Befehl {\color{red}\texttt{plot(A)}} zeichnet für jede eine Kurve aus \texttt{n} Punkten. Die Matrix \texttt{A} Element von \texttt{$\mathbb{R}^{n\times m}$} je eine Kurve aus \texttt{n} Punkten. Die \texttt{x}-Achse zeigt wieder die Indizes 1 bis \texttt{n}. Die LInien werden zur Unterscheidung verschiedene Stile bzw. verschiedene Farben haben.
\newline\newline
Beim Befehl {\color{red}\texttt{plot(x,A)}} wird jede der \texttt{m} Kolonnen der Matrix \texttt{A} Element von \texttt{$\mathbb{R}^{n\times m}$} gegen die gemeinsame unabhängige Variable \texttt{x} aufgezeichnet. Der Vektor \texttt{x} muss die Dimension \texttt{n} haben: \texttt{x} ist ELement von \texttt{$\mathbb{R}^{n\times 1}$}
\newline\newline
Beim Befehl {\color{red}\texttt{plot(A,B)}} nilden je eine Kolonne der Matrix \texttt{A} Element von \texttt{$\mathbb{R}^{n\times m}$} und der Matrix \texttt{B} Element von \texttt{$\mathbb{R}^{n\times m}$} ein \texttt{x}-\texttt{y}-Vektorpaar. 
\newline\newline
Der Befehl \boxed{\textbf{\texttt{subplot(n,m,p)}}} unterteilt ein Graphikfenster in \texttt{n} Zeilen von je \texttt{m} Bildern. Damit können \texttt{n$\times$m} Bilder in ein Graphikfenster eingebettet werden. \texttt{p} ist der Laufindex der \texttt{n$\times$m} Bilder, wobei die Numerierung zeilenweise von links nach rechts erfolgt. Für jedes neue Bild im Graphikfenster wird der Befehl \texttt{subplot} wiederholt, jedesmal mit dem neuen Index \texttt{p}. Der eigentliche Befehl \texttt{plot} mit seinen Parametern muss dann natürlich auc noch kommen. 
\newline\newline
Die Befehle \boxed{\textbf{\texttt{semilogx}}} und {\color{red}\texttt{plot}} sind bis auf die Skalierung der \texttt{x}-Achse identisch. Beim Befehl {\color{red}\texttt{plot}} hat die Abszisse eine lineare, beim Befehl \texttt{semilogx} aber eine logarithmische Skala (Basis 10). Der Befehl {\color{red}\texttt{semilogx(x,y)}} entspricht dem Befehl {\color{red}\texttt{plot(log10(x), y)}}, doch MATLAB gibt bei \texttt{semilogx} für \texttt{x=0} keine Warnung "log for zero". Der Nullpunkt der \texttt{x}-Achse wird unterdrückt, er kann nicht gezeichnet werden. Mit dem Befehl {\color{red}\texttt{semilogy}} wird die Ordinate mit dem Zehnerlogarithmus skaliert.    
\newline\newline
Der Befehl \boxed{\textbf{\texttt{loglog}}} plottet eine zweidimensionale Graphik in einem doppeltlogarithmischen Koordinatensystem (Basis 10). Der Befehl {\color{red}\texttt{loglog(x,y),log10(y)}} entspricht dem Befehl {\color{red}\texttt{plot(log10(x))}}, doch MATLAB gibt bei \texttt{loglog} keine Warnung "log of zero", falls \texttt{x} oder \texttt{y} gleich Null ist. Der Koordinatennullpunkt wird unterdrückt.  
\newline\newline
Der Befehl \boxed{\textbf{\texttt{polar(phi,r)}}} zeichnet die beiden Vektoren \texttt{phi} und \texttt{r} in ein polares Koordinatensystem. Die Werte des Vektors \texttt{phi} sind im Bogenmass angegeben. Die WErte des Vektors \texttt{r} entsprechen dem Radius, d.h. dem Abstand zwischen dem Ursprung und dem betreffenden Punkt der Funktion.
\newline\newline
Der Befehl \boxed{\textbf{\texttt{plot(x1,x2,y1,y2)}}} versieht die Graphik mit zwei Ordinaten. Die linke \texttt{y}-Achse bezieht sich auf \texttt{y1} in Funktion von \texttt{x1} und die rechte \texttt{y}-Achse auf \texttt{y2} in Funktion von \texttt{x2}.
\subsection{Massstab}
Mit dem Befehl \boxed{\textbf{\texttt{axis([xmin xmax ymin ymax])}}} lassen sich die Grenzen der \texttt{x}- und der \texttt{y}-Achse neu definieren. Mit dem Befehl {\color{red}\texttt{axis auto}} setzt für die Achsen wieder dir ursprünglichen Werte ein. Mit dem Befehl {\color{red}\texttt{axis equal}} erhalten alle Achsen die gleiche Skalierung. Mit dem Befehl {\color{red}\texttt{axis ij}} wechselt das Vorzeichen der \texttt{y}-Achse. Die positive \texttt{y}-Achse zeigt nun nach unten. Der Befehl {\color{red}\texttt{axis xy}} macht {\color{red}\texttt{axis ij}} wieder rückgängig. Der Befehl {\color{red}\texttt{axis tight}} passt die Achsenlänge exakt dem Bild an. Der Befehl {\color{red}\texttt{axis off}} schaltet alle axis-Definitionen, die "tick marks" und den Hintergrund aus. Mit dem Befehl {\color{red}\texttt{axis on}} werden sie wieder aktiviert.
\newline\newline
Der Befehl \boxed{\textbf{\texttt{zoom on}}} aktiviert in der aktuellen Graphik die Zoom-Funktion, er kann direkt im Command Window eingegeben werden. Mit "clic and drag" wird dann im Bild ein beliebiger Ausschnitt näher herangebracht. Mit der linken Maustaste wird der Graphikausschnitt vergrössert und mit der rechten Maustaste verkleinert. Der Befehl {\color{red}\texttt{zoom(factor)}} zoomt die Achsen um den für "factor" gewählten Wert. Der Befehl {\color{red}\texttt{zoom out}} führt die Graphik in ihre default-Fenstergrösse zurück. Der Befehl {\color{red}\texttt{zoom xon}} bzw. {\color{red}\texttt{zoom yon}} aktiviert in re aktuellen Graphik die Zoom-Funktion nur für die betreffenden Achsen. Der Befehl {\color{red}\texttt{zoom(figurename, option)}} versieht die Graphik "figurename" mit einer Zoom-Funktion. Als Option kann eine der oben genannten Zoom-Funktionen gewählt werden.
\newline\newline
Der Befehl \boxed{\textbf{\texttt{grid on}}} versieht die aktuelle Graphik mit ienem Liniennetz. Mit dem Befehl {\color{red}\texttt{grid off}} werden die Linien wieder deaktiviert. Der Befehl {\color{red}\texttt{box on}} umrahmt das aktuelle Bild mit einem Rahmen aus dünnen, schwarzen LInien, der Befehl {\color{red}\texttt{box off}} entfernt ihn wieder.
\newline\newline
Der Befehl \boxed{\textbf{\texttt{hold on}}} fixiert das aktuelle Graphikfenster, so dass weitere Funktionen im bereits bestehenden Graphikfenster positioniert werden können. Die ursprünglichen Achseinstellungen bleiben unverändert, selbst dann, wenn die neue Funktion nicht gut in den Rahmen passt. Der Befehl {\color{red}\texttt{hold off}} führt zum Normalbetrieb zurück, d.h. ein neuer plot-Befehl löscht die aktuelle Funktion und fügt die neue Funktion in das Fenster ein, falls nicht mit dem Befehl {\color{red}\texttt{figure}} ein weiteres Graphikfenster geöffnet wurde.
\newline\newline
Der Befehl \boxed{\textbf{\texttt{axes('position', [links unten Breite Höhe])}}} beschreibt im Graphikfenster die POsition der linken unteren Ecke des Bildes und seine Abmessung. Mit Werten zwischen 0 und 1 kann nun die Position und dir Grösse des Bildes festgelegt werden. Das default-Graphikfenster hat eine Breite und eine Höhe 1. Der Befehl {\color{red}\texttt{axex}} generiert in einem Graphikfenster ein Koordinatensystem, in das mit {\color{red}\texttt{plot}} ein beliebiges Bild hineingelegt werden kann. Über mehrere {\color{red}\texttt{axex}}-Definitionen können im Fenster mehrere Bilder erzeugt werden.
\subsection{Beschriften von Bilder} 
Der Befehl \boxed{\textbf{\texttt{legend('st1', 'st2', ...)}}} versieht im Fenster ein Bild mit einer Legende. Er erhält die einzelnen Strings "st1", "st2", usw. Für jedes Bild in einem Graphikfenster kann ein eigener Titel gewählt werden. Der zu schreibende Text wird in Anführungszeichen genommen. Der Befehl {\color{red}\texttt{legend off}} entfernt die Legende aus dem aktuellen Bild. Der Befehl {\color{red}\texttt{legend('st1', 'st2', ..., position)}} positioniert die Legende mit der Angabe einer Position an einen definierten Ort in der Graphik: 0-beste, 1-oben-rechts, 2-oben links, 3-unten-links, 4-unten-rechts, -1-rechts. Die Legende kann verschoben werden, indem die Legende mit der rechten Maustaste angeklickt und an den gewünschten Ort gezogen wird.
\newline\newline
Der Befehl \boxed{\textbf{\texttt{title('text')}}} fügt einen Titel oberhalb der Graphik hinzu. Ein Text kann hoch \texttt{("\^\,")} und tiefgestellte \texttt{"\_\,"}, kursiv \texttt{"it"} geschrieben oder griechische Zeichen enthalten. 
\begin{equation}
\boxed{A_1e^{-\alpha t}\sin \beta t\quad \texttt{'$\backslash$itA\_\{1\}e\^\,\{$\backslash$alpha$\backslash$itt\}sin$\backslash$beta$\backslash$itt'}}
\end{equation}
Der Befehl \boxed{\textbf{\texttt{xlabel('text')}}} schreibt die Zeile "text" unter die Abszisse. Der Befehl \texttt{ylabel} ist für die Beschriftung der Ordinate.
\newline\newline
Der Befehl \boxed{\textbf{\texttt{text(x,y,'text')}}} kann innerhalb des Bildrahmes eine beliebige Textzeile an der Stelle \texttt{(x,y)} angebracht werden. \texttt{x} und \texttt{y} sind in den Koordinaten der Achsen anzugehen. Dagegen verwendet der Befehl {\color{red}\texttt{text(x,y,'text,'sc')}} die Koordinaten des Graphikfensters, nämlich (0,0) in der unteren linken Ecke und (1,1) ub der oberen rechten Ecke. Geht ein Text über mehrere Zeilen, so kann er in eine Text-Variable geschrieben werden. Ein Text kann hoch- und tiefgestellte, kursiv geschriebene oder griechische Zeichen enthalten.
\newline\newline
Mit dem Befehl \boxed{\textbf{\texttt{gtext('text')}}} kann mit der Maus im Bild irgendein Text an beliebigen Ort eingefügt werden. Im Graphifenster erscheint der Mauspfeil als Fadenkreuz. An der gewünschten Stelle kann durch Betätigung einer Maustaste der Text eingefügt werden.
\subsection{Graphiken speichern oder drucken}
Der Befehl \boxed{\textbf{\texttt{print}}} sendet eine Kopie des aktuellen Graphikfensters an den Drucker. Der Befehl {\color{red}\texttt{print filename}} speichert eine Kopie des aktuellen Graphikfensters als PostScript-Datei im aktiven Directory unter dem Dateinamen "filename". Mit dem Befehl {\color{red}\texttt{print path}} kann der genaue Pfad angegeben werden, wo das aktuelle Graphikfenster als POst-Script-Datei abgelegt werden soll. Der Befehl {\color{red}\texttt{print[-ddevice][-options]$<$filenmae$>$}} speichert die aktuelle Graphik im Format des speziell gewählten Druckertreibers und de rzusätzlichen Option im aktiven Directory unter "filename" ab. Der Befehl {\color{red}\texttt{help print}} zeigt im Command-Window alle möglichen \texttt{[-ddevice]} und \texttt{[-options]}.
\newline\newline
Der Befehl \boxed{\textbf{\texttt{orient landscape}}} druckt bei print-Befehlen die Graphikfenster im Querformat. Mit dem Befehl {\color{red}\texttt{orient portrait}} wird das Graphikfenster im Hochformat grdruckt. Der Befehl {\color{red}\texttt{orient tall}} setzt das Blattformat auf Hochformat. Zusätzliche wird das Graphikfenster auf das ganze Papierblatt vergrössert bzw. verkleinert. Der Befehl {\color{red}\texttt{orient}} sagt im Command Window, welche Orientierung momentan aktiv ist.
\section{Dreidimensionale Graphik}
\subsection{Elementare dreidimensionale Graphik}
Der Befehl \boxed{\textbf{\texttt{plot3(x,y,z)}}} plottet im dreidimensionalen Raum die Graphen, die durch die Vektoren \texttt{x}, \texttt{y} und \texttt{z} gegeben sind. Die Vektoren müssen alle dieselbe Länge haben. Mit {\color{red}\texttt{plot3(X,Y,Z)}} zeichnet pro Kolonne einen Graphen. Die Matrizen müssen alle dieselbe Grösse haben. Bei {\color{red}\texttt{plot3(x,y,z,'style')}} können zusätzlich noch der Linientyp, die Plot Symbole und die Farbe des Graphen geändert werden. {\color{red}\texttt{help plot}} listet im Command Window eine Answahl von möglichen Liniendefinitionen auf.
\newline\newline
Bei \boxed{\textbf{\texttt{mesh(Z)}}} entsprechen die Werte der Matrix \texttt{Z} Element von \texttt{$\mathbb{R}^{n\times m}$} den \texttt{z}-Werten des Netzes. Für die \texttt{x}- und \texttt{y}-Werte verwendet \texttt{mesh} die Kolonnen- bzw. die Zeilennummer.
\newline\newline
Mit dem Befehl {\color{red}\texttt{mesh(X,Y,Z,C)}} zeichnet ein Netz un der Vogelperspektive mit \texttt{Z} als Funktion von \texttt{X} und \texttt{Y}. Es handelt sich hierbei um eine FUnktion mit zwei Variablen. \texttt{X}, \texttt{Y} und \texttt{Z} sind Matrizen mit den Werten für die \texttt{x}-, \texttt{y}- und \texttt{z}-Koordinaten. \texttt{X} und \texttt{Y} können aber auch Vektoren der Länge \texttt{m} und \texttt{n} sein. Jeder \texttt{z}-Koordinate aus der Matrix \texttt{Z} werden dann die entsprechenden WErte des \texttt{x}- und \texttt{y}-Vektors zugewiesen. \texttt{C} ist ebenfalls eine Matrix und beinhaltet die Farbskala für die Graphik. Ohne \texttt{C} wird \texttt{C=Z} gesetzt.
\newline\newline
Mit dem Befehl \boxed{\textbf{\texttt{fill(X,Y,Z,C)}}} wird in den Farben der Matrix \texttt{C} ein dreidimensionales Polygon geplottet. Sind \texttt{X}, \texttt{Y} und \texttt{Z} Vektoren, so wird die Fläche unterhalb des Graphen mit Farbe ausgefüllt. Ist \texttt{C} ein Skalar, so wird die Fläche monochrom. Folgende Farben sind möglich: {\color{red}\texttt{'r'}}, {\color{red}\texttt{'g'}}, {\color{red}\texttt{'b'}}, {\color{red}\texttt{'c'}}, {\color{red}\texttt{'m'}}, {\color{red}\texttt{'y'}}, {\color{red}\texttt{'w'}} und {\color{red}\texttt{'k'}}. Mit dem Vektor {\color{red}\texttt{[rot grün blau]}} kann eine neue Farbe gemischt werden. Die Werte von liegen zwischen 0 und 1. Je nach Anteil ergibt sich eine Farbkombination. Ist \texttt{C} ein Vektor, so hat er die gleiche Länge wie \texttt{X}, \texttt{Y} und \texttt{Z}. Wird für \texttt{C} einer der drei Vektoren gewählt, dann ist die Farbabstufung der momentan aktive \texttt{colormap} proportional zur betreffenden Koordinatenachse. 
\newline\newline
Sind \texttt{X}, \texttt{Y} und \texttt{Z} Matrizen, so zeichnet \boxed{\textbf{\texttt{fill3}}} pro Kolonne ein Polygon und füllt es mit der entsprechenden Farbe aus. Die Farbgebung bleibt gleich. Falls \texttt{C} ein Zeilenvektor ist, dann hat das Polygon die Schattierung \texttt{shading flat}. Für eine Matrix wird sie {\color{red}\texttt{shading interp}}. {\color{red}\texttt{shading}} shattiert die Objekt-Oberfläche, {\color{red}\texttt{shading flat}} berechnet für jede Teilfläche einer Oberfläche, die mit den Befehlen \texttt{surf}, \texttt{mesh}, \texttt{polar}, \texttt{fill} oder \texttt{fill3} gebildet wurden, die entsprechende Farbabstufung. {\color{red}\texttt{shading interp}} interpoliert über die Farbabstufung. {\color{red}\texttt{shading faceted}} entspricht dem {\color{red}\texttt{shading flat}}. Die 3D Graphik wird jecoh zusätzlich mit schwarzen Linien versehen. 
\subsection{Projektionsarten einer Graphik}
Mit \boxed{\textbf{\texttt{view(az,el)}}} kann in einem dreidimensionalen Plot der Blickwinkel beliebig eingestellt werden, bzw. die Graphik-Box ist um zwei Achsen drehbar. {\color{red}\texttt{az}} steht für azimuth und definiert die horizontale Rotation im Grad. Für einen positiven Winkel dreht sich die Graphik entgegen dem Uhrzeigersinn um die \texttt{z}-Achse. {\color{red}\texttt{el}} beschreibt die Anheben bzw. Senken der Gtraphik in Grad. Bei einem positiven Winkel befindet sich der Betrachter in der Vogelperspektive, bei negativem Winkel in der Froschperspektive. {\color{red}\texttt{view([x y z])}} setzt den Blickwinkel in kartesischen Koordinaten. {\color{red}\texttt{view(2)}} stellt für die 2D Ansicht den vordefinierten Blickwinkel {\color{red}\texttt{view(0, 90)}} ein. {\color{red}\texttt{view(3)}} stellt für die 3D Ansicht den vordefinierten Blickwinkel {\color{red}\texttt{view(-37.5,30)}} ein. {\color{red}\texttt{T=view}} speichert diew \texttt{view} der aktuiellen Graphik in der Variable \texttt{T} als \texttt{4x4}-Matrix. {\color{red}\texttt{view(T)}} weist einer aktuellen Graphik die in der Variable \texttt{T} gespeicherte view zu.
\newline\newline
\boxed{\textbf{\texttt{T=viewmtx(az,el)}}} weist wie bei {\color{red}\texttt{T=view(az, el)}} der Variable \texttt{T} die 4x4-Transformationsmatrix zu. Die Ansicht der aktuellen Graphik wird dabei nicht verändert. Mit {\color{red}\texttt{T=viewmtx(az, el, phi)}} wird die Graphik durch ein Objektiv betrachtet. Der Linsenwinkel wird in Grad angegeben. \texttt{phi=0} Grad definiert die orthogonale Projektion. 10 Grad entspricht einem Teleobjektiv, 25 Grad einem Normalobjektiv und 60 Grad einem Weitwinkelobjektiv. Bei {\color{red}\texttt{T=viewmtx(az, el, phi, tp)}} wird mit \texttt{tp=[xp, yp, zp]} einen Fluchtpunkt gesetzt.
\newline\newline
\boxed{\textbf{\texttt{rotate3d on}}} aktiviert in der aktuellen Graphik die Maus gesteuerte 3D-Rotation. Die Graphik-Ansicht kann damit beliebig verändert werden. Mit {\color{red}\texttt{rotate3d off}} wird sie wieder deaktiviert. 
\subsection{Dreidimensionale Graphik beschriften}
\boxed{\textbf{\texttt{zlabel('text')}}} versieht die \texttt{z}-Achse mit der Aufschrift "text". Mit dem Befehl {\color{red}\texttt{colorbar('vert')}} erscheint in der aktuellen Graphik eine vertikale Farbskala. In einem 3D-Plot bezieht sie sich auf die Werte der \texttt{z}-Achse. {\color{red}\texttt{colorbar('horiz')}} zeichnet eine vertikale Farbskala. Im 3D-Plot ist die Farbgebung ebenfalls auf die \texttt{z}-Achse abgestimmt. \boxed{\textbf{\texttt{colorbar}}} alleine fügt der Graphik entweder eine vertikale Farbskala hinzu, oder die bestehende Farbskala wird aktualisiert.
\section{Spezielle Graphen}
\boxed{\textbf{\texttt{fill(x,y,c)}}} füllt diejenige Fläche mit der Farbe \texttt{c} aus, welche von der Geraden, die den Endpunkt von \texttt{f(x)} mit dem Anfang verbindet, und der Funktion \texttt{y=f(x)} selbst umgeben wird. Ist \texttt{c} ein Vektor derselben Länge wie \texttt{x} und \texttt{y}, dann verwendet MATLAB entweder die im Vektor \texttt{c} definierten Farben, oder für \texttt{c=x} bzw. \texttt{c=y} die Farbpalette der aktuellen \texttt{colormap}. Sind in \boxed{\textbf{\texttt{fill(X,Y,C)}}} \texttt{X} und \texttt{Y} Matrizen derselben Grösse, so wird pro Kolonne ein Polygon gezeichnet. \texttt{C} kann ein Vektor aber auch eine Matrix sein. Beim Vektor ist die Schattierung der Fläche {\color{red}\texttt{shading flat}}, bei der Matrix {\color{red}\texttt{shading interp}}.
\newline\newline
Mit dem Befehl \boxed{\textbf{\texttt{fplot('f',lim)}}} zeichnet eine beliebige Funktion \texttt{f=f(x)} im Bereich \texttt{lim=[x$_{\texttt{min}}$ x$_{\texttt{max}}$]}. Mit \texttt{lim=[x$_{\texttt{min}}$ x$_{\texttt{max}}$ y$_{\texttt{min}}$ y$_{\texttt{max}}$]} werden zusätzliche Schranken gesetzt. Mit dem Befehl {\color{red}\texttt{fplot('f', lim, tol)}} mit \texttt{tol$<$1} definiert die Toleranz des relativen Fehlers. Die voreingestellte Toleranz ist 2e-3 bzw. 0.2\%. Mit dem Befehl {\color{red}\texttt{fplot('f', lim, N)}} berechnet zwischen \texttt{x$_{\texttt{min}}$} für die Funktion \texttt{f=f(x)} \texttt{N+1} Punkte. Mit dem Befehl {\color{red}\texttt{fplot('f', lim, 'LineSpec')}} definiert mit LineSpec den Linien-Typ der Funktion. Alle möglichen "line specifications" werden mit \texttt{help plot} aufgelistet.    
\newline\newline
Mit dem Befehl \boxed{\textbf{\texttt{hist(x)}}} zeichnet MATLAB ein Histogramm mit den in \texttt{x} gespeicherten Daten. Es ist in 10 gleichmässig verteilte Intervalle unterteilt. Pro Intervall gibt es die Anzahl Elemente an, die es enthält. Wenn \texttt{x} eine Matrix ist, plottet \texttt{hist} pro Kolonne ein Histogramm. Mit dem Befehl {\color{red}\texttt{hist(x,n)}} definiert man zusätzlich die Anzahl \texttt{n} Intervalle. Mit dem Befehl {\color{red}\texttt{hist(x,y)}} berechnet MATLAB die Verteilung von \texttt{x} bezüglich \texttt{y}. \texttt{y} ist ein Vektor, deren Elemente in aufsteigender Ordnung aufgelistet sind. Jedes einzelne Element von \texttt{y} entspricht einem Zentrum.     
\newline\newline
Mit dem Befehl \boxed{\textbf{\texttt{pie(x)}}} stellt die Daten aus dem Vektor \texttt{x} in einem Kuchendiagramm dar. Die Elemente von \texttt{x} werden mit der Summe der \texttt{x}-Werte dividiert. Damit ist die Grösse von jedem einzelnen Kuchenstück in \% gegeben. Mit dem Befehl {\color{red}\texttt{pie(x,explode)}} zieht mit "explode" die gewünschte Stücke aus dem Kuchen, \texttt{explode} ist ein Vektor derselben Länge wie \texttt{x}. Seine Elemente haben entweder den Betrag 0, d.h. das entsprechende Stück von \texttt{x} verbleibt im Kuchen, ode rden Betrag 1, d.h. das dazugehörende Stück von \texttt{x} wird aus dem Kuchen herausgezogen.
\newline\newline
Mit dem Befehl \boxed{\textbf{\texttt{stem(y)}}} zeichnet eine Verteilung der Daten aus dem Vektor \texttt{y}. Jeder Wert aus \texttt{y} im Plot mit einem Kreis und einer Linie versehen. Auf der Abszisse wird jedes Element aus \texttt{x} fortlaufend eingereiht. Auf der Ordinate kann sein Wert abgelesen werden. Der entsprechende Wert ist mit ienem Kreis gekennzeichnet. Bei {\color{red}\texttt{stem(x,y)}} entsprechen die Elemente aus dem \texttt{x}-Vektor den \texttt{x}-Werten und diejenigen aus dem \texttt{y}-Vektor den \texttt{y}-Werten. Mit dem Befehl {\color{red}\texttt{stem(..., 'filled')}} malt den Kreis mit der entsprechenden Farbe aus. Mit dem Befehl {\color{red}\texttt{stem(..., 'linespec')}} kann der Linien-Typ gewählt werden. Alle möglichen "line specifications" werden mit {\color{red}\texttt{help plot}} aufgelistet.    
\newline\newline
Mit dem Befehl \boxed{\textbf{\texttt{contour(Z)}}} zeichnet einen Konturplot mit den WErten aus der Matrix \texttt{Z}. Die Elemente der Matrix \texttt{Z} entsprechen den Werten auf der \texttt{z}-Achse. Die Kolonnenzahlen ergeben die Koordinaten auf der \texttt{x}-Achse und die Zeilenzahlen die WErte auf der \texttt{y}-Achse. Die Höhen für die einzelnen Höhenlinien werden automatisch ausgewählt. Mit {\color{red}\texttt{contour(x,y,z)}} werden die \texttt{x}- und \texttt{y}-Koordinaten explizit mitgeliefert. {\color{red}\texttt{contour(z,N)}} und {\color{red}\texttt{contour(x,y,z,N)}} verwendet für den Konturplot \texttt{N} Höhenlinien. {\color{red}\texttt{contour(Z,v)}} und {\color{red}\texttt{contour(x,y,Z,v)}} zeichnet all jene Höhenlinien, deren Höhen im Vektor \texttt{v} angegeben werden. Mit {\color{red}\texttt{contour(Z,[v v])}} plottet MATLAB eine einzige Höhenlinie bei der Höhe \texttt{v}. Mit {\color{red}\texttt{contour(...,'linespec')}} kann der Linien-Typ gewählt werden. Alle möglichen "line specifications" werden mit {\color{red}\texttt{help plot}} aufgelistet...  