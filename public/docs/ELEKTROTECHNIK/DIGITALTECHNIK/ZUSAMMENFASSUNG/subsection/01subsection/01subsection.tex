\section{Das digitale Signal}
Ein {\color{red}\textbf{digitales Signal}} ist sowohl wert- als auch zeitdiskret. Es kann endlich viele Zustände annehmen und sind von der Zeit abhängig. Ein binäre Signal kann nur 2 Zustände annehmen, während ein digitales Signal kann $N$ Zustände annehmen. Muss also Informationenals mit mehr als zwei Werte fliessen, so muss dies durch sequentielles Aneinanderreihen von zwei-wertigen Einzelsignalen erfolgen.
\section{Binäre Signale}
{\color{red}\textbf{Binäre Signale}} werden durch Spannungen abgebildet. Die Aussagenlogik stellt diese binäre Signale durch "true" oder "false", die Schaltungstechnik durch "high" oder "flow" und die Digitaltechnik durch "1" oder "0".
\section{Vorteile der Digitaltechnik}
Folgende sind {\color{red}\textbf{Vorteile der Digitaltechnik}}: 
\begin{enumerate}[$(1)$]
\item Digitale Signale lassen sich einfacher fehlerfrei übertragen und verarbeiten. 
\item Keine Fortsetzungsfehler bei der Verarbeitung und Übertragung wie bei analogen Techniken. 
\item Digitale Information kann man einfacher fehlerfrei abspeichern.
\end{enumerate}
\section{Nachteile der Digitaltechnik}
Folgende sind {\color{red}\textbf{Nachteile der Digitaltechnik}}: 
\begin{enumerate}[$(1)$]
\item Informationsverlust bei der Umwandlung von analogen Signalen in digitale Signale. 
\item Rundungsfehler bei der Verarbeitung mit ungenügender Wortlänge. 
\item Langsamer als analoge Verarbeitung. Vgl. HF-Technik, wo analog gearbeitet wird. 
\item Höherer Schaltungsaufwand als bei analoger Schaltungstechnik. Dank Miniaturisierung tritt dieser Nachteil immer mehr in den Hintergrund.
\end{enumerate}