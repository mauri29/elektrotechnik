Bezüglich des Ursprungs $O$ entsteht den \textbf{Ortsvektor} vom Massenpunkt $M$. Dieser Ortsvektor $\overrightarrow{r}_M$ ist unabhängig vom gewählten Koordinatensystem und wird durch seinen Betrag und Richtung charakterisiert. Der Bewegungszustand eines Massenpunktes zwischen einem Zeitintervall wird durch seine Lage beschrieben. 
\begin{equation}
\boxed{\overrightarrow{r}_M(t):=\overrightarrow{f}(t)=\overrightarrow{OM}(t)}
\end{equation}
Eine Vektorfunktion der Zeit ist die Zuordnung einer vektoriellen abhängigen variablen zu gegebenen Werten der skalaren unabhängigen Variablen wie die Zeit. Die Funktion stellt eien Abbildung von Zahlen in Vektoren dar. Eine Vektorfunktion verändert den Betrag und die Richtung bezüglich eines Bezugspunktes. 
\newline\newline
Ist der Betrag $\Big\vert\overrightarrow{r}\Big\vert$ veränderlich, die Richtung aber konstant, so liegt eine \textbf{geradlinige Bewegung} vor und die Bahnkurve liegt auf einer Gerade, die durch den Ursprung $O$ geht. 
\newline\newline
Wird die Richtung verändert und den Betrag $\Big\vert\overrightarrow{r}\Big\vert$ konstant gehalten, so liegt eine \textbf{krummlinige Bewegung} mit Bahnkurve auf einer Kugel. 
\newline\newline
Zur Beschreibung von Vektorfunktionen führt man \textbf{Basen} ein, welche aus drei Vektoren $\overrightarrow{b}_1$, $\overrightarrow{b}_2$ und $\overrightarrow{b}_3$, die linear unabhängig sind. 
\newline\newline
Der Ortsvektor $\overrightarrow{r}$ und die Vektorfunktion $\overrightarrow{f}$ lassen sich bezüglich dieser Basis nach Parallelogramregel zerlegen wobei $r_i$ die skalaren und $r_i\overrightarrow{b}_i$ die vektorielle Komponenten sind. Sind die drei Vektoren der Basis orthogonal zueinander und normiert, so heissen sie \textbf{orthonormierte Basisvektoren} oder orthogonale Einheitsvektoren.
\begin{equation}
\boxed{\overrightarrow{r}=r_1\cdot\overrightarrow{b}_1+r_2\cdot\overrightarrow{b}_2+r_3\cdot\overrightarrow{b}_3}
\end{equation}
Orthogonale Einheitsvektoren werden in Richtung zu den entsprechenden Koordinatenlinien aller Koordinatenarten eingeführt. Auf diese Weise entstehen kartesische, zylindrische und sphärische Basisvektoren.
%%%%%%%%%%%%%%%%%%%%%%%%%%%%%%%%%%%%%%%%%%%%%%%%%%%%%%%%%%%%%%%%%%%%%%%%%%%%%%%%%%%%%%%%%%%%
\subsection{Kartesische Komponenten der Bewegung}
Die Zerlegung des Ortsvektors bezüglich der \textbf{kartesischen Basis} ergibt
\begin{equation} 
\boxed{\overrightarrow{r}\left(t\right)=x\left(t\right)\cdot\overrightarrow{e}_x+y\left(t\right)\cdot\overrightarrow{e}_y+z\left(t\right)\cdot\overrightarrow{e}_z}
\end{equation} 
%%%%%%%%%%%%%%%%%%%%%%%%%%%%%%%%%%%%%%%%%%%%%%%%%%%%%%%%%%%%%%%%%%%%%%%%%%%%%%%%%%%%%%%%%%%%
\subsection{Zylindrische Komponenten der Bewegung}
Den tangentialen Richtungen der Koordinatenlinien sind zylindrisch radial und azimutal gerichtet. Die Richtungen von $\overrightarrow{e}_{\rho}$ und $\overrightarrow{e}_{\varphi}$ sind bezüglich $O_{xyz}$ veränderlich und von der jeweiligen Lage von $M$ abhängig. Diese Vektoren sind nicht konstante Vektorfunktionen, welche durch die enzige skalare Winkelfunktion $\varphi=f_{\varphi}(t)$ festgelegt wird. Die Zerlegung des Ortsvektors bezüglich der \textbf{zylindrischen Basis} ergibt
\begin{equation}
\boxed{\overrightarrow{r}\left(t\right)=\rho\left(t\right)\cdot\overrightarrow{e}_{\rho}\left(\varphi(t)\right)+z\left(t\right)\cdot\overrightarrow{e}_z}
\end{equation}
%%%%%%%%%%%%%%%%%%%%%%%%%%%%%%%%%%%%%%%%%%%%%%%%%%%%%%%%%%%%%%%%%%%%%%%%%%%%%%%%%%%%%%%%%%%%
\subsection{Sphärische Komponenten der Bewegung}
Die Einheitsvektoren $\overrightarrow{e}_r$, $\overrightarrow{e}_{\varphi}$ und $\overrightarrow{e}_{\psi}$ sind polar-radial, meridional und azimutal. Alle Einheitsvektoren haben veränderliche Richtungen. Die Zerlegung des Ortsvektors bezüglich der \textbf{sphärischen Basis} ergibt
\begin{equation}
\boxed{\overrightarrow{r}\left(t\right)=r\left(t\right)\cdot\overrightarrow{e}_r\left(\theta\left(t\right),\psi\left(t\right)\right)}
\end{equation}
%%%%%%%%%%%%%%%%%%%%%%%%%%%%%%%%%%%%%%%%%%%%%%%%%%%%%%%%%%%%%%%%%%%%%%%%%%%%%%%%%%%%%%%%%%%%
\subsection{Verschiebung}
Beschreiben $\overrightarrow{r}\left(t_1\right)$ und $\overrightarrow{r}\left(t_2\right)$ zu den Zeitpunkten $t_1$ und $t_2$ den Ort eines Massenpunktes bezüglich eines Koordinatensystems, so ist die Verschiebung $\triangle \overrightarrow{r}$ des Massenpunktes im Zeitintervall $[t_1, t_2]$ gegeben durch
\begin{equation}
\boxed{\triangle \overrightarrow{r}\left(t\right)=\overrightarrow{r}\left(t_2\right)-\overrightarrow{r}\left(t_1\right)=\Big\vert \triangle \overrightarrow{r}\left(t\right)\Big\vert\cdot \overrightarrow{e}_{\triangle \overrightarrow{r}}}
\end{equation}
Jeder Verschiebungsvektor $\triangle \overrightarrow{r}$ kann als Produkt seiner Länge $\Big\vert \triangle \overrightarrow{r}\Big\vert$ und des Richtungsvektors$\overrightarrow{e}_{\triangle \overrightarrow{r}}$ der Verschiebung formuliert werden.  
%%%%%%%%%%%%%%%%%%%%%%%%%%%%%%%%%%%%%%%%%%%%%%%%%%%%%%%%%%%%%%%%%%%%%%%%%%%%%%%%%%%%%%%%%%%%