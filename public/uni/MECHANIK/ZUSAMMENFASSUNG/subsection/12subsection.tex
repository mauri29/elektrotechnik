\subsection{Physikalische Grundlage}
\textbf{Normal- und Reibungskräften} treten bei der Berührung zweier materiellen Flächen auf. Die \textbf{Normalkräfte} charakterisieren den Widerstand der festen Körper gegen Eindringen und liegen längs der gemeinsamen Normalen durch die Berührungspunkte und ergeben keine Leistung bei zulässigen virtuellen Bewegungen beider Körpern. Die \textbf{Reibungskräfte} liegen dagegen in der gemeinsamen Tangentialebene $\tau$ der beiden Berührungsflächen und ergeben Beiträge zur virtuellen Leistung, selbst bei zulässigen virtuellen Bewegungen.
\newline\newline
Die \textbf{Rauigkeit} der Berührungsflächen, welche zumindest lokal zu unterschie-dlichen Richtungen der gemeinsamen Flächennormalen führt und sie tendenziell in Richtung der Verhinderung einer etwaigen Bewegung verdreht. Die Tendenz der Moleküle ihre Bindungen durch externe Faktoren wie Wärme, Druck verursachen \textbf{lokale Verschweissungen}, welche eine Erhöhung des Haftvermögens, also des Gleitwiderstandes. Dies äussert sich wiederum in einer zusätzlichen Schubkomponente der Kontaktkraft, d.h. einer Komponente in der gemeinsamen tangentialen Ebene $\tau$.
\newline\newline
Ein Körper 1 mit Gesamtgewicht $G$ durch Hinzufügen oder Wegnehmen von Standardgewichten kann verändert werden, ruhe vorerst auf einer ebenen horizontalen Unterlage 2 und sei einer horizontalen Kraft mit Kraftvektor $\overrightarrow{F}$
\begin{equation}
\boxed{\overrightarrow{F}=F\cdot \overrightarrow{e}_x}\quad \boxed{\overrightarrow{F}_R=F_R\cdot \overrightarrow{e}_x=-F\cdot \overrightarrow{e}_x}\quad \boxed{F_R=-F}
\end{equation}
ausgesetzt, dessen Betrag allmählich erhöht wird. Der Körper 1 bleibt wegen der REibung an seiner Berührungsfläche in Ruhe, bis der Kraftbetrag $\Big\vert \overrightarrow{F}\Big\vert$ einen Grenzwert erreicht. Anschliessend steuert man den Kraftbetrag so, dass die Geschwindigkeit $\overrightarrow{v}=v\cdot \overrightarrow{e}_x$ der geradlinigen Translation des Körpers konstant bleibt. Man erzeugt sowohl positive als auch negative Werte von $F$ mit entsprechenden Werte von $v$ und registriert die Funktion $F=F\left(v\right)$ für verschiedene Werte des Gewichtes $G$. Da sowohl in der Ruhelage als auch bei der gleichförmigen geradlinigen Translation die äusseren Kräfte im Gleichgewicht sein müssen, ist der Reibungskraftvektor $\overrightarrow{F}_R$ und der Normalvektor $\overrightarrow{N}$.
\begin{equation}
\boxed{\overrightarrow{N}=N\cdot \overrightarrow{e}_y=-\overrightarrow{G}=G\cdot \overrightarrow{e}_y}\quad \boxed{N=G} 
\end{equation}
Man beachte, dass die Vektoren so definiert sind, dass $N$ und $G$ immer positives Vorzeichen haben, während die Vorzeichen von $F_R$ und $F$ stets verschieden sind. Wirkt $\overrightarrow{F}$ in der positiven $x$-Richtung, so ist $F$ positiv und $F_R$ negativ, während für $\overrightarrow{F}$ in der negativen $x$-Richtung $F_R$ positiv wird. 
\newline\newline
Die Funktion $F_R\left(v\right)$ ist proportional zum Normalkraftbetrag $\Big\vert \overrightarrow{N}\Big\vert$. Die Funktion $f\left(v\right)=\dfrac{F_R\left(v\right)}{\Big\vert \overrightarrow{N}\Big\vert}$ stellt  für alle Werte von $\Big\vert \overrightarrow{N}\Big\vert$ einen einzigen Graphen dar. Gemäss diesen Graph setzt sich der Körper in Bewegung, sobald $\Big\vert f\left(v\right)\Big\vert$ einen Grenzwert $\mu_0$ erreicht. Dann aber nimmt das Verhältnis $\Big\vert f\left(v\right)\Big\vert$ mit zunehmender Schnelligkeit $\Big\vert v\Big\vert$ ab bis zu einem stationären Wert $\mu_1$, der für alle grösseren Werte von $\Big\vert v\Big\vert$ in erster Näherung konstant bleibt.
\newline\newline
Der Grenzwert $\mu_0$ heisst \textbf{Haftreibungszahl} und der stationäre Wert $\mu_1$ \textbf{Gleit-reibungszahl}. Beide zahlen hängen hauptsächlich von der Art der Materialpaarung und von den Schmierverhältnissen an den Berührungsflächen ab. 
\subsection{Haftreibung}
Ein Körper setzt sich in Bewegung erst dann, wenn der Betrag der Reibungskraft $\Big\vert \overrightarrow{F}_R\Big\vert$ den Grenzwert $\mu_0\Big\vert \overrightarrow{N}\Big\vert$ erreicht hat.
\newline\newline
An den materiellen Berührungspunkten zwischen zwei Körpern 1 und 2 bleibt die relative Geschwindigkeit $\overrightarrow{v}=\overrightarrow{0}$, solange die Haftbedingung erfüllt ist.
\begin{equation}
\boxed{\Big\vert \overrightarrow{F}_R\Big\vert<\mu_0\cdot \Big\vert \overrightarrow{N}\Big\vert}
\end{equation}
Die Haftreibungsgesetz ist eine Ungleichung, welche nicht zur Bestimmung der Haftreibungskraft benutzt werden kann. In einer Bindung mit Haftreibung muss deshalb vorerst eine Haftreibungskraft oder ein Haftreibungsmoment von unbekanntem Betrag und unbekannter Richtung eingeführt werden. Bei einem statisch bestimmten System ergibt sich die Haftreibungskraft und die Normalkraft aus den Gleichgewichtsbedingungen. Die so berechneten Kräfte werden in das Haftreibungsgesetz eingesetzt, um im Rahmen der Diskussion der Resultate zu entscheiden, ob das System in Ruhe sein kann.
\subsection{Gleireibung}
Verlangen die Gleichgewichtsbedingungen Lagerkomponenten, die der Haftbedingung nicht genügen, so tritt Gleiten ein. Nimmt man einfachheitshalber an, dass einer der beiden sich berührenden Körper in Ruhe ist, dann greift im Berührungspunkt am bewegten Körper eine Reibungskraft an, die der Geschwin-digkeit des materiellen Massenangriffspunktes entgegengesetzt ist. Der Reibungsvektor besitzt die Geschwindigkeit $\overrightarrow{v}$ und die Gleitreibungszahl $\mu_1$.
\begin{equation}
\boxed{\overrightarrow{F}_R=\-\mu_1\cdot \Big\vert N\Big\vert\cdot \dfrac{\overrightarrow{v}}{\Big\vert \overrightarrow{v}\Big\vert}}
\end{equation}
Die Gleitreibung wird meist nur mit trockener Reibung, d.h. Gleitreibung zwischen ungefetteten oder nur mässig gefetteten Flächen in Verbindung gebracht. Die Gleichgewichtsbedingungen sind für einen bewegten starren Körper erfüllt, dessen Kinemate bezüglich seines Massenmittelpunktes konstant bleibt. In einem solchen Fall liefert das Gleitreibungsgesetzt im Gegensatz zum Haftreibungsgesetz eine zusätzliche Gleichung zur Ermittlung der Unbekannten.  
\subsection{Gelenk- und Lagerreibung}
