\documentclass[10pt, a4paper]{report}
%%%%%%%%%%%%%%%%%%%%%%%% INPUT %%%%%%%%%%%%%%%%%%%%%%%%
\usepackage[utf8]{inputenc}
\usepackage[german]{babel}

%MATHEMATIK
\usepackage{amssymb}
\usepackage{amsmath}
\usepackage{cancel}
\usepackage{enumerate}
\usepackage{esint}
\DeclareMathOperator{\arsinh}{arsinh}
\DeclareMathOperator{\arccot}{arccot}

%LAYOUT
\usepackage[T1]{fontenc}
\usepackage{charter}
\usepackage{float}
\usepackage[margin=1cm, justification=centering, singlelinecheck=no,tablename=Tab., figurename=Abb.]{caption}
\usepackage{comment}
\usepackage{graphicx}
%\usepackage{hyperref}
\usepackage{xcolor}
\usepackage{wrapfig}

% BOLD
\let\oldtextbf\textbf
\renewcommand{\textbf}[1]{\textcolor{cyan}{\oldtextbf{#1}}}

% COLORS
\usepackage{sectsty}
\chapterfont{\color{cyan}}
\sectionfont{\color{orange}}
\subsectionfont{\color{blue}}
\subsubsectionfont{\color{violet}}

%BOXED
\newcommand*{\boxedcolor}{orange}
\makeatletter
\renewcommand{\boxed}[1]{\textcolor{\boxedcolor}{%
  \fbox{\normalcolor\m@th$\displaystyle#1$}}}


\title{Allgemeine Elektrotechnik}
\author{Ramirez Robayo, Oscar Mauricio}
\date{\today}

\begin{document}

%\maketitle

%\tableofcontents

%\part{First Part of this document}

\chapter{Einführung}
\section{Physikalische Grössen}
\subsection{Zahlwert und Einheit}
Eine physikalische Grösse beinhaltet eine Zahl und eine Einheit. Beinhaltet eine physikalische Grösse einen grossen Zahlenwert, so kann man diese als ein Vielfaches dieser Einheit ausdrücken. Die Dimension gibt Auskunft auf eine detaillierte Charakterisierung der Grösse wie die Höhe, der Abstand der die Strecke mit Einheit Meter.
\subsection{Grundeinheiten und abgeleitete Einheiten}
Die Grundeinheiten besteht aus sieben Grundeinheiten: \textbf{Länge} (Meter $\text{m}$), \textbf{Masse} (Kilogramm $\text{kg}$), Zeit (Sekunde $\text{s}$), \textbf{Stromstärke} (Ampere $\text{A}$), \textbf{Temperatur} (Kelvin $\text{K}$), \textbf{Stoffmenge} (Mol $\text{mol}$) und \textbf{Lichtstärke} (Candela $\text{cd}$).
\newline\newline
Die abgeleitete Einheiten entstehen durch Beziehungen zwischen der Grundeinheiten wie die Geschwindigkeit oder die Beschleunigung.
\newline\newline
Werden Gleichungen mit den Grundeinheiten gerechnet, so wird das Resultat auch in einer Grundeinheit ausgedrückt. Das Ziel besteht auch darin, Resultate mit Hilfe von Zehnerpotenzen zu schreiben.
\begin{table}[H]
\centering
\begin{tabular}{llll}
\hline
Grösse&Zeichen&Name&Symbol\\\hline
Länge&$l$&Meter&$\text{m}$\\
Masse&$m$&Kilogramm&$\text{kg}$\\
Zeit&$t$&Sekunde&$\text{s}$\\
Stromstärke&$I, i$&Ampere&$\text{A}$\\\hline
\end{tabular}
\caption{Grundeinheiten}
\end{table}
\begin{table}[H]
\centering
\begin{tabular}{lllll}
\hline
Grösse&Zeichen&Name&Symbol&Ausdruck\\\hline
Kraft&$F$&Newton&$\text{N}$&$\text{N}=\text{m}\,\text{kg}\, \text{s}^{-2}$\\
Leistung&$P$&Watt&$\text{W}$&$\text{W}=\text{V}\,\text{A}=\text{N}\,\text{m}\,\text{s}^{-1}$\\
Arbeit, Energie&$W$&Joule&$\text{J}$&$\text{J}=\text{W}\,\text{s}=\text{N}\,\text{m}$\\
Spannung&$U, u$&Volt&$\text{V}$&$\text{V}=\text{W}\,\text{A}^{-1}$\\
Widerstand&$R$&Ohm&$\Omega$&$\Omega=\text{V}\,\text{A}^{-1}$\\
Spezifischer Widerstand&$\rho$&&$\Omega\,\text{m}$&$\Omega\,\text{m}=\text{V}\,\text{m}\,\text{A}^{-1}$\\
Leitwert&$G$&Siemens&$\text{S}$&$\text{S}=\Omega^{-1}$\\
Spezifische Leitfähigkeit&$\sigma$&&$\text{S}\,\text{m}^{-1}$&$\text{S}\,\text{m}^{-1}=\Omega^{-1}\,\text{m}^{-1}$\\
Ladung&$Q$&Coulomb&$\text{C}$&$\text{C}=\text{A}\,\text{s}$\\
Elektr. Verschiebungsdichte&$D$&&$\text{A}\,\text{s}\,\text{m}^2$&$\text{C}\,\text{m}^{-2}=\text{A}\,\text{s}\,\text{m}^{-2}$\\
Elektr. Feldstärke&$E$&&$\text{V}\,\text{m}^{-1}$&\\
Kapazität&$C$&Farad&F&$\text{F}=\text{A}\,\text{s}\,\text{V}^{-1}=\text{s}\,\Omega^{-1}$\\
Induktionsfluss&$\phi$&Weber&$\text{Wb}$&$\text{Wb}=\text{V}\,\text{s}$\\
Magn. Induktionsdichte&$B$&Tesla&$\text{T}$&$\text{T}=\text{V}\,\text{s}\,\text{m}^{-2}$\\
Magn. Feldstärke&$H$&&$\text{A}\,\text{m}^{-1}$&\\
Induktivität&$L$&Henry&$\text{H}$&$\text{H}=\text{V}\,\text{s}\,\text{A}^{-1}=\Omega\,\text{s}$\\\hline
\end{tabular}
\caption{Abgeleitete Einheiten}
\end{table}
\begin{table}[H]
\centering
\begin{tabular}{ll}
\hline
Definition&Wert\\\hline
Lichtgeschwindigkeit im Vakuum&$c_0=299'792'458\,\text{m}\,\text{s}^{-1}$\\
Elementarladung&$e=1.6022\cdot 10^{-19}\,\text{C}$\\
Ruhemasse Elektron&$m_0=9.1096\cdot 10^{-31}\,\text{kg}$\\
Ruhemasse Proton&$m_p=1.6726\cdot 10^{-27}\,\text{kg}$\\
Permeabilität im Vakuum&$\mu_0=4\pi\cdot 10^{-7}\,\text{V}\,\text{s}\,\text{A}^{-1}\,\text{m}^{-1}$\\
Permittivität im Vakuum&$\epsilon_0=c_0^{-2}\mu_0^{-1}=8.85419\cdot 10^{-12}\,\text{A}\,\text{s}\,\text{V}^{-1}\,\text{m}^{-1}$\\
Wellenimpendanz des freien Raumes&$\nu_0=\sqrt{\mu_0/\epsilon_0}=376.73\,\Omega$\\\hline
\end{tabular}
\caption{Universelle Konstanten}
\end{table}
\subsection{Skalare und vektorielle Grössen}
Zahlenwerte mit Einheiten bezeichnet man als skalare Grössen. Zahlenwerte mit Enheiten, die in einer bestimmten Richtung des Raumes wirken bezeichnet man als vektorielle Grössen und haben einen Vektorpfeil über das Symbol und ihr Betrag ist die Länge des Vektors.
\section{Elektrizität und ihre Wirkungen}
\subsection{Elektrische Ladung und elektrischer Strom}
Die elektrische Ladung $Q$ ist eine physikalische Grösse und benötigt einen Träger. Der Raum befindet sich in einem elektrischen Feld. Körper, die eine elektrische Ladung tragen, üben eine Kraftwirkung aufeinander aus.
\newline\newline
Elektrische Ladungen könenn sich dabei anziehen oder abstossen. Man unterscheidet zwischen positiver und negativer Ladung. Gleichnamige Ladungen stossen sich ab, während ungleichnamige Ladungen ziehen sich an. Elektrizität entsteht durch Trennen von Ladungen verschiedenen Vorzeichens.
\newline\newline
Im elektrischen neutralen Zustand heben sich die Wirkungen positiver und negativer Ladungen gegenseitig auf. Elektrische Ladungen lassen sich durch Berührung übertragen. Die Elektrizität besteht aus Elektronen.
\newline\newline
Elektrizitätsträger können sich je nach Material übertragen. Ladungsträger bewegen sich in einem Leitungsstrom bzw. elektrischen Strom. Dieser elektrischen Strom ist das Verhältnis der elektrischen Ladung pro Zeit.
\begin{equation}
\boxed{I=\dfrac{Q}{t}}
\end{equation}
Bewegte Ladungen haben thermische (Leiter erwärmt sich bis zum Schmelztemperatur), magnetische (Kräfte entstehen durch Magnete auf Leiter) und chemische Wirkungen (Durch Stromvorgang werden Stoffe verändert).
\subsection{Aufbau der Materie}
Moleküle können in Atome aufgeteilt werden. Um den positiven Atomkern kreisen die negativ geladenen Elektronen. Diese Elektronen bilden die Elektronenhülle, welche Elektronen durch ihre Energie zu Gruppen (Elektronenschalen) aufgeteilt werden. Ein Elektron kann sich nur auf eine Quantenbahn aufhalten. Ein Atom ist elektrisch neutral. Die Elektronen in der äusseren Schale sind Valenzelektronen und bestimmen das chemische Verhalten des Atoms.
\newline\newline
Das Atomkern besteht aus Protonen und Neutronen bzw. aus Nukleonen. Neutronen sind für die Kernspaltung von grösser Bedeutung. Die Beeinflussung der Elektronenhülle erfolgt durch Ionisierungsenergie und bildet Ionen durch Elektronenabgabe oder -zugabe. Ein Ion ist ein elektrisch geladenes Atom.
\subsection{Leiter und Nichtteiler}
Materialien werden in Leiter und Nichtteiler unterteilt. Zu den Leiter zählen die Metalle, aber verändern nicht das Material durch Stromdurchgang. Säuren, Basen und Salzlösungen werden beim Stromdurchgang verändert.
\newline\newline
Zu den Nichtleiter zählen Gummi, Seide, Kunststoffe, Porzellan, Glas, Glimmer, usw. In diesen Stoffen stehen nahezu keine Elektronen zur Verfügung.




\chapter{Grundbegriffe und Grundgesetze}
\section{Elemente der Programmiersprache Java}
\subsection{Bytecode}
Der Java-Compiler erzeugt aus den Quellcode-Dateien den so genannten \textbf{Bytecode}. Dieser Code ist binär und Ausgangspunkt für die virtuelle Machine zur Ausführung. Der Bytecode ist wie ein Prozessor, der Anweiungen wie arithmetische Operationen, Sprünge und Weiteres kennt.
\subsection{Java Virtual Machine}
Die Java Virtual Machine (JVM) kümmert sich um den Bytecode, den Quellcode auszuführen. Die Laufzeitumgebung lädt den Bytecode, prüft ihn und führt ihn in einer kontrollierten Umgebung aus. Java ist Plattform- und Betriebssystemunabhängig. Zu der JVM und der Programmiersprache kommen Standardbibliotheken für Datenstrukturen, Zeichenkettenverarbeitung, Datumverarbeitung, grafische Oberflächen, Ein- und Ausgabe, Netzwerkoperationen und mehr.
\subsection{Objektorientierung}
Eine Laufzeitumgebung eliminiert viele Fehler. Objektorientierte Programmierung versucht, die Komplexität des Softwareproblems besser zu modellieren. Menschen denken objektorientiert, darum Java bildet diese ab. Objekte bestehen aus \textbf{Eigenschaften}, also Dinge, die ein Objekt ``hat'' und ``kann''. Objekte entstehen aus \textbf{Klassen}, das sind Beschreibungen für den Aufbau von Objekten.
\\\\
Primitive Datentypen für numerische Zahlen oder Unicode-Zeichen werden nicht als Objekte betrachtet. Das \textbf{Java-Security-Modell} sicherstellt den Programmablauf. Der \textbf{Verifier} liest Code und überprüft die Korrektheit und Typsicherheit. Treten Sicherheitsprobleme auf, werden sie durch Exceptions zur Laufzeit gemeldet. Das Security-Manager überwächt Zugriffe auf das Dateisystem, die Netzwerk-Ports, externe Prozesse und weitere Systemressourcen.
\\\\
In Java gibt es keine Zeiger auf Speicherbereiche, dagegen führt Java \textbf{Referenzen} ein. Eine Referenz repräsentiert ein Objekt, und eine Variable speichert diese Referenz, sie wird Referenzvariable genannt. JVM verbindet die Referenz mit einem Speicherbereich und einem Referenztyp; der Zugriff, Dereferenzierung genannt, ist indirekt. Referenz und Speicherblock sind getrennt.
\\\\
In Java gibt es keine benutzerdefinierten überladenen Operatoren. Da das Operatorzeichen auf unterschiedlichen Datentypen gültig ist, nennt sich so ein Operator \textbf{Überladen}. Bei Zeichenketten werden Pluszeichen als \textbf{Konkatenation} angewendet. Java braucht keine \textbf{Präprozessoren}.
\subsection{Java Platform}
Mit dem Java Development Kit (JDK) lassen sich Java SE-Applikationen entwickeln. Dem JDK sind Hilfsprogramme beigelegt, die für die Java-Entwicklung nötig sind. Dazu zählen der essenzielle Compiler, aber auch andere Hilfsprogramme, etwa zur Signierung von Java-Archiven oder zum Start einer Management-Konsole.
\\\\
Das Java SE Runtime (JRE) enthält genau das, was zur Ausführung von Java-Programmen nötig ist. Die Distribution umfasst nur die JVM und Java-Bibliotheken, aber weder den Quellcode der Java-Bibliotheken noch Tools wie Management-Tools.
\subsection{Das erste Programm compilieren und testen}
\lstinputlisting[language=Java]{../../PROJEKTE/000001HelloWorld/src/Squares.java}
Ein Compiler übersetzt bzw. transformiert das geschriebene Programm in eine andere Repräsentation nämlich den Bytecode und erzeugt aus dem Program mit Endung \texttt{.java} die Datei \texttt{.class}, welche Bytecode enthält.
\\\\
Wenn der Compiler aufgrund eines syntaktischen Fehlers eine Übersetzung in Java-Bytecode nicht durchführen kann, spricht man von einem Compilerfehler.
\\\\
Eine Laufzeitumgebung liest die Bytecode-Datei Anweisung für Anweisung aus und führt sie auf den konkreten Mikroprozessor aus. Der Interpreter bringt das Programm zur Ausführung.
\\\\
Ein Java-Projekt braucht eine ordentliche Ordnerstruktur, und hier gibt es zur Organisation der Dateien unterschiedliche Ansätze. Die einfachste Form ist, Quellen, Klassendateien und Ressourcen in ein Verzeichnis zu setzen. Es gibt zwei Verzeichnisse \texttt{src} für die Quellen und \texttt{bin} für die erzeugten Klassendateien. Ein eigener Ordner \texttt{lib} ist sinnvoll für Java-Bibliotheken.
\\\\
Das Programm sitzt in einer Klasse, die drei Methoden enthält. Die Methode $\texttt{quadrat(int)}$, bekommt als Übergangsparameter eine ganze Zahl und berechnet daraus die Quadratzahl, die sie anschliessend zurückgibt. Eine weitere Methode übernimmt die Ausgabe der Quadratzahlen bis zu einer vorgegebenen Grenze. Die Methode \texttt{main()}, als Anfangspunkt, ruft die Methode \texttt{ausgabe(int)} auf.
\section{Imperative Sprachkonzepte}
\subsection{Elemente der Programmiersprache Java}
Unter dem Begriff \textbf{Semantik} versteht man die Lexikalik, Syntax und Semantik eines Programms. Der Compiler verläuft diese Schritte bevor er den Bytecode erzeugt.
\\\\
Ein \textbf{Token} ist eine lexikalische Einheit, die dem Compiler die Bausteine des Programms liefert. Der Compiler erkennt an der Grammatik einer Sprache, welche Folgen von Zeichen ein Token bilden.
\\\\
\textbf{Whitespaces} sind Leerzeichen, Tabulatoren, Zeilenvorschub und Seitenvorschubzeichen.
\\\\
Neben den Trennern gibt es noch zwölf ASCII-Zeichen geformte Tokens, die als \textbf{Separator} definiert werden: \texttt{( ) \{ \} [ ] ; , . ... @ ::}
\\\\
Für Variablen, Methoden, Klassen und Schnittstellen werden \textbf{Bezeichner}, auch \textbf{Identifizierer} genannt, vergeben. Unter \textbf{Variablen} sind dann Daten verfügbar. \textbf{Methoden} sind die Unterprogramme in objektorientierten Programmiersprachen, und \textbf{Klassen} sind die Bausteine objektorientierter Programme. Ein Bezeichner ist eine Folge von Zeichen, die fast beliebig sein kann. Die Zeichen sind Elemente aus dem Unicode-Zeichensatz. Der Bezeichner muss mit einem Java-Buchstaben beginnen. String ist eine Klasse und kein Datentyp.
\\\\
Ein Java-Buchstabe umfasst unsere lateinische Buchstaben ``A'' bis ``Z'', ``a'' bis ``z'', sondern auch viele Zeichen aus dem Unicode-Alphabet, den Unterstrich, Währungszeichen, griechische oder arabische Buchstaben, Akzente. Java unterscheidet zwischen Gross- und Kleinschreibung. Nicht erlaubt sind Zahlen am Anfang, Leerzeichen, Ausrufezeichen, reservierte Wörter oder reservierte Schlüsselwörter.
\\\\
Ein \textbf{Literal} ist ein konstanter Ausdruck wie die Wahrheitswerte \texttt{true} und \texttt{false}, Integrale Literale für Zahlen, Fliesskommaliterale, Zeichenliterale wie $\backslash$n, String-Literale für Zeichenketten wie ``Hello World'', Referenztypen wie \texttt{null}.
\\\\
Bestimmte Wörter sind reservierte Schlüsselwörter com Compiler besonders behandelt. Schlüsselwörter bestimmen die Sprache eines Compilers. Es können keine eigenen Schlüsselwörter hinzugefügt werden. Schlüsselwörter sind:
\\\\
\texttt{abstract}, \texttt{assert}, \texttt{boolean}, \texttt{break}, \texttt{byte}, \texttt{case}, \texttt{catch}, \texttt{char}, \texttt{class}, \texttt{const}, \texttt{continue}, \texttt{default}, \texttt{do}, \texttt{double}, \texttt{else}, \texttt{enum}, \texttt{extends}, \texttt{final}, \texttt{finally}, \texttt{float}, \texttt{for}, \texttt{goto}, \texttt{if}, \texttt{implements}, \texttt{import}, \texttt{instanceof}, \texttt{int}, \texttt{interface}, \texttt{long}, \texttt{native}, \texttt{new}, \texttt{package}, \texttt{private}, \texttt{protected}, \texttt{public}, \texttt{return}, \texttt{short}, \texttt{static}, \texttt{strictfp}, \texttt{super}, \texttt{switch}, \texttt{synchronized}, \texttt{this}, \texttt{throw}, \texttt{throws}, \texttt{transient}, \texttt{try}, \texttt{void}, \texttt{volatile}, \texttt{while}
\\\\
Der Compiler überliest alle Kommentare und die Trennzeichen bringen den Compiler von Token zu Token. \textbf{Zeilenkommentare} kann man mit Schrägsstrichen \boxed{\textbf{\texttt{//}}} und kommentieren den Rest einer Zeile bis Zeilenumbruchzeichen aus. \textbf{Blockkommentare} (``Wie'') kommentiert in Blöcke mit \boxed{\textbf{\texttt{/* */}}} aus. \textbf{Javadoc-Kommentare} (``Was'') sind besondere Blockkommentare mit \boxed{\textbf{\texttt{/** */}}} und beschreibt die Methode oder die Parameter, aus denen sich später die API generieren lässt. Kein Kommentar kommt in den Bytecode.
\subsection{Anweisungen}
Programme sind Ablauffolgen, die im Kern aus \textbf{Anweisungen} bestehen. Sie werden zu grösseren Bausteinen zusammengesetzt, den Methoden, die wiederum Klassen bilden. Klassen selbst werden in Paketen gesammelt, und eine Sammlung von Paketen wird als Java-Archiv ausgeliefert.
\\\\
Durch Anweisungen werden \textbf{Algorithmen} geschrieben. Anweisungen können Ausdrucksanweisungen für Zuweisungen oder Methodenaufrufe, auch  Fallunterscheidungen, oder Schleifen für Wiederholungen sein.
\\\\
Anweisungen müssen in einen Rahmen gepackt werden. Dieser Rahmen heisst \textbf{Kompilationseinheit} und deklariert eine Klasse mit ihren Methoden und Variablen. Anweisungen ausserhalb von Klassen sind nicht erlaubt. Der Klassenname ist ein Bezeichner und beinhaltet die gleiche Dateiname. Klassennamen beginnen mit Grossbuchstabe und Methoden sind kleingeschrieben. Zwischen den geschweiften Klammern folgen Deklarationen von Methoden und zwischen den Methoden die Anweisungen.
\\\\
Eine besondere Methode ist \boxed{\textbf{\texttt{public static void(String[] args)\{\}}}}. Die Methode ist für die Laufzeitumgebung etwas Besonders, denn beim Aufruf des Java-Interpreters mit einem Klassennamen wird diese Methode als Erstes ausgeführt. Demnach werden die Anweisungen ausgeführt, die innerhalb der geschweiften Klammern stehen. Der Parameter \texttt{args} wird immer verwendet.
\\\\
Haltet man sich nicht an die Syntax für den Startpunkt, so kann der Interpreter die Ausführung nicht beginnen und man hätte einen semantischen Fehler produziert, obwohl die Methode korrekt gebildet ist.
\\\\
Die Methode \boxed{\textbf{\texttt{println(...)}}} gibt Meldungen auf der Konsole aus. Innerhalb der Klammern können Argumente angegeben werden wie Zeichenketten oder \textbf{Strings} oder eine Folge von Buchstaben, Ziffern oder Sonderzeichen in doppelten Anführungszeichen. Die Methode \texttt{println(...)} gehört zum Typ \textbf{\texttt{out}} und diese zu \textbf{\texttt{System}}.
\lstinputlisting[language=Java]{../../PROJEKTE/000001HelloWorld/src/PrimeraClase.java}
Java erlaubt Methoden, die gleich heissen, denen aber unterschiedliche Dinge übergeben werden können; diese Methoden nennt man \textbf{überladen}. Viele \texttt{println()}-Methoden akzeptieren zahlartige Argumente und sind überladen.
\lstinputlisting[language=Java]{../../PROJEKTE/000001HelloWorld/src/OverloadedPrintln.java}
Die Methode \boxed{\textbf{\texttt{printf()}}} ermöglicht variable Argumentenlisten gemäss einer Formatierungsanweisung. Die Formatierungsanweisung \boxed{\textbf{\texttt{$\backslash$n}}} setzt einen Zeilenumbruch, \boxed{\textbf{\texttt{$\backslash$d}}} ist ein Platzhalter für eine ganze Zahl, \boxed{\textbf{\texttt{$\backslash$f}}} ist ein Platzhalter für eine Fliesskommazahl, \boxed{\textbf{\texttt{$\backslash$s}}} ist eine Zeichenkette oder etwas, was in einen String konvertiert werden soll.
\lstinputlisting[language=Java]{../../PROJEKTE/000001HelloWorld/src/VarArgs.java}
Methodenaufrufe lassen sich als Anweisungen einsetzen, wenn sie mit einem Semikolon abegschlossen sind, man spricht von einer \textbf{Ausdrucksanweisung} (expression statement). Jeder Methodenaufruf mit Semikolon bildet eine Ausdrucksanweisung. Dabei ist es egal, ob die Methode selbst eine Rückgabe liefert oder nicht.
\\\\
Die Methode \boxed{\textbf{\texttt{Math.random()}}} liefert eine Fliesskommazahl zwischen 0 (inklusiv) und 1 (exklusiv). In einer objektorientierte Programmiersprache sind alle Methoden an bestimmte Objekte mit einem Zustand gebunden. Alle Operationen und Zustände sind an Objekte bzw. Klassen gebunden. Der Aufruf einer Methode auf einem Objekt richtet die Anfrage genau an dieses bestimmte Objekt.
\\\\
Die Deklaration einer Klasse oder Methode kann einen oder mehrere \textbf{Modifizierer} enthalten, die zum Beispiel die Nutzung einschränken oder parallelen Zugriff synchronisieren. Der Modifizierer \boxed{\textbf{\texttt{public}}} ist ein Sichtbarkeitsmodifizierer. Er bestimmt, onb die Klasse bzw. die Methode für Programmcode anderer Klassen sichtbar ist oder nicht. Der Modifizierer \boxed{\textbf{\texttt{static}}} zwingt den Programmierer nicht dazu, vor dem Methodenaufruf ein Objekt der Klasse zu bilden. Dieser Modifizierer bestimmt die Eigenschaft, ob sich eine Methode nur über ein konkretes Objekt aufrufen lässt oder eine Eigenschaft der Klasse ist, sodass für den Aufruf kein Objekt der Klasse nötig wird.
\\\\
Ein \textbf{Block} fasst eine Gruppe von Anweisungen,die hintereinander ausgeführt werden. Ein Block \boxed{\textbf{\texttt{\{\}}}} ist eine Anweisung, die in geschweiften Klammern eine Folge von Anweisungen zu einer neuen Anweisung zusammenfasst. Ein Block kann überall dort verwendet werden, wo auch eine einzelne Anweisung stehen kann. Der neue Block hat jedoch eine Besonderheit in Bezug auf Variablen, da er einen lokalen Bereich für die darin befindlichen Anweisungen inklusive der Variablen bildet.
\\\\
Ein Block ohne Anweisung nennt sich ein leerer Block. Er verhält sich wie eine leere Anweisung, also wie ein Semikolon. Es gibt innere und äussere Blöcke. Blöcke fassen Anweisungen zusammen.
\section{Datentypen, Variablen und Zuweisungen}
Java speichert Variablen. Eine Variable ist ein reservierter Speicherbereich und belegt eine feste Anzahl von Bytes. Variablen und Ausdrücke haben einen \textbf{Datentyp} und einen \textbf{Datenwert}. Der Datentyp bestimmt die zulässigen Operationen. Java ist eine streng typisierte Programmiersprache. Datentypen werden unterteilt in \textbf{primitive Datentypen} (Zahlen, Unicode-Zeichen und Wahrheitswerte) und \textbf{Referenztypen} (Zeichenketten, Datenstrukturen, Zwergpinscher) und Bytecode durch den Compiler einfacher erzeugt.
\begin{table}[H]
\centering
\begin{tabular}{lll}
\hline
Typ&Grösse&Belegung (Wertebereich)\\\hline
boolean&1 Bit&\texttt{true} oder \texttt{false}\\
char& 16Bit&$\text{0x0000 \dotso 0xFFFF}$\\\hline
byte*&8 Bit&$-2^7$ bis $2^7-1$\\
short*&16 Bit&$-2^{15}$ bis $2^{15}-1$\\
int*&32 Bit&$-2^{31}$ bis $2^{31}-1$ \\
long*&64 Bit&$-2^{63}$ bis $2^{63}-1$\\\hline
float&32 Bit&$1,4023\cdot 10^{-45} \dotso 3,4028\cdot10^{38}$\\
double&64 Bit&$4,9406\cdot 10^{-324} \dotso 1,7976\cdot 10^{308}$\\\hline
\end{tabular}
\caption{Java-Datentypen, Grössen und Formate. *\textit{Zweierkomplement}}
\end{table}
\noindent Es gibt mehr negative Werte als positive Werte, das liegt an der Kodierung im Zweierkomplement. Bei \textbf{\texttt{float}} und \textbf{\texttt{double}} ist das Vorzeichen nicht angegeben, die Wertebereiche unterscheiden sich nicht, die kleinsten und grössten darstellbaren Zahlen können sowohl positiv als auch negativ sein.
\subsection{Variablendeklarationen}
Mit Variablen lassen sich Daten speichern, die vom Programm gelesen und geschrieben werden können. Variablen müssen deklariert werden. Hinter dem Typnamen folgt der Name der Variablen. Die \textbf{Deklaration} ist eine Anweisung und wird daher mit einem Semikolon abgeschlossen.
\lstinputlisting[language=Java]{../../PROJEKTE/000001HelloWorld/src/FirstVariable.java}
Gleich bei der Deklaration lassen sich Variablen mit einem Anfangswert initialisieren. Hinter einem Gleichheitszeichen steht der Wert, der oft ein Literal ist.
\newline\newline
Eine Konsoleneingabe. Eine Variante ist die Klasse \texttt{java.util.Scanner}. Folgende Tabelle zeigt die Eingabe von drei verschiedenen Datentypen.
\begin{table}[H]
\centering
\begin{tabular}{ll}
\hline
Eingabe & Anweisung\\\hline
\texttt{String}&\texttt{String s = new java.util.Scanner(System.in).nextLine();}\\
\texttt{int}&\texttt{int i = new java.util.Scanner(System.in).nextInt();}\\
\texttt{double}&\texttt{double d = new java.util.Scanner(System.in).nextDouble();}\\\hline
\end{tabular}
\caption{Einlesen einer Zeichenkette, Ganz- und Fliesskommazahl von der Konsole}
\end{table}
\noindent Folgendes Beispiel zeigt eine Anwendung aller Eingabenmöglichkeiten mit der Klasse \texttt{java.util.Scanner}.
\lstinputlisting[language=Java]{../../PROJEKTE/000001HelloWorld/src/SmallConversation.java}
\subsection{Fliesskommazahlen}
Java bietet die Datentypen \textbf{\texttt{float}} und \textbf{\texttt{double}}. Fliesskommazahl können einen Vorkommateil und einen Nachkommateil besitzen, die durch einen Dezimalzahl getrennt sind. Standardmässig sind die Fliesskommaliterale vom Typ \textbf{\texttt{double}}. Ein nachgestelltes \textbf{\texttt{f}} oder \textbf{\texttt{F}} zeigt dem Computer an, dass es sich um einen \texttt{float} handelt.
\newline\newline
So ist beispielsweise \texttt{1+2+4.0} eine Addition aus \texttt{1+2} dann in \texttt{double} transformiert und anschliessend \texttt{3.0+4.0}. Die Standardbibliothek \textbf{\texttt{java.math}} bietet die Klasse \textbf{\texttt{BigDecimal}} an. Diese Klasse eignet sich gut für gute Genauigkeit wie Währungen.
\subsection{Ganzzahlige Datentypen}
Java stellt fünf ganzzahlige Datentypen zur Verfügung: \textbf{\texttt{byte}}, \textbf{\texttt{short}}, \textbf{\texttt{char}}, \textbf{\texttt{int}} und \textbf{\texttt{long}}. Ganzzahlige Datentypen sind immer vorzeichenbehaftet (mit Ausnahme von \texttt{char}). Einen Modifizierer \texttt{unsigned} gibt es nicht. Java reserviert nicht so viele Bits wie benötigt und wählt nicht automatisch den passenden Wertebereich. Dabei ist \textbf{\texttt{System.out.println( 122323423434525345345435);}} fehlerbehaftet. Der Datentyp \textbf{\texttt{int}} ist in Java standardmässig.
\newline\newline
An das Ende von Ganzzahlliteralen vom Typ \textbf{\texttt{long}} wird ein \textbf{\texttt{l}} oder ein \textbf{\texttt{L}} gesetzt. Dabei wird \textbf{\texttt{System.out.println( 122323423434525345345435L);}} gültig.
\newline\newline
Ein \textbf{\texttt{byte}} ist ein Datentyp mit einem kleineren Wertebereich. Eine Initialisierung \textbf{\texttt{byte b = 200;}} ist fehlerbehaftet. Eine explizite Typumwandlung lässt Zahlen in einem \textbf{\texttt{byte}} speichern und zwar \textbf{\texttt{byte b = (byte) 200;$\Longrightarrow$ -56}}.
\newline\newline
Der Datentyp \textbf{\texttt{short}} stehen 16 Bits, (1 Bit für das Vorzeichen und 15 Bit für die Zahlen) Speicher zur Verfügung. Ein \texttt{short} ohne Vorzeichen kann folgendermassen initialisiert werden: \textbf{\texttt{short s = (short) 3300;$\Longrightarrow$ -32536}}
\subsection{Wahrheitswerte}
Der Datentyp \textbf{\texttt{boolean}} beschreibt einen Wahrheitswert, der entweder \textbf{\texttt{true}} oder \textbf{\texttt{false}} ist. Diese sind reservierte Wörter und bilden neben konstanten Strings und primitiven Datentypen Literale. Numerische Werte werden nicht als Wahrheitswerte interpretiert. Der boolesche Typ wird für Bedingungen, Verzweigungen oder Schleifen benötigt. Ein Wahrheitswerte ergibt sich aus Vergleichen.
\subsection{Unterstriche in Zahlen}
Eine Variante um grosse Zahlen mit viele Nullen zu schreiben ist es, Unterstriche in Zahlen einzusetzen, denn ein Unterstrich gliedert die Zahl in Blöcke. Unterstriche machen Tausender-Blöcke gut sichtbar. Hilfreich ist die Schreibweise auch bei Literalen in Binär- und Hexadezimaldarstellung. Mit \textbf{\texttt{0b}} beginnt ein Literal in Binärschreibweise und mit \textbf{\texttt{0x}} beginnt ein Literal in Hexadezimalschreibweise. Zwei aufeinanderfolgende Unterstriche sind aber nicht erlaubt und er darf nicht am Anfang stehen.
\subsection{Alphanumerische Zeichen}
Der alphanumerische Datentyp \textbf{\texttt{char}} ist 2 Byte gross und nimmt ein Unicode-Zeichen auf. Ein \texttt{char} ist nicht vorzeichenbehaftet. Die Literale werden in Hochkommata (nicht Anführungszeichen) gesetzt. Ein \texttt{char} kann automatisch in ein \texttt{int} konvertiert werden.
\subsection{Initialisierung von lokalen Variablen}
Die Laufzeitumgebung bzw. der Compiler initialisiert lokale Variablen nicht automatisch mit einem Nullwert bzw. einen \texttt{false}. Sind Variablen nicht initialisiert, so gibt es Fehlermeldungen.
\section{Ausdrücke, Operanden und Operatoren}
Mathematische Ausdrücke bestehen aus \textbf{Operanden} und \textbf{Operatoren}. Ein Operand ist eine Varaible, ein Literal oder Rückgabe eines Methodenaufrufs. Die Operatoren verknüpfen die Operanden. Je nach Anzahl der Operanden unterscheidet man folgende Arten von Operatoren:
\begin{itemize}
\item Ist ein Operator auf genau einem Operanden definiert, so nennt er sich unärer Operator. Bsp: Negatives Vorzeichen.
\item Die üblichen Operatoren für mathematische Ausdrücke sind binäre Operatoren.
\item Das Fragezeichen-Operator für bedingte Ausdrücke ist ein tertiäres Operator.
\end{itemize}
\subsection{Zuweisungsoperator}
Das Gleichheitszeichen \textbf{\texttt{=}} dient in Java der Zuweisung. Der Zuweisungsoperator ist ein binärer Operator, bei dem auf der linken Seite due zu belegende Variable steht und auf der rechten Seite ein Ausdruck. Erst nach dem Auswerten des Ausdrucks kopiert der Zuweisungsoperator das Ergebnis in die Variable. Division durch Null, so gibt es keinen Schreibzugriff auf die Variable. Zuweisungen können geschachtelt werden.
\subsection{Arithmetische Operatoren}
Ein arithmetischer Operator verknüpft die Operanden mit den Operatoren Addition (\textbf{\texttt{+}}), Subtraktion (\textbf{\texttt{-}}), Multiplikation (\textbf{\texttt{*}}), Division (\textbf{\texttt{/}}) und den Rest-Operator (\textbf{\texttt{\%}}). Die arithmetische Operatoren sind binär.
\newline\newline
Bei Ausdrücken mit unterschiedlichen numerischen Datentypen, bringt der Compiler vor der Anwendung der Operation alle Operanden auf den umfassenderen Typ. Vor der Auswertung von \texttt{1+2.0} wird die Ganzzahl \texttt{1} in ein \texttt{double} konvertiert und dann die Addition vorgenommen - das Ergebnis ist auch vom Typ \texttt{double}. Das nennt sich \textbf{numerische Umwandlung}. Die Operation wird ausgeführt, und der Ergebnistyp entspricht dem umfassenden Typ.
\newline\newline
Der binäre Operator bildet den Quotienten aus Dividend und Divisor. Die Division ist für Ganzzahlen und für Fliesskomazahlen definiert. Bei der Ganzzahldivision wird zu null hin gerundet und das Ergebnis ist keine Fliesskomazahl. Den Datentyp des Ergebnisses bestimmen die Operanden und nicht der Operator. Soll das Ergebnis vom Typ \texttt{double} sein, muss mindestens ein Operand ebenfalls \text{double} sein.
\subsection{Der Restwert-Operator \%}
Der Restwert-Operator liefert der Rest einer Division zweier Ganzzahlen und Fliesskomazahlen. Die DIvision und der Restwert richten sich nach einer einfachen Formel: \texttt{(int)(a/b)*b+(a\%b)=a}. Das Ergebnis ist nur dann negativ, wenn der Dividend negativ ist; das Ergebnis ist nur dann positiv, wenn der Dividend positiv ist. Um mit \texttt{value\%2 == 1} zu testem, ob \texttt{value} eine ungerade Zahl ist, muss \texttt{value} positiv sein.
\subsection{Präfix- oder Postfix-Inkrement und -Dekrement}
Die Operatoren \textbf{\texttt{++}} und \textbf{\texttt{--}} kürzen die Programmzeilen zum Inkrement und Dekrement ab. Eine lokale Variable muss allerdings vorher initialisiert sein, da ein Lesezugriff vor einem Schreibzugriff stattfindet. Beide Operatoren erfüllen somit zwei Aufgaben: Neben der Wertrückgabe gibt es eine Veränderung der Variablen.
\begin{table}[H]
\centering
\begin{tabular}{lll}
\hline
&Präfix&Postfix\\\hline
Inkrement & Prä-Inkrement, \textbf{\texttt{++i}}&Post-Inkrement, \textbf{\texttt{i++}}\\
Dekrement & Prä-Dekrement, \textbf{\texttt{--i}}&Post-Dekrement, \textbf{\texttt{i--}}\\\hline
\end{tabular}
\caption{Präfix- oder Postfix-Inkrement und -Dekrement}
\end{table}
\noindent Die beiden Operatoren liefern einen Ausdruck und geben daher einen Wert zurück. Es macht jedoch einen feinen Unterschied, wo dieser Operator platziert wird: Er kann vor der Variablen stehen, wie \texttt{++i} oder dahinter wie \texttt{i++}. Der \textbf{Präfix-Operator} verändert die Variable vor der Auswertung des Ausdrucks, und der \textbf{Postfix-Operator} verändert die Variable nach der Auswertung des Ausdrucks.
\lstinputlisting[language=Java]{../../PROJEKTE/000001HelloWorld/src/Prefixen.java}
\subsection{Auswertung bei Array-Zugriffen}
Falls die linke Seite beim Verbundoperator ein Array-Zugriff ist, wird die Indexberechnung nur einmal vorgenommen. Dies ist wichtig beim Einsatz vom Präfix-/Postfix-Operator oder von Methodenaufrufen, die Nebenwirkungen besitzen, also etwa Zustände wie einen Zähler verändern.
\subsection{Zuweisung mit Operation (Verbundoperator)}
Zuweisungen lassen sich mit numerischen Operatoren kombinieren. Für einen binären Operator (symbolisch \textbf{\texttt{\#}} genannt) im Ausdruck \textbf{\texttt{a = a\#(b)}} kürzt der Verbundoperator den Ausdruck zu \textbf{\texttt{a\#b}} ab. Der Verbundoperator erlaubt eine kompakte Schreibweise.
\subsection{Relationale und Gleichheitsoperatoren}
Relationale Operatoren sind Vergleichsoperatoren, die Ausdrücke miteinander vergleichen und einen Wahrheitswert vom Typ \texttt{boolean} ergeben. Die numerische Vergleiche sind: grösser (\textbf{\texttt{>}}), kleiner (\textbf{\texttt{<}}), grösser/gleich (\textbf{\texttt{$\geq$}}), kleiner/gleich (\textbf{\texttt{$\leq$}}), Gleichheit (\textbf{\texttt{==}}), Ungleichheit (\textbf{\texttt{!=}}).
\subsection{Logische Operatoren}
Die Programmierung ist an Bedingungen verknüpft. Diese Bedingungen sind komplex zusammengesetzt, wobei drei Operatoren am häufigsten vorkommen. \textbf{Nicht \texttt{!}:} (Negation) dreht die Aussage um; aus \texttt{wahr} wird \texttt{falsch} und aus \texttt{falsch} wird \texttt{wahr}. \textbf{Und \texttt{\&\&}:} (Konjunktion) beide Aussagen müssen \texttt{wahr} sein, damit die Gesamtaussage \texttt{wahr} wird. \textbf{Oder \texttt{||}:} (Disjunktion) eine der beiden Aussagen muss \texttt{wahr} sein, damit die Gesamtaussage \texttt{wahr} wird. \textbf{Xor \texttt{\^}:} (Exklusives Oder) Operation, die nur dann \texttt{wahr} liefert, wenn genau einer der beiden Operanden \texttt{wahr} ist. Sind beide Operanden gleich, so ist das Ergebnis \texttt{false}.
\begin{table}[H]
\centering
\begin{tabular}{llllll}
\hline
boolean a& boolean b&\texttt{!a}&\texttt{a\&\&b}&\texttt{a||b}&\texttt{a\^\,b}\\\hline
\texttt{true}&\texttt{true}&\texttt{false}&\texttt{true}&\texttt{true}&\texttt{false}\\
\texttt{true}&\texttt{false}&\texttt{false}&\texttt{false}&\texttt{true}&\texttt{true}\\
\texttt{false}&\texttt{true}&\texttt{true}&\texttt{false}&\texttt{true}&\texttt{true}\\
\texttt{false}&\texttt{false}&\texttt{true}&\texttt{false}&\texttt{false}&\texttt{true}\\\hline
\end{tabular}
\caption{Verknüpfungen der logischen Operatoren.}
\end{table}
\subsection{Rang der Operatoren}
Neben Plus und Mail gibt es eine Vielzahl von Operatoren., die alle ihre eigenenn Vorrangregeln besitzen. Der Multiplikationsoperator besitzt eine höhere Priorität als der Plus-Operator. Der \textbf{arithmetische Typ} steht für Ganz- und Fliesskommazahlen, der \textbf{integrale Typ} für \texttt{char} und Ganzzahlen und der Eintrag primitiv für jegliche primitiven Datentypen.
\begin{table}[H]
\centering
\begin{tabular}{llll}
\hline
Operator&Rang&Typ&Beschreibung\\\hline
\texttt{++}, \texttt{--}&1&arithmetisch&Inkrement und Dekrement\\
\texttt{+}, \texttt{-}&1&arithmetisch&unäres Plus und Minus\\
\texttt{\~}&1&integral&bitweises Komplement\\
\texttt{!}&1&boolean&logisches Komplement\\
\texttt{(Typ)}&1&jeder&Cast\\\hline
\texttt{*}, \texttt{/}, \texttt{\%}&2&arithmetisch&Multiplikation, Division, Rest\\\hline
\texttt{+}, \texttt{-}&3&arithmetisch&Addition und Subtraktion\\
\texttt{+}&3&String&String-Konkatenation\\
\texttt{<<}&4&integral&Verschiebung links\\
\texttt{>>}&4&integral&Rechtsverschiebung mit Vorzeichenerweiterung\\
\texttt{>>>}&4&integral&Rechtsverschiebung ohne Vorzeichenerweiterung\\\hline
\texttt{<}, \texttt{<=}, \texttt{>}, \texttt{>=}&5&arithmetisch&Numerische Vergleiche\\
\texttt{instanceof}&5&Objekt&Typvergleich\\
\texttt{==}, \texttt{!=}&6&primitiv&Gleich-/Ungleichheit von Werten\\
\texttt{==}, \texttt{!=}&6&Objekt&Gleich-/Ungleichheit von Referenzen\\
\texttt{\&}&7&integral&bitweises Und\\
\texttt{\&}&7&boolean&logisches Und\\\hline
\texttt{\^}&8&integral&bitweises XOR\\
\texttt{\^}&8&boolean&logisches XOR\\
\texttt{|}&9&integral&bitweises Oder\\
\texttt{|}&9&boolean&logisches Oder\\\hline
\texttt{\&\&}&10&boolean&logisches konditionales Und, Kurzschluss\\\hline
\texttt{||}&11&boolean&logisches konditionales Oder, Kurzschluss\\
\texttt{?:}&12&jeder&Bedingungsoperator\\
\texttt{=}&13&jeder&Zuweisung\\
\texttt{*=}, \texttt{/=}, \texttt{\%=}&13&arithmetisch&Zuweisung mit Operation\\
\texttt{+=}, \texttt{=}, \texttt{<<=}&13&arithmetisch&Zuweisung mit Operation\\
\texttt{>>=}, \texttt{>>>=}, \texttt{\&=}&13&arithmetisch&Zuweisung mit Operation\\
\texttt{\^=}, \texttt{|=}&13&arithmetisch&Zuweisung mit Operation\\
\texttt{+=}&14&String&Zuweisung mit String-Konkatenation\\\hline
\end{tabular}
\caption{Operatoren mit Rangordnung}
\end{table}
\subsection{Die Typumwandlung (Casting)}
Datentypen können konvertiert werden, dies nennt sich \textbf{Typumwandlung}. Java unterscheidet zwischen zwei Arten der Typumwandlung. EIne Typumwandlung hat eine sehr hohe Priorität. DAher muss der Ausdruck gegebenfalls geklammert werden.
\begin{itemize}
\item \textbf{Implizite Typumwandlung:} Daten eines kleineren Datentyps werden automatisch dem grösseren angepasst. Der Compiler nimmt die Anpassung selbständig vor.
\item \textbf{Explizite Typumwandlung:} Ein grösserer Typ kann einem kleineren Typ mit möglichem Verlust von Informationen zugewiesen werden.
\end{itemize}
Werte der Datentypen \texttt{byte} und \texttt{short} werden bei Rechenoperationen automatisch in den Datentyp \texttt{int} umgewandelt. Ist ein Operand vom Datentyp \texttt{long}, dann werden alle Operanden auf \texttt{long} erweitert. Wird aber \texttt{short} oder \texttt{byte} als Ergebnis verlangt, dann ist dieses durch einen expliziten Typecast anzugeben, und nur die niederwertigen Bits des Ergebniswerts werden übergeben.
\begin{table}[H]
\centering
\begin{tabular}{ll}
\hline
Vom Typ&in den Typ\\\hline
\texttt{byte} (8 Bit)&\texttt{short}, \texttt{int}, \texttt{long}, \texttt{float}, \texttt{double}\\
\texttt{short} (16 Bit)&\texttt{int}, \texttt{long}, \texttt{float}, \texttt{double}\\
\texttt{char} (16 Bit)&\texttt{int},\texttt{long}, \texttt{float}, \texttt{double}\\
\texttt{int} (32 Bit)&\texttt{long}, \texttt{float}, \texttt{double}\\
\texttt{long} (64 Bit)&\texttt{float}, \texttt{double}\\
\texttt{float} (32 Bit)&\texttt{double}\\
\texttt{double} (64 Bit)&\texttt{double}\\\hline
\end{tabular}
\caption{Implizite Typumwandlungen}
\end{table}
\noindent Die Anpassung ist eine Erweiterung des Wertebereichs (widening conversion). Der Typ \texttt{boolean} taucht nicht auf, er lässt sich in keinen anderen primitiven Typ konvertieren. Dass ein \texttt{long} auf ein \texttt{double} gebracht werden kann bzw. ein \texttt{int} auf ein \texttt{float} ist als Fehler in der Java zu sehen, denn es gehen Informationen verloren. Ein \texttt{double} kann die 64 Bit für Ganzzahlen nicht effizient nutzen wie ein \texttt{long}.
\newline\newline
Die explizite Anpassung engt einen Typ ein (narrowing conversion). Der gewünschte Typ für eine Typumwandlung wird vor den umzuwandelnden Datentyp in Klammern gesetzt. Bei jeder expliziten Typumwandlung geht Information verloren.
\newline\newline
Bei der expliziten Typumwandlung von \texttt{double} und \texttt{float} in einen Ganzzahltyp kann es selbstverständlich zum Verlust von Genauigkeit kommen sowie zur Einschränkung des Wertebereichs. Bei der konvertierung von Fliesskommazahlen verwendet Java eine Rundung gegen null, schneidet also schlicht den Nachkommaanteil ab.
\lstinputlisting[language=Java]{../../PROJEKTE/000001HelloWorld/src/Typumwandlung.java}
Die String-Konkatenation ist strikt von links nach rechts und natürlich nicht kommutativ wie die numerische Addition. Besteht der Auddruck aus mehreren Teilen, so muss die Auswertungsreihenfolge beachtet werden, andernfalls kommt es zu seltsamen Zusammensetzungen.
\lstinputlisting[language=Java]{../../PROJEKTE/000001HelloWorld/src/PlusString.java}
Nur eine Zeichenkette in doppelten Anführungszeichen ist ein String, und der Plus-Operator entfaltet seine besondere Wirkung. Ein einzelnes Zeichen in einfachen Hochkommata konvertiert Java nach den Regeln der Typumwandlung bei Berechnungen in ein \texttt{int} und Additionen sind Ganzzahl-Additionen.
\lstinputlisting[language=Java]{../../PROJEKTE/000001HelloWorld/src/PlusZeichen.java}
\section{Bedingte Anweisungen}
\subsection{Verzweigung mit der \texttt{if}-Anweisung}
Die \textbf{\texttt{if}}-Anweisung besteht aus dem Schlüsselwort \texttt{if}, dem zwingend ein Ausdruck mit dem Typ \texttt{boolean} in Klammern folgt. Es folgt eine Anweisung, die oft eine Blockanweisung ist.
\lstinputlisting[language=Java]{../../PROJEKTE/000001HelloWorld/src/WhatsYourNumber.java}
Ist das Ergebnis in der Bedingung \texttt{true}, so werden die Anweisungen in der Fallunterscheidung ausgeführt, sonst werden die \text{else}-Anweisungen ausgeführt. Eine Fallunterscheidung hat kein Semikolon. \texttt{if} und \texttt{if-else}-Anweisungen werden geschachtelt (kaskadiert).
\lstinputlisting[language=Java]{../../PROJEKTE/000001HelloWorld/src/IsLeapYear.java}
Die eingerückten Verzweigungen nennen sich auch angehäufte \texttt{if}-Anweisungen oder \texttt{if}-Kaskade.
\subsection{Der Bedingungsoperator}
Der Bedingungsoperator, auch Konditionaloperator, erlaubt es, den Wert eines Ausdrucks von einer Bedingung abhängig zu machen, ohne dass dazu eine \texttt{if}-Anweisung verwendet werden muss. Die Operanden sind durch \textbf{\texttt{?}} und \textbf{\texttt{:}} voneinander getrennt.
\lstinputlisting[language=Java]{../../PROJEKTE/000001HelloWorld/src/BedingungsOperator.java}
Drei Ausdrücke kommen in Bedingunsoperator vor. Der erste Ausdruck muss vom Typ \texttt{boolean} sein. Der Bedingungsoperator kann eingesetzt werden, wenn der zweite und dritte Operand ein numerischer Typ, boolescher Typ oder Referenztyp sind.
\subsection{Die \texttt{switch}-Anweisung}
Eine Kurzform für speziell gebaute, angehäufte \texttt{if}-Anweisungen bietet \textbf{\texttt{switch}}. Es gibt eine Reihe von unterschiedlichen Sprungzeilen, die mit \textbf{\texttt{case}} markiert sind. Die \texttt{switch}-Anweisung erlaubt die Auswahl vin Ganzzahlen, Wrapper-Typen, Aufzählungen und Strings.
\lstinputlisting[language=Java]{../../PROJEKTE/000001HelloWorld/src/Calculator.java}













\chapter{Einfache RLC-Netzwerke im Zeitbereich}
\section{Zweidimensionale Graphik}
\subsection{Elementare zweidimensionale Graphik}
Der Befehl \boxed{\textbf{\texttt{plot}}} öffnet ein Graphikfenster namens "figure" mit einer Nummer, in das eine Graphik engebettet werden kann. Falls für die Abszisse und für die Ordinate keine Schranken gesetzt werden, passt sich die Skalierung des Koordinatensystems den Daten automatisch an. Für jedes weitere Bild mus mit dem Befehl \boxed{\textbf{\texttt{figure}}} ein neues Graphikfenster geöffnet werden, es erhält eine neue Nummer. Andernfalls wird das alte Bild im Graphikfenster durch das neue Bild überschrieben. 
\newline\newline
Bekanntlich basiert die Grundstruktur von MATLAB auf einer \texttt{n$\times$m}-Matrix aus reellen oder komplexen Elementen. Auch Daten werden ja in Matrizen abgelegt. Für ein zweidimensaionales Bild benötigt MATLAB also mindestens zwei Kolonnenvektoren gleicher Länge.
\newline\newline
Der Befehl {\color{red}\texttt{plot(x,y)}} zeichnet den Datensatz \texttt{y} in Funktion von Datensatz \texttt{x} auf. Wie üblich wird jedem Wert von \texttt{x} ein Wert von \texttt{y} zugeordnet. \texttt{x} sind die Werte der Abszisse und \texttt{y} diejenigen auf der Ordinate. Die daraus resultierende Punkte werden mit geraden Linien verbunden (lineare Interpolation). Beide Achsen haben eine lineare Skala.
\newline\newline
Mit {\color{red}\texttt{plot(x,y,s)}} werden im String \texttt{s} der Linientyp und die Farbe der Kurve definiert. Der Befehl {\color{red}\texttt{plot(x,y,'c+:')}} plottet eine rot punktierte Linie, die jedem Datenpunkt ein rotes Plus-Zeichen hat. Der Befehl {\color{red}\texttt{plot(y)}} enthält nur einen Kolonnenvektor \texttt{y} Element von \texttt{$\mathbb{R}^{n\times 1}$}. In diesem Fall generiert MATLAB für die \texttt{x}-Achse automatisch Werte, nämlich 1 bis \texttt{n}, die Indizes der \texttt{n} Kolonnenwerte. 
\newline\newline
Handelt es sich jedoch bei \texttt{y} um einen Vektor mit komplexen Zahlen, so werden beim Befehl {\color{red}\texttt{plot(y)}} die Realteile auf der \texttt{x}-Achse und die Imaginärteile auf der \texttt{y}-Achse aufgetragen. 
\newline\newline
Der Befehl {\color{red}\texttt{plot(A)}} zeichnet für jede eine Kurve aus \texttt{n} Punkten. Die Matrix \texttt{A} Element von \texttt{$\mathbb{R}^{n\times m}$} je eine Kurve aus \texttt{n} Punkten. Die \texttt{x}-Achse zeigt wieder die Indizes 1 bis \texttt{n}. Die LInien werden zur Unterscheidung verschiedene Stile bzw. verschiedene Farben haben.
\newline\newline
Beim Befehl {\color{red}\texttt{plot(x,A)}} wird jede der \texttt{m} Kolonnen der Matrix \texttt{A} Element von \texttt{$\mathbb{R}^{n\times m}$} gegen die gemeinsame unabhängige Variable \texttt{x} aufgezeichnet. Der Vektor \texttt{x} muss die Dimension \texttt{n} haben: \texttt{x} ist ELement von \texttt{$\mathbb{R}^{n\times 1}$}
\newline\newline
Beim Befehl {\color{red}\texttt{plot(A,B)}} nilden je eine Kolonne der Matrix \texttt{A} Element von \texttt{$\mathbb{R}^{n\times m}$} und der Matrix \texttt{B} Element von \texttt{$\mathbb{R}^{n\times m}$} ein \texttt{x}-\texttt{y}-Vektorpaar. 
\newline\newline
Der Befehl \boxed{\textbf{\texttt{subplot(n,m,p)}}} unterteilt ein Graphikfenster in \texttt{n} Zeilen von je \texttt{m} Bildern. Damit können \texttt{n$\times$m} Bilder in ein Graphikfenster eingebettet werden. \texttt{p} ist der Laufindex der \texttt{n$\times$m} Bilder, wobei die Numerierung zeilenweise von links nach rechts erfolgt. Für jedes neue Bild im Graphikfenster wird der Befehl \texttt{subplot} wiederholt, jedesmal mit dem neuen Index \texttt{p}. Der eigentliche Befehl \texttt{plot} mit seinen Parametern muss dann natürlich auc noch kommen. 
\newline\newline
Die Befehle \boxed{\textbf{\texttt{semilogx}}} und {\color{red}\texttt{plot}} sind bis auf die Skalierung der \texttt{x}-Achse identisch. Beim Befehl {\color{red}\texttt{plot}} hat die Abszisse eine lineare, beim Befehl \texttt{semilogx} aber eine logarithmische Skala (Basis 10). Der Befehl {\color{red}\texttt{semilogx(x,y)}} entspricht dem Befehl {\color{red}\texttt{plot(log10(x), y)}}, doch MATLAB gibt bei \texttt{semilogx} für \texttt{x=0} keine Warnung "log for zero". Der Nullpunkt der \texttt{x}-Achse wird unterdrückt, er kann nicht gezeichnet werden. Mit dem Befehl {\color{red}\texttt{semilogy}} wird die Ordinate mit dem Zehnerlogarithmus skaliert.    
\newline\newline
Der Befehl \boxed{\textbf{\texttt{loglog}}} plottet eine zweidimensionale Graphik in einem doppeltlogarithmischen Koordinatensystem (Basis 10). Der Befehl {\color{red}\texttt{loglog(x,y),log10(y)}} entspricht dem Befehl {\color{red}\texttt{plot(log10(x))}}, doch MATLAB gibt bei \texttt{loglog} keine Warnung "log of zero", falls \texttt{x} oder \texttt{y} gleich Null ist. Der Koordinatennullpunkt wird unterdrückt.  
\newline\newline
Der Befehl \boxed{\textbf{\texttt{polar(phi,r)}}} zeichnet die beiden Vektoren \texttt{phi} und \texttt{r} in ein polares Koordinatensystem. Die Werte des Vektors \texttt{phi} sind im Bogenmass angegeben. Die WErte des Vektors \texttt{r} entsprechen dem Radius, d.h. dem Abstand zwischen dem Ursprung und dem betreffenden Punkt der Funktion.
\newline\newline
Der Befehl \boxed{\textbf{\texttt{plot(x1,x2,y1,y2)}}} versieht die Graphik mit zwei Ordinaten. Die linke \texttt{y}-Achse bezieht sich auf \texttt{y1} in Funktion von \texttt{x1} und die rechte \texttt{y}-Achse auf \texttt{y2} in Funktion von \texttt{x2}.
\subsection{Massstab}
Mit dem Befehl \boxed{\textbf{\texttt{axis([xmin xmax ymin ymax])}}} lassen sich die Grenzen der \texttt{x}- und der \texttt{y}-Achse neu definieren. Mit dem Befehl {\color{red}\texttt{axis auto}} setzt für die Achsen wieder dir ursprünglichen Werte ein. Mit dem Befehl {\color{red}\texttt{axis equal}} erhalten alle Achsen die gleiche Skalierung. Mit dem Befehl {\color{red}\texttt{axis ij}} wechselt das Vorzeichen der \texttt{y}-Achse. Die positive \texttt{y}-Achse zeigt nun nach unten. Der Befehl {\color{red}\texttt{axis xy}} macht {\color{red}\texttt{axis ij}} wieder rückgängig. Der Befehl {\color{red}\texttt{axis tight}} passt die Achsenlänge exakt dem Bild an. Der Befehl {\color{red}\texttt{axis off}} schaltet alle axis-Definitionen, die "tick marks" und den Hintergrund aus. Mit dem Befehl {\color{red}\texttt{axis on}} werden sie wieder aktiviert.
\newline\newline
Der Befehl \boxed{\textbf{\texttt{zoom on}}} aktiviert in der aktuellen Graphik die Zoom-Funktion, er kann direkt im Command Window eingegeben werden. Mit "clic and drag" wird dann im Bild ein beliebiger Ausschnitt näher herangebracht. Mit der linken Maustaste wird der Graphikausschnitt vergrössert und mit der rechten Maustaste verkleinert. Der Befehl {\color{red}\texttt{zoom(factor)}} zoomt die Achsen um den für "factor" gewählten Wert. Der Befehl {\color{red}\texttt{zoom out}} führt die Graphik in ihre default-Fenstergrösse zurück. Der Befehl {\color{red}\texttt{zoom xon}} bzw. {\color{red}\texttt{zoom yon}} aktiviert in re aktuellen Graphik die Zoom-Funktion nur für die betreffenden Achsen. Der Befehl {\color{red}\texttt{zoom(figurename, option)}} versieht die Graphik "figurename" mit einer Zoom-Funktion. Als Option kann eine der oben genannten Zoom-Funktionen gewählt werden.
\newline\newline
Der Befehl \boxed{\textbf{\texttt{grid on}}} versieht die aktuelle Graphik mit ienem Liniennetz. Mit dem Befehl {\color{red}\texttt{grid off}} werden die Linien wieder deaktiviert. Der Befehl {\color{red}\texttt{box on}} umrahmt das aktuelle Bild mit einem Rahmen aus dünnen, schwarzen LInien, der Befehl {\color{red}\texttt{box off}} entfernt ihn wieder.
\newline\newline
Der Befehl \boxed{\textbf{\texttt{hold on}}} fixiert das aktuelle Graphikfenster, so dass weitere Funktionen im bereits bestehenden Graphikfenster positioniert werden können. Die ursprünglichen Achseinstellungen bleiben unverändert, selbst dann, wenn die neue Funktion nicht gut in den Rahmen passt. Der Befehl {\color{red}\texttt{hold off}} führt zum Normalbetrieb zurück, d.h. ein neuer plot-Befehl löscht die aktuelle Funktion und fügt die neue Funktion in das Fenster ein, falls nicht mit dem Befehl {\color{red}\texttt{figure}} ein weiteres Graphikfenster geöffnet wurde.
\newline\newline
Der Befehl \boxed{\textbf{\texttt{axes('position', [links unten Breite Höhe])}}} beschreibt im Graphikfenster die POsition der linken unteren Ecke des Bildes und seine Abmessung. Mit Werten zwischen 0 und 1 kann nun die Position und dir Grösse des Bildes festgelegt werden. Das default-Graphikfenster hat eine Breite und eine Höhe 1. Der Befehl {\color{red}\texttt{axex}} generiert in einem Graphikfenster ein Koordinatensystem, in das mit {\color{red}\texttt{plot}} ein beliebiges Bild hineingelegt werden kann. Über mehrere {\color{red}\texttt{axex}}-Definitionen können im Fenster mehrere Bilder erzeugt werden.
\subsection{Beschriften von Bilder} 
Der Befehl \boxed{\textbf{\texttt{legend('st1', 'st2', ...)}}} versieht im Fenster ein Bild mit einer Legende. Er erhält die einzelnen Strings "st1", "st2", usw. Für jedes Bild in einem Graphikfenster kann ein eigener Titel gewählt werden. Der zu schreibende Text wird in Anführungszeichen genommen. Der Befehl {\color{red}\texttt{legend off}} entfernt die Legende aus dem aktuellen Bild. Der Befehl {\color{red}\texttt{legend('st1', 'st2', ..., position)}} positioniert die Legende mit der Angabe einer Position an einen definierten Ort in der Graphik: 0-beste, 1-oben-rechts, 2-oben links, 3-unten-links, 4-unten-rechts, -1-rechts. Die Legende kann verschoben werden, indem die Legende mit der rechten Maustaste angeklickt und an den gewünschten Ort gezogen wird.
\newline\newline
Der Befehl \boxed{\textbf{\texttt{title('text')}}} fügt einen Titel oberhalb der Graphik hinzu. Ein Text kann hoch \texttt{("\^\,")} und tiefgestellte \texttt{"\_\,"}, kursiv \texttt{"it"} geschrieben oder griechische Zeichen enthalten. 
\begin{equation}
\boxed{A_1e^{-\alpha t}\sin \beta t\quad \texttt{'$\backslash$itA\_\{1\}e\^\,\{$\backslash$alpha$\backslash$itt\}sin$\backslash$beta$\backslash$itt'}}
\end{equation}
Der Befehl \boxed{\textbf{\texttt{xlabel('text')}}} schreibt die Zeile "text" unter die Abszisse. Der Befehl \texttt{ylabel} ist für die Beschriftung der Ordinate.
\newline\newline
Der Befehl \boxed{\textbf{\texttt{text(x,y,'text')}}} kann innerhalb des Bildrahmes eine beliebige Textzeile an der Stelle \texttt{(x,y)} angebracht werden. \texttt{x} und \texttt{y} sind in den Koordinaten der Achsen anzugehen. Dagegen verwendet der Befehl {\color{red}\texttt{text(x,y,'text,'sc')}} die Koordinaten des Graphikfensters, nämlich (0,0) in der unteren linken Ecke und (1,1) ub der oberen rechten Ecke. Geht ein Text über mehrere Zeilen, so kann er in eine Text-Variable geschrieben werden. Ein Text kann hoch- und tiefgestellte, kursiv geschriebene oder griechische Zeichen enthalten.
\newline\newline
Mit dem Befehl \boxed{\textbf{\texttt{gtext('text')}}} kann mit der Maus im Bild irgendein Text an beliebigen Ort eingefügt werden. Im Graphifenster erscheint der Mauspfeil als Fadenkreuz. An der gewünschten Stelle kann durch Betätigung einer Maustaste der Text eingefügt werden.
\subsection{Graphiken speichern oder drucken}
Der Befehl \boxed{\textbf{\texttt{print}}} sendet eine Kopie des aktuellen Graphikfensters an den Drucker. Der Befehl {\color{red}\texttt{print filename}} speichert eine Kopie des aktuellen Graphikfensters als PostScript-Datei im aktiven Directory unter dem Dateinamen "filename". Mit dem Befehl {\color{red}\texttt{print path}} kann der genaue Pfad angegeben werden, wo das aktuelle Graphikfenster als POst-Script-Datei abgelegt werden soll. Der Befehl {\color{red}\texttt{print[-ddevice][-options]$<$filenmae$>$}} speichert die aktuelle Graphik im Format des speziell gewählten Druckertreibers und de rzusätzlichen Option im aktiven Directory unter "filename" ab. Der Befehl {\color{red}\texttt{help print}} zeigt im Command-Window alle möglichen \texttt{[-ddevice]} und \texttt{[-options]}.
\newline\newline
Der Befehl \boxed{\textbf{\texttt{orient landscape}}} druckt bei print-Befehlen die Graphikfenster im Querformat. Mit dem Befehl {\color{red}\texttt{orient portrait}} wird das Graphikfenster im Hochformat grdruckt. Der Befehl {\color{red}\texttt{orient tall}} setzt das Blattformat auf Hochformat. Zusätzliche wird das Graphikfenster auf das ganze Papierblatt vergrössert bzw. verkleinert. Der Befehl {\color{red}\texttt{orient}} sagt im Command Window, welche Orientierung momentan aktiv ist.
\section{Dreidimensionale Graphik}
\subsection{Elementare dreidimensionale Graphik}
Der Befehl \boxed{\textbf{\texttt{plot3(x,y,z)}}} plottet im dreidimensionalen Raum die Graphen, die durch die Vektoren \texttt{x}, \texttt{y} und \texttt{z} gegeben sind. Die Vektoren müssen alle dieselbe Länge haben. Mit {\color{red}\texttt{plot3(X,Y,Z)}} zeichnet pro Kolonne einen Graphen. Die Matrizen müssen alle dieselbe Grösse haben. Bei {\color{red}\texttt{plot3(x,y,z,'style')}} können zusätzlich noch der Linientyp, die Plot Symbole und die Farbe des Graphen geändert werden. {\color{red}\texttt{help plot}} listet im Command Window eine Answahl von möglichen Liniendefinitionen auf.
\newline\newline
Bei \boxed{\textbf{\texttt{mesh(Z)}}} entsprechen die Werte der Matrix \texttt{Z} Element von \texttt{$\mathbb{R}^{n\times m}$} den \texttt{z}-Werten des Netzes. Für die \texttt{x}- und \texttt{y}-Werte verwendet \texttt{mesh} die Kolonnen- bzw. die Zeilennummer.
\newline\newline
Mit dem Befehl {\color{red}\texttt{mesh(X,Y,Z,C)}} zeichnet ein Netz un der Vogelperspektive mit \texttt{Z} als Funktion von \texttt{X} und \texttt{Y}. Es handelt sich hierbei um eine FUnktion mit zwei Variablen. \texttt{X}, \texttt{Y} und \texttt{Z} sind Matrizen mit den Werten für die \texttt{x}-, \texttt{y}- und \texttt{z}-Koordinaten. \texttt{X} und \texttt{Y} können aber auch Vektoren der Länge \texttt{m} und \texttt{n} sein. Jeder \texttt{z}-Koordinate aus der Matrix \texttt{Z} werden dann die entsprechenden WErte des \texttt{x}- und \texttt{y}-Vektors zugewiesen. \texttt{C} ist ebenfalls eine Matrix und beinhaltet die Farbskala für die Graphik. Ohne \texttt{C} wird \texttt{C=Z} gesetzt.
\newline\newline
Mit dem Befehl \boxed{\textbf{\texttt{fill(X,Y,Z,C)}}} wird in den Farben der Matrix \texttt{C} ein dreidimensionales Polygon geplottet. Sind \texttt{X}, \texttt{Y} und \texttt{Z} Vektoren, so wird die Fläche unterhalb des Graphen mit Farbe ausgefüllt. Ist \texttt{C} ein Skalar, so wird die Fläche monochrom. Folgende Farben sind möglich: {\color{red}\texttt{'r'}}, {\color{red}\texttt{'g'}}, {\color{red}\texttt{'b'}}, {\color{red}\texttt{'c'}}, {\color{red}\texttt{'m'}}, {\color{red}\texttt{'y'}}, {\color{red}\texttt{'w'}} und {\color{red}\texttt{'k'}}. Mit dem Vektor {\color{red}\texttt{[rot grün blau]}} kann eine neue Farbe gemischt werden. Die Werte von liegen zwischen 0 und 1. Je nach Anteil ergibt sich eine Farbkombination. Ist \texttt{C} ein Vektor, so hat er die gleiche Länge wie \texttt{X}, \texttt{Y} und \texttt{Z}. Wird für \texttt{C} einer der drei Vektoren gewählt, dann ist die Farbabstufung der momentan aktive \texttt{colormap} proportional zur betreffenden Koordinatenachse. 
\newline\newline
Sind \texttt{X}, \texttt{Y} und \texttt{Z} Matrizen, so zeichnet \boxed{\textbf{\texttt{fill3}}} pro Kolonne ein Polygon und füllt es mit der entsprechenden Farbe aus. Die Farbgebung bleibt gleich. Falls \texttt{C} ein Zeilenvektor ist, dann hat das Polygon die Schattierung \texttt{shading flat}. Für eine Matrix wird sie {\color{red}\texttt{shading interp}}. {\color{red}\texttt{shading}} shattiert die Objekt-Oberfläche, {\color{red}\texttt{shading flat}} berechnet für jede Teilfläche einer Oberfläche, die mit den Befehlen \texttt{surf}, \texttt{mesh}, \texttt{polar}, \texttt{fill} oder \texttt{fill3} gebildet wurden, die entsprechende Farbabstufung. {\color{red}\texttt{shading interp}} interpoliert über die Farbabstufung. {\color{red}\texttt{shading faceted}} entspricht dem {\color{red}\texttt{shading flat}}. Die 3D Graphik wird jecoh zusätzlich mit schwarzen Linien versehen. 
\subsection{Projektionsarten einer Graphik}
Mit \boxed{\textbf{\texttt{view(az,el)}}} kann in einem dreidimensionalen Plot der Blickwinkel beliebig eingestellt werden, bzw. die Graphik-Box ist um zwei Achsen drehbar. {\color{red}\texttt{az}} steht für azimuth und definiert die horizontale Rotation im Grad. Für einen positiven Winkel dreht sich die Graphik entgegen dem Uhrzeigersinn um die \texttt{z}-Achse. {\color{red}\texttt{el}} beschreibt die Anheben bzw. Senken der Gtraphik in Grad. Bei einem positiven Winkel befindet sich der Betrachter in der Vogelperspektive, bei negativem Winkel in der Froschperspektive. {\color{red}\texttt{view([x y z])}} setzt den Blickwinkel in kartesischen Koordinaten. {\color{red}\texttt{view(2)}} stellt für die 2D Ansicht den vordefinierten Blickwinkel {\color{red}\texttt{view(0, 90)}} ein. {\color{red}\texttt{view(3)}} stellt für die 3D Ansicht den vordefinierten Blickwinkel {\color{red}\texttt{view(-37.5,30)}} ein. {\color{red}\texttt{T=view}} speichert diew \texttt{view} der aktuiellen Graphik in der Variable \texttt{T} als \texttt{4x4}-Matrix. {\color{red}\texttt{view(T)}} weist einer aktuellen Graphik die in der Variable \texttt{T} gespeicherte view zu.
\newline\newline
\boxed{\textbf{\texttt{T=viewmtx(az,el)}}} weist wie bei {\color{red}\texttt{T=view(az, el)}} der Variable \texttt{T} die 4x4-Transformationsmatrix zu. Die Ansicht der aktuellen Graphik wird dabei nicht verändert. Mit {\color{red}\texttt{T=viewmtx(az, el, phi)}} wird die Graphik durch ein Objektiv betrachtet. Der Linsenwinkel wird in Grad angegeben. \texttt{phi=0} Grad definiert die orthogonale Projektion. 10 Grad entspricht einem Teleobjektiv, 25 Grad einem Normalobjektiv und 60 Grad einem Weitwinkelobjektiv. Bei {\color{red}\texttt{T=viewmtx(az, el, phi, tp)}} wird mit \texttt{tp=[xp, yp, zp]} einen Fluchtpunkt gesetzt.
\newline\newline
\boxed{\textbf{\texttt{rotate3d on}}} aktiviert in der aktuellen Graphik die Maus gesteuerte 3D-Rotation. Die Graphik-Ansicht kann damit beliebig verändert werden. Mit {\color{red}\texttt{rotate3d off}} wird sie wieder deaktiviert. 
\subsection{Dreidimensionale Graphik beschriften}
\boxed{\textbf{\texttt{zlabel('text')}}} versieht die \texttt{z}-Achse mit der Aufschrift "text". Mit dem Befehl {\color{red}\texttt{colorbar('vert')}} erscheint in der aktuellen Graphik eine vertikale Farbskala. In einem 3D-Plot bezieht sie sich auf die Werte der \texttt{z}-Achse. {\color{red}\texttt{colorbar('horiz')}} zeichnet eine vertikale Farbskala. Im 3D-Plot ist die Farbgebung ebenfalls auf die \texttt{z}-Achse abgestimmt. \boxed{\textbf{\texttt{colorbar}}} alleine fügt der Graphik entweder eine vertikale Farbskala hinzu, oder die bestehende Farbskala wird aktualisiert.
\section{Spezielle Graphen}
\boxed{\textbf{\texttt{fill(x,y,c)}}} füllt diejenige Fläche mit der Farbe \texttt{c} aus, welche von der Geraden, die den Endpunkt von \texttt{f(x)} mit dem Anfang verbindet, und der Funktion \texttt{y=f(x)} selbst umgeben wird. Ist \texttt{c} ein Vektor derselben Länge wie \texttt{x} und \texttt{y}, dann verwendet MATLAB entweder die im Vektor \texttt{c} definierten Farben, oder für \texttt{c=x} bzw. \texttt{c=y} die Farbpalette der aktuellen \texttt{colormap}. Sind in \boxed{\textbf{\texttt{fill(X,Y,C)}}} \texttt{X} und \texttt{Y} Matrizen derselben Grösse, so wird pro Kolonne ein Polygon gezeichnet. \texttt{C} kann ein Vektor aber auch eine Matrix sein. Beim Vektor ist die Schattierung der Fläche {\color{red}\texttt{shading flat}}, bei der Matrix {\color{red}\texttt{shading interp}}.
\newline\newline
Mit dem Befehl \boxed{\textbf{\texttt{fplot('f',lim)}}} zeichnet eine beliebige Funktion \texttt{f=f(x)} im Bereich \texttt{lim=[x$_{\texttt{min}}$ x$_{\texttt{max}}$]}. Mit \texttt{lim=[x$_{\texttt{min}}$ x$_{\texttt{max}}$ y$_{\texttt{min}}$ y$_{\texttt{max}}$]} werden zusätzliche Schranken gesetzt. Mit dem Befehl {\color{red}\texttt{fplot('f', lim, tol)}} mit \texttt{tol$<$1} definiert die Toleranz des relativen Fehlers. Die voreingestellte Toleranz ist 2e-3 bzw. 0.2\%. Mit dem Befehl {\color{red}\texttt{fplot('f', lim, N)}} berechnet zwischen \texttt{x$_{\texttt{min}}$} für die Funktion \texttt{f=f(x)} \texttt{N+1} Punkte. Mit dem Befehl {\color{red}\texttt{fplot('f', lim, 'LineSpec')}} definiert mit LineSpec den Linien-Typ der Funktion. Alle möglichen "line specifications" werden mit \texttt{help plot} aufgelistet.    
\newline\newline
Mit dem Befehl \boxed{\textbf{\texttt{hist(x)}}} zeichnet MATLAB ein Histogramm mit den in \texttt{x} gespeicherten Daten. Es ist in 10 gleichmässig verteilte Intervalle unterteilt. Pro Intervall gibt es die Anzahl Elemente an, die es enthält. Wenn \texttt{x} eine Matrix ist, plottet \texttt{hist} pro Kolonne ein Histogramm. Mit dem Befehl {\color{red}\texttt{hist(x,n)}} definiert man zusätzlich die Anzahl \texttt{n} Intervalle. Mit dem Befehl {\color{red}\texttt{hist(x,y)}} berechnet MATLAB die Verteilung von \texttt{x} bezüglich \texttt{y}. \texttt{y} ist ein Vektor, deren Elemente in aufsteigender Ordnung aufgelistet sind. Jedes einzelne Element von \texttt{y} entspricht einem Zentrum.     
\newline\newline
Mit dem Befehl \boxed{\textbf{\texttt{pie(x)}}} stellt die Daten aus dem Vektor \texttt{x} in einem Kuchendiagramm dar. Die Elemente von \texttt{x} werden mit der Summe der \texttt{x}-Werte dividiert. Damit ist die Grösse von jedem einzelnen Kuchenstück in \% gegeben. Mit dem Befehl {\color{red}\texttt{pie(x,explode)}} zieht mit "explode" die gewünschte Stücke aus dem Kuchen, \texttt{explode} ist ein Vektor derselben Länge wie \texttt{x}. Seine Elemente haben entweder den Betrag 0, d.h. das entsprechende Stück von \texttt{x} verbleibt im Kuchen, ode rden Betrag 1, d.h. das dazugehörende Stück von \texttt{x} wird aus dem Kuchen herausgezogen.
\newline\newline
Mit dem Befehl \boxed{\textbf{\texttt{stem(y)}}} zeichnet eine Verteilung der Daten aus dem Vektor \texttt{y}. Jeder Wert aus \texttt{y} im Plot mit einem Kreis und einer Linie versehen. Auf der Abszisse wird jedes Element aus \texttt{x} fortlaufend eingereiht. Auf der Ordinate kann sein Wert abgelesen werden. Der entsprechende Wert ist mit ienem Kreis gekennzeichnet. Bei {\color{red}\texttt{stem(x,y)}} entsprechen die Elemente aus dem \texttt{x}-Vektor den \texttt{x}-Werten und diejenigen aus dem \texttt{y}-Vektor den \texttt{y}-Werten. Mit dem Befehl {\color{red}\texttt{stem(..., 'filled')}} malt den Kreis mit der entsprechenden Farbe aus. Mit dem Befehl {\color{red}\texttt{stem(..., 'linespec')}} kann der Linien-Typ gewählt werden. Alle möglichen "line specifications" werden mit {\color{red}\texttt{help plot}} aufgelistet.    
\newline\newline
Mit dem Befehl \boxed{\textbf{\texttt{contour(Z)}}} zeichnet einen Konturplot mit den WErten aus der Matrix \texttt{Z}. Die Elemente der Matrix \texttt{Z} entsprechen den Werten auf der \texttt{z}-Achse. Die Kolonnenzahlen ergeben die Koordinaten auf der \texttt{x}-Achse und die Zeilenzahlen die WErte auf der \texttt{y}-Achse. Die Höhen für die einzelnen Höhenlinien werden automatisch ausgewählt. Mit {\color{red}\texttt{contour(x,y,z)}} werden die \texttt{x}- und \texttt{y}-Koordinaten explizit mitgeliefert. {\color{red}\texttt{contour(z,N)}} und {\color{red}\texttt{contour(x,y,z,N)}} verwendet für den Konturplot \texttt{N} Höhenlinien. {\color{red}\texttt{contour(Z,v)}} und {\color{red}\texttt{contour(x,y,Z,v)}} zeichnet all jene Höhenlinien, deren Höhen im Vektor \texttt{v} angegeben werden. Mit {\color{red}\texttt{contour(Z,[v v])}} plottet MATLAB eine einzige Höhenlinie bei der Höhe \texttt{v}. Mit {\color{red}\texttt{contour(...,'linespec')}} kann der Linien-Typ gewählt werden. Alle möglichen "line specifications" werden mit {\color{red}\texttt{help plot}} aufgelistet...  

\end{document}
