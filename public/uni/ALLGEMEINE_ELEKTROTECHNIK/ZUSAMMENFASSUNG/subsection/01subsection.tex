\section{Physikalische Grössen}
\subsection{Zahlwert und Einheit}
Eine physikalische Grösse beinhaltet eine Zahl und eine Einheit. Beinhaltet eine physikalische Grösse einen grossen Zahlenwert, so kann man diese als ein Vielfaches dieser Einheit ausdrücken. Die Dimension gibt Auskunft auf eine detaillierte Charakterisierung der Grösse wie die Höhe, der Abstand der die Strecke mit Einheit Meter.
\subsection{Grundeinheiten und abgeleitete Einheiten}
Die Grundeinheiten besteht aus sieben Grundeinheiten: \textbf{Länge} (Meter $\text{m}$), \textbf{Masse} (Kilogramm $\text{kg}$), Zeit (Sekunde $\text{s}$), \textbf{Stromstärke} (Ampere $\text{A}$), \textbf{Temperatur} (Kelvin $\text{K}$), \textbf{Stoffmenge} (Mol $\text{mol}$) und \textbf{Lichtstärke} (Candela $\text{cd}$).
\newline\newline
Die abgeleitete Einheiten entstehen durch Beziehungen zwischen der Grundeinheiten wie die Geschwindigkeit oder die Beschleunigung.
\newline\newline
Werden Gleichungen mit den Grundeinheiten gerechnet, so wird das Resultat auch in einer Grundeinheit ausgedrückt. Das Ziel besteht auch darin, Resultate mit Hilfe von Zehnerpotenzen zu schreiben.
\begin{table}[H]
\centering
\begin{tabular}{llll}
\hline
Grösse&Zeichen&Name&Symbol\\\hline
Länge&$l$&Meter&$\text{m}$\\
Masse&$m$&Kilogramm&$\text{kg}$\\
Zeit&$t$&Sekunde&$\text{s}$\\
Stromstärke&$I, i$&Ampere&$\text{A}$\\\hline
\end{tabular}
\caption{Grundeinheiten}
\end{table}
\begin{table}[H]
\centering
\begin{tabular}{lllll}
\hline
Grösse&Zeichen&Name&Symbol&Ausdruck\\\hline
Kraft&$F$&Newton&$\text{N}$&$\text{N}=\text{m}\,\text{kg}\, \text{s}^{-2}$\\
Leistung&$P$&Watt&$\text{W}$&$\text{W}=\text{V}\,\text{A}=\text{N}\,\text{m}\,\text{s}^{-1}$\\
Arbeit, Energie&$W$&Joule&$\text{J}$&$\text{J}=\text{W}\,\text{s}=\text{N}\,\text{m}$\\
Spannung&$U, u$&Volt&$\text{V}$&$\text{V}=\text{W}\,\text{A}^{-1}$\\
Widerstand&$R$&Ohm&$\Omega$&$\Omega=\text{V}\,\text{A}^{-1}$\\
Spezifischer Widerstand&$\rho$&&$\Omega\,\text{m}$&$\Omega\,\text{m}=\text{V}\,\text{m}\,\text{A}^{-1}$\\
Leitwert&$G$&Siemens&$\text{S}$&$\text{S}=\Omega^{-1}$\\
Spezifische Leitfähigkeit&$\sigma$&&$\text{S}\,\text{m}^{-1}$&$\text{S}\,\text{m}^{-1}=\Omega^{-1}\,\text{m}^{-1}$\\
Ladung&$Q$&Coulomb&$\text{C}$&$\text{C}=\text{A}\,\text{s}$\\
Elektr. Verschiebungsdichte&$D$&&$\text{A}\,\text{s}\,\text{m}^2$&$\text{C}\,\text{m}^{-2}=\text{A}\,\text{s}\,\text{m}^{-2}$\\
Elektr. Feldstärke&$E$&&$\text{V}\,\text{m}^{-1}$&\\
Kapazität&$C$&Farad&F&$\text{F}=\text{A}\,\text{s}\,\text{V}^{-1}=\text{s}\,\Omega^{-1}$\\
Induktionsfluss&$\phi$&Weber&$\text{Wb}$&$\text{Wb}=\text{V}\,\text{s}$\\
Magn. Induktionsdichte&$B$&Tesla&$\text{T}$&$\text{T}=\text{V}\,\text{s}\,\text{m}^{-2}$\\
Magn. Feldstärke&$H$&&$\text{A}\,\text{m}^{-1}$&\\
Induktivität&$L$&Henry&$\text{H}$&$\text{H}=\text{V}\,\text{s}\,\text{A}^{-1}=\Omega\,\text{s}$\\\hline
\end{tabular}
\caption{Abgeleitete Einheiten}
\end{table}
\begin{table}[H]
\centering
\begin{tabular}{ll}
\hline
Definition&Wert\\\hline
Lichtgeschwindigkeit im Vakuum&$c_0=299'792'458\,\text{m}\,\text{s}^{-1}$\\
Elementarladung&$e=1.6022\cdot 10^{-19}\,\text{C}$\\
Ruhemasse Elektron&$m_0=9.1096\cdot 10^{-31}\,\text{kg}$\\
Ruhemasse Proton&$m_p=1.6726\cdot 10^{-27}\,\text{kg}$\\
Permeabilität im Vakuum&$\mu_0=4\pi\cdot 10^{-7}\,\text{V}\,\text{s}\,\text{A}^{-1}\,\text{m}^{-1}$\\
Permittivität im Vakuum&$\epsilon_0=c_0^{-2}\mu_0^{-1}=8.85419\cdot 10^{-12}\,\text{A}\,\text{s}\,\text{V}^{-1}\,\text{m}^{-1}$\\
Wellenimpendanz des freien Raumes&$\nu_0=\sqrt{\mu_0/\epsilon_0}=376.73\,\Omega$\\\hline
\end{tabular}
\caption{Universelle Konstanten}
\end{table}
\subsection{Skalare und vektorielle Grössen}
Zahlenwerte mit Einheiten bezeichnet man als skalare Grössen. Zahlenwerte mit Enheiten, die in einer bestimmten Richtung des Raumes wirken bezeichnet man als vektorielle Grössen und haben einen Vektorpfeil über das Symbol und ihr Betrag ist die Länge des Vektors.
\section{Elektrizität und ihre Wirkungen}
\subsection{Elektrische Ladung und elektrischer Strom}
Die elektrische Ladung $Q$ ist eine physikalische Grösse und benötigt einen Träger. Der Raum befindet sich in einem elektrischen Feld. Körper, die eine elektrische Ladung tragen, üben eine Kraftwirkung aufeinander aus.
\newline\newline
Elektrische Ladungen könenn sich dabei anziehen oder abstossen. Man unterscheidet zwischen positiver und negativer Ladung. Gleichnamige Ladungen stossen sich ab, während ungleichnamige Ladungen ziehen sich an. Elektrizität entsteht durch Trennen von Ladungen verschiedenen Vorzeichens.
\newline\newline
Im elektrischen neutralen Zustand heben sich die Wirkungen positiver und negativer Ladungen gegenseitig auf. Elektrische Ladungen lassen sich durch Berührung übertragen. Die Elektrizität besteht aus Elektronen.
\newline\newline
Elektrizitätsträger können sich je nach Material übertragen. Ladungsträger bewegen sich in einem Leitungsstrom bzw. elektrischen Strom. Dieser elektrischen Strom ist das Verhältnis der elektrischen Ladung pro Zeit.
\begin{equation}
\boxed{I=\dfrac{Q}{t}}
\end{equation}
Bewegte Ladungen haben thermische (Leiter erwärmt sich bis zum Schmelztemperatur), magnetische (Kräfte entstehen durch Magnete auf Leiter) und chemische Wirkungen (Durch Stromvorgang werden Stoffe verändert).
\subsection{Aufbau der Materie}
Moleküle können in Atome aufgeteilt werden. Um den positiven Atomkern kreisen die negativ geladenen Elektronen. Diese Elektronen bilden die Elektronenhülle, welche Elektronen durch ihre Energie zu Gruppen (Elektronenschalen) aufgeteilt werden. Ein Elektron kann sich nur auf eine Quantenbahn aufhalten. Ein Atom ist elektrisch neutral. Die Elektronen in der äusseren Schale sind Valenzelektronen und bestimmen das chemische Verhalten des Atoms.
\newline\newline
Das Atomkern besteht aus Protonen und Neutronen bzw. aus Nukleonen. Neutronen sind für die Kernspaltung von grösser Bedeutung. Die Beeinflussung der Elektronenhülle erfolgt durch Ionisierungsenergie und bildet Ionen durch Elektronenabgabe oder -zugabe. Ein Ion ist ein elektrisch geladenes Atom.
\subsection{Leiter und Nichtteiler}
Materialien werden in Leiter und Nichtteiler unterteilt. Zu den Leiter zählen die Metalle, aber verändern nicht das Material durch Stromdurchgang. Säuren, Basen und Salzlösungen werden beim Stromdurchgang verändert.
\newline\newline
Zu den Nichtleiter zählen Gummi, Seide, Kunststoffe, Porzellan, Glas, Glimmer, usw. In diesen Stoffen stehen nahezu keine Elektronen zur Verfügung.



