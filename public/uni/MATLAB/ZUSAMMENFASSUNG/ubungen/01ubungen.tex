\textbf{Drehbuch}
\begin{enumerate}[$\bullet$]
\item Administration Unterricht, Leistungsbewertung, Unterlagen, Homepage.
\item Matlab und Toolboxen: eine Einführung.
\item Erster Start von MATLAB, Fensterorganisation und Erklärungen.
\item Gastreferenten aus der Praxis zeigen den Umgang mit Matlab
\end{enumerate}
\textbf{Lernziele}
Die Studierenden...
\begin{enumerate}[$\bullet$]
\item haben Klarheit über die Unterrichtsorganisation und die Leistungsbeurteilung.
\item lernen Matlab kennen mit den wichtigsten Fenstern.
\item kennen die wichtigsten Toolboxen für ihr Studium mit einem Beispiel.
\item und wissen, wie man überprüfen kann, welche installiert sind
\end{enumerate}
\section{Getting started in Matlab}
\begin{enumerate}
\item Sie definieren eine Variable \texttt{a} mit dem Wert 5 $\Longrightarrow$ {\color{magenta}\texttt{a=5}}
\item Sie definieren einen Vektor mit 100 Werten $\Longrightarrow$ {\color{magenta}\texttt{linspace(0,1)}}
\item Sie löschen den Bildschirm $\Longrightarrow {\color{magenta}\texttt{clc}}$
\end{enumerate}
\section{Die Matlab Homepage}
Matlab besitzt eine gute Homepage. Sie finden diese unter folgendem Link:
%\hyperlink{https://ch.mathworks.com/de/}{https://ch.mathworks.com/de/}
Unter der Mathworks Community finden Sie viele Antworten auf Ihre Probleme.
\\\\
\textbf{Aufgabe:}
Suchen Sie in der Community Hilfe für den Befehl {\color{red}\texttt{rotate surface}}.
Wählen Sie das erste Beispiel aus.
Dieses Beispiel kann direkt im Webbrowser (ohne Benutzung von Matlab) ausgeführt werden.
\\\\
Probieren Sie dies aus:
$\Rightarrow$ Try this example $\Rightarrow$ Try it in your browser
ändern Sie den Code auf
{\color{red}\texttt{surf(peaks(100),'Parent',t)}}
ab und führen Sie diesen aus.
Waren Sie erfolgreich? Nein
\section{Was ist eigentlich Simulink?}
Simulink basiert auf Matlab und ist eine zusätzliche Umgebung. Während Matlab ein textbasiertes Programm ist, ist Simulink grafisch orientiert. Man kann komplexe Gleichungen einfach grafisch darstellen.
\begin{enumerate}
\item Die Programmierung in Matlab erfolgt: {\color{magenta}mittels Text}.
\item Die Programmierung in Simulink ist: {\color{magenta}grafisch aufgebaut}.
\item Simulationen in Simulink: {\color{magenta}sind immer in Funktion der Zeit}.
\item Komplexe differentialgleichungen können: {\color{magenta}einfach grafisch dargestellt werden}.
\item Programmblöcke werden mittels: {\color{magenta}grafischer Linien miteinander verbunden}.
\end{enumerate}
\section{Toolboxen in Matlab und Simulink}
\begin{enumerate}
\item Eigentlich ist Matlab ein numerisches Programm. Die {\color{magenta}Symbolic Math Toolbox} bildet eine Ausnahme. Diese Toolbox wurde vor einigen Jahren von einer externen Firma eingekauft und integriert.
\item {\color{magenta}Simscape} erlaubt die Simulation von physikalischen Systemen.
\item Die {\color{magenta}Signal Processing Toolbox} ist ein Zusatz in Matlab und erlaubt die Verarbeitung von z.B. Audiosignalen.
\item Die {\color{magenta}Control System Toolbox} war eine der ersten Toolboxen in Matlab. Sie ist extrem mächtig. Sie wird häufig in der Regelungstechnik verwendet und beihaltet Auslegungen von z.B. einem PID-Regler.
\item {\color{magenta}Matlab} kennen Sie bereits: dies ist ein leistungsstarker texbasierter Editor.
\item {\color{magenta}Simulink} ist dagegen grafisch orientiert.
\end{enumerate}
\section{Model Based Design}
Ein Design wird folgendermassen umgesetzt:
\begin{enumerate}
\item {\color{magenta}Aufbau des Modells:}
Aufbau des Modells (Physik) vor allem in Simulink z.T. auch in Matlab (embedded Functions)
Regelungen werden direkt in Simulink gezeichnet
\item {\color{magenta}Simulation des Modells:}
Das System wird in Simulink simuliert.
Für die Physik wird häufig Simscape verwendet (z.B. Umrichter, Motoren, Kondensatoren, ...)
\item {\color{magenta}Generation von C-Code}
Falls die Simulation erfolgreich ist, wird direkt aus dem Simulink Modell der C-Code compiliert und auf die Target-hardware heruntergeladen.
Dies geht übrigens auch für kleine Prozessoren und DSPs.
\item {\color{magenta}Optional: Simulation des Systems mit einem RTS (real time simulator)}
Es gibt Leistungsrechner  (dSpace). Mit denen kann die programmierte Hardware in Echtzeit getestet werden.
Auf diesen Rechnern wird die ``Physik'' implementiert... und auch diese Rechner werden in Simulink programmiert.
Dies ist ``State of the art'' für die Automobilindustrie, Flugzeugindustrie und Traction.
\item {\color{magenta}Testen und watchen}
Die Software wird auf der Target-hardware und im Feld getestet.
\end{enumerate}







