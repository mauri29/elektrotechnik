\section{Funktionen von mehreren Variablen und ihre Darstellung}
\subsection{Definition einer Funktion von mehreren Variablen}
Unter einer Funktion von zwei unabhängigen Variablen versteht man eine Vorschrift, die jedem geordneten Zahlenpaar $\left(x;y\right)$ aus einer Menge $D$ genau ein Element $z$ aus einer Menge $W$ zuordnet. Symbolische Schreibweise: $z=f\left(x;y\right)$.
\newline\newline
Seien $x,y$ die unabhängige Variable, $z$ die abhängige Variable, $D$ der Definitionsbereich und $W$ der Wertebereich der Funktion. 
\newline\newline
Eine Funktion von $n$ unabhängige Variablen lautet
\begin{equation}
\boxed{y=f\left(x_1; x_2;\dotso; x_n\right)}
\end{equation}
\subsection{Darstellungsformen einer Funktion von zwei Variablen}
\subsubsection{Analytische Darstellung}
Die Funktion wird explizit durch eine Funktionsgleichung dargestellt.
\begin{equation}
\boxed{z=f\left(x;y\right)}
\end{equation}
Die Funktion wird implizit durch eine Funktionsgleichung dargestellt.
\begin{equation}
\boxed{F\left(x;y;z\right)=0}
\end{equation}
\subsubsection{Graphische Darstellung}
Die Variablen $x$, $y$ und $z$ einer Funktion $z=f\left(x;y\right)$ werden als rechtwinklige oder kartesische Koordinaten eines Raumpunktes $P$ gedeutet: $P\left(x;y;z\right)$. Der Funktionswert $z=f\left(x;y\right)$ ist dabei die Höhenkoordinate der zugeordneten Bildpunktes. Man erhält als Bild der Funktion eine über dem Definitionsbereich liegende Fläche.
\newline\newline
Die Schnittkurvendiagramme einer Funktion $z=f\left(x;y\right)$ erhält man durch Schnitte der zugehörigen Bildfläche mit Ebenen, die parallel zu einer der drei koordinatenebenen verlaufen. Die Schnittkurven werden noch in die jeweilige Koordinatenebene projiziert und repräsentieren einparametrige Kurvenscharen. Ihre Gleichungen erhält man aus der Funktionsgleichung $z=f\left(x;y\right)$, indem man der Reihe nach jeweils eine der drei Variablen als Parameter betrachtet. In den naturwissenschaftlich-technischer Anwendungen werden die Schnittkurvendiagramme als Kennlinienfelder bezeichnet.
\newline\newline
Das Höhenliniendiagramm ist ein spezielles Schnittkurvendiagramm mit der Höhenkoordinate $z$ als Kurvenparameter
\begin{equation}
\boxed{f\left(x;y\right)=c}
\end{equation}
\subsection{Ebenen im Raum}
\subsubsection{Ebenen}
Die Bildfläche einer linearen Funktion ist eine Ebene
\begin{equation}
\boxed{ax+by+cz+d=0}
\end{equation}
\subsubsection{Koordinatenebenen}
Die $xy$-Ebene ist $z=0$. Die $xz$-Ebene ist $y=0$. Die $yz$-Ebene ist $x=0$.
\subsubsection{Parallelebenen}
Eine Ebene parallel zur $xy$-Ebene lautet: $z=a$. Eine Ebene parallel zur $xz$-Ebene lautet: $y=a$. Eine Ebene parallel zur $yz$-Ebene lautet: $x=a$
\subsection{Rotationsflächen}
\subsubsection{Gleichung einer Rotationsfläche}
Eine Rotationsfläche entsteht durch Drehung einer ebenen Kurve $z=f\left(x\right)$ um die $z$-Achse für $a\leq r\leq b$. Diese Rotationsfläche lautet in zylindrische und kartesische  Koordinaten
\begin{equation}
\boxed{z=f\left(r\right)=f\left(\sqrt{x^2+y^2}\right)}
\end{equation}
\subsubsection{Spezielle Rotationsfläche}
\textbf{Kugel:} Obere bzw. untere Halbkugel in kartesische  bzw. Zylinderkoordinaten $z=\pm\sqrt{R^2-x^2-y^2}$ bzw. $z=\pm\sqrt{R^2-r^2}$
\begin{equation}
\boxed{x^2+y^2+z^2=R^2}\quad \boxed{r^2+z^2=R^2}
\end{equation}
\textbf{Kreiskegel:} Folgende Gleichungen beschreiben einen Doppelkegel. Für $z\geq 0$ erhält man den Mantel des gezeichneten Kegels mit der Funktionsgleichung $z=\dfrac{H}{R}\sqrt{x^2+y^2}$
\begin{equation}
\boxed{x^2+y^2=\dfrac{R^2}{H^2}z^2}\quad \boxed{z=\dfrac{H}{R}r}
\end{equation}
\textbf{Kreiszylinder:} Sei $z\in \mathbb{R}$ die Höhenkoordinate
\begin{equation}
\boxed{x^2+y^2=R^2}\quad \boxed{r=R}
\end{equation}
\textbf{Ellipsoid:} Durch Auflösen nach $z$ erhält man zwei Funktionen.
\begin{equation}
\boxed{\dfrac{x^2}{a^2}+\dfrac{y^2}{b^2}+\dfrac{z^2}{c^2}=1}
\end{equation}
\subsection{Schnittkurvendiagramme}
Die Struktur einer Funktion $z=f\left(x; y\right)$ kann durch Schnittkurven- pder Schnittliniendiagramme durch ebene Schnitte der zugehörigen Bildfläche dargestellt. Schnittebenen werden parallel zu einer der drei Koordinatenebenen gewählt. Das Schnittliniendiagramm ist das Höhenliniendiagramm, bei dem alle auf der Fläche $z=f\left(x; y\right)$ gelegenen Punkte gleicher Höhe $z=c$ zu einer Flächekurve zusammengefasst werden.
\newline \newline
Diese Kurve lässt sich auch als Schnitt der Fläche $z=f\left(x; y\right)$ mit der zur $x,y$-Ebene parallelen Ebene $z=c$ auffassen. Die Höhenlinien einer Funktion genügen folgender impliziten Kurvengleichung
\begin{equation}
\boxed{z=f\left(x; y\right)=\text{const.}=c}
\end{equation}
Die Höhenlinien repräsentieren eine einparametrige Kurvenschar mit der Höhenkoordinate $z=c$ als Parameter. Zu jedem zulässigen Parameterwert gehört dabei genau eine Höhenlinie. Die Höhenlinie sind die Projektionen der Linien gleicher Höhe in die $x$, $y$-Koordinatenebene. 
\newline\newline
Folgene Schnittkurvendiagramme der Funktion $z=f\left(x; y\right)$ ergeben sich durch Schnitte der zugehörigen Bildfläche mit Ebenen parallel zu einer der drei Koordinatenebenen:
\begin{itemize}
\item Schnitte parallel zur $x$, $y$-Ebene: $z=f\left(x; y\right)=\text{const.}=c$
\item Schnitte parallel zur $y$, $z$-Ebene: $z=f\left(x=c; y\right)$
\item Schnitte parallel zur $x$, $z$-Ebene: $z=f\left(x; y=c\right)$
\end{itemize}
Die Schnittkurvendiagramme repräsentieren somit einparametrige Kurvenscharen. IHre Gleichungen erhält man aus der Funktionsgleichung $z=\left(x; y\right)$, indem man der Reihe nach eine der drei Variablen festhält, d.h. als Parameter betrachtet. Das Höhenliniendiagramm ist ein spezielles Schnittkurvendiagramm mit der Höhenkoordinate $z$ als Kurvenparameter. In den physikalisch-technisch Anwendungen wird das Schnittliniendiagramm einer Funktion meist als \textbf{Kennlinienfeld} bezeichnet.
\section{Grenzwert und Stetigkeit einer Funktion}
Die Begriffe Grenzwert und Stetigkeit einer Funktion lassen sich sinngemäss auch für Funktionen von mehreren Variablen übertragen. Mit dem Grenzwert einer Funktion $z=f\left(x; y\right)$ an der Stelle $\left(x_0; y_0\right)$ lässt sich das Verhalten der Funktion untersuchen, wenn man sich dieser Stelle beliebig nähert. Wr gehen dabei davon aus, dass die Funktion in einer gewissen Umgebung von $\left(x_0; y_0\right)$ definiert ist, eventuell mit Ausnahme dieser Stelle selbst. Diese Funktion hat dann definitionsgemäss an der Stelle $\left(x_0; y_0\right)$ den \textbf{Grenzwert} $g$, wenn sich die Funktionswerte $f\left(x; y\right)$ beim Grenzübergang dem Wert $g$ beliebig nähern.
\begin{equation}
\boxed{\displaystyle \lim_{\left(x; y\right)\rightarrow \left(x_0; y_0\right)}f\left(x; y\right)=g}
\end{equation}
Aus $\left(x; y\right)\rightarrow \left(x_0; y_0\right)$ folgt stets $f\left(x; y\right)\rightarrow g$, und zwar unabhängig vom eingeschlagenen Weg für jede Folge von Zahlenpaaren $\left(x; y\right)$, die sich beliebig der Stelle $\left(x_0; y_0\right)$ nähern. Eine Funktion $f\left(x; y\right)$ kann auch in einer Definitionslücke $\left(x_0; y_0\right)$ einen Grenzwert haben, obwohl sie dort nicht definiert ist. Anschauliche Deutung des Grenzwertes auf der Bildfläche von $z=f\left(x; y\right)$: bewegt man sich auf dieser Fläche in Richtung der Stelle $\left(x_0; y_0\right)$, so unterscheidet sich die erreichte Höhe immer weniger vom Grenzwert $g$.
\newline\newline
Eine in $\left(x_0; y_0\right)$ und einer gewissen Umgebung von $\left(x_0; y_0\right)$ definierte Funktion $z=f\left(x; y\right)$ heisst an der Stelle $\left(x_0; y_0\right)$ \textbf{stetig}, wenn der Grenzwert der Funktion an dieser Stelle vorhanden ist und mit dem dortigen Funktionswert übereinstimmt.
\begin{equation}
\boxed{\displaystyle \lim_{\left(x; y\right)\rightarrow \left(x_0; y_0\right)}f\left(x; y\right)=f\left(x_0; y_0\right)}
\end{equation}
Die Stetigkeit an einer bestimmten Stelle setzt voraus, dass die Funktion dort auch definiert ist. Ferner muss der Grenzwert an dieser Stelle existieren und mit dem Funktionswert übereinstimmen. Eine Funktion $z=f\left(x; y\right)$ heisst dagegen an der Stelle $\left(x_0; y_0\right)$ \textbf{unstetig}, wenn $f\left(x_0; y_0\right)$ nicht vorhanden ist oder $f\left(x_0; y_0\right)$ vom Grenzwert verschieden ist oder dieser nicht existiert. Eine Funktion, die an jeder Stelle ihres Definitionsbereiches stetig ist, wird als stetige Funktion bezeichnet.  
\section{Partielle Differentiation}
\subsection{Partielle Ableitungen 1. Ordnung}
Die Ableitung einer Funktion von einer Variable an der Stelle $x_P$ wird durch den Grenzwert definiert und lässt sich als Steigung $m$ der im Punkt $P=\left(x_P; y_P\right)$ errichteten Kurventangente deuten
\begin{equation}
\boxed{\begin{array}{lll}
m_{\text{Tan},P}=\tan\left(\alpha\right)&=&\displaystyle \lim_{\triangle x\rightarrow 0}\dfrac{\triangle y}{\triangle x}=\displaystyle \lim_{x_Q\rightarrow x_P}\dfrac{f\left(x_Q\right)-f\left(x_P\right)}{x_Q-x_P}\\
&=&\displaystyle \lim_{\triangle x\rightarrow 0}\dfrac{f\left(x_P+\triangle x\right)-f\left(x_P\right)}{\triangle x}\\
&=&\dfrac{\text{d}}{\text{d}x}\Big[f\left(x_P\right)\Big]=f'\left(x_P\right)
\end{array}}
\end{equation}
Analoge Überlegungen führen bei einer Funktion von zwei Variablen, die sich bildlich als Fläche im Raum darstellen lässt, zum Begriff der partiellen Ableitung einer Funktion. Man geht von einem Punkt $P=\left(x_P; y_P; z_P\right)$ auf einer Fläche $z=f\left(x; y\right)$ aus. Durch diesen Flächenpunkt legt man zwei Schnittebenen, die parallel zur $x$, $z$- bzw. $y$, $z$-Koordinatenebene verlaufen. Als Schnittkurven erhält man dann zwei Flächenkurven $K_1$ und $K_2$, mit denen man sich jetzt näher befassen wird.
\newline\newline
Die auf der Schnittkurve $K_1$ gelegenen Punkte stimmen in ihrer $y$-Koordinate miteinander überein: $y=y_P$. Die Höhenkoordinate $z$ dieser Punkte hängt somit nur von der Variablen $x$, d.h. der $x$-Koordinate ab. Diese Funktionsgleichung der Schnittkurve $K_1$ lautet daher
\begin{equation} 
\boxed{K_1:\quad z=f\left(x; y_P\right)=g\left(x\right)}
\end{equation} 
Das Steigungsverhalten dieser Kurve lässt sich besser untersuchen, wenn man die Kurve in die $x$, $z$-Ebene projiziert. Dabei wird die Gestalt der Kurve in kleinster Weise verändert. Für die Steigung $m_x$, der in $P$ errichteten Kurventangente gilt dann definitionsgemäss
\begin{equation}
\boxed{m_x=\tan\left(\alpha\right)=g'\left(x_P\right)=\displaystyle \lim_{\triangle x\rightarrow 0}\dfrac{g\left(x_P+\triangle x\right)-g\left(x_P\right)}{\triangle x}}
\end{equation}
Beachtet man dabei noch, dass $g\left(x\right)=f\left(x; y_P\right)$ ist, so kann man diesen Grenzwert auch wie folgt schreiben
\begin{equation}
\boxed{m_x=\displaystyle \lim_{\triangle x\rightarrow 0}\dfrac{f\left(x_P+\triangle x\right)-f\left(x_P; y_P\right)}{\triangle x}}
\end{equation}
Die erste partielle Ableitung nach $x$ an der Stelle $\left(x_P, y_P\right)$ erfolgt durch Betrachtung der $y$-Variable als Parameter und wird durch das Symbol $f_x\left(x_P, y_P\right)$ oder $z_x\left(x_P, y_P\right)$ oder $\dfrac{\partial z}{\partial x}$ gekennzeichnet.
\newline\newline
Analog für die Schnittkurve $K_2$ gilt 
\begin{equation} 
\boxed{K_2:\quad z=f\left(x_P; y\right)=h\left(y\right)}
\end{equation} 
\begin{equation}
\boxed{m_y=\tan\left(\beta\right)=h'\left(y_P\right)=\displaystyle \lim_{\triangle y\rightarrow 0}\dfrac{h\left(y_P+\triangle y\right)-h\left(y_P\right)}{\triangle y}}
\end{equation}
\begin{equation}
\boxed{m_y=\displaystyle \lim_{\triangle y\rightarrow 0}\dfrac{f\left(x_P; y_P+\triangle y\right)-f\left(x_P; y_P\right)}{\triangle y}}
\end{equation}
Die erste partielle Ableitung nach $y$ an der Stelle $\left(x_P; y_P\right)$ erfolgt durch Betrachtung der $x$-Variable als Parameter und wird durch das Symbol $f_y\left(x_P; y_P\right)$ oder $z_y\left(x_P; y_P\right)$ oder $\dfrac{\partial z}{\partial y}$ gekennzeichnet.
\newline\newline
Unter der partiellen Ableitungen erster Ordnung einer Funktion $z=f\left(x; y\right)$ an der Stelle $\left(x; y\right)$ werden die folgenden Grenzwerte verstanden
\begin{equation}
\boxed{f_x\left(x; y\right)=\displaystyle \lim_{\triangle x\rightarrow 0}\dfrac{f\left(x+\triangle x; y\right)-f\left(x; y\right)}{\triangle x}}\quad \boxed{f_y\left(x; y\right)=\displaystyle \lim_{\triangle y\rightarrow 0}\dfrac{f\left(x; y+\triangle y\right)-f\left(x; y\right)}{\triangle y}}
\end{equation}
Die partielle Ableitung $f_x\left(x_P; y_P\right)$ bzw. $f_y\left(x_P; y_P\right)$ ist der \textbf{Anstieg der Flächentangente im Flächenpunkt} $P$ in der positiven $x$ bzw. $y$-Richtung.
\begin{equation}
\boxed{\dfrac{\partial}{\partial x}\Big[f\left(x; y\right)\Big]=f_x\left(x; y\right)}\quad \boxed{\dfrac{\partial}{\partial y}\Big[f\left(x; y\right)\Big]=f_y\left(x; y\right)}
\end{equation}
\subsection{Partielle Ableitungen höherer Ordnung}
Partielle Ableitungen höherer Ordnung lassen sich auch in Form partieller Differentialquotienten darstellen. So lautet beispielsweose die Schreibweise für partielle Differentialquotienten 2. Ordnung
\begin{equation}
\boxed{f_{xx}=\dfrac{\partial}{\partial x}\Big(\dfrac{\partial}{\partial x}\Big[f\left(x; y\right)\Big]\Big)=\dfrac{\partial^2}{\partial x}\Big[f\left(x; y\right)\Big]}\quad \boxed{f_{yy}=\dfrac{\partial}{\partial y}\Big(\dfrac{\partial}{\partial y}\Big[f\left(x; y\right)\Big]\Big)=\dfrac{\partial^2}{\partial y}\Big[f\left(x; y\right)\Big]}
\end{equation}
\begin{equation}
\boxed{f_{xy}=\dfrac{\partial}{\partial y}\Big(\dfrac{\partial}{\partial x}\Big[f\left(x; y\right)\Big]\Big)=\dfrac{\partial^2}{\partial x \partial y}\Big[f\left(x; y\right)\Big]}\quad \boxed{f_{yx}=\dfrac{\partial}{\partial x}\Big(\dfrac{\partial}{\partial y}\Big[f\left(x; y\right)\Big]\Big)=\dfrac{\partial^2}{\partial y\partial x}\Big[f\left(x; y\right)\Big]}
\end{equation}
Unter bestimmten Voraussetzungen, auf die man im Rahmen dieser Darstellung nur flüchtig eingehen kann, ist bei den gemischten partiellen Ableitungen die Reihenfolge der Differentiationen vertauschbar. Sind nämlich die partiellen Ableitungen $k$-ter Ordnung stetige Funktionen, so gilt der folgende \textbf{Satz von Schwarz}. 
\begin{equation}
\boxed{f_{xy}=f_{yx}}\quad \boxed{f_{xxy}=f_{yxx}=f_{xyx}}\quad \boxed{f_{yxx}=f_{xyy}=f_{yxy}}
\end{equation}
Bei einer gemischten partiellen Ableitung $k$-ter Ordnung darf die Reihenfolge der einzelnen Differentiationsschritte vertauscht werden, wenn die partiellen Ableitungen $k$-ter Ordnung stetige Funktionen sind.
\subsection{Differentiation nach einem Parameter - Verallgemeinerte Kettenregel}
Man betrachte Funktionen von zwei oder mehreren unabhängigen Variablen, die selbst noch von einem Parameter $t_1\leq t\leq t_2$ abhängen. Insbesondere interessiert man sich für die Ableitungen dieser Funktionen nach dem Parameter.
\begin{equation}
\boxed{z=f\Big(x\left(t\right), y\left(t\right)\Big)}
\end{equation}
Diese Funktion wird als eine zusammengesetzte, verkettete oder mittelbare Funktion dieses Parameters bezeichnet. Ihre Ableitung erhält man nach der folgenden, als \textbf{Kettenregel} bezeichnetes Vorschrift.
\begin{equation}
\boxed{\dfrac{\text{d}z}{\text{d}t}=\dfrac{\partial z}{\partial x}\cdot \dfrac{\text{d}x}{\text{d}t}+\dfrac{\partial z}{\partial y}\cdot \dfrac{\text{d}y}{\text{d}t}}
\end{equation}
\subsection{Das totale Differential einer Funktion}
Die Rolle, die die Kurventangente bei einer Funktion von einer Variablen spielt, übernimmt bei einer Funktion $z=f\left(x; y\right)$ von zwei Variablen die sogenannte \textbf{Tangentialebene}. Sie enthält sämtliche im Flächenpunkt $P=\left(x_P; y_P; z_P\right)$ an die Bildfläche von $z=f\left(x; y\right)$ angelegten Tangenten. In der unmittelbaren Umgebung ihres Berührungspunktes $P$ besitzen Fläche und Tangentialebene im Allgemeinen keinen weiteren gemeinsamen Punkt.
\newline\newline
Die Funktionsgleichung dieser Tangentialebene lautet
\begin{equation}
\boxed{z=ax+by+c}
\end{equation}
Die unbekannten Koeffizienten bestimmt man aus den bekannten Eigenschaften der Tangentialebene. So besitzen Fläche und Tangentialebene im Berührungspunkt $P$ den gleichen Anstieg. Dies bedeutet, dass dort die ersten partiellen Ableitungen übereinstimmen müssen. Die partiellen Ableitungen der linearen Funktion sind $f_x\left(x_P; y_P\right)=a$ und $f_y\left(x_P; y_P\right)=b$. Da $P$ ein gemeinsamer Punkt von Fläche und Tangentialebene ist, so erhält man durch Einsetzen der Koordinaten von $P$ in die Gleichung der Tangentialebene den Parameter $c=z_P-ax_P-by_P$.
\newline\newline
Die Gleichung der Tangentialebene an die Fläche $z=f\left(x; y\right)$ im Flächenpunkt $P$ mit $z_P=f\left(x_P; y_P\right)$ lautet in symmetrischer Schreibweise
\begin{equation}
\boxed{z=\underbrace{f_x\left(x_P; y_P\right)}_{a}\cdot x+\underbrace{f_y\left(x_P; y_P\right)}_{b}\cdot y+\underbrace{z_P-\underbrace{f_x\left(x_P; y_P\right)}_{a}\cdot x_P-\underbrace{f_y\left(x_P; y_P\right)}_{b}\cdot y_P}_{c}}
\end{equation}
Die \textbf{Gleichung der Tangentialebene} an die Fläche $z=f\left(x; y\right)$ im Flächenpunkt $P=\left(x_P; y_P; z_P\right)$ mit $z_P=f\left(x_P; y_P\right)$ lautet
\begin{equation}
\boxed{z=\dfrac{\partial}{\partial x}\Big[f\left(x_P; y_P\right)\Big]\cdot \left(x-x_P\right)+\dfrac{\partial}{\partial y}\Big[f\left(x_P; y_P\right)\Big]\cdot \left(y-y_P\right)+z_P}
\end{equation}
Durch das totale Differential kann man Probleme lösen wie die Linearisierung einer Funktion bzw. eines Kennlinienfeldes, implizite Differentiation und Fehlerfortpflanzung lösen. Dabei betrachtet man eine Funktion von zwei unabhängigen Variablen $z=f\left(x; y\right)$ aus auf der sich eine punktförmige Masse $P$ befindet. 
\newline\newline
Die Problemstellung lautet dann welche Änderung erfährt die Höhenkoordinate $z$ des Massenpunktes bei einer Verschiebung auf der Fläche selbst oder auf der zugehörigen Tangentialebene.
\subsubsection{Verschiebung des Massenpunktes auf der Fläche}
Bezogen auf den Ausgangspunkt $P$ bezeichnet man die Reihe nach mit $\triangle x$, $\triangle y$ und $\triangle z$. Die Masse wird nun so auf der Fläche verschoben, dass sich seine beiden unabhängigen Koordinaten $x$ und $y$ um $\triangle x$ bzw. $\triangle y$ ändern. Dabei ändert sich die Höhenkoordinate $z$, d.h. der Funktionswert um
\begin{equation} 
\boxed{\triangle z=f\left(x_P+\triangle x; y_P+\triangle y\right)-f\left(x_P; y_P\right)}
\end{equation} 
Diese Grösse beschreibt den Zuwachs der Höhenkoordinate und damit des Funktionswertes bei einer Verschiebung auf der Fläche. Der Massenpunkt ist dabei vom Punkt $P$ in den Punkt $Q$ gewandert.
\begin{equation}
\boxed{P=\left(x_P; y_P; z_P\right)\Longrightarrow Q=\left(x_P+\triangle x; y_P+\triangle y; z_P+\triangle z\right)}
\end{equation}
\subsubsection{Verschiebung des Massenpunktes auf der Tangtentialebene}
Die mit einer Verschiebung der Masse auf der Tangentialebene verbundenen Koordinatenänderungen bezeichnet man jetzt der Reihe nach mit $\text{d}x$, $\text{d}y$ und $\text{d}z$. Dabei soll der Massenpunkt so auf der Tangentialebene verschoben werden, dass sich seine beiden unabhängigen Koordinaten wiederum um $\triangle x$ bzw. $\triangle y$ ändern.
\newline\newline 
Die Änderung der Höhenkoordinate des Massenpunktes lässt sich dann leicht aus der Funktionsgleichung der Tangentialebene berechnen. So setzt man
\begin{equation}
\boxed{x-x_P=\text{d}x}\quad \boxed{y-y_P=\text{d}y}\quad \boxed{z-z_P=\text{d}z}
\end{equation}
\begin{equation}
\boxed{\text{d}z=\dfrac{\partial}{\partial x}\Big[f\left(x_P; y_P\right)\Big]\cdot \text{d}x+\dfrac{\partial}{\partial y}\Big[f\left(x_P; y_P\right)\Big]\cdot \text{d}y}
\end{equation}
Diese Grösse beschreibt den Zuwachs der Höhenkoordinate $z$ bei einer Verschiebung auf der Tangentialebene. Der Massenpunkt ist dabei vom Ausgangspunkt $P$ in den Punkt $Q'$ gewandert, der zwar auf der Tangentialebene, im Allgemeinen aber nicht auf der Fläche liegt
\begin{equation} 
\boxed{P=\left(x_P; y_P; z_P\right)\Longrightarrow Q'=\left(x_P+\text{d}x; y_P+\text{d}y; z_P+\text{d}z\right)}
\end{equation} 
Es ist somit $\triangle x = \text{d}x$ und $\triangle y = \text{d}y$ aber $\triangle z \neq \text{d}z$. Bei geringfügigen Verschiebungen, d.h. für kleine Werte von $\triangle x = \text{d}x$ und $\triangle y = \text{d}y$ gilt dann näherungsweise
\begin{equation}
\boxed{\triangle z\approx \text{d}z=\dfrac{\partial}{\partial x}\Big[f\left(x_P; y_P\right)\Big]\cdot \triangle x+\dfrac{\partial}{\partial y}\Big[f\left(x_P; y_P\right)\Big]\cdot \triangle y}
\end{equation}
Man darf unter diesen Voraussetzungen die Fläche $z=f\left(x; y\right)$ in der unmittelbaren Umgebung des Berührungspunktes $P$ durch die zugehörige Tangentialebene ersetzen. Diese Näherung wird in der Linearisierung von Funktionen und Kennlinienfeldern und bei der Fehlerfortpflanzung in Gebrauch gemacht.
\newline\newline
Unter dem \textbf{totalen Differential einer Funktion} $z=f\left(x; y\right)$ von zwei unabhängigen Variablen wird folgender lineare Differentialausdruck verstanden.  
\begin{equation}
\boxed{\text{d}z=\dfrac{\partial}{\partial x}\Big[f\left(x; y\right)\Big]\cdot \text{d}x+\dfrac{\partial}{\partial y}\Big[f\left(x; y\right)\Big]\cdot \text{d}y}
\end{equation}
Das totale Differential einer Funktion $z=f\left(x; y\right)$ beschreibt die Änderung der Höhenkoordinate bzw. des Funktionswertes $z$ auf der im Berührungspunkt $P$ errichteten Tangentialebene. Dabei sind die Differentiale $\text{d}x$, $\text{d}y$ und $\text{d}z$ die Koordinaten eines beliebigen Punktes auf der Tangentialebene.
\subsection{Anwendungen}
\subsubsection{Implizite Differentiation}
Hiermit werden ausgehend von impliziten Funktionen der Form $F\left(x; y\right)=0$ ausgegangen und die durch diese Gleichung definierte Kurve als Schnittlinie der Fläche $z=F\left(x; y\right)$ mit der $x$-, $y$-Ebene $z=0$ aufgefasst. Die Kurve $F\left(x; y\right)=0$ ist die \textbf{Schnittkante} der Fläche $z=F\left(x; y\right)$ mit der $x$, $y$-Ebene $z=0$. 
\newline\newline
Unter bestimmten Voraussetzungen ist es dann möglich, den Kurvenanstieg durch die partiellen Ableitungen erster Ordnung von $z=F\left(x; y\right)$ auszudrücken. Zu diesem Zweck bildet man das totale Differential der Funktion $z=F\left(x; y\right)$
\begin{equation}
\boxed{\text{d}z=\dfrac{\partial}{\partial x}\Big[F\left(x; y\right)\Big]\cdot \text{d}x+\dfrac{\partial}{\partial y}\Big[F\left(x; y\right)\Big]\cdot \text{d}y}
\end{equation}
Für die Punkte der \textbf{Schnittpunkte} ist $z=0$ und somit auch $\text{d}z=0$. Dann folgen
\begin{equation}
\boxed{\dfrac{\partial}{\partial x}\Big[F\left(x; y\right)\Big]\cdot \text{d}x+\dfrac{\partial}{\partial y}\Big[F\left(x; y\right)\Big]\cdot \text{d}y=0}
\end{equation}
Teilt man das Polynom durch $\text{d}x$ und durch Auflösen erhält man
\begin{equation}
\boxed{\dfrac{-\dfrac{\partial}{\partial x}\Big[F\left(x; y\right)\Big]}{\dfrac{\partial}{\partial y}\Big[F\left(x; y\right)\Big]}=\dfrac{\text{d}y}{\text{d}x}\quad \left(\dfrac{\partial}{\partial y}\Big[F\left(x; y\right)\Big]\neq 0\right)}
\end{equation}
Der Anstieg einer in der impliziten Form $F\left(x; y\right)=0$ dargestellten Funktionskurve im Kurvenpunkt $P=\left(x_P; y_P\right)$ lässt sich mit Hilfe der partiellen Fifferentiation bestimmen
\begin{equation}
\boxed{\dfrac{\text{d}y}{\text{d}x}\left(x_P; y_P\right)=\dfrac{-\dfrac{\partial}{\partial x}\Big[F\left(x_P; y_P\right)\Big]}{\dfrac{\partial}{\partial y}\Big[F\left(x_P; y_P\right)\Big]}}
\end{equation}
\subsubsection{Linearisierung einer Funktion}
Eine nichtlineare Funktion $y=f\left(x\right)$ lässt sich in einer Umgebung eines Punktes $P=\left(x_P; y_P\right)$ durch eine lineare Funktion bzw. durch eine Kurventangente annähern. Eine Funktion $z=f\left(x; y\right)$ von zwei unabhängigen Variablen lässt sich unter bestimmten Voraussetzungen in der unmittelbaren Umgebung eines Flächenpunktes $P=\left(x_P; y_P; z_P\right)$ linearisieren bzw. durch eine lineare Funktion vom Typ $z=ax+by+cz$ näherungsweise ersetzt werden. Als Ersatzfunktion wählt man die Tangentialebene in $P$. Der Punkt $P$ wird in naturwissenschatlich-technischen Bereich als \textbf{Arbeitspunkt} bekannt.
\newline\newline
Linearisierung einer Funktion $z=f\left(x; y\right)$ bedeutet also, dass man die gekrümmte Bildfläche von $z=f\left(x; y\right)$ in der unmittelbaren Umgebung des Arbeitspunktes $P$ durch die dortige Tangentialebene ersetzt werden kann. Die nichtlineare Funktion $z=f\left(x; y\right)$ wird durch die \textbf{Tangentialebene} bzw. durch das \textbf{totale Differential} angenähert, wobei $\triangle x$, $\triangle y$ und $\triangle z$ die Abweichungen eines beliebigen Flächenpunktes gegenüber dem Arbeitspunkt $P$ sind.
\begin{equation}
\boxed{
\begin{array}{lll}
z-z_P&=&\dfrac{\partial}{\partial x}\left(x_P; y_p\right)\cdot \left(x-x_P\right)+\dfrac{\partial}{\partial y}\left(x_P; y_P\right)\cdot \left(y-y_P\right)\\
\triangle z&=&\dfrac{\partial}{\partial x}\left(x_P; y_p\right)\cdot \triangle x+\dfrac{\partial}{\partial y}\left(x_P; y_P\right)\cdot \triangle y\\
\underbrace{\triangle z}_{w}&=&\underbrace{\left(\dfrac{\partial f}{\partial x}\right)_P}_{a}\cdot \underbrace{\triangle x}_{u}+\underbrace{\left(\dfrac{\partial f}{\partial y}\right)_P}_{b}\cdot \underbrace{\triangle y}_{v}\\
\end{array}
}
\end{equation}
In der Automation sind $u$, $v$ und $w$ die Abweichungen gegenüber dem Arbeitspunkt $P$, d.h. $P$ ist Koordinatenursprung des neuen Koordinatensystems, also die Relativkoordinaten, während $a$ und $b$ die Werte der beiden partiellen Ableitungen erster Ordnung im Arbeitspunkt $P$. 
\newline\newline
Eine Funktion von $n$ unabhängigen Variablen lässt sich linearisieren. In der unmittelbaren Umgebung des Arbeitspunktes $P$ kann die Funktion $y=f\left(x_1; x_2;\dotso; x_n\right)$ näherungsweise durch das \textbf{totale Differential} ersetzt werden. Die Grössen $\triangle x_i$ sind die Änderungen der unabhängigen Variablen bezogen auf den Arbeitspunkt.
\begin{equation}
\boxed{\triangle y=\left(\dfrac{\partial f}{\partial x_1}\right)_P\triangle x_1+\left(\dfrac{\partial f}{\partial x_2}\right)_P\triangle x_2+\dotso+\left(\dfrac{\partial f}{\partial x_n}\right)_P\triangle x_n}
\end{equation}
\subsubsection{Relative oder lokale Extremwerte}
Eine Funktion $z=f\left(x; y\right)$ besitzt an der Stelle $\left(x_P; y_P\right)$ ein \textbf{relatives Maximum} bzw. \textbf{relatives Minimum}, wenn in einer gewissen Umgebung von $\left(x_P; y_P\right)$ stets gilt
\begin{equation}
\boxed{f\left(x_P; y_P\right)>f\left(x; y\right)}\quad \boxed{f\left(x_P; y_P\right)<f\left(x; y\right)}
\end{equation}
Die relativen Maxima und Minima einer Funktion werden unter dem Sammelbegriff ``Relative Extremwerte'' zusammengefasst. Die den Extremwerten entsprechenden Flächenpunkte heissen \textbf{Hoch-} bzw- \textbf{Tiefpunkte}. Ein relativer Extremwert wird auch als lokaler Extremwert bezeichnet, da die extreme Lage meist nur in der unmittelbaren Umgebung, also im lokalen Bereich zutrifft. Ist die Ungleichung an jeder Stelle $\left(x; y\right)$ des Definitionsbereiches von $z=f\left(x; y\right)$ erfüllt, so liegt an der Stelle $\left(x_P; y_P\right)$ ein \textbf{absolutes Maximum} bzw. \textbf{absolutes Minimum} vor. 
\newline\newline
In einem relativen Extremum $\left(x_P; y_P\right)$ der Funktion $z=f\left(x; y\right)$ besitzt die zugehörige Bildfläche stets eine zur $x$, $y$-Ebene parallele Tangentialebene. Folgende Bedingungen sind \textbf{notwendige Voraussetzungen} für die Existenz eines relativen Extremwertes an der Stelle $\left(x_P; y_P\right)$
\begin{equation}
\boxed{\dfrac{\partial f}{\partial x}\left(x_P; y_P\right)=0}\quad \boxed{\dfrac{\partial f}{\partial y}\left(x_P; y_P\right)=0}
\end{equation}
Eine Funktion $z=f\left(x; y\right)$ besitzt an der Stelle $\left(x_P; y_P\right)$ mit Sicherheit einen relativen Extremwert, wenn folgenden \textbf{hinreichenden Voraussetzungen} erfüllt sind
\begin{equation}
\boxed{\triangle =\dfrac{\partial^2 f}{\partial x^2}\left(x_P; y_P\right)\cdot \dfrac{\partial^2 f}{\partial y^2}\left(x_P; y_P\right)-\dfrac{\partial^2 f}{\partial x\partial y}\left(x_P; y_P\right)>0}
\end{equation}
Das Vorzeichen von $\dfrac{\partial^2 f}{\partial x^2}\left(x_P; y_P\right)$ entscheidet dann über die Art des Extremwertes
\begin{equation}
\boxed{\dfrac{\partial^2 f}{\partial x^2}\left(x_P; y_P\right)<0\Longrightarrow \text{Relatives Maximum}}
\end{equation}
\begin{equation}
\boxed{\dfrac{\partial^2 f}{\partial x^2}\left(x_P; y_P\right)>0\Longrightarrow \text{Relatives Minimum}}
\end{equation}
Für $\triangle =0$ liegt kein Extremwert, sondern ein Sattelpunkt liegt vor. Für $\triangle =0$ versagt das Kriterium, d.h. an der Stelle $\left(x_P; y_P\right)$ kann nicht darüber diskutiert werden, ob es sich dabei um einen relativen Extremwert handelt oder nicht.
\subsubsection{Lagrangesches Multiplikatorverfahren zur Lösung einer Extremwertaufgabe mit Nebenbedingungen}
Die Extremwerte einer Funktion $z=f\left(x; y\right)$, deren unabhängige Variable $x$ und $y$ einer Nebenbedingung $\varphi\left(x; y\right)=0$ unterworfen sind, lassen sich mit Hilfe des Lagrangeschen Multiplikatorverfahrens schrittweise wie folgt bestimmen.
\newline \newline
Aus der Funktionsgleichung $z=f\left(x; y\right)$ und der Nebenbedingung $\varphi\left(x; y\right)=0$ wird zunächst die Hilfsfunktion gebildet. Der noch unbekannte Faktor $\lambda$ heisst \textbf{Lagrangescher Multiplikator}.
\begin{equation} 
\boxed{F\left(x; y; \lambda\right)=f\left(x; y\right)+\lambda\cdot \varphi\left(x; y\right)}
\end{equation} 
Dann werden die partiellen Ableitungen erster Ordnung dieser Hilfsfunktion gebildet und gleich Null gesetzt
\begin{equation}
\boxed{
\begin{array}{lllll}
\dfrac{\partial}{\partial x}\Big[F\left(x; y\right)\Big]&=&\dfrac{\partial}{\partial x}\Big[f\left(x; y\right)\Big]+\lambda\cdot \dfrac{\partial}{\partial x}\Big[\varphi\left(x; y\right)\Big]&=&0\\\\
\dfrac{\partial}{\partial y}\Big[F\left(x; y\right)\Big]&=&\dfrac{\partial}{\partial y}\Big[f\left(x; y\right)\Big]+\lambda\cdot \dfrac{\partial}{\partial y}\Big[\varphi\left(x; y\right)\Big]&=&0\\\\
\dfrac{\partial}{\partial \lambda}\Big[F\left(x; y\right)\Big]&=&\varphi\left(x; y\right)&=&0\\\\
\end{array}
}
\end{equation}
Der Lagrangeschen Multiplikator $\lambda$ ist eine Hilfsgrösse und daher meist ohne nähere Bedeutung. Er sollte daher möglichst früh aus den Rechnungen eliminiert werden. Die obigen Bedingungen sind nicht hinreichend für die Existenz eines Extremwertes unter der Nebenbedingung $\varphi\left(x; y\right)=0$.
\newline\newline
Für Funktionen von $n$ Variablen $x_1$, $x_2$, $\dotso$, $x_n$ mit $m$ Nebenbedingungen mit $\left(m<n\right)$ bildet man die Hilfsfunktion
\begin{equation}
\boxed{F\left(x_1; \dotso; x_n; \lambda_1; \dotso; \lambda_m \right)=f\left(x_1; \dotso; x_n\right)+\displaystyle \sum_{i=1}^n\lambda_i\cdot \varphi_i\left(x_1; \dotso; x_n\right)}
\end{equation}
und setzt man die $\left(n+m\right)$ partiellen Ableitungen erster Ordnung dieser Funktion der Reihe nach gleich Null
\begin{equation}
\boxed{\dfrac{\partial}{\partial x_i}\Big[F\left(x_1; \dotso; x_n; \lambda_1; \dotso; \lambda_m \right)\Big]=0}\quad \boxed{\dfrac{\partial}{\partial \lambda_i}\Big[F\left(x_1; \dotso; x_n; \lambda_1; \dotso; \lambda_m \right)\Big]=0}
\end{equation}
Aus diesen $\left(n+m\right)$ Gleichungen lassen sich dann die $\left(n+m\right)$ Unbekannten $x_1$, $x_2$, $\dotso$, $x_n$, $\lambda_1$, $\lambda_2$, $\dotso$, $\lambda_m$ berechnen. 
\subsubsection{Lineare Fortpflanzung - Direkte Messung}
Das Messergebnis einer aus $n$ Meswerten bestehenden Messreihe $x_1$, $x_2$, $\dotso$, $x_n$ wird in folgender Form ausgedrückt
\begin{equation}
\boxed{x=\overline{x}\pm \triangle x}\quad \boxed{\overline{x}=\dfrac{1}{n}\cdot \displaystyle \sum_{i=1}^nx_i}\quad \boxed{\triangle x=s_{\overline{x}}=\sqrt{\dfrac{1}{n\left(n-1\right)}\cdot \displaystyle \sum_{i=1}^n\left(x_i-\overline{x}\right)^2}}
\end{equation}
wobei $\overline{x}$ das arithmetische Mittel der $n$ Einzelwerte, $s_{\overline{x}}$ die Standardabweichung des Mittelwertes und $\triangle x$ die \textbf{Messunsicherheit} der Grösse $x$. Die auftretende Summe $\displaystyle \sum_{i=1}^n\left(x_i-\overline{x}\right)^2$ heisst Summe der \textbf{Abweichungsquadrate}.
\subsubsection{Lineare Fortpflanzung - Indirekte Messung}
Das Messergebnis zweier direkt gemessener Grössen $x$ und $y$ liege in der Form
\begin{equation}
\boxed{x=\overline{x}\pm\triangle x=\overline{x}\pm s_{\overline{x}}}\quad \boxed{y=\overline{y}\pm\triangle y=\overline{y}\pm s_{\overline{y}}}
\end{equation}
Dabei sind $\overline{x}$ und $\overline{y}$ die arithmetischen Mittelwerte und $\triangle x$ und $\triangle y$ die Messunsicherheiten der beiden Grössen, für die man in diesem Zusammenhang meist die Standardabweichung $s_{\overline{x}}$ und $s_{\overline{y}}$ der beide Mittelwerte heranzieht. 
\newline\newline
Die von den direkten Messgrössen $x$ und $y$ abhängige indirekte Messgrösse $z=f\left(x; y\right)$ besitzt dann den Mittelwert
\begin{equation}
\boxed{\overline{z}=f\left(\overline{x}; \overline{y}\right)}
\end{equation}
Mit Hilfe des totalen Differentials der Funktion gelingt es dann, ein sogenannter Fehlerfortpflanzungsgesetz herzuleiten, d.h. eine Beziehung, die darüber Aufschluss gibt, wie sich die Messunsicherheiten $\triangle x$ und $\triangle y$ der beiden unabhängigen Messgrössen $x$ und $y$ auf die Messunsicherheit $\triangle z$ der abhängigen Grösse $z=f\left(x; y\right)$ auswirken. Zu diesem Zweck bildet man das totale Differential der Funktion $z=f\left(x; y\right)$ an der Stelle $x=\overline{x}$ und $y=\overline{y}$
\begin{equation}
\boxed{\text{d}z=\dfrac{\partial f}{\partial x}\left(\overline{x}; \overline{y}\right)\text{}d x+\dfrac{\partial f}{\partial y}\left(\overline{x}; \overline{y}\right)\text{d} y}
\end{equation}
Die Differentiale $\text{d}x$ und $\text{d}y$ deuten als Messunsicherheiten $\triangle x$ und $\triangle y$ der beiden unabhängigem Messgrössen $x$ und $y$. Das totale Differential $\text{d}z$ liefert einen Näherungswert für die Messunsicherheit $\triangle z$.
\begin{equation}
\boxed{\triangle z_{\text{max}}=\Big\vert \dfrac{\partial f}{\partial x}\left(\overline{x}; \overline{y}\right)\triangle x\Big\vert+\Big\vert \dfrac{\partial f}{\partial y}\left(\overline{x}; \overline{y}\right)\triangle y\Big\vert}
\end{equation}
Das Messergebnis für die indirekte Messgrösse $z=f\left(x; y\right)$ wird dann in der Form
\begin{equation}
\boxed{z=\overline{z}\pm \triangle z_{\text{max}}}
\end{equation}
\section{Doppelintegrale}
Ein Mehrfachintegral besteht aush mehreren nacheinander auszuführende gewöhnliche Integrationen. Legt man ein Koordinatensystem zugrunde, das sich der Symmetrie des Problems in besonders günstiger Weise anpasst, so verainfacht sich die Berechnung der Integrale oft erheblich. Bei ebenen Problemen mit Kreissymmetrie etwa wird man daher vorzugsweise \textbf{Polarkoordinaten}, bei rotationssymmetrischen Problemen zweckmässigerweise \textbf{Zylinderkoordinaten} verwenden.
\newline\newline
Der Begriff des Doppelintegrals lässt sich anhand eines geometrischen Problems einführen. Sei $z=f\left(x; y\right)$ im Bereich $\left(A\right)$ eine definierte und stetige Funktion mit $f\left(x; y\right)\geq 0$. Sein Boden besteht aus dem Bereich $\left(A\right)$ und die Mantellinien verlaufen parallel zur $z$-Achse. Man ist hier für das Volumen $V$ interessiert.
\newline\newline
Der Bereich $\left(A\right)$ wird in $n$ Teilbereiche it den Flächeninalten $\triangle A_i$ zerlegt. Der Zylinder selbst erfällt dabei in eine leich grosse Anzahl von Röhren. Betrachte man den $k$-ten Rohr mit flachen Boden $\triangle A_k$ und gekrümmter Deckel als Funktion $z=f\left(x; y\right)$. Das Volumen diesen Rohr stimmt dann mit dem Volumen einer Säule überein, die über der gleichen Grundfläche errichtet wird und deren Höhe durch die Höhenkoordinate $z_k=f\left(x_k; y_k\right)$ des Flächenpunktes $P_k=\left(x_k; y_k\right)$ gegeben ist.
\begin{equation}
\boxed{\triangle V_k\approx \left(\triangle A_k\right)z_k=z_k\triangle A_k=f\left(x_k; y_k\right)\triangle A_k}
\end{equation}
Dieser Näherungswert lässt sich noch verbessern, wenn man in geeigneter Weise die Anzahl der Röhren vergrössert. Die Anzahl der Teilbereiche $n$ wachsen unbegrenzt $\left(n\rightarrow \infty\right)$, wobei gleichzeitig der Durchmesser eines jeden Teilbereiches gegen Null streben soll. Bei diesem Grenzübergang strebt die Summe gegen einen Grenzwert über den Bereich $\left(A\right)$.
\newline\newline
Bei diesem Grenzübergang strebt die Summe gegen einen Bereich $\left(A\right)$ oder kurz \textbf{Doppelintegral}, \textbf{Flächenintegral} oder \textbf{2-dimensionales Gebietsintegral} und darf als Volumen $V$ des Körpers interpretiert werden. Hier sind $x$ und $y$ die Integrationsvariablen, $f\left(x; y\right)$ die Integrandfunktion oder der Integrand, $\text{d}A$ das Flächendifferential oder Flächenelement und $\left(A\right)$ der Integrationsbereich.
\begin{equation}
\boxed{\displaystyle \lim_{\substack{n\rightarrow \infty\\\triangle A_k\rightarrow 0}} \displaystyle \sum_{k=1}^nf\left(x_k; y_K\right)\triangle A_k}
\end{equation}
\subsection{Doppelintegral in kartesische Koordinaten}
Die Berechnung des Doppelintegrals wird in kartesische Koordinaten betrachtet. Ein Integrationsbereich lässt sich durch die Ungleichungen
\begin{equation}  
\boxed{f_u\left(x\right)\leq y\leq f_o\left(x\right),\quad a\leq x\leq b}
\end{equation}
wobei $f_u\left(x\right)$ die untere und $f_o\left(x\right)$ die obere Randkurve ist und die seitlichen Begrenzungen aus zwei Parallelen zur $y$-Achsen mit den Funktionsgleichungen $x=a$ und $x=b$ bestehen. Das Flächenelement $\text{d}A$ besitzt in der kartesischen Darstellung die geometrische Form eines achsenparallelen Rechtecks mit den infinitesimal kleinen Seitenlängen $\text{d}x$ und $\text{d}y$. Somit gilt
\begin{equation}
\boxed{\text{d}A=\text{d}x\,\text{d}y=\text{d}y\,\text{d}x}
\end{equation}
Über den Flächenelement liegt eine quaderförmige Säule mit dem infinitesimal kleinen Rauminhalt. Das Volumen $V$ des Zylinders berechnet man schrittweise durch Summation der Säulenvolumina.
\begin{equation}
\boxed{\text{d}V=z\,\text{d}A=f\left(x; y\right)\,\text{d}x\,\text{d}y=f\left(x; y\right)\,\text{d}y\,\text{d}x}
\end{equation}
Die Berechnung des Doppelintegrals erfolgt durch zwei nacheinander auszuführende gewöhnliche Integrationen. 
\newline\newline
\textbf{Innere Integration:} Die Variable $x$ wird zunächst als eine Art Konstante betrachtet und die Funktion $f\left(x; y\right)$ unter Verwendung der für gewöhnliche Integrale gültigen Regeln nach der Variablen $y$ integriert. In die ermittelte Stammfunktion setzt man dann für $y$ die Integrationsgrenzen $f_o\left(x\right)$ bzw. $f_u\left(x\right)$ ein und bildet die entsprechende Differenz. 
\newline\newline
\textbf{Äussere Integration:} Die als Ergebnis der inneren Integration erhaltene, nur noch von der Variablen $x$ abhängige Funktion wird nun in den Grenzen von $x=a$ bis $x=b$ integriert.
\begin{equation}
\boxed{V=\displaystyle \iint_{\left(A\right)}f\left(x; y\right)\,\text{d}A=\displaystyle \int_{x=a}^{b}\text{d}V_{\text{Scheibe}}=\underbrace{\displaystyle \int_{x=a}^b\underbrace{\left(\displaystyle \int_{y=f_a\left(x\right)}^{f_o\left(x\right)}f\left(x; y\right)\,\text{d}y\right)}_{\text{inneres Integral}}\,\text{d}x}_{\text{äusseres Integral}}}
\end{equation}
im Allgemeinen gilt: bei einer Vertauschung der Integrationsreihenfolge müssen die Integrationsgrenzen jeweils neu bestimmt werden.
\subsection{Doppelintegral in Polarkoordinaten}
Die Berechnung des Doppelintegrals vereinfacht sich, wenn man an Stelle der kartesischen Koordinaten $x$ und $y$ die Polarkoordinaten $r$ und $\varphi$ verwendet. Zwischen ihnen besteht dabei der folgende Zusammenhang
\begin{equation} 
\boxed{x=r\cdot \cos\left(\varphi\right)}\quad \boxed{y=r\cdot \sin\left(\varphi\right)}\quad \boxed{r\geq 0}\quad \boxed{0\leq \varphi\leq 2\pi}
\end{equation}
Die Gleichung einer Kurve lautet in Polarkoordinaten $r=f\left(\varphi\right)$ oder $r=r\left(\varphi\right)$. Eine von zwei Variablen $x$ und $y$ abhängige Funktion $z=f\left(x; y\right)$ geht bei der \textbf{Koordinatentransformation} in die von $r$ und $\varphi$ abhängige Funktion über
\begin{equation}
\boxed{z=f\left(x; y\right)=f\left(r\cos\left(\varphi\right); r\sin\left(\varphi\right)\right)=F\left(r; \varphi\right)}
\end{equation}
Die bei Doppelintegralen in Polarkoordinatendarstellung auftretenden Integrationsbereiche $\left(A\right)$ besitzt die Gestalt zwei Strahlen $\varphi=\varphi_1$ und $\varphi=\varphi_2$ sowie einer inneren Kurve $r=r_i\left(\varphi\right)$ und einer äusseren Kurve $r=r_a\left(\varphi\right)$ begrenzt und lassen sich durch die Ungleichungen beschreiben
\begin{equation}
\boxed{r_i\left(\varphi\right)\leq r\leq r_a\left(\varphi\right)}\quad \boxed{\varphi_1\leq \varphi\leq \varphi_2} 
\end{equation}
Das Flächenelement $\text{d}A$ wird in der Polarkoordinaten von zwei infinitesimal benachbarten Kreisen mit den Radien $r$ und $r+\text{d}r$ und zwei infinitesimal benachbarten Strahlen mit den Polarwinkeln $\varphi$ und $\varphi+\text{d}\varphi$ berandet. Dabei gilt
\begin{equation}
\boxed{\text{d}A=\left(r\,\text{d}\varphi\right)\text{d}r=r\,\text{d}r\,\text{d}\varphi}
\end{equation}
Das Doppelintegral besitzt somit in Polarkoordinaten das folgende Aussehen und die Berechnung erfolgt wiederum von innen nach aussen, zuerst radial zwischen den Randkurven $r=r_i\left(\varphi\right)$ und $r=r_a\left(\varphi\right)$ integriert und anschliessend in Winkelrichtung $\varphi=\varphi_1$ und $\varphi=\varphi_2$. Wird die Reihenfolge der Integrationen geändert, dann müssen auch die Integrationsgrenzen neu bestimmt werden.
\begin{equation} 
\boxed{\displaystyle \iint_{\left(A\right)}f\left(x; y\right)\,\text{d}A=\displaystyle \int_{\varphi=\varphi_1}^{\varphi=\varphi_2}\displaystyle \int_{r=r_i\left(\varphi\right)}^{r_a\left(\varphi\right)}f\left(r\cdot \cos\left(\varphi\right); r\cdot \sin\left(\varphi\right)\right)\cdot \underbrace{r\,\text{d}r\,\text{d}\varphi}_{\text{d}A}}
\end{equation} 
\subsection{Anwendungen der Doppelintegrale}
\subsubsection{Flächeninhalt}
Der Flächeninhalt $A$ eines Normalbereichs $\left(A\right)$ lässtsich nach dem Baukastenprinzip aus infinitesimal kleinen rechteckigen Flächenelementen vom Flächeninhalt $\text{d}A=\text{d}y\,\text{d}x$ zusammensetzen. Man betrachte einen in der Fläche liegenden und zur $y$-Achse parallelen Streifen der Breite $\text{d}x$. Der Flächeninhalt eines Streifens erhält man, indem man den Flächeninhalt sämtlicher im Streifen gelegener Flächenelemente aufsummiert. Die Summation der Flächenelemente bedeutet eine Integration in der $y$-Richtung zwischen der unteren Grenze $y=f_u\left(x\right)$ und der oberen Grneze $y=f_o\left(x\right)$. Der Flächeninhalt eines solchen infinitesimal schmalen Streifens beträgt
\begin{equation}
\boxed{\text{d}A_{\text{Streifen}}=\displaystyle \int_{y=f_u\left(x\right)}^{f_o\left(x\right)}\text{d}A=\left(\displaystyle \int_{y=f_u\left(x\right)}^{f_o\left(x\right)}\text{d}y\right)\text{d}x}
\end{equation}
Jetzt summieren über sämtliche Streifenelemente $x=a$ und $x=b$. Der Flächeninhalt in \textbf{kartesische Koordinaten} $A$ lautet
\begin{equation}
\boxed{A=\displaystyle \iint_{\left(A\right)}\text{d}A=\displaystyle \int_{x=a}^{b}\text{d}A_{\text{Streifen}}=\displaystyle \int_{x=a}^b\displaystyle \int_{y=f_u\left(x\right)}^{f_o\left(x\right)}1\,\text{d}y\,\text{d}x}
\end{equation}
Bei Verwendung von \textbf{Polarkoordinaten} lautet 
\begin{equation}
\boxed{A=\displaystyle \int_{\varphi=\varphi_1}^{\varphi=\varphi_2}\displaystyle \int_{r=r_i\left(\varphi\right)}^{r=r_a\left(\varphi\right)}r\,\text{d}r\,\text{d}\varphi}
\end{equation}
\subsubsection{Schwerpunkt einer homogenen Fläche}
Für die Schwerpunktskoordinaten $x_S$ und $y_S$ einer homogenen ebenen Fläche vom Flächenihalt $A$ bei Verwendung kartesischer Koordinaten ist für das Flächenelement $\text{d}A=\text{dy}\,\text{d}x$ zu setzen, bei Verwendung von Polarkoordinaten setzt man $x=r\cdot \cos\left(\varphi\right)$, $y=r\cdot \sin\left(\varphi\right)$ und $\text{d}A=r\,\text{d}r\,\text{d}\varphi$
\begin{equation}
\boxed{x_S=\dfrac{1}{A}\displaystyle \iint_{\left(A\right)}x\,\text{d}A}\quad \boxed{y_S=\dfrac{1}{A}\displaystyle \iint_{\left(A\right)}y\,\text{d}A}
\end{equation}
Für \textbf{kartesische Koordinaten} lauten die Koordinaten des Schwerpunktes
\begin{equation}
\boxed{x_S=\dfrac{1}{A}\displaystyle \int_{x=a}^{b}\displaystyle \int_{y=f_u\left(x\right)}^{f_o\left(x\right)}x\,\text{d}y\,\text{d}x}\quad \boxed{y_S=\dfrac{1}{A}\displaystyle \int_{x=a}^{b}\displaystyle \int_{y=f_u\left(x\right)}^{f_o\left(x\right)}y\,\text{d}y\,\text{d}x}
\end{equation}
Für \textbf{Polarkoordinaten} lauten die Koordinaten des Schwerpunktes
\begin{equation}
\boxed{x_S=\dfrac{1}{A}\displaystyle \int_{\varphi=\varphi_1}^{\varphi_2}\displaystyle \int_{r=r_i\left(\varphi\right)}^{r_a\left(\varphi\right)}r^2\cdot \cos\left(\varphi\right)\,\text{d}r\,\text{d}\varphi}\quad \boxed{y_S=\dfrac{1}{A}\displaystyle \int_{\varphi=\varphi_1}^{\varphi_2}\displaystyle \int_{r=r_i\left(\varphi\right)}^{r_a\left(\varphi\right)}r^2\cdot \sin\left(\varphi\right)\,\text{d}r\,\text{d}\varphi}
\end{equation}
\subsubsection{Flächenmomente}
Flächenmomente sind Grössen, die im Zusammenhang mit Biegeproblemen bei Balken und Trägern auftreten. Dabei wird noch zwischen axialen oder äquatorialen und polaren Flächenmomenten unterschieden. Bei einem axialen Flächenmoment liegt die Bezugsachse in der Flächenebene, während sie bei einem polaren Flächenmoment senkrecht zur Flächenebene orientiert ist.
\begin{equation} 
\boxed{\text{d}I_x=y^2\,\text{d}A}
\end{equation}
Man behandle zunächst die auf die Koordinatenachsen bezogenen Flächenmomente. Definitionsgemäss wird dabei die infinitesiml kleine Grösse als axiales Flächenelement von $\text{d}A$ bezüglich der $x$-Achse bezeichnet. Sie ist das Produkt aus dem Flächenelement $\text{d}A$ und dem Quadrat des Abstandes, den dieses Flächenelemtn von der $x$-Achse besitzt.
\newline\newline
Durch Integration über die Gesamtfläche $A$ erhält man hieraus das axiale Flächenmoment $I_x$ der Fläche $A$ bezüglich der $x$-Achse
\begin{equation}
\boxed{I_x=\displaystyle \int_{\left(A\right)}\text{d}I_x=\displaystyle \iint_{\left(A\right)}y^2\,\text{d}A}
\end{equation}
Analog wird das axiale Flächenmoment $I_y$ der Fläche $A$ bezüglich der $y$-Achse als Bezugsachse definiert. Ausgehend von dem Beitrag $\text{d}I_y=x^2\,\text{d}A$ eines Flächenelements $\text{d}A$ erhält man durch Integration das Flächenmoment der Gesamtfläche
\begin{equation}
\boxed{I_y=\displaystyle \iint_{\left(A\right)}\,\text{d}I_y=\displaystyle \iint_{\left(A\right)}x^2\,\text{d}A}
\end{equation}
Unter dem \textbf{polaren Flächenmoment} $I_p$ einer Fläche $A$, bezogen auf eine durch den Koordinatenursprung senkrecht zur Flächenebene verlaufende Bezugsachse ($z$-Achse), wird die wie folgt definierte Grösse verstanden wobei $\text{d}I_p=r^2\,\text{d}A$
\begin{equation}
\boxed{I_p=\displaystyle \iint_{\left(A\right)}\text{d}I_p=\displaystyle \iint_{\left(A\right)}r^2\,\text{d}A}
\end{equation}
Wegen des Satzes von Pythagoras besteht zwischen den beiden axialen und dem polaren Flächenmoment stets die Beziehung
\begin{equation}
\boxed{I_p=I_x+I_y}
\end{equation}
So gelten in \textbf{kartesische Koordinaten} 
\begin{equation}
\boxed{I_x=\displaystyle \int_{x=a}^{b}\displaystyle \int_{y=f_u\left(x\right)}^{f_o\left(x\right)}y^2\,\text{d}y\,\text{d}x}\quad \boxed{I_y=\displaystyle \int_{x=a}^{b}\displaystyle \int_{y=f_u\left(x\right)}^{f_o\left(x\right)}x^2\,\text{d}y\,\text{d}x}
\end{equation}
\begin{equation}
\boxed{I_p=\displaystyle \int_{x=a}^{b}\displaystyle \int_{y=f_u\left(x\right)}^{f_o\left(x\right)}\left(x^2+y^2\right)\,\text{d}y\,\text{d}x}
\end{equation}
und in \textbf{Polarkoordinaten}
\begin{equation}
\boxed{I_x=\displaystyle \int_{\varphi=\varphi_1}^{\varphi_2}\displaystyle \int_{r=r_i\left(x\right)}^{r_a\left(\varphi\right)}r^3\cdot \sin^2\left(\varphi\right)\,\text{d}r\,\text{d}\varphi}\quad \boxed{I_y=\displaystyle \int_{\varphi=\varphi_1}^{\varphi_2}\displaystyle \int_{r=r_i\left(x\right)}^{r_a\left(\varphi\right)}r^3\cdot \cos^2\left(\varphi\right)\,\text{d}r\,\text{d}\varphi}
\end{equation}
\begin{equation}
\boxed{I_p=\displaystyle \int_{\varphi=\varphi_1}^{\varphi_2}\displaystyle \int_{r=r_i\left(x\right)}^{r_a\left(\varphi\right)}r^3\,\text{d}r\,\text{d}\varphi}
\end{equation}
\section{Dreifachintegrale}
Die Integration einer Funtion von drei unabhängigen Variablen hat keine geometrische Interpretation.Sei $u=f\left(x; y; z\right)$ eine im räumlichen Bereich $\left(V\right)$ definierte und stetige Funktion. Den Bereich unterteilt man zunächst in $n$ Teilbereiche. Mit dem $k$-ten Teilbereich vom Volumen $\triangle V_k$ wird man sich eingehender befassen. In diesem Teilbereich wählt man einen beliebigen Punkt $P_k=\left(x_k; y_K; z_k\right)$ aus, berechnet an dieser Stelle den Funktionswert $u_k=f\left(x_k; y_k; z_k\right)$ und bildet schliesslich das Produkt aus Funktionswert und Volumen: $f\left(x_k; y_k; z_k\right)\cdot \triangle V_k$.
\newline\newline
Die Summe aller dieser Produkte wird Zwischensumme $Z_n$ genannt
\begin{equation}
\boxed{Z_n=\displaystyle \sum_{k=1}^n f\left(x_k; y_k; z_k\right)\triangle V_k}
\end{equation}
Man lässt die Anzahl $n$ der Teilbereiche unbegrenzt wachsen $\left(n\rightarrow \infty\right)$, wobei gleichzeitig der Durchmesser eines jeden Teilbereiches gegen Null gehen soll. 
\begin{equation}
\boxed{\displaystyle \iiint_{\left(V\right)}f\left(x; y; z\right)\,\text{d}V=\displaystyle \lim_{n\rightarrow \infty}Z_n=\displaystyle \lim_{n\rightarrow \infty}\displaystyle \sum_{k=1}^n f\left(x_k; y_k; z_k\right)\triangle V_k}
\end{equation}
Bei diesem Grenzübergang strebt die Zwischensumme $Z_n$ gegen einen Grenzwert, der als \textbf{3-dimensionales Bereichsintegral} von $f\left(x; y; z\right)$ über $\left(V\right)$ oder \textbf{Dreifachintegral} bezeichnet wird, wobei $x$, $y$ und $z$ die Integrationsvariablen, $f\left(x; y; z\right)$ die Integrandfunktion, $\text{d}V$ das Volumendifferential oder Volumenelement und $\left(V\right)$ der räumlicher Integrationsbereich oder Körper.
\subsection{Dreifachintegral in kartesische Koordinaten}
Der Berechnung eines Dreifachintegrals legt man zunächst ein kartesisches Koordinatensystem und einen zylinderischen Integrationsbereich $\left(V\right)$ zugrunde, der unten durch eine Fläche $z=z_u\left(x; y\right)$ und oben durch eine Fläche $z=z_o\left(x; y\right)$ begrenzt wird. Die Projektion dieser Begrenzungsflächen in die $x$, $y$-Ebene führt zu einem Bereich $\left(A\right)$, der durch die Kurven $y=f_u\left(x\right)$ und $y=f_o\left(x\right)$ sowie die Parallelen $x=a$ und $x=b$ berandet wird.
\newline\newline
Der zylindrische Integrationsbereich $\left(V\right)$ kann somit durch die Ungleichungen beschrieben werden
\begin{equation}
\boxed{z_u\left(x; y\right)\leq z\leq z_o\left(x; y\right)}\quad \boxed{f_u\left(x\right)\leq y\leq f_o\left(x\right)}\quad \boxed{a\leq x\leq b}
\end{equation}
Das Volumenelement $\text{d}V$ besitzt in der kartesischen Darstellung die geometrische Form eines Quaders mit den infinitesimal kleinen Seitenlängen $\text{d}x$, $\text{d}y$ und $\text{d}z$. Ein Dreifachintegral lässt sich durch drei nacheinander auszuführende gewöhnliche Integrationen durchführen
\begin{equation}
\boxed{\displaystyle \iiint_{\left(V\right)}f\left(x; y; z\right)\,\text{d}V=\underbrace{\displaystyle \int_{x=a}^b\underbrace{\left(\displaystyle \int_{y=f_a\left(x\right)}^{f_o\left(x\right)}\underbrace{\left(\displaystyle \int_{y=f_a\left(x\right)}^{f_o\left(x\right)}\,\text{d}y\right)}_{\text{Integration 1}}\,\text{d}y\right)}_{\text{Integration 2}}\,\text{d}x}_{\text{Integration 3}}}
\end{equation}
Dabei wird wie bei den Doppelintegralen von innen nach aussen integriert. Die Integrationsreihenfolge ist nur dann vertauschbar, wenn sämtliche Integrationsgrenzen konstant sind. Bei der Vertauschung der Integrationsreihenfolge müssen die Integrationsgrenzen neu bestimmt werden.
\subsection{Dreifachintegral in Zylinderkoordinaten}
In technischen Anwendungen treten häufig Körper mit Rotationssymmetrie auf. Zu ihrer Beschreibung verwendet man zweckmässigerweise Zylinderkoordinaten $\left(r; \varphi; z\right)$, die sich der Symmetrie des Körpers in besonderem Masse anpassen. Man spricht in diesem Zusammenhang auch von symmetriegerechten Koordinaten. Auch die Berechnung eines Dreifachintegrals lässt sich in zahlreichen Fällen bei Verwendung von Zylinderkoordinaten erheblich vereinfachen.
\newline\newline
Sei $P=\left(x; y; z\right)$ ein beliebiger Punkt des kartesischen Raumes, $P'=\left(x; y\right)$ der durch senkrechte Projektion von $P$ in die $x$, $y$-Ebene erhaltene Bildpunkt. Die Lage von $P'$ kann man auch durch die Polarkoordinaten $r$ und $\varphi$ beschreiben. Sie bestimmen zugleich zusammen mit der Höhenkoordinate $z$ in eindeutiger Weise die Lage des Raumpunktes $P$. Die drei Koordinaten $r$, $\varphi$ und $z$ werden als Zylinderkoordinaten von $P$ bezeichnet.
\newline\newline
Zwischen den kartesischen Koordinate und den Zylinderkoordinaten bestehen dabei die folgenden Umrechnungen
\begin{equation} 
\boxed{x=r\cdot \cos\left(\varphi\right)}\quad \boxed{y=r\cdot \sin\left(\varphi\right)}\quad \boxed{z=z}
\end{equation}
\begin{equation}
\boxed{r=\sqrt{x^2+y^2}}\quad \boxed{\tan\left(\varphi\right)=\dfrac{y}{x}}\quad \boxed{z=z}
\end{equation}
Das Volumenelement $\text{d}V$ besitzt in Zylinderkoordinaten die Form
\begin{equation}
\boxed{\text{d}V=\left(\text{d}A\right)\text{d}z=\left(r\,\text{d}r\,\text{d}\varphi\right)\text{d}z=r\,\text{d}z\,\text{d}r\,\text{d}\varphi}
\end{equation}
Für ein Dreifachintegral erhält man dann in Zylinderkoordinaten die Darstellung
\begin{equation}
\boxed{\displaystyle \iiint_{\left(V\right)}f\left(x; y; z\right)\,\text{d}V=\displaystyle \iiint_{\left(V\right)}f\left(r\cdot \cos\left(\varphi\right); r\cdot \sin\left(\varphi\right); z\right)\cdot \underbrace{r\,\text{d}z\,\text{d}r\,\text{d}\varphi}_{\text{d}V}}
\end{equation}
Die Mantelfläche eines rotationssymmetrischen Körpers entsteht durch Drehung einer Kurve $z=f\left(x\right)$ und die $z$-Achse, die damitauch zugleich Symmetrieachse ist. Bei der Rotation wird aus der kartesischen Koordinate $x$ die Zylinderkoordinate $r$ und die Kurvengleichung $z=f\left(x\right)$ geht dabei in die Funktionsgleichung $z=f\left(r\right)$ der Rotationsfläche $\left(x\rightarrow r\right)$ über.
\subsection{Anwendungen des Dreifachintegrals}
\subsubsection{Volumen und Masse eines Körpers}
Das Volumen eines zylindrischen Körpers mit Grundfläche $z=z_u\left(x; y\right)$ und Deckelfläche $z=z_o\left(x\right)$ kann man durch ein Dreifachintegral berechnen. Durch Projektion des Zylinders in die $x$, $y$-Ebene erhält man den Normalbereich $\left(A\right)$ durch die Kurven $y_u=f_u\left(x\right)$ und $y=f_o\left(x\right)$ sowie die Parallelen $x=a$ und $x=b$.
\begin{equation}
\boxed{V=\displaystyle \iiint_{\left(V\right)}\text{d}V=\displaystyle \iiint_{\left(V\right)}\text{d}z\,\text{d}y\,\text{d}x}
\end{equation}
Betrachte man nun ein infinitesimal kleines, im Körper gelegenes Volumenelement $\text{d}V$. Es besitzt in einem karteischen Koordinatensystem bekanntlich die Gestalt eines achsenparallelen Quaders mit den Kantenlängen $\text{d}x$, $\text{d}y$ und $\text{d}z$. Sein Volumen beträgt $\text{d}V=\text{d}x\,\text{d}y\,\text{d}z=\text{d}z\,\text{d}y\,\text{d}x$
\newline\newline
\textbf{Volumen einer Säule:} Betrachte man eine zur $z$-Achse parallele Säule mit der infinitesimal kleiinen Querschnittsfläche $\text{d}A=\text{d}x\,\text{d}y=\text{d}y\,\text{d}x$, indem Volumenelement über Volumenelement gesetzt wird, bis man an die beiden Begrenzungsflächen des Körpers stösst. Das Volumen dieser Säule erhält man durch Summation sämtlicher in der Säule gelegener Volumenelemente, d.h. durch Integration der Volumenelemente $\text{d}V$ in der $z$-Richtung zwischen den Grenzen $z=z_u\left(x; y\right)$ und $z=z_o\left(x; y\right)$. Das infinitesimal kleine Säulenvolumen beträgt
\begin{equation}
\boxed{\text{d}V_{\text{Säule}}=\displaystyle \int_{z=z_u\left(\right)}^{z=z_o\left(\right)}\text{d}V=\left(\displaystyle \int_{z=z_u\left(x; y\right)}^{z=z_o\left(x; y\right)}\text{d}z\right)\text{d}y\,\text{d}x}
\end{equation}
\textbf{Volumen einer Scheibe:} Legt man zur $y$-Richtung Säule an Säule, bis man an die Randkurven $y=f_u\left(x\right)$ bzw. $y=f_o\left(x\right)$ des Bereiches $\left(A\right)$ in der $x$, $y$-Ebene stossen und erhält eine Volumenschicht der Breite oder Dicke $\text{d}x$. Das Volumen $\text{d}V_{\text{Scheibe}}$ dieser infinitesimal dünnen Scheibe ergibt sich dann durch Summation der Säulenvolumina, d.h. durch INtegration der Säulenvolumina $\text{d}V_{\text{Säule}}$ in der $y$-Richtung zwischen den Grenzen $y=f_u\left(x\right)$ und $y=f_o\left(x\right)$.  
\begin{equation}
\boxed{\text{d}V_{\text{Säule}}=\displaystyle \int_{y=f_u\left(x\right)}^{f_o\left(x\right)}\text{d}V_{\text{Säule}}=\left(\displaystyle \int_{y=f_u\left(x\right)}^{f_o\left(x\right)}\left(\displaystyle \int_{z=z_u\left(x; y\right)}^{z_o\left(x; y\right)}\text{d}z\right)\text{d}y\right)\text{d}x}
\end{equation}
\textbf{Volumen des Zylinders:} Es wird in der $x$-Richtung Scheibe an Scheibe gelegt, bis der zylindrische Körper vollständig ausgefüllt ist. Das Zylindervolumen $V$ erhält man dann durch Summation über die Volumina sämtlicher Scheiben, d.h. durch Integration in der $x$-Richtung zwischen den Grenzen $x=a$ und $x=b$. Es gilt
\begin{equation}
\boxed{V=\displaystyle \iiint_{\left(V\right)}\text{d}V=\displaystyle \int_{x=a}^{b}\text{d}V_{\text{Scheibe}}=\displaystyle \int_{x=a}^b\displaystyle \int_{y=f_u\left(x\right)}^{f_o\left(x\right)}\displaystyle \int_{z=z_u\left(x; y\right)}^{z_o\left(x; y\right)}\text{d}z\,\text{d}y\,\text{d}x}
\end{equation}
\subsubsection{Schwerpunkt eines Körpers}
Die Berechnung des Schwerpunktes eines homogenen Körpers für die Schwerpunktkoordinaten $x_S$, $y_S$ und $z_S$ berechnet sich wie folgt
\begin{equation}
\boxed{x_S=\dfrac{1}{V}\cdot \displaystyle \iiint_{\left(V\right)}x\,\text{d}V}\quad \boxed{y_S=\dfrac{1}{V}\cdot \displaystyle \iiint_{\left(V\right)}y\,\text{d}V}\quad \boxed{z_S=\dfrac{1}{V}\cdot \displaystyle \iiint_{\left(V\right)}z\,\text{d}V}
\end{equation}
Somit handelt es sich um Dreifachintegrale. Die Schwerpunktkoordinaten in \textbf{kartesischen Koordinaten} sind
\begin{equation}
\boxed{x_S=\dfrac{1}{V}\cdot \displaystyle \int_{x=a}^{b}\displaystyle \int_{y=f_u\left(x\right)}^{f_o\left(x\right)}\displaystyle \int_{z=z_u\left(x; y\right)}^{z_o\left(x;y\right)}x\,\text{d}z\,\text{d}y\,\text{d}x}
\end{equation}
\begin{equation}
\boxed{y_S=\dfrac{1}{V}\cdot \displaystyle \int_{x=a}^{b}\displaystyle \int_{y=f_u\left(x\right)}^{f_o\left(x\right)}\displaystyle \int_{z=z_u\left(x; y\right)}^{z_o\left(x;y\right)}y\,\text{d}z\,\text{d}y\,\text{d}x}
\end{equation}
\begin{equation}
\boxed{z_S=\dfrac{1}{V}\cdot \displaystyle \int_{x=a}^{b}\displaystyle \int_{y=f_u\left(x\right)}^{f_o\left(x\right)}\displaystyle \int_{z=z_u\left(x; y\right)}^{z_o\left(x;y\right)}z\,\text{d}z\,\text{d}y\,\text{d}x}
\end{equation}
Bei einem rotationssymmetrischen Körper liegt der Schwerpunkt $S$ auf der Rotationsachse. Legt man das Koordinatensystem so, dass die Rotationsachse in die $z$-Achse fällt, dann  gilt in der \textbf{Zylinderkoordinaten}
\begin{equation}
\boxed{x_S=0}\quad \boxed{y_S=0}\quad \boxed{z_S=\dfrac{1}{V}\cdot \displaystyle \iiint_{\left(V\right)}z\,\text{d}z\,\text{d}r\,\text{d}\varphi}
\end{equation}
\subsubsection{Massenträgheitsmomente}
Das Massenträgheitsmoment ist eine von der Masse und der räumliche Verteilung um die Drehachse physikalische Grösse. Definitionsgemäss liefert ein Massenelement $\text{d}m$ des Körpers den infinitesimalen kleinen Beitrag zum Massenträgheitsmoment $J$ des Gesamtkörpers, bezogen auf eine bestimmte Achse $A$, wobei $r_A$ der senkrechte Abstand des Massen- bzw. Volumenelementes von der Bezugsachse $A$, $\rho$ die konstante Dichte des Körpers mit $\text{d}m=\rho\,\text{d}V$ 
\begin{equation}
\boxed{\text{d}J=r_A^2\,\text{d}m=r_A^2\left(\rho\,\text{d}V\right)=\rho \,r_A^2\,\text{d}V}
\end{equation}
Für das Massenträgheitsmoment eines homogenen Körpers erhält man dann durch Aufsummieren, d.h. Integration der Gleichung folgende Integralformel
\begin{equation}
\boxed{J=\rho\cdot \displaystyle \iiint_{\left(V\right)}r_A^2\,\text{d}V}
\end{equation}
In \textbf{kartesischen Koordinaten}
\begin{equation}
\boxed{J=\rho\cdot \displaystyle \int_{x=a}^{b}\displaystyle \int_{y=f_u\left(x\right)}^{f_o\left(x\right)}\displaystyle \int_{z=z_u\left(x; y\right)}^{z_o\left(x; y\right)}\left(x^2+y^2\right)\,\text{d}z\,\text{d}y\,\text{d}x}
\end{equation}
In \textbf{Zylinderkoordinaten} bezogen auf die Rotationsachse bzw. $z$-Achse
\begin{equation}
\boxed{J_z=\rho\cdot \displaystyle \iiint_{\left(V\right)}r^3\,\text{d}z\,\text{d}r\,\text{d}\varphi}
\end{equation}